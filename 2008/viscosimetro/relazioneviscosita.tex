\documentclass[italian,a4paper]{article}
\usepackage[tight,nice]{units}
\usepackage{babel,amsmath,amssymb,amsthm,graphicx,url}
\usepackage[text={6in,9in},centering]{geometry}
\usepackage[utf8x]{inputenc}
\usepackage[T1]{fontenc}
\usepackage{ae,aecompl}
\usepackage[Euler]{upgreek}
\usepackage[footnotesize,bf]{caption}
\usepackage[usenames]{color}
\include{pstricks}
\frenchspacing
\pagestyle{plain}
%------------- eliminare prime e ultime linee isolate
\clubpenalty=9999%
\widowpenalty=9999
%--- definizione numerazioni
\renewcommand{\theequation}{\thesection.\arabic{equation}}
\renewcommand{\thefigure}{\thesection.\arabic{figure}}
\renewcommand{\thetable}{\thesection.\arabic{table}}
\addto\captionsitalian{%
  \renewcommand{\figurename}%
{Grafico}%
}
%
%------------- ridefinizione simbolo per elenchi puntati: en dash
%\renewcommand{\labelitemi}{\textbf{--}}
\renewcommand{\labelenumi}{\textbf{\arabic{enumi}.}}
\setlength{\abovecaptionskip}{\baselineskip}   % 0.5cm as an example
\setlength{\floatsep}{2\baselineskip}
\setlength{\belowcaptionskip}{\baselineskip}   % 0.5cm as an example
%------------- nuovi environment senza spazi
%\newenvironment{packed_item}{
%\begin{itemize}
%  \setlength{\itemsep}{1pt}
%  \setlength{\parskip}{0pt}
%  \setlength{\parsep}{0pt}
%}{\end{itemize}}
%\newenvironment{packed_enum}{
%\begin{enumerate}
%  \setlength{\itemsep}{1pt}
%  \setlength{\parskip}{0pt}
%  \setlength{\parsep}{0pt}
%}{\end{enumerate}}
%\newenvironment{packed_description}{
%\begin{enumerate}
%   \setlength{\itemsep}{1pt}
%   \setlength{\parskip}{0pt}
%   \setlength{\parsep}{0pt}
% }{\end{enumerate}}
%--------- comandi insiemi numeri complessi, naturali, reali e altre abbreviazioni
\newcommand{\e}{\mathrm{e}} %numero di nepero
\newcommand{\di}{\mathrm{d}} %simbolo di differenziale
\renewcommand{\leq}{\leqslant}
\renewcommand{\pi}{\uppi} % costante pi greco
\renewcommand{\tau}{\uptau} %momento della forza
\newcommand{\pas}{\ensuremath{Pa\,s}} %pascal per secondo
\newcommand{\coloneqq}{\mathrel{\mathop:}=} % := ``per definizione''
\newcommand{\ms}{(\unitfrac{m}{s})}
%--------- porzione dedicata ai float in una pagina:
\renewcommand{\textfraction}{0.05}
\renewcommand{\topfraction}{0.95}
\renewcommand{\bottomfraction}{0.95}
\renewcommand{\floatpagefraction}{0.35}
\setcounter{totalnumber}{5}
%---------
%
%---------
\begin{document}
\title{Relazione di laboratorio: il viscosimetro}
\author{\normalsize Ilaria Brivio (582116)\\%
\normalsize \url{brivio.ilaria@tiscali.it}%
\and %
\normalsize Matteo Abis (584206)\\ %
\normalsize \url{webmaster@latinblog.org}}
\date{\today}
\maketitle
%------------------
\section{Obiettivo dell'esperienza}
Obiettivo dell'esperienza è misurare la viscosità di un liquido osservando la caduta libera di un grave attraverso il liquido stesso e verificare la legge di Stokes.
\section{Descrizione dell'apparato strumentale}
Lo strumento impiegato è un tubo di raggio interno $\unit[4.5]{cm}$ che contiene un sapone liquido di densità $\rho_0 = \unitfrac[(1.032\pm0.001)\cdot10^{3}]{kg}{m^3}$. All'esterno sono presenti undici tacche, a intervalli di cinque centimetri. Sono state utilizzate sfere di acciaio con densità $\rho=\unitfrac[(7.870\pm0.005)\cdot10^{3}]{kg}{m^3}$ di vari diametri:
\begin{table}[h]
\centering
\begin{tabular}{r@{ $=$ }l r@{ $=$ }l}
$D_1$ & $\unit[1.5]{mm}$ & $D_6$ & $5/32''$\\
$D_2$ & $2/32''$ &  $D_7$ & $6/32''$\\
$D_3$ & $\unit[2]{mm}$ &  $D_8$ & $7/32''$\\
$D_4$ & $3/32''$ &  $D_9$ & $8/32''$\\
$D_5$ & $4/32''$& $D_{10}$ & $9/32''$\\
\end{tabular}
\end{table}\\
\`E stato utilizzato il cronometro manuale con sensibilità $\unit[10^4]{s^{-1}}$, mentre come errore sulla posizione è stato assunto un terzo dello spessore della tacca sul bordo esterno del tubo (che misura circa \unit[1]{mm}, e risulta essere quindi $\unit[3\cdot10^{-4}]{m}$.
\section{Descrizione della metodologia di misura}
Si vuole stimare il tempo impiegato da sfere di vari raggi a percorrere una certa altezza di liquido, una volta raggiunta la velocità limite.
Sono stati rilevati i tempi di passaggio di cinque sfere per ogni diametro attraverso i traguardi sull'esterno del viscosimetro ogni cinque centimetri. Tuttavia, le sfere con diametro da $D_7$ a $D_{10}$ sono troppo veloci per poter misurare i tempi di percorrenza di singoli tratti di \unit[5]{cm}. Perciò è stato misurato solo il tempo di percorrenza dell'intero tratto di \unit[50]{cm}. Per lo stesso motivo, per $D_5$ e $D_6$ sono stati rilevati i tempi ogni \unit[10]{cm}.
\section{Risultati sperimentali ed elaborazione dati}
Sono state impiegate due procedure diverse per analizzare i dati. Come prima cosa da un grafico con i tempi medi delle misure ripetute in ordinata e lo spazio percorso in ascissa  è stato ricavato, per interpolazione lineare $y=ax+b$ il coefficiente $a=1/v_L$. Da questo si possono ottenere $\eta_i=D_i^2ga_i(\rho-\rho_0)/18$ con errore per propagazione e quindi $\eta_1$ come media pesata.
Alternativamente si può riportare in grafico $1/v_{L,i}$ in funzione dell'inverso del quadrato dei diametri $D_i^{-2}$, ottenendo una retta con pendenza $A=D^2/v_L$, secondo la legge di Stokes. Da cui $\eta_2 = Ag(\rho-\rho_0)/18$.
Dalle interpolazioni effettuate per le prime sei misure (grafici~\ref{piccole} e~\ref{medie}) e dal calcolo diretto con propagazione degli errori per le ultime quattro risultano i valori di $a$ riportati in tabella~\ref{a}.
\begin{table}[t]\caption{Valori di $a=\nicefrac{1}{v_L}$, in \unitfrac{s}{cm}.}\label{a}\centering
 \begin{tabular}{rr@{$\pm$}l rr@{$\pm$}l}
 $a_1$ & 4.4010&0.0006 &$a_6$ & 0.5992&0.0010\\
 $a_2$ & 3.8554&0.0020&$a_7$ & 0.4090&0.0005\\
 $a_3$ & 2.4136&0.0006&$a_8$ & 0.2956&0.0016\\
 $a_4$ & 1.6956&0.0005&$a_9$ & 0.2169&0.0010\\
 $a_5$ & 0.9453&0.0010&$a_{10}$ & 0.1619&0.0010\\
\end{tabular}
\end{table}
Da questi si ottengono degli $\eta_i$, riportati in tabella~\ref{eta}, la cui media pesata risulta:
\begin{equation*}
 \eta_1 = \unit[3.418\pm0.007]{\pas}
\end{equation*}
\begin{table}[t]\caption{Valori di $\eta$ ($\unit{\pas}$) ricavati direttamente dagli $a_i$ relativi ai vari diametri.}\label{eta}\centering
 \begin{tabular}{rr@{$\pm$}l rr@{$\pm$}l}
$\eta_1$ &3.689 &0.049 &$\eta_6$ &3.516 &0.019\\
$\eta_2$ &3.619 &0.046 &$\eta_7$ &3.456 &0.015\\
$\eta_3$ &3.596 &0.036 &$\eta_8$ &3.399 &0.022\\
$\eta_4$ &3.582 &0.030 &$\eta_{9}$ &3.259 &0.019\\
$\eta_5$ &3.550 &0.023 &$\eta_{10}$ &3.077 &0.022
\end{tabular}
\end{table}
Dall'interpolazione effettuata nel grafico~\ref{alter}, risulta invece $A=\unit[(9.877\pm0.046)\cdot 10^{-4}]{ms}$ e quindi $\eta_2 = \unit[3.679\pm0.017]{\pas}$, che ha compatibilità estremamente scarsa ($\lambda = 13.98$) con il valore $\eta_1$ precedentemente ricavato.
\section{Discussione dei risultati}
Si nota che la media $\eta_1$ attribuisce peso maggiore alle ultime misure effettuate. Queste tuttavia sono state valutate con l'elaborazione di meno dati (cinque o un solo tempo) e c'è ragione di ritenere che presentino un valore di viscosità inferiore per l'aumento di temperatura nel laboratorio e per la presenza di bolle nel liquido dopo la discesa di numerose sfere. Al contrario, l'interpolazione di $\eta_2$ interpreta in modo più profondo e diretto la legge di Stokes e per la determinazione dei parametri dà maggiore importanza alle zone con maggiore densità di dati sperimentali, in questo caso, come si vede dal grafico~\ref{alter}, quelle riferite ai diametri più piccoli. 
Per gli errori sistematici valutiamo innanzitutto il numero di Reynolds. Per la sfera più grande vale $\mathcal{R}=2\rho v R / \eta = 0.06 \ll 0.2$, quindi il regime laminare è nettamente prevalente, come si voleva. Stimiamo anche lo spazio (o equivalentemente il tempo) necessario perché il corpo abbia una velocità prossima a $v_L$, detto $\tau=2\rho R^2 / 9\eta$ il tempo caratteristico.
Se $t$ è dell'ordine di $3\tau$, $v= v_L$ a meno del 5\%. In termini di spazio, basta percorrere uno spazio di $\unit[0.35]{cm}$. Si scende a un errore di meno dell'1\% per un tempo $t=5\tau$. Quindi nel nostro caso, a una decina di centimetri sotto il pelo del liquido l'errore sistematico è certamente trascurabile. Infine, esistono un'effetto parete e un effetto terminale che incidono sulla velocità di discesa. Detta infatti $v_\infty$ la velocità limite teorica, si ha $v_\infty = 1.21 v_L$ per un raggio della sfera di $\unit[0.375]{cm}$ o  $v_\infty = 1.04 v_L$ per un raggio della sfera di $\unit[0.15]{cm}$. Questo errore sistematico non è quindi trascurabile, ma non lo tratteremo comunque in queste pagine.
\section{Conclusioni}
Per i motivi discussi nella sezione precedente, si ritiene più affidabile la seconda stima della viscosità del fluido. Inoltre, la legge di Stokes appare evidentemente verificata.
\section{Appendice}
\begin{figure}[hp]\caption{Medie dei tempi in ordinata (\unit{s}), posizione lungo il viscosimetro in ascissa (\unit{cm}) con rette interpolanti, diametri da $D_1$ a $D_4$. Le rette più pendenti si riferiscono ovviamente alle sfere con diametro minore. Gli errori sono troppo piccoli per poter essere rappresentati in questa scala.}\label{piccole}
\centering
% GNUPLOT: LaTeX picture using PSTRICKS macros
% Define new PST objects, if not already defined
\ifx\PSTloaded\undefined
\def\PSTloaded{t}
\psset{arrowsize=.01 3.2 1.4 .3}
\psset{dotsize=.08}
\catcode`@=11

\newpsobject{PST@Border}{psline}{linewidth=.0015,linestyle=solid}
\newpsobject{PST@Axes}{psline}{linewidth=.0015,linestyle=dotted,dotsep=.004}
\newpsobject{PST@Solid}{psline}{linewidth=.0015,linestyle=solid}
\newpsobject{PST@Dashed}{psline}{linewidth=.0015,linestyle=dashed,dash=.01 .01}
\newpsobject{PST@Dotted}{psline}{linewidth=.0025,linestyle=dotted,dotsep=.008}
\newpsobject{PST@LongDash}{psline}{linewidth=.0015,linestyle=dashed,dash=.02 .01}
\newpsobject{PST@Diamond}{psdots}{linewidth=.001,linestyle=solid,dotstyle=square,dotangle=45}
\newpsobject{PST@Filldiamond}{psdots}{linewidth=.001,linestyle=solid,dotstyle=square*,dotangle=45}
\newpsobject{PST@Cross}{psdots}{linewidth=.001,linestyle=solid,dotstyle=+,dotangle=45}
\newpsobject{PST@Plus}{psdots}{linewidth=.001,linestyle=solid,dotstyle=+}
\newpsobject{PST@Square}{psdots}{linewidth=.001,linestyle=solid,dotstyle=square}
\newpsobject{PST@Circle}{psdots}{linewidth=.001,linestyle=solid,dotstyle=o}
\newpsobject{PST@Triangle}{psdots}{linewidth=.001,linestyle=solid,dotstyle=triangle}
\newpsobject{PST@Pentagon}{psdots}{linewidth=.001,linestyle=solid,dotstyle=pentagon}
\newpsobject{PST@Fillsquare}{psdots}{linewidth=.001,linestyle=solid,dotstyle=square*}
\newpsobject{PST@Fillcircle}{psdots}{linewidth=.001,linestyle=solid,dotstyle=*}
\newpsobject{PST@Filltriangle}{psdots}{linewidth=.001,linestyle=solid,dotstyle=triangle*}
\newpsobject{PST@Fillpentagon}{psdots}{linewidth=.001,linestyle=solid,dotstyle=pentagon*}
\newpsobject{PST@Arrow}{psline}{linewidth=.001,linestyle=solid}
\catcode`@=12

\fi
\psset{unit=5.0in,xunit=5.0in,yunit=3.0in}
\pspicture(0.000000,0.000000)(1.000000,1.000000)
\ifx\nofigs\undefined
\catcode`@=11

\PST@Border(0.1430,0.1260)
(0.1580,0.1260)

\rput[r](0.1270,0.1260){0}
\PST@Border(0.1430,0.2944)
(0.1580,0.2944)

\rput[r](0.1270,0.2944){50}
\PST@Border(0.1430,0.4628)
(0.1580,0.4628)

\rput[r](0.1270,0.4628){100}
\PST@Border(0.1430,0.6312)
(0.1580,0.6312)

\rput[r](0.1270,0.6312){150}
\PST@Border(0.1430,0.7996)
(0.1580,0.7996)

\rput[r](0.1270,0.7996){200}
\PST@Border(0.1430,0.9680)
(0.1580,0.9680)

\rput[r](0.1270,0.9680){250}
\PST@Border(0.2234,0.1260)
(0.2234,0.1460)

\rput(0.2234,0.0840){10}
\PST@Border(0.3842,0.1260)
(0.3842,0.1460)

\rput(0.3842,0.0840){20}
\PST@Border(0.5450,0.1260)
(0.5450,0.1460)

\rput(0.5450,0.0840){30}
\PST@Border(0.7058,0.1260)
(0.7058,0.1460)

\rput(0.7058,0.0840){40}
\PST@Border(0.8666,0.1260)
(0.8666,0.1460)

\rput(0.8666,0.0840){50}
\PST@Border(0.1430,0.9680)
(0.1430,0.1260)
(0.9470,0.1260)
(0.9470,0.9680)
(0.1430,0.9680)

\rput{L}(0.0420,0.5470){tempo ($\unit{s})$}
\rput(0.5450,0.0210){posizione (\unit{cm})}
\PST@Solid(0.1430,0.1999)
(0.1430,0.1999)
(0.1511,0.2074)
(0.1592,0.2149)
(0.1674,0.2224)
(0.1755,0.2298)
(0.1836,0.2373)
(0.1917,0.2448)
(0.1998,0.2523)
(0.2080,0.2598)
(0.2161,0.2673)
(0.2242,0.2748)
(0.2323,0.2822)
(0.2405,0.2897)
(0.2486,0.2972)
(0.2567,0.3047)
(0.2648,0.3122)
(0.2729,0.3197)
(0.2811,0.3272)
(0.2892,0.3346)
(0.2973,0.3421)
(0.3054,0.3496)
(0.3135,0.3571)
(0.3217,0.3646)
(0.3298,0.3721)
(0.3379,0.3796)
(0.3460,0.3870)
(0.3542,0.3945)
(0.3623,0.4020)
(0.3704,0.4095)
(0.3785,0.4170)
(0.3866,0.4245)
(0.3948,0.4320)
(0.4029,0.4395)
(0.4110,0.4469)
(0.4191,0.4544)
(0.4272,0.4619)
(0.4354,0.4694)
(0.4435,0.4769)
(0.4516,0.4844)
(0.4597,0.4919)
(0.4678,0.4993)
(0.4760,0.5068)
(0.4841,0.5143)
(0.4922,0.5218)
(0.5003,0.5293)
(0.5085,0.5368)
(0.5166,0.5443)
(0.5247,0.5517)
(0.5328,0.5592)
(0.5409,0.5667)
(0.5491,0.5742)
(0.5572,0.5817)
(0.5653,0.5892)
(0.5734,0.5967)
(0.5815,0.6041)
(0.5897,0.6116)
(0.5978,0.6191)
(0.6059,0.6266)
(0.6140,0.6341)
(0.6222,0.6416)
(0.6303,0.6491)
(0.6384,0.6565)
(0.6465,0.6640)
(0.6546,0.6715)
(0.6628,0.6790)
(0.6709,0.6865)
(0.6790,0.6940)
(0.6871,0.7015)
(0.6952,0.7089)
(0.7034,0.7164)
(0.7115,0.7239)
(0.7196,0.7314)
(0.7277,0.7389)
(0.7358,0.7464)
(0.7440,0.7539)
(0.7521,0.7614)
(0.7602,0.7688)
(0.7683,0.7763)
(0.7765,0.7838)
(0.7846,0.7913)
(0.7927,0.7988)
(0.8008,0.8063)
(0.8089,0.8138)
(0.8171,0.8212)
(0.8252,0.8287)
(0.8333,0.8362)
(0.8414,0.8437)
(0.8495,0.8512)
(0.8577,0.8587)
(0.8658,0.8662)
(0.8739,0.8736)
(0.8820,0.8811)
(0.8902,0.8886)
(0.8983,0.8961)
(0.9064,0.9036)
(0.9145,0.9111)
(0.9226,0.9186)
(0.9308,0.9260)
(0.9389,0.9335)
(0.9470,0.9410)

\PST@Dashed(0.1430,0.1908)
(0.1430,0.1908)
(0.1511,0.1973)
(0.1592,0.2039)
(0.1674,0.2105)
(0.1755,0.2170)
(0.1836,0.2236)
(0.1917,0.2301)
(0.1998,0.2367)
(0.2080,0.2433)
(0.2161,0.2498)
(0.2242,0.2564)
(0.2323,0.2629)
(0.2405,0.2695)
(0.2486,0.2760)
(0.2567,0.2826)
(0.2648,0.2892)
(0.2729,0.2957)
(0.2811,0.3023)
(0.2892,0.3088)
(0.2973,0.3154)
(0.3054,0.3219)
(0.3135,0.3285)
(0.3217,0.3351)
(0.3298,0.3416)
(0.3379,0.3482)
(0.3460,0.3547)
(0.3542,0.3613)
(0.3623,0.3679)
(0.3704,0.3744)
(0.3785,0.3810)
(0.3866,0.3875)
(0.3948,0.3941)
(0.4029,0.4006)
(0.4110,0.4072)
(0.4191,0.4138)
(0.4272,0.4203)
(0.4354,0.4269)
(0.4435,0.4334)
(0.4516,0.4400)
(0.4597,0.4466)
(0.4678,0.4531)
(0.4760,0.4597)
(0.4841,0.4662)
(0.4922,0.4728)
(0.5003,0.4793)
(0.5085,0.4859)
(0.5166,0.4925)
(0.5247,0.4990)
(0.5328,0.5056)
(0.5409,0.5121)
(0.5491,0.5187)
(0.5572,0.5252)
(0.5653,0.5318)
(0.5734,0.5384)
(0.5815,0.5449)
(0.5897,0.5515)
(0.5978,0.5580)
(0.6059,0.5646)
(0.6140,0.5712)
(0.6222,0.5777)
(0.6303,0.5843)
(0.6384,0.5908)
(0.6465,0.5974)
(0.6546,0.6039)
(0.6628,0.6105)
(0.6709,0.6171)
(0.6790,0.6236)
(0.6871,0.6302)
(0.6952,0.6367)
(0.7034,0.6433)
(0.7115,0.6498)
(0.7196,0.6564)
(0.7277,0.6630)
(0.7358,0.6695)
(0.7440,0.6761)
(0.7521,0.6826)
(0.7602,0.6892)
(0.7683,0.6958)
(0.7765,0.7023)
(0.7846,0.7089)
(0.7927,0.7154)
(0.8008,0.7220)
(0.8089,0.7285)
(0.8171,0.7351)
(0.8252,0.7417)
(0.8333,0.7482)
(0.8414,0.7548)
(0.8495,0.7613)
(0.8577,0.7679)
(0.8658,0.7745)
(0.8739,0.7810)
(0.8820,0.7876)
(0.8902,0.7941)
(0.8983,0.8007)
(0.9064,0.8072)
(0.9145,0.8138)
(0.9226,0.8204)
(0.9308,0.8269)
(0.9389,0.8335)
(0.9470,0.8400)

\PST@Dotted(0.1430,0.1666)
(0.1430,0.1666)
(0.1511,0.1707)
(0.1592,0.1748)
(0.1674,0.1789)
(0.1755,0.1830)
(0.1836,0.1871)
(0.1917,0.1912)
(0.1998,0.1953)
(0.2080,0.1994)
(0.2161,0.2035)
(0.2242,0.2077)
(0.2323,0.2118)
(0.2405,0.2159)
(0.2486,0.2200)
(0.2567,0.2241)
(0.2648,0.2282)
(0.2729,0.2323)
(0.2811,0.2364)
(0.2892,0.2405)
(0.2973,0.2446)
(0.3054,0.2487)
(0.3135,0.2528)
(0.3217,0.2569)
(0.3298,0.2610)
(0.3379,0.2651)
(0.3460,0.2692)
(0.3542,0.2733)
(0.3623,0.2774)
(0.3704,0.2815)
(0.3785,0.2857)
(0.3866,0.2898)
(0.3948,0.2939)
(0.4029,0.2980)
(0.4110,0.3021)
(0.4191,0.3062)
(0.4272,0.3103)
(0.4354,0.3144)
(0.4435,0.3185)
(0.4516,0.3226)
(0.4597,0.3267)
(0.4678,0.3308)
(0.4760,0.3349)
(0.4841,0.3390)
(0.4922,0.3431)
(0.5003,0.3472)
(0.5085,0.3513)
(0.5166,0.3554)
(0.5247,0.3596)
(0.5328,0.3637)
(0.5409,0.3678)
(0.5491,0.3719)
(0.5572,0.3760)
(0.5653,0.3801)
(0.5734,0.3842)
(0.5815,0.3883)
(0.5897,0.3924)
(0.5978,0.3965)
(0.6059,0.4006)
(0.6140,0.4047)
(0.6222,0.4088)
(0.6303,0.4129)
(0.6384,0.4170)
(0.6465,0.4211)
(0.6546,0.4252)
(0.6628,0.4293)
(0.6709,0.4335)
(0.6790,0.4376)
(0.6871,0.4417)
(0.6952,0.4458)
(0.7034,0.4499)
(0.7115,0.4540)
(0.7196,0.4581)
(0.7277,0.4622)
(0.7358,0.4663)
(0.7440,0.4704)
(0.7521,0.4745)
(0.7602,0.4786)
(0.7683,0.4827)
(0.7765,0.4868)
(0.7846,0.4909)
(0.7927,0.4950)
(0.8008,0.4991)
(0.8089,0.5032)
(0.8171,0.5074)
(0.8252,0.5115)
(0.8333,0.5156)
(0.8414,0.5197)
(0.8495,0.5238)
(0.8577,0.5279)
(0.8658,0.5320)
(0.8739,0.5361)
(0.8820,0.5402)
(0.8902,0.5443)
(0.8983,0.5484)
(0.9064,0.5525)
(0.9145,0.5566)
(0.9226,0.5607)
(0.9308,0.5648)
(0.9389,0.5689)
(0.9470,0.5730)

\PST@LongDash(0.1430,0.1540)
(0.1430,0.1540)
(0.1511,0.1569)
(0.1592,0.1598)
(0.1674,0.1627)
(0.1755,0.1655)
(0.1836,0.1684)
(0.1917,0.1713)
(0.1998,0.1742)
(0.2080,0.1771)
(0.2161,0.1800)
(0.2242,0.1829)
(0.2323,0.1857)
(0.2405,0.1886)
(0.2486,0.1915)
(0.2567,0.1944)
(0.2648,0.1973)
(0.2729,0.2002)
(0.2811,0.2030)
(0.2892,0.2059)
(0.2973,0.2088)
(0.3054,0.2117)
(0.3135,0.2146)
(0.3217,0.2175)
(0.3298,0.2203)
(0.3379,0.2232)
(0.3460,0.2261)
(0.3542,0.2290)
(0.3623,0.2319)
(0.3704,0.2348)
(0.3785,0.2377)
(0.3866,0.2405)
(0.3948,0.2434)
(0.4029,0.2463)
(0.4110,0.2492)
(0.4191,0.2521)
(0.4272,0.2550)
(0.4354,0.2578)
(0.4435,0.2607)
(0.4516,0.2636)
(0.4597,0.2665)
(0.4678,0.2694)
(0.4760,0.2723)
(0.4841,0.2751)
(0.4922,0.2780)
(0.5003,0.2809)
(0.5085,0.2838)
(0.5166,0.2867)
(0.5247,0.2896)
(0.5328,0.2925)
(0.5409,0.2953)
(0.5491,0.2982)
(0.5572,0.3011)
(0.5653,0.3040)
(0.5734,0.3069)
(0.5815,0.3098)
(0.5897,0.3126)
(0.5978,0.3155)
(0.6059,0.3184)
(0.6140,0.3213)
(0.6222,0.3242)
(0.6303,0.3271)
(0.6384,0.3299)
(0.6465,0.3328)
(0.6546,0.3357)
(0.6628,0.3386)
(0.6709,0.3415)
(0.6790,0.3444)
(0.6871,0.3473)
(0.6952,0.3501)
(0.7034,0.3530)
(0.7115,0.3559)
(0.7196,0.3588)
(0.7277,0.3617)
(0.7358,0.3646)
(0.7440,0.3674)
(0.7521,0.3703)
(0.7602,0.3732)
(0.7683,0.3761)
(0.7765,0.3790)
(0.7846,0.3819)
(0.7927,0.3847)
(0.8008,0.3876)
(0.8089,0.3905)
(0.8171,0.3934)
(0.8252,0.3963)
(0.8333,0.3992)
(0.8414,0.4020)
(0.8495,0.4049)
(0.8577,0.4078)
(0.8658,0.4107)
(0.8739,0.4136)
(0.8820,0.4165)
(0.8902,0.4194)
(0.8983,0.4222)
(0.9064,0.4251)
(0.9145,0.4280)
(0.9226,0.4309)
(0.9308,0.4338)
(0.9389,0.4367)
(0.9470,0.4395)

\PST@Diamond(0.1430,0.2000)
\PST@Diamond(0.2234,0.2740)
\PST@Diamond(0.3038,0.3481)
\PST@Diamond(0.3842,0.4221)
\PST@Diamond(0.4646,0.4963)
\PST@Diamond(0.5450,0.5703)
\PST@Diamond(0.6254,0.6446)
\PST@Diamond(0.7058,0.7187)
\PST@Diamond(0.7862,0.7927)
\PST@Diamond(0.8666,0.8670)
\PST@Plus(0.1430,0.1906)
\PST@Plus(0.2234,0.2553)
\PST@Plus(0.3038,0.3211)
\PST@Plus(0.3842,0.3856)
\PST@Plus(0.4646,0.4504)
\PST@Plus(0.5450,0.5154)
\PST@Plus(0.6254,0.5809)
\PST@Plus(0.7058,0.6453)
\PST@Plus(0.7862,0.7100)
\PST@Plus(0.8666,0.7749)
\PST@Square(0.1430,0.1665)
\PST@Square(0.2234,0.2073)
\PST@Square(0.3038,0.2480)
\PST@Square(0.3842,0.2886)
\PST@Square(0.4646,0.3291)
\PST@Square(0.5450,0.3697)
\PST@Square(0.6254,0.4106)
\PST@Square(0.7058,0.4510)
\PST@Square(0.7862,0.4918)
\PST@Square(0.8666,0.5324)
\PST@Cross(0.1430,0.1541)
\PST@Cross(0.2234,0.1826)
\PST@Cross(0.3038,0.2110)
\PST@Cross(0.3842,0.2397)
\PST@Cross(0.4646,0.2681)
\PST@Cross(0.5450,0.2967)
\PST@Cross(0.6254,0.3254)
\PST@Cross(0.7058,0.3538)
\PST@Cross(0.7862,0.3824)
\PST@Cross(0.8666,0.4111)
\PST@Border(0.1430,0.9680)
(0.1430,0.1260)
(0.9470,0.1260)
(0.9470,0.9680)
(0.1430,0.9680)

\catcode`@=12
\fi
\endpspicture

\end{figure}
\begin{figure}[hp]\caption{Medie dei tempi in ordinata (\unit{s}), posizione lungo il viscosimetro in ascissa (\unit{cm}) con rette interpolanti, diametri $D_5$ e $D_6$. Gli errori sono troppo piccoli per poter essere rappresentati in questa scala.}\label{medie}
\centering
% GNUPLOT: LaTeX picture using PSTRICKS macros
% Define new PST objects, if not already defined
\ifx\PSTloaded\undefined
\def\PSTloaded{t}
\psset{arrowsize=.01 3.2 1.4 .3}
\psset{dotsize=.08}
\catcode`@=11

\newpsobject{PST@Border}{psline}{linewidth=.0015,linestyle=solid}
\newpsobject{PST@Axes}{psline}{linewidth=.0015,linestyle=dotted,dotsep=.004}
\newpsobject{PST@Solid}{psline}{linewidth=.0015,linestyle=solid}
\newpsobject{PST@Dashed}{psline}{linewidth=.0015,linestyle=dashed,dash=.01 .01}
\newpsobject{PST@Dotted}{psline}{linewidth=.0025,linestyle=dotted,dotsep=.008}
\newpsobject{PST@LongDash}{psline}{linewidth=.0015,linestyle=dashed,dash=.02 .01}
\newpsobject{PST@Diamond}{psdots}{linewidth=.001,linestyle=solid,dotstyle=square,dotangle=45}
\newpsobject{PST@Filldiamond}{psdots}{linewidth=.001,linestyle=solid,dotstyle=square*,dotangle=45}
\newpsobject{PST@Cross}{psdots}{linewidth=.001,linestyle=solid,dotstyle=+,dotangle=45}
\newpsobject{PST@Plus}{psdots}{linewidth=.001,linestyle=solid,dotstyle=+}
\newpsobject{PST@Square}{psdots}{linewidth=.001,linestyle=solid,dotstyle=square}
\newpsobject{PST@Circle}{psdots}{linewidth=.001,linestyle=solid,dotstyle=o}
\newpsobject{PST@Triangle}{psdots}{linewidth=.001,linestyle=solid,dotstyle=triangle}
\newpsobject{PST@Pentagon}{psdots}{linewidth=.001,linestyle=solid,dotstyle=pentagon}
\newpsobject{PST@Fillsquare}{psdots}{linewidth=.001,linestyle=solid,dotstyle=square*}
\newpsobject{PST@Fillcircle}{psdots}{linewidth=.001,linestyle=solid,dotstyle=*}
\newpsobject{PST@Filltriangle}{psdots}{linewidth=.001,linestyle=solid,dotstyle=triangle*}
\newpsobject{PST@Fillpentagon}{psdots}{linewidth=.001,linestyle=solid,dotstyle=pentagon*}
\newpsobject{PST@Arrow}{psline}{linewidth=.001,linestyle=solid}
\catcode`@=12

\fi
\psset{unit=5.0in,xunit=5.0in,yunit=3.0in}
\pspicture(0.000000,0.000000)(1.000000,1.000000)
\ifx\nofigs\undefined
\catcode`@=11

\PST@Border(0.1270,0.1260)
(0.1420,0.1260)

\rput[r](0.1110,0.1260){0}
\PST@Border(0.1270,0.2025)
(0.1420,0.2025)

\rput[r](0.1110,0.2025){5}
\PST@Border(0.1270,0.2791)
(0.1420,0.2791)

\rput[r](0.1110,0.2791){10}
\PST@Border(0.1270,0.3556)
(0.1420,0.3556)

\rput[r](0.1110,0.3556){15}
\PST@Border(0.1270,0.4322)
(0.1420,0.4322)

\rput[r](0.1110,0.4322){20}
\PST@Border(0.1270,0.5087)
(0.1420,0.5087)

\rput[r](0.1110,0.5087){25}
\PST@Border(0.1270,0.5853)
(0.1420,0.5853)

\rput[r](0.1110,0.5853){30}
\PST@Border(0.1270,0.6618)
(0.1420,0.6618)

\rput[r](0.1110,0.6618){35}
\PST@Border(0.1270,0.7384)
(0.1420,0.7384)

\rput[r](0.1110,0.7384){40}
\PST@Border(0.1270,0.8149)
(0.1420,0.8149)

\rput[r](0.1110,0.8149){45}
\PST@Border(0.1270,0.8915)
(0.1420,0.8915)

\rput[r](0.1110,0.8915){50}
\PST@Border(0.1270,0.9680)
(0.1420,0.9680)

\rput[r](0.1110,0.9680){55}
\PST@Border(0.2090,0.1260)
(0.2090,0.1460)

\rput(0.2090,0.0840){10}
\PST@Border(0.3730,0.1260)
(0.3730,0.1460)

\rput(0.3730,0.0840){20}
\PST@Border(0.5370,0.1260)
(0.5370,0.1460)

\rput(0.5370,0.0840){30}
\PST@Border(0.7010,0.1260)
(0.7010,0.1460)

\rput(0.7010,0.0840){40}
\PST@Border(0.8650,0.1260)
(0.8650,0.1460)

\rput(0.8650,0.0840){50}
\PST@Border(0.1270,0.9680)
(0.1270,0.1260)
(0.9470,0.1260)
(0.9470,0.9680)
(0.1270,0.9680)

\rput{L}(0.0420,0.5470){tempo ($\unit{s})$}
\rput(0.5370,0.0210){posizione (\unit{cm})}
\PST@Solid(0.1270,0.1957)
(0.1270,0.1957)
(0.1353,0.2030)
(0.1436,0.2103)
(0.1518,0.2176)
(0.1601,0.2249)
(0.1684,0.2322)
(0.1767,0.2395)
(0.1850,0.2468)
(0.1933,0.2542)
(0.2015,0.2615)
(0.2098,0.2688)
(0.2181,0.2761)
(0.2264,0.2834)
(0.2347,0.2907)
(0.2430,0.2980)
(0.2512,0.3053)
(0.2595,0.3126)
(0.2678,0.3199)
(0.2761,0.3272)
(0.2844,0.3346)
(0.2927,0.3419)
(0.3009,0.3492)
(0.3092,0.3565)
(0.3175,0.3638)
(0.3258,0.3711)
(0.3341,0.3784)
(0.3424,0.3857)
(0.3506,0.3930)
(0.3589,0.4003)
(0.3672,0.4076)
(0.3755,0.4150)
(0.3838,0.4223)
(0.3921,0.4296)
(0.4003,0.4369)
(0.4086,0.4442)
(0.4169,0.4515)
(0.4252,0.4588)
(0.4335,0.4661)
(0.4417,0.4734)
(0.4500,0.4807)
(0.4583,0.4880)
(0.4666,0.4954)
(0.4749,0.5027)
(0.4832,0.5100)
(0.4914,0.5173)
(0.4997,0.5246)
(0.5080,0.5319)
(0.5163,0.5392)
(0.5246,0.5465)
(0.5329,0.5538)
(0.5411,0.5611)
(0.5494,0.5685)
(0.5577,0.5758)
(0.5660,0.5831)
(0.5743,0.5904)
(0.5826,0.5977)
(0.5908,0.6050)
(0.5991,0.6123)
(0.6074,0.6196)
(0.6157,0.6269)
(0.6240,0.6342)
(0.6323,0.6415)
(0.6405,0.6489)
(0.6488,0.6562)
(0.6571,0.6635)
(0.6654,0.6708)
(0.6737,0.6781)
(0.6819,0.6854)
(0.6902,0.6927)
(0.6985,0.7000)
(0.7068,0.7073)
(0.7151,0.7146)
(0.7234,0.7219)
(0.7316,0.7293)
(0.7399,0.7366)
(0.7482,0.7439)
(0.7565,0.7512)
(0.7648,0.7585)
(0.7731,0.7658)
(0.7813,0.7731)
(0.7896,0.7804)
(0.7979,0.7877)
(0.8062,0.7950)
(0.8145,0.8023)
(0.8228,0.8097)
(0.8310,0.8170)
(0.8393,0.8243)
(0.8476,0.8316)
(0.8559,0.8389)
(0.8642,0.8462)
(0.8725,0.8535)
(0.8807,0.8608)
(0.8890,0.8681)
(0.8973,0.8754)
(0.9056,0.8828)
(0.9139,0.8901)
(0.9222,0.8974)
(0.9304,0.9047)
(0.9387,0.9120)
(0.9470,0.9193)

\PST@Dashed(0.1270,0.1711)
(0.1270,0.1711)
(0.1353,0.1757)
(0.1436,0.1803)
(0.1518,0.1850)
(0.1601,0.1896)
(0.1684,0.1942)
(0.1767,0.1989)
(0.1850,0.2035)
(0.1933,0.2081)
(0.2015,0.2128)
(0.2098,0.2174)
(0.2181,0.2220)
(0.2264,0.2267)
(0.2347,0.2313)
(0.2430,0.2359)
(0.2512,0.2406)
(0.2595,0.2452)
(0.2678,0.2498)
(0.2761,0.2545)
(0.2844,0.2591)
(0.2927,0.2637)
(0.3009,0.2684)
(0.3092,0.2730)
(0.3175,0.2776)
(0.3258,0.2823)
(0.3341,0.2869)
(0.3424,0.2915)
(0.3506,0.2962)
(0.3589,0.3008)
(0.3672,0.3054)
(0.3755,0.3101)
(0.3838,0.3147)
(0.3921,0.3193)
(0.4003,0.3240)
(0.4086,0.3286)
(0.4169,0.3332)
(0.4252,0.3379)
(0.4335,0.3425)
(0.4417,0.3471)
(0.4500,0.3518)
(0.4583,0.3564)
(0.4666,0.3610)
(0.4749,0.3657)
(0.4832,0.3703)
(0.4914,0.3749)
(0.4997,0.3796)
(0.5080,0.3842)
(0.5163,0.3888)
(0.5246,0.3935)
(0.5329,0.3981)
(0.5411,0.4027)
(0.5494,0.4074)
(0.5577,0.4120)
(0.5660,0.4166)
(0.5743,0.4213)
(0.5826,0.4259)
(0.5908,0.4305)
(0.5991,0.4352)
(0.6074,0.4398)
(0.6157,0.4444)
(0.6240,0.4490)
(0.6323,0.4537)
(0.6405,0.4583)
(0.6488,0.4629)
(0.6571,0.4676)
(0.6654,0.4722)
(0.6737,0.4768)
(0.6819,0.4815)
(0.6902,0.4861)
(0.6985,0.4907)
(0.7068,0.4954)
(0.7151,0.5000)
(0.7234,0.5046)
(0.7316,0.5093)
(0.7399,0.5139)
(0.7482,0.5185)
(0.7565,0.5232)
(0.7648,0.5278)
(0.7731,0.5324)
(0.7813,0.5371)
(0.7896,0.5417)
(0.7979,0.5463)
(0.8062,0.5510)
(0.8145,0.5556)
(0.8228,0.5602)
(0.8310,0.5649)
(0.8393,0.5695)
(0.8476,0.5741)
(0.8559,0.5788)
(0.8642,0.5834)
(0.8725,0.5880)
(0.8807,0.5927)
(0.8890,0.5973)
(0.8973,0.6019)
(0.9056,0.6066)
(0.9139,0.6112)
(0.9222,0.6158)
(0.9304,0.6205)
(0.9387,0.6251)
(0.9470,0.6297)

\PST@Diamond(0.2090,0.2681)
\PST@Diamond(0.3730,0.4126)
\PST@Diamond(0.5370,0.5575)
\PST@Diamond(0.7010,0.7022)
\PST@Diamond(0.8650,0.8470)
\PST@Plus(0.2090,0.2168)
\PST@Plus(0.3730,0.3083)
\PST@Plus(0.5370,0.4011)
\PST@Plus(0.7010,0.4923)
\PST@Plus(0.8650,0.5835)
\PST@Border(0.1270,0.9680)
(0.1270,0.1260)
(0.9470,0.1260)
(0.9470,0.9680)
(0.1270,0.9680)

\catcode`@=12
\fi
\endpspicture

\end{figure}
\begin{figure}[p]\caption{Inverso dei quadrati dei diametri in ascissa ($\unit{\cdot 10^4 m^{-2}}$), inverso della velocità limite in ordinata (\unitfrac{s}{m}) con retta interpolante.}\label{alter}
\centering
\include{alternativo}
\end{figure}
\begin{table}[hp]\caption{Tempi (\unit{s}) rilevati per il passaggio della sfera sui vari traguardi (\unit{cm}). Cinque sfere di diametro \unit[1.5]{mm}.}
\centering \small
\begin{tabular}{r*5c}
5 &22.0633 &21.9768 &21.9622 &22.0995 &21.8152\\
10 &44.2588 &43.9159 &44.0570 &43.9366 &43.4889\\
15 &66.3605 &66.2565 &66.0354 &65.8532 &65.2545\\
20 &88.5749 &88.2890 &87.9168 &87.8935 &86.9652\\
25 &110.9141 &110.3737 &109.9505 &109.6918 &108.8561\\
30 &133.1851 &132.3357 &132.1190 &131.4671 &130.5232\\
35 &155.3989 &154.5919 &154.1245 &153.4904 &152.3414\\
40 &177.6639 &176.5493 &176.0020 &175.4550 &174.2094\\
45 &199.7706 &198.6755 &198.0071 &197.4141 &195.9174\\
50 &221.9873 &221.0572 &220.0171 &219.4166 &217.6295
\end{tabular}
\end{table}
\begin{table}[hp]\caption{Tempi (\unit{s}) rilevati per il passaggio della sfera sui vari traguardi (\unit{cm}). Cinque sfere di diametro \unit[2/32]{''}.}
\centering \small
\begin{tabular}{r*5c}
5 &19.3441 &19.1003 &19.1785 &19.0366 &19.2417\\
10 &38.5618 &38.4871 &38.4383 &38.2734 &38.2671\\
15 &58.0700 &57.8006 &58.1617 &57.4505 &58.1468\\
20 &77.4698 &77.2979 &77.4469 &76.6469 &76.5524\\
25 &96.6133 &96.4791 &96.9129 &95.8414 &95.7619\\
30 &116.0165 &115.6841 &116.2029 &114.9287 &115.2599\\
35 &135.3374 &135.2231 &135.5667 &134.2188 &134.9346\\
40 &154.7546 &154.5461 &154.8837 &153.4309 &153.3084\\
45 &173.8791 &173.8185 &174.3125 &172.4481 &172.4526\\
50 &193.1626 &193.0940 &193.5777 &191.8238 &191.6567
\end{tabular}
\end{table}
\begin{table}[hp]\caption{Tempi (\unit{s}) rilevati per il passaggio della sfera sui vari traguardi (\unit{cm}). Cinque sfere di diametro \unit[2]{mm}.}
\centering \small
\begin{tabular}{r*5c}
5 &12.0735 &12.0563 &12.0795 &12.0336 &11.9043\\
10 &24.2311 &24.1561 &24.1574 &24.1013 &23.9867\\
15 &36.2779 &36.5074 &36.2528 &36.1132 &35.9502\\
20 &48.5869 &48.2169 &48.3942 &48.1237 &48.0338\\
25 &60.7042 &60.2633 &60.3286 &60.2714 &60.0204\\
30 &72.6668 &72.4050 &72.2811 &72.3240 &72.1130\\
35 &84.8895 &84.5837 &84.4658 &84.3816 &84.1202\\
40 &96.9701 &96.7455 &96.2148 &96.3929 &96.1234\\
45 &109.2424 &108.7270 &108.4763 &108.5751 &108.1043\\
50 &121.2001 &120.8799 &120.5727 &120.6063 &120.0331
\end{tabular}
\end{table}
\begin{table}[hp]\caption{Tempi (\unit{s}) rilevati per il passaggio della sfera sui vari traguardi (\unit{cm}). Cinque sfere di diametro \unit[3/32]{''}.}
\centering \small
\begin{tabular}{r*5c}
5 &8.3659 &8.2929 &8.5119 &8.1874 &8.2944\\
10 &16.9555 &16.8747 &17.0141 &16.5488 &16.6416\\
15 &25.3888 &25.4030 &25.5628 &24.9299 &24.9519\\
20 &34.0958 &33.8457 &34.0994 &33.3936 &33.4111\\
25 &42.5096 &42.4109 &42.5701 &41.8353 &41.6701\\
30 &51.0556 &51.0103 &51.1092 &50.2093 &50.0831\\
35 &59.6284 &59.4389 &59.7534 &58.5955 &58.6774\\
40 &68.1454 &67.9336 &68.1824 &66.9764 &66.9438\\
45 &76.7282 &76.3866 &76.8270 &75.4457 &75.2839\\
50 &85.3041 &84.9327 &85.3860 &83.8293 &83.7272
\end{tabular}
\end{table}
\begin{table}[hp]\caption{Tempi (\unit{s}) rilevati per il passaggio della sfera sui vari traguardi (\unit{cm}). Cinque sfere di diametro \unit[4/32]{''}.}
\centering \small
\begin{tabular}{r*5c}
10 &9.1740 &9.2749 &9.3356 &9.4129 &9.2203\\
20 &18.7734 &18.6987 &18.7404 &18.8126 &18.5922\\
30 &28.3608 &28.1444 &28.1084 &28.3104 &28.0037\\
40 &37.8503 &37.6021 &37.6675 &37.6935 &37.3833\\
50 &47.3757 &47.0648 &47.0555 &47.2037 &46.7653
\end{tabular}
\end{table}
\begin{table}[hp]\caption{Tempi (\unit{s}) rilevati per il passaggio della sfera sui vari traguardi (\unit{cm}). Cinque sfere di diametro \unit[5/32]{''}.}
\centering \small
\begin{tabular}{r*5c}
10 &5.8941 &5.9632 &5.9886 &5.9362 &5.8888\\
20 &11.9636 &11.9059 &12.0151 &11.9613 &11.6978\\
30 &17.9970 &17.9487 &18.0514 &18.0878 &17.7542\\
40 &24.0087 &23.8471 &24.0976 &23.8800 &23.7861\\
50 &29.9813 &29.8997 &30.0141 &29.7883 &29.7493
\end{tabular}
\end{table}
\begin{table}[hp]\caption{Tempi (\unit{s}) rilevati per il passaggio delle sfere da $D_7$ a $D_{10}$ sul traguardo dei \unit[50]{cm}.}
\centering \small
\begin{tabular}{r*5c}
$D_7$ &20.4155 &20.4998 &20.3931 &20.4337 &20.5051\\
$D_8$ &15.0105 &14.5210 &14.7475 &14.7935 &14.8191\\
$D_9$ &10.8372 &10.8669 &10.8057 &11.0179 &10.7077\\
$D_{10}$ &8.0948 &8.0813 &7.9078 &8.1931 &8.1917
\end{tabular}
\end{table}
\end{document}
