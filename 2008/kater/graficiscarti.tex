\documentclass[italian,a4paper]{article}
\usepackage[tight,nice]{units}
\usepackage{babel,amsmath,amssymb,amsthm,graphicx,url,wrapfig,multirow}
\usepackage[text={6in,9in},centering]{geometry}
\usepackage[utf8x]{inputenc}
\usepackage[T1]{fontenc}
\usepackage{ae,aecompl}
\usepackage[Euler]{upgreek}
\usepackage[footnotesize,bf]{caption}
\usepackage[usenames]{color}
\include{pstricks}
% \include{pstricks}
\frenchspacing
\pagestyle{plain}
%------------- eliminare prime e ultime linee isolate
\clubpenalty=9999%
\widowpenalty=9999
%--- definizione numerazioni
\renewcommand{\theequation}{\thesection.\arabic{equation}}
\renewcommand{\thefigure}{\thesection.\arabic{figure}}
\renewcommand{\thetable}{\thesection.\arabic{table}}
\addto\captionsitalian{%
  \renewcommand{\figurename}%
{Grafico}%
}
%
%------------- ridefinizione simbolo per elenchi puntati: en dash
%\renewcommand{\labelitemi}{\textbf{--}}
\renewcommand{\labelenumi}{\textbf{\arabic{enumi}.}}
\setlength{\abovecaptionskip}{\baselineskip}   % 0.5cm as an example
\setlength{\floatsep}{2\baselineskip}
\setlength{\belowcaptionskip}{\baselineskip}   % 0.5cm as an example
%------------- nuovi environment senza spazi
%\newenvironment{packed_item}{
%\begin{itemize}
%  \setlength{\itemsep}{1pt}
%  \setlength{\parskip}{0pt}
%  \setlength{\parsep}{0pt}
%}{\end{itemize}}
%\newenvironment{packed_enum}{
%\begin{enumerate}
%  \setlength{\itemsep}{1pt}
%  \setlength{\parskip}{0pt}
%  \setlength{\parsep}{0pt}
%}{\end{enumerate}}
%\newenvironment{packed_description}{
%\begin{enumerate}
%   \setlength{\itemsep}{1pt}
%   \setlength{\parskip}{0pt}
%   \setlength{\parsep}{0pt}
% }{\end{enumerate}}
%--------- comandi insiemi numeri complessi, naturali, reali e altre abbreviazioni
\newcommand{\micro}{\ensuremath{\upmu}} %prefisso micro
\newcommand{\e}{\mathrm{e}} %numero di nepero
\newcommand{\di}{\mathrm{d}} %simbolo di differenziale
\renewcommand{\leq}{\leqslant}
\renewcommand{\pi}{\uppi} % costante pi greco
\renewcommand{\tau}{\uptau} %momento della forza
\newcommand{\coloneqq}{\mathrel{\mathop:}=} % := ``per definizione''
\newcommand{\ms}{(\unitfrac{m}{s})}
%--------- porzione dedicata ai float in una pagina:
\renewcommand{\textfraction}{0.05}
\renewcommand{\topfraction}{0.95}
\renewcommand{\bottomfraction}{0.95}
\renewcommand{\floatpagefraction}{0.35}
\setcounter{totalnumber}{5}
%---------
%
%---------
\begin{document}
Da un calcolo numerico si trova, per un angolo di $5^\circ$:
\begin{equation*}
 I(5^\circ)=\dfrac 2 \pi \int_0^{5^\circ} \dfrac{\di x}{\sqrt{\cos x - \cos 5^\circ}} = 1.0004058\dots
\end{equation*}
Quindi l'approssimazione usata ($I(\alpha) \equiv 1$) risulta buona a meno di un'approssimazione dell'ordine di $4\cdot10^{-4}$. I seguenti grafici confermano questa influenza dell'errore sistematico.
\begin{figure}[h]
\centering
\caption{Prima serie di cento misure, cronometro automatico. Scarti $x_i - \bar{x}$ rispetto al numero di oscillazioni.}\label{scarti1}
 \include{scarti1}
\end{figure}
\begin{figure}[h]
\centering
\caption{Seconda serie di cento misure, cronometro automatico. Scarti $x_i - \bar{x}$ rispetto al numero di oscillazioni.}\label{scarti2}
 \include{scarti2}
\end{figure}
\begin{figure}[h]
\centering
\caption{Terza serie di cento misure, cronometro automatico. Scarti $x_i - \bar{x}$ rispetto al numero di oscillazioni.}\label{scarti3}
 \include{scarti3}
\end{figure}
\begin{figure}[h]
\centering
\caption{Quarta serie di cento misure, cronometro automatico. Scarti $x_i - \bar{x}$ rispetto al numero di oscillazioni.}\label{scarti4}
 \include{scarti4}
\end{figure}
\begin{figure}[h]
\centering
\caption{Quinta serie di cento misure, cronometro automatico. Scarti $x_i - \bar{x}$ rispetto al numero di oscillazioni.}\label{scarti5}
 \include{scarti5}
\end{figure}
\end{document}
