\documentclass[italian,a4paper]{article}
\usepackage[tight,nice]{units}
\usepackage{babel,amsmath,amssymb,amsthm,graphicx,url,gensymb}
\usepackage[text={5.5in,9in},centering]{geometry}
\usepackage[utf8x]{inputenc}
%\usepackage[T1]{fontenc}
\usepackage{ae,aecompl}
\usepackage[footnotesize,bf]{caption}
\usepackage[usenames]{color}
\include{pstricks}
\frenchspacing
\pagestyle{plain}
%------------- eliminare prime e ultime linee isolate
\clubpenalty=9999%
\widowpenalty=9999
%--- definizione numerazioni
\renewcommand{\theequation}{\thesection.\arabic{equation}}
\renewcommand{\thefigure}{\arabic{figure}}
\renewcommand{\thetable}{\arabic{table}}
\addto\captionsitalian{%
  \renewcommand{\figurename}%
{Grafico}%
}
%
%------------- ridefinizione simbolo per elenchi puntati: en dash
%\renewcommand{\labelitemi}{\textbf{--}}
\renewcommand{\labelenumi}{\textbf{\arabic{enumi}.}}
\newcommand{\mol}{mol}
\setlength{\abovecaptionskip}{\baselineskip}   % 0.5cm as an example
\setlength{\floatsep}{2\baselineskip}
\setlength{\belowcaptionskip}{\baselineskip}   % 0.5cm as an example
%--------- comandi insiemi numeri complessi, naturali, reali e altre abbreviazioni
\renewcommand{\leq}{\leqslant}
%--------- porzione dedicata ai float in una pagina:
\renewcommand{\textfraction}{0.05}
\renewcommand{\topfraction}{0.95}
\renewcommand{\bottomfraction}{0.95}
\renewcommand{\floatpagefraction}{0.35}
\setcounter{totalnumber}{5}
%---------
%
%---------
\begin{document}
\title{Relazione di laboratorio: la legge dei gas perfetti}
\author{\normalsize Ilaria Brivio (582116)\\%
\normalsize \url{brivio.ilaria@tiscali.it}%
\and %
\normalsize Matteo Abis (584206)\\ %
\normalsize \url{webmaster@latinblog.org}}
\date{\today}
\maketitle
%------------------
\section{Obiettivo dell'esperienza}
Obiettivo dell'esperienza è la verifica dell'equazione di stato dei gas perfetti e i casi particolari della legge di Boyle per le trasformazioni isoterme e di Gay-Lussac per le isocore.
\section{Descrizione dell'apparato strumentale}
Il sistema utilizzato è costituito da un Dewar cilindrico isolato termicamente dall'ambiente esterno, contenente acqua e ghiaccio in equilibrio termico. Attraverso il coperchio è introdotto un agitatore per rendere più possibile omogenea la temperatura dell'acqua. All'interno del cilindro è fissato un recipiente a forma di siringa monouso di volume massimo $V_{\text{max}}$ pari a circa $\unit[25]{cm^3}$, il cui pistone è collegato a un sistema a manovella che permette di variarne gradualmente la compressione. L'aria viene introdotta nella siringa attraverso due tubicini sottili collegati a una valvola a tre vie. Tale valvola viene aperta una sola volta nella fase iniziale dell'esperienza e con il pistone sollevato il più possibile. Il volume $V_0$ dei tubicini è molto piccolo ma non trascurabile ed è quindi causa di un errore sistematico nella stima del volume di gas contenuto all'interno della siringa. Volume, pressione e temperatura del gas sono misurati rispettivamente con un reostato di sensibilità $\unit[10]{cm^{-3}}$ collegato al pistone, un sensore di pressione di sensibilità $\unitfrac[10^2]{dm^3}{Kg}$ collegato alla valvola e una termocoppia di sensibilità $\unit[10]{\celsius^{-1}}$ posta vicino alla siringa. All'interno del Dewar è presente inoltre un riscaldatore costituito da una resistenza elettrica alimentata con un potenziale di \unit[30]{V} e una potenza di \unit[90]{W}.
Tutti gli strumenti sono controllati dal programma PiPerVi, attraverso il quale è possibile impostare la temperatura del sistema a meno di \unit[1]{\celsius}. 
\section{Descrizione della metodologia di misura}
Il gas è stato sottoposto a una compressione isoterma fino al raggiungimento del volume minimo $V_{\text{min}}$, ruotando lentamente la manovella collegata al pistone della siringa, ed è stato successivamente riportato allo stesso modo al volume iniziale. Il procedimento è stato ripetuto a diversi valori di temperatura impostata: $T_{\text{imp}} =$ 0, 15, 25, 35, 45, 55 \unit{\celsius}. Durante la trasformazione il programma PiPerVi ha registrato a intervalli di tempo regolari i valori di $V$, $\nicefrac{1}{P}$ e $T$ rilevati dai vari strumenti. Per ragioni tecniche il reostato non misura correttamente il volume a fine e inzio corsa, pertanto sono stati scartati i dati esterni all'intervallo da 5.91 a  $\unit[21.50]{cm^3}$.
Infine sono stati convertiti i dati in unità SI, secondo i fattori di conversione: $\unit[1]{cm^3}=\unit[10^{-6}]{m^3}$ e $\unitfrac[1]{Kg}{dm^3}=\unit[98060]{Pa}$.
\section{Risultati sperimentali ed elaborazione dati}
Per ogni temperatura $T_{\text{imp}}$ sono stati analizzati separatamente i due campioni di dati relativi alla compressione e all'espansione del gas.
Sono stati inizialmente interpolati con delle rette del tipo $y=ax+b$ e $y=cx+d$ i punti della curva $(\nicefrac{1}{P},V)$ (grafici in appendice) rispettivamente in compressione e in espansione. La proporzionalità diretta di $V$ da $\nicefrac{1}{P}$ verifica la legge dei gas $PV=nRT$, dove $n$ è il numero di moli di gas presenti nel contenitore, $R=\unitfrac[8.3136]{J}{K\mol}$ la costante di stato dei gas e $T$ la temperatura in Kelvin. In particolare, il coefficiente angolare della retta ($a$ e $c$) corrisponde al prodotto $nRT$, mentre l'ordinata all'origine rappresenta il volume di gas $V_0$ contenuto nei tubicini che collegano la siringa alla valvola e al sensore di pressione. I risultati ottenuti sono riportati nella tabella seguente (\emph{c} indica la fase di compressione, \emph{e} quella di espansione):
\begin{table}[h]
\centering
 \begin{tabular}{c r@{$\pm$}l r@{$\pm$}l}
 T (\unit{\celsius}) &  \multicolumn{2}{c}{$-V_0\ (\unit{cm^3}$)}  & \multicolumn{2}{c}{$nRT$ (\unit{J})}\\\hline
   0c & -0.9996 &  0.0016 &  2.31210 &  0.00027\\
   0e & -1.0976 &  0.0020 &  2.31991 &  0.00027\\
  15c & -1.0879 &  0.0018 &  2.45135 &  0.00027\\
  15e & -1.2297 &  0.0030 &  2.46484 &  0.00045\\
  25c & -1.1352 &  0.0022 &  2.52983 &  0.00036\\
  25e & -1.2594 &  0.0022 &  2.54258 &  0.00034\\
  35c & -1.1677 &  0.0018 &  2.60281 &  0.00030\\
  35e & -1.2966 &  0.0024 &  2.61611 &  0.00040\\
  45c & -1.2738 &  0.0021 &  2.68964 &  0.00038\\
  45e & -1.3551 &  0.0016 &  2.69543 &  0.00028\\
  55c & -1.4656 &  0.0027 &  2.78554 &  0.00049\\
  55e & -1.3749 &  0.0035 &  2.77694 &  0.00063\\
 \end{tabular}
\end{table}\\
Dei valori corrispondenti alle stesse temperature sono state calcolate le medie pesate. Il valore di temperatura riportato è anch'esso la media aritmetica delle temperature osservate:
\begin{table}[h]
\centering
 \begin{tabular}{c r@{$\pm$}l r@{$\pm$}l}
 T (\unit{\celsius}) &  \multicolumn{2}{c}{$V_0\ (\unit{cm^3}$)}  & \multicolumn{2}{c}{$nRT$ (\unit{J})}\\\hline
   0.363 & -1.0390 &  0.0013 &  2.31593 &  0.00019\\
  15.964 & -1.1257 &  0.0016 &  2.45490 &  0.00023\\
  25.508 & -1.1983 &  0.0015 &  2.53654 &  0.00025\\
  35.421 & -1.2153 &  0.0014 &  2.60769 &  0.00024\\
  45.309 & -1.3257 &  0.0013 &  2.69341 &  0.00022\\
  55.232 & -1.4307 &  0.0022 &  2.78235 &  0.00038\\
 \end{tabular}
\end{table}\\
Interpolando con una retta $T_0 + 1/nR \cdot nRT$, si possono ricavare la temperatura dello zero assoluto $T_0$ e il numero di moli di gas presenti nella siringa, con errore per propagazione. In ordinata si pone la temperatura media perché affetta da un errore relativo maggiore:
\begin{table}[h]
\centering
 \begin{tabular}{r@{$\pm$}l r@{$\pm$}l}
\multicolumn{2}{c}{$T_0\ (\celsius)$}  & \multicolumn{2}{c}{$1/nR$ (\unitfrac{K}{J})}\\\hline
-275.9 &  4.5 &  119.1 & 1.7\\
 \end{tabular}
\end{table}\\
Da questi si ricava subito $n = 0.001010\pm0.000015\unit{\mol}$. La compatibilità del valore $T_0$ con $\unit[-273.15\pm0.01]{\celsius}$ risulta 0.61, quindi buona. Infine, si è stimato il volume massimo della siringa riportando in grafico i valori minimi della pressione in funzione della temperatura media (grafici~\ref{t-pmincomp} e~\ref{t-pminesp}) e interpolando con una retta del tipo $P_{\text{min}} + nR/V_{\text{max}} T$. Le interpolazioni sono state eseguite separatamente per le fasi di espansione e compressione in quanto condizioni sperimentali diverse, ed è stata calcolata poi la media pesata dei volumi massimi ricavati, con errore per propagazione (vedi tabella~\ref{vmax}).
\begin{table}[ht]\caption{Interpolazione lineare per la determinazione del volume massimo.}\label{vmax}
\centering
 \begin{tabular}{r r@{$\pm$}l}
&\multicolumn{2}{c}{$nR/V_{\text{max}}\ (\unitfrac{J}{K m^3})$}\\\hline
compressione & $336.0$ & $6.9$\\
espansione  & $337.2$ & $7.4$\\
\\
&\multicolumn{2}{c}{$V_{\text{max}}\ (\unit{cm^3})$}\\\hline
compressione& 24.99&0.63\\
espansione&   24.90&0.66\\
media pesata& 24.95&0.46
 \end{tabular}
\end{table}\\
\section{Discussione dei risultati}
I risultati ottenuti si possono ritenere soddisfacenti, nonostante l'esperienza sia influenzata da alcuni errori sistematici. Il più importante è la presenza di gas nei tubi che collegano la siringa alla valvola di apertura e al sensore di pressione. Il volume di questi tubi non è del tutto trascurabile, ed è stato misurato come $V_0$ nella prima interpolazione. Inoltre, i tubi non sono alla stessa temperatura della siringa. Un altro tipo di errore deriva dal fatto che la trasformazione non è perfettamente isoterma, ma ci sono variazioni di temperatura anche di un grado. La siringa, così come i tubi (grafico~\ref{t-v0}), poi è soggetta a variazioni di volume dovute all'aumento di temperatura e pressione. Infine, la legge dei gas perfetti introduce un'approssimazione che però nel nostro caso si può ritenere molto buona, e quindi sostanzialmente ininfluente.
\section{Conclusioni}
La legge dei gas perfetti per trasformazioni isoterme e isocore è verificata. Inoltre la stima della temperatura di zero assoluto è molto vicina al valore atteso, soprattutto considerando le numerose approssimazioni effettuate nell'esperimento.
\newpage
\section{Appendice}
Nei grafici è disegnato un punto ogni dieci punti osservati. Il grande numero di punti infatti supera le capacità di memoria del \LaTeX. Tale quantità di dati è eccessiva per poter essere riportata sotto forma di tabella, ma si è ritenuto di doverli comunque rendere disponibili su internet all'indirizzo \url{http://www.latinblog.org/latex/misure.tar.gz}.
Gli errori nei grafici sono ovunque troppo piccoli per essere rappresentati.
\subsection*{Formule}
\begin{description}
 \item[Propagazione dell'errore, due variabili]
\begin{equation*}
 \sigma_f^2 = \left(\dfrac{\partial f}{\partial x}\sigma_x\right)^2 + \left(\dfrac{\partial f}{\partial y}\sigma_y\right)^2
\end{equation*}
 \item[Media pesata]
\begin{equation*}
 \bar{x}=\left(\sum_i \dfrac{x_i}{\sigma_{x_i}^2} \right)\left(\sum_i \dfrac{1}{\sigma_{x_i}^2} \right)^{-1} \qquad \sigma_{\bar{x}} = \left(\sum_i \dfrac{1}{\sigma_{x_i}^2} \right)^{-\nicefrac 1 2}
\end{equation*}
\item[Interpolazione lineare $y=kx+y_0$]
\begin{align*}
y_0 &= \dfrac{1}{\Delta}\left[ \left(\sum_{i=1}^Nx_i^2\right) \left(\sum_{i=1}^Ny_i\right)-\left(\sum_{i=1}^Nx_i\right)\left(\sum_{i=1}^Nx_iy_i\right)\right]\\[3pt]
k &= \dfrac{1}{\Delta}\left[N \left(\sum_{i=1}^Nx_iy_i\right)-\left(\sum_{i=1}^Nx_i\right)\left(\sum_{i=1}^Ny_i\right)\right]\\[3pt]
\Delta &= N\sum_{i=1}^Nx_i^2 - \left(\sum_{i=1}^Nx_i\right)^2\\[6pt]
\sigma_{y_0}^2 &= \dfrac{\sigma_y^2}{\Delta}\sum_{i=1}^Nx_i^2\\[3pt]
\sigma_k^2 &= \dfrac{N\sigma_y^2}{\Delta}
\end{align*}
\end{description}
\begin{figure}[p]\caption{Interpolazione dei dati con inverso della pressione in ascissa e volume in ordinata. Temperatura di 0 \celsius, fase di compressione.}\label{0c}
% GNUPLOT: LaTeX picture using PSTRICKS macros
% Define new PST objects, if not already defined
\ifx\PSTloaded\undefined
\def\PSTloaded{t}
\psset{arrowsize=.01 3.2 1.4 .3}
\psset{dotsize=.04}
\catcode`@=11

\newpsobject{PST@Border}{psline}{linewidth=.0015,linestyle=solid}
\newpsobject{PST@Axes}{psline}{linewidth=.0015,linestyle=dotted,dotsep=.004}
\newpsobject{PST@Solid}{psline}{linewidth=.0015,linestyle=solid}
\newpsobject{PST@Dashed}{psline}{linewidth=.0015,linestyle=dashed,dash=.01 .01}
\newpsobject{PST@Dotted}{psline}{linewidth=.0025,linestyle=dotted,dotsep=.008}
\newpsobject{PST@LongDash}{psline}{linewidth=.0015,linestyle=dashed,dash=.02 .01}
\newpsobject{PST@Diamond}{psdots}{linewidth=.001,linestyle=solid,dotstyle=square,dotangle=45}
\newpsobject{PST@Filldiamond}{psdots}{linewidth=.001,linestyle=solid,dotstyle=square*,dotangle=45}
\newpsobject{PST@Cross}{psdots}{linewidth=.001,linestyle=solid,dotstyle=+,dotangle=45}
\newpsobject{PST@Plus}{psdots}{linewidth=.001,linestyle=solid,dotstyle=+}
\newpsobject{PST@Square}{psdots}{linewidth=.001,linestyle=solid,dotstyle=square}
\newpsobject{PST@Circle}{psdots}{linewidth=.001,linestyle=solid,dotstyle=o}
\newpsobject{PST@Triangle}{psdots}{linewidth=.001,linestyle=solid,dotstyle=triangle}
\newpsobject{PST@Pentagon}{psdots}{linewidth=.001,linestyle=solid,dotstyle=pentagon}
\newpsobject{PST@Fillsquare}{psdots}{linewidth=.001,linestyle=solid,dotstyle=square*}
\newpsobject{PST@Fillcircle}{psdots}{linewidth=.001,linestyle=solid,dotstyle=*}
\newpsobject{PST@Filltriangle}{psdots}{linewidth=.001,linestyle=solid,dotstyle=triangle*}
\newpsobject{PST@Fillpentagon}{psdots}{linewidth=.001,linestyle=solid,dotstyle=pentagon*}
\newpsobject{PST@Arrow}{psline}{linewidth=.001,linestyle=solid}
\catcode`@=12

\fi
\psset{unit=5.0in,xunit=5.0in,yunit=3.0in}
\pspicture(0.000000,0.000000)(1.000000,1.000000)
\ifx\nofigs\undefined
\catcode`@=11

\PST@Border(0.1270,0.1260)
(0.1420,0.1260)

\rput[r](0.1110,0.1260){4}
\PST@Border(0.1270,0.2196)
(0.1420,0.2196)

\rput[r](0.1110,0.2196){6}
\PST@Border(0.1270,0.3131)
(0.1420,0.3131)

\rput[r](0.1110,0.3131){8}
\PST@Border(0.1270,0.4067)
(0.1420,0.4067)

\rput[r](0.1110,0.4067){10}
\PST@Border(0.1270,0.5002)
(0.1420,0.5002)

\rput[r](0.1110,0.5002){12}
\PST@Border(0.1270,0.5938)
(0.1420,0.5938)

\rput[r](0.1110,0.5938){14}
\PST@Border(0.1270,0.6873)
(0.1420,0.6873)

\rput[r](0.1110,0.6873){16}
\PST@Border(0.1270,0.7809)
(0.1420,0.7809)

\rput[r](0.1110,0.7809){18}
\PST@Border(0.1270,0.8744)
(0.1420,0.8744)

\rput[r](0.1110,0.8744){20}
\PST@Border(0.1270,0.9680)
(0.1420,0.9680)

\rput[r](0.1110,0.9680){22}
\PST@Border(0.1270,0.1260)
(0.1270,0.1460)

\rput(0.1270,0.0840){3}
\PST@Border(0.2441,0.1260)
(0.2441,0.1460)

\rput(0.2441,0.0840){4}
\PST@Border(0.3613,0.1260)
(0.3613,0.1460)

\rput(0.3613,0.0840){5}
\PST@Border(0.4784,0.1260)
(0.4784,0.1460)

\rput(0.4784,0.0840){6}
\PST@Border(0.5956,0.1260)
(0.5956,0.1460)

\rput(0.5956,0.0840){7}
\PST@Border(0.7127,0.1260)
(0.7127,0.1460)

\rput(0.7127,0.0840){8}
\PST@Border(0.8299,0.1260)
(0.8299,0.1460)

\rput(0.8299,0.0840){9}
\PST@Border(0.9470,0.1260)
(0.9470,0.1460)

\rput(0.9470,0.0840){10}
\PST@Border(0.1270,0.9680)
(0.1270,0.1260)
(0.9470,0.1260)
(0.9470,0.9680)
(0.1270,0.9680)

\rput{L}(0.0420,0.5470){$V\ (\unit{cm^3})$}
\rput(0.5370,0.0210){$1/P\ (\unit{Pa^{-1}})$}
\PST@Diamond(0.9164,0.9446)
\PST@Diamond(0.9056,0.9348)
\PST@Diamond(0.8949,0.9259)
\PST@Diamond(0.8841,0.9156)
\PST@Diamond(0.8745,0.9063)
\PST@Diamond(0.8614,0.8941)
\PST@Diamond(0.8446,0.8791)
\PST@Diamond(0.8303,0.8660)
\PST@Diamond(0.8255,0.8599)
\PST@Diamond(0.8219,0.8576)
\PST@Diamond(0.8171,0.8539)
\PST@Diamond(0.8124,0.8487)
\PST@Diamond(0.8088,0.8450)
\PST@Diamond(0.8028,0.8398)
\PST@Diamond(0.7980,0.8352)
\PST@Diamond(0.7920,0.8305)
\PST@Diamond(0.7849,0.8235)
\PST@Diamond(0.7741,0.8118)
\PST@Diamond(0.7657,0.8061)
\PST@Diamond(0.7573,0.7982)
\PST@Diamond(0.7454,0.7884)
\PST@Diamond(0.7334,0.7762)
\PST@Diamond(0.7215,0.7659)
\PST@Diamond(0.7083,0.7528)
\PST@Diamond(0.6976,0.7435)
\PST@Diamond(0.6916,0.7379)
\PST@Diamond(0.6784,0.7262)
\PST@Diamond(0.6676,0.7168)
\PST@Diamond(0.6581,0.7065)
\PST@Diamond(0.6509,0.7009)
\PST@Diamond(0.6414,0.6920)
\PST@Diamond(0.6306,0.6822)
\PST@Diamond(0.6162,0.6691)
\PST@Diamond(0.6007,0.6555)
\PST@Diamond(0.5875,0.6424)
\PST@Diamond(0.5780,0.6340)
\PST@Diamond(0.5661,0.6228)
\PST@Diamond(0.5588,0.6153)
\PST@Diamond(0.5433,0.6013)
\PST@Diamond(0.5277,0.5877)
\PST@Diamond(0.5158,0.5760)
\PST@Diamond(0.5062,0.5676)
\PST@Diamond(0.4942,0.5564)
\PST@Diamond(0.4846,0.5470)
\PST@Diamond(0.4692,0.5334)
\PST@Diamond(0.4644,0.5278)
\PST@Diamond(0.4548,0.5203)
\PST@Diamond(0.4500,0.5157)
\PST@Diamond(0.4453,0.5100)
\PST@Diamond(0.4416,0.5063)
\PST@Diamond(0.4357,0.5016)
\PST@Diamond(0.4320,0.4984)
\PST@Diamond(0.4272,0.4937)
\PST@Diamond(0.4213,0.4876)
\PST@Diamond(0.4153,0.4829)
\PST@Diamond(0.4118,0.4796)
\PST@Diamond(0.4070,0.4754)
\PST@Diamond(0.4046,0.4736)
\PST@Diamond(0.3985,0.4675)
\PST@Diamond(0.3914,0.4605)
\PST@Diamond(0.3879,0.4572)
\PST@Diamond(0.3831,0.4534)
\PST@Diamond(0.3783,0.4492)
\PST@Diamond(0.3687,0.4403)
\PST@Diamond(0.3627,0.4343)
\PST@Diamond(0.3567,0.4296)
\PST@Diamond(0.3532,0.4254)
\PST@Diamond(0.3472,0.4202)
\PST@Diamond(0.3411,0.4151)
\PST@Diamond(0.3376,0.4113)
\PST@Diamond(0.3352,0.4090)
\PST@Diamond(0.3328,0.4067)
\PST@Diamond(0.3268,0.4013)
\PST@Diamond(0.3220,0.3967)
\PST@Diamond(0.3185,0.3933)
\PST@Diamond(0.3137,0.3894)
\PST@Diamond(0.3101,0.3862)
\PST@Diamond(0.3089,0.3844)
\PST@Diamond(0.3053,0.3812)
\PST@Diamond(0.3005,0.3765)
\PST@Diamond(0.2970,0.3733)
\PST@Diamond(0.2933,0.3702)
\PST@Diamond(0.2910,0.3679)
\PST@Diamond(0.2885,0.3656)
\PST@Diamond(0.2862,0.3635)
\PST@Diamond(0.2839,0.3616)
\PST@Diamond(0.2826,0.3599)
\PST@Diamond(0.2791,0.3568)
\PST@Diamond(0.2766,0.3543)
\PST@Diamond(0.2742,0.3515)
\PST@Diamond(0.2706,0.3487)
\PST@Diamond(0.2671,0.3455)
\PST@Diamond(0.2658,0.3448)
\PST@Diamond(0.2646,0.3436)
\PST@Diamond(0.2623,0.3414)
\PST@Diamond(0.2575,0.3372)
\PST@Diamond(0.2539,0.3336)
\PST@Diamond(0.2504,0.3293)
\PST@Diamond(0.2455,0.3252)
\PST@Diamond(0.2431,0.3228)
\PST@Diamond(0.2359,0.3169)
\PST@Diamond(0.2324,0.3136)
\PST@Diamond(0.2252,0.3068)
\PST@Diamond(0.2217,0.3024)
\PST@Diamond(0.2144,0.2967)
\PST@Diamond(0.2097,0.2927)
\PST@Diamond(0.2072,0.2899)
\PST@Diamond(0.2024,0.2852)
\PST@Diamond(0.1989,0.2813)
\PST@Diamond(0.1917,0.2760)
\PST@Diamond(0.1881,0.2718)
\PST@Diamond(0.1833,0.2681)
\PST@Diamond(0.1762,0.2614)
\PST@Diamond(0.1678,0.2548)
\PST@Diamond(0.1570,0.2447)
\PST@Diamond(0.1463,0.2349)
\PST@Diamond(0.1296,0.2202)
\PST@Dashed(0.1296,0.2190)
(0.1296,0.2190)
(0.1375,0.2263)
(0.1455,0.2337)
(0.1534,0.2410)
(0.1614,0.2483)
(0.1693,0.2557)
(0.1773,0.2630)
(0.1852,0.2703)
(0.1932,0.2777)
(0.2011,0.2850)
(0.2091,0.2924)
(0.2170,0.2997)
(0.2250,0.3070)
(0.2329,0.3144)
(0.2408,0.3217)
(0.2488,0.3290)
(0.2567,0.3364)
(0.2647,0.3437)
(0.2726,0.3511)
(0.2806,0.3584)
(0.2885,0.3657)
(0.2965,0.3731)
(0.3044,0.3804)
(0.3124,0.3878)
(0.3203,0.3951)
(0.3283,0.4024)
(0.3362,0.4098)
(0.3442,0.4171)
(0.3521,0.4244)
(0.3601,0.4318)
(0.3680,0.4391)
(0.3760,0.4465)
(0.3839,0.4538)
(0.3919,0.4611)
(0.3998,0.4685)
(0.4078,0.4758)
(0.4157,0.4831)
(0.4237,0.4905)
(0.4316,0.4978)
(0.4395,0.5052)
(0.4475,0.5125)
(0.4554,0.5198)
(0.4634,0.5272)
(0.4713,0.5345)
(0.4793,0.5419)
(0.4872,0.5492)
(0.4952,0.5565)
(0.5031,0.5639)
(0.5111,0.5712)
(0.5190,0.5785)
(0.5270,0.5859)
(0.5349,0.5932)
(0.5429,0.6006)
(0.5508,0.6079)
(0.5588,0.6152)
(0.5667,0.6226)
(0.5747,0.6299)
(0.5826,0.6372)
(0.5906,0.6446)
(0.5985,0.6519)
(0.6065,0.6593)
(0.6144,0.6666)
(0.6224,0.6739)
(0.6303,0.6813)
(0.6382,0.6886)
(0.6462,0.6960)
(0.6541,0.7033)
(0.6621,0.7106)
(0.6700,0.7180)
(0.6780,0.7253)
(0.6859,0.7326)
(0.6939,0.7400)
(0.7018,0.7473)
(0.7098,0.7547)
(0.7177,0.7620)
(0.7257,0.7693)
(0.7336,0.7767)
(0.7416,0.7840)
(0.7495,0.7913)
(0.7575,0.7987)
(0.7654,0.8060)
(0.7734,0.8134)
(0.7813,0.8207)
(0.7893,0.8280)
(0.7972,0.8354)
(0.8052,0.8427)
(0.8131,0.8501)
(0.8211,0.8574)
(0.8290,0.8647)
(0.8369,0.8721)
(0.8449,0.8794)
(0.8528,0.8867)
(0.8608,0.8941)
(0.8687,0.9014)
(0.8767,0.9088)
(0.8846,0.9161)
(0.8926,0.9234)
(0.9005,0.9308)
(0.9085,0.9381)
(0.9164,0.9455)

\PST@Border(0.1270,0.9680)
(0.1270,0.1260)
(0.9470,0.1260)
(0.9470,0.9680)
(0.1270,0.9680)

\catcode`@=12
\fi
\endpspicture

\end{figure}
\begin{figure}[p]\caption{Interpolazione dei dati con inverso della pressione in ascissa e volume in ordinata. Temperatura di 0 \celsius, fase di espansione.}\label{0e}
% GNUPLOT: LaTeX picture using PSTRICKS macros
% Define new PST objects, if not already defined
\ifx\PSTloaded\undefined
\def\PSTloaded{t}
\psset{arrowsize=.01 3.2 1.4 .3}
\psset{dotsize=.04}
\catcode`@=11

\newpsobject{PST@Border}{psline}{linewidth=.0015,linestyle=solid}
\newpsobject{PST@Axes}{psline}{linewidth=.0015,linestyle=dotted,dotsep=.004}
\newpsobject{PST@Solid}{psline}{linewidth=.0015,linestyle=solid}
\newpsobject{PST@Dashed}{psline}{linewidth=.0015,linestyle=dashed,dash=.01 .01}
\newpsobject{PST@Dotted}{psline}{linewidth=.0025,linestyle=dotted,dotsep=.008}
\newpsobject{PST@LongDash}{psline}{linewidth=.0015,linestyle=dashed,dash=.02 .01}
\newpsobject{PST@Diamond}{psdots}{linewidth=.001,linestyle=solid,dotstyle=square,dotangle=45}
\newpsobject{PST@Filldiamond}{psdots}{linewidth=.001,linestyle=solid,dotstyle=square*,dotangle=45}
\newpsobject{PST@Cross}{psdots}{linewidth=.001,linestyle=solid,dotstyle=+,dotangle=45}
\newpsobject{PST@Plus}{psdots}{linewidth=.001,linestyle=solid,dotstyle=+}
\newpsobject{PST@Square}{psdots}{linewidth=.001,linestyle=solid,dotstyle=square}
\newpsobject{PST@Circle}{psdots}{linewidth=.001,linestyle=solid,dotstyle=o}
\newpsobject{PST@Triangle}{psdots}{linewidth=.001,linestyle=solid,dotstyle=triangle}
\newpsobject{PST@Pentagon}{psdots}{linewidth=.001,linestyle=solid,dotstyle=pentagon}
\newpsobject{PST@Fillsquare}{psdots}{linewidth=.001,linestyle=solid,dotstyle=square*}
\newpsobject{PST@Fillcircle}{psdots}{linewidth=.001,linestyle=solid,dotstyle=*}
\newpsobject{PST@Filltriangle}{psdots}{linewidth=.001,linestyle=solid,dotstyle=triangle*}
\newpsobject{PST@Fillpentagon}{psdots}{linewidth=.001,linestyle=solid,dotstyle=pentagon*}
\newpsobject{PST@Arrow}{psline}{linewidth=.001,linestyle=solid}
\catcode`@=12

\fi
\psset{unit=5.0in,xunit=5.0in,yunit=3.0in}
\pspicture(0.000000,0.000000)(1.000000,1.000000)
\ifx\nofigs\undefined
\catcode`@=11

\PST@Border(0.1270,0.1260)
(0.1420,0.1260)

\rput[r](0.1110,0.1260){4}
\PST@Border(0.1270,0.2196)
(0.1420,0.2196)

\rput[r](0.1110,0.2196){6}
\PST@Border(0.1270,0.3131)
(0.1420,0.3131)

\rput[r](0.1110,0.3131){8}
\PST@Border(0.1270,0.4067)
(0.1420,0.4067)

\rput[r](0.1110,0.4067){10}
\PST@Border(0.1270,0.5002)
(0.1420,0.5002)

\rput[r](0.1110,0.5002){12}
\PST@Border(0.1270,0.5938)
(0.1420,0.5938)

\rput[r](0.1110,0.5938){14}
\PST@Border(0.1270,0.6873)
(0.1420,0.6873)

\rput[r](0.1110,0.6873){16}
\PST@Border(0.1270,0.7809)
(0.1420,0.7809)

\rput[r](0.1110,0.7809){18}
\PST@Border(0.1270,0.8744)
(0.1420,0.8744)

\rput[r](0.1110,0.8744){20}
\PST@Border(0.1270,0.9680)
(0.1420,0.9680)

\rput[r](0.1110,0.9680){22}
\PST@Border(0.1270,0.1260)
(0.1270,0.1460)

\rput(0.1270,0.0840){2}
\PST@Border(0.2295,0.1260)
(0.2295,0.1460)

\rput(0.2295,0.0840){3}
\PST@Border(0.3320,0.1260)
(0.3320,0.1460)

\rput(0.3320,0.0840){4}
\PST@Border(0.4345,0.1260)
(0.4345,0.1460)

\rput(0.4345,0.0840){5}
\PST@Border(0.5370,0.1260)
(0.5370,0.1460)

\rput(0.5370,0.0840){6}
\PST@Border(0.6395,0.1260)
(0.6395,0.1460)

\rput(0.6395,0.0840){7}
\PST@Border(0.7420,0.1260)
(0.7420,0.1460)

\rput(0.7420,0.0840){8}
\PST@Border(0.8445,0.1260)
(0.8445,0.1460)

\rput(0.8445,0.0840){9}
\PST@Border(0.9470,0.1260)
(0.9470,0.1460)

\rput(0.9470,0.0840){10}
\PST@Border(0.1270,0.9680)
(0.1270,0.1260)
(0.9470,0.1260)
(0.9470,0.9680)
(0.1270,0.9680)

\rput{L}(0.0420,0.5470){$V\ (\unit{cm^3})$}
\rput(0.5370,0.0210){$1/P\ (\unit{Pa^{-1}})$}
\PST@Diamond(0.2254,0.2157)
\PST@Diamond(0.2359,0.2204)
\PST@Diamond(0.2453,0.2303)
\PST@Diamond(0.2557,0.2409)
\PST@Diamond(0.2621,0.2480)
\PST@Diamond(0.2736,0.2589)
\PST@Diamond(0.2820,0.2684)
\PST@Diamond(0.2914,0.2784)
\PST@Diamond(0.3019,0.2897)
\PST@Diamond(0.3081,0.2973)
\PST@Diamond(0.3154,0.3040)
\PST@Diamond(0.3248,0.3144)
\PST@Diamond(0.3363,0.3264)
\PST@Diamond(0.3489,0.3401)
\PST@Diamond(0.3573,0.3486)
\PST@Diamond(0.3646,0.3568)
\PST@Diamond(0.3740,0.3665)
\PST@Diamond(0.3845,0.3773)
\PST@Diamond(0.3907,0.3846)
\PST@Diamond(0.3960,0.3898)
\PST@Diamond(0.4044,0.3993)
\PST@Diamond(0.4096,0.4046)
\PST@Diamond(0.4158,0.4113)
\PST@Diamond(0.4211,0.4174)
\PST@Diamond(0.4285,0.4249)
\PST@Diamond(0.4347,0.4310)
\PST@Diamond(0.4441,0.4389)
\PST@Diamond(0.4504,0.4474)
\PST@Diamond(0.4588,0.4558)
\PST@Diamond(0.4671,0.4637)
\PST@Diamond(0.4724,0.4708)
\PST@Diamond(0.4787,0.4768)
\PST@Diamond(0.4839,0.4824)
\PST@Diamond(0.4881,0.4876)
\PST@Diamond(0.4912,0.4909)
\PST@Diamond(0.5006,0.4993)
\PST@Diamond(0.5090,0.5082)
\PST@Diamond(0.5111,0.5114)
\PST@Diamond(0.5184,0.5180)
\PST@Diamond(0.5227,0.5231)
\PST@Diamond(0.5289,0.5288)
\PST@Diamond(0.5341,0.5348)
\PST@Diamond(0.5394,0.5400)
\PST@Diamond(0.5424,0.5442)
\PST@Diamond(0.5478,0.5493)
\PST@Diamond(0.5530,0.5554)
\PST@Diamond(0.5592,0.5615)
\PST@Diamond(0.5665,0.5690)
\PST@Diamond(0.5717,0.5746)
\PST@Diamond(0.5760,0.5793)
\PST@Diamond(0.5822,0.5863)
\PST@Diamond(0.5864,0.5900)
\PST@Diamond(0.5906,0.5942)
\PST@Diamond(0.5958,0.5994)
\PST@Diamond(0.5938,0.5980)
\PST@Diamond(0.6011,0.6050)
\PST@Diamond(0.6084,0.6130)
\PST@Diamond(0.6125,0.6181)
\PST@Diamond(0.6199,0.6261)
\PST@Diamond(0.6241,0.6303)
\PST@Diamond(0.6294,0.6359)
\PST@Diamond(0.6366,0.6438)
\PST@Diamond(0.6429,0.6504)
\PST@Diamond(0.6513,0.6593)
\PST@Diamond(0.6597,0.6677)
\PST@Diamond(0.6659,0.6747)
\PST@Diamond(0.6744,0.6827)
\PST@Diamond(0.6848,0.6939)
\PST@Diamond(0.6879,0.6986)
\PST@Diamond(0.6879,0.6990)
\PST@Diamond(0.6879,0.6990)
\PST@Diamond(0.6879,0.6990)
\PST@Diamond(0.6879,0.6990)
\PST@Diamond(0.6879,0.6990)
\PST@Diamond(0.6879,0.6990)
\PST@Diamond(0.6879,0.6995)
\PST@Diamond(0.6879,0.6990)
\PST@Diamond(0.7361,0.7435)
\PST@Diamond(0.7319,0.7435)
\PST@Diamond(0.7308,0.7439)
\PST@Diamond(0.7308,0.7439)
\PST@Diamond(0.7308,0.7439)
\PST@Diamond(0.7308,0.7439)
\PST@Diamond(0.7308,0.7439)
\PST@Diamond(0.7308,0.7439)
\PST@Diamond(0.7308,0.7435)
\PST@Diamond(0.7319,0.7458)
\PST@Diamond(0.7465,0.7598)
\PST@Diamond(0.7455,0.7594)
\PST@Diamond(0.7465,0.7608)
\PST@Diamond(0.7507,0.7650)
\PST@Diamond(0.7528,0.7673)
\PST@Diamond(0.7549,0.7692)
\PST@Diamond(0.7559,0.7711)
\PST@Diamond(0.7591,0.7734)
\PST@Diamond(0.7632,0.7776)
\PST@Diamond(0.7664,0.7814)
\PST@Diamond(0.7696,0.7860)
\PST@Diamond(0.7716,0.7879)
\PST@Diamond(0.7738,0.7898)
\PST@Diamond(0.7769,0.7935)
\PST@Diamond(0.7811,0.7973)
\PST@Diamond(0.7853,0.8010)
\PST@Diamond(0.7883,0.8038)
\PST@Diamond(0.7925,0.8090)
\PST@Diamond(0.7936,0.8113)
\PST@Diamond(0.7999,0.8164)
\PST@Diamond(0.8051,0.8221)
\PST@Diamond(0.8124,0.8295)
\PST@Diamond(0.8146,0.8323)
\PST@Diamond(0.8176,0.8356)
\PST@Diamond(0.8198,0.8380)
\PST@Diamond(0.8240,0.8417)
\PST@Diamond(0.8281,0.8464)
\PST@Diamond(0.8345,0.8534)
\PST@Diamond(0.8375,0.8567)
\PST@Diamond(0.8417,0.8609)
\PST@Diamond(0.8459,0.8660)
\PST@Diamond(0.8501,0.8693)
\PST@Diamond(0.8522,0.8735)
\PST@Diamond(0.8564,0.8773)
\PST@Diamond(0.8606,0.8815)
\PST@Diamond(0.8658,0.8866)
\PST@Diamond(0.8689,0.8903)
\PST@Diamond(0.8721,0.8941)
\PST@Diamond(0.8742,0.8960)
\PST@Diamond(0.8763,0.8988)
\PST@Diamond(0.8783,0.9002)
\PST@Diamond(0.8795,0.9020)
\PST@Diamond(0.8825,0.9044)
\PST@Diamond(0.8847,0.9067)
\PST@Diamond(0.8878,0.9100)
\PST@Diamond(0.8909,0.9137)
\PST@Diamond(0.8930,0.9170)
\PST@Diamond(0.8982,0.9212)
\PST@Diamond(0.9004,0.9245)
\PST@Diamond(0.9034,0.9268)
\PST@Diamond(0.9056,0.9301)
\PST@Diamond(0.9066,0.9315)
\PST@Diamond(0.9108,0.9348)
\PST@Diamond(0.9139,0.9381)
\PST@Diamond(0.9181,0.9437)
\PST@Dashed(0.2254,0.2088)
(0.2254,0.2088)
(0.2324,0.2162)
(0.2394,0.2236)
(0.2464,0.2310)
(0.2534,0.2384)
(0.2604,0.2458)
(0.2674,0.2532)
(0.2744,0.2606)
(0.2814,0.2680)
(0.2884,0.2754)
(0.2954,0.2828)
(0.3024,0.2903)
(0.3094,0.2977)
(0.3164,0.3051)
(0.3234,0.3125)
(0.3304,0.3199)
(0.3374,0.3273)
(0.3443,0.3347)
(0.3513,0.3421)
(0.3583,0.3495)
(0.3653,0.3569)
(0.3723,0.3643)
(0.3793,0.3717)
(0.3863,0.3791)
(0.3933,0.3866)
(0.4003,0.3940)
(0.4073,0.4014)
(0.4143,0.4088)
(0.4213,0.4162)
(0.4283,0.4236)
(0.4353,0.4310)
(0.4423,0.4384)
(0.4493,0.4458)
(0.4563,0.4532)
(0.4633,0.4606)
(0.4703,0.4680)
(0.4773,0.4754)
(0.4843,0.4829)
(0.4913,0.4903)
(0.4983,0.4977)
(0.5053,0.5051)
(0.5123,0.5125)
(0.5193,0.5199)
(0.5263,0.5273)
(0.5333,0.5347)
(0.5403,0.5421)
(0.5473,0.5495)
(0.5543,0.5569)
(0.5613,0.5643)
(0.5682,0.5718)
(0.5752,0.5792)
(0.5822,0.5866)
(0.5892,0.5940)
(0.5962,0.6014)
(0.6032,0.6088)
(0.6102,0.6162)
(0.6172,0.6236)
(0.6242,0.6310)
(0.6312,0.6384)
(0.6382,0.6458)
(0.6452,0.6532)
(0.6522,0.6606)
(0.6592,0.6681)
(0.6662,0.6755)
(0.6732,0.6829)
(0.6802,0.6903)
(0.6872,0.6977)
(0.6942,0.7051)
(0.7012,0.7125)
(0.7082,0.7199)
(0.7152,0.7273)
(0.7222,0.7347)
(0.7292,0.7421)
(0.7362,0.7495)
(0.7432,0.7569)
(0.7502,0.7644)
(0.7572,0.7718)
(0.7642,0.7792)
(0.7712,0.7866)
(0.7782,0.7940)
(0.7852,0.8014)
(0.7922,0.8088)
(0.7991,0.8162)
(0.8061,0.8236)
(0.8131,0.8310)
(0.8201,0.8384)
(0.8271,0.8458)
(0.8341,0.8532)
(0.8411,0.8607)
(0.8481,0.8681)
(0.8551,0.8755)
(0.8621,0.8829)
(0.8691,0.8903)
(0.8761,0.8977)
(0.8831,0.9051)
(0.8901,0.9125)
(0.8971,0.9199)
(0.9041,0.9273)
(0.9111,0.9347)
(0.9181,0.9421)

\PST@Border(0.1270,0.9680)
(0.1270,0.1260)
(0.9470,0.1260)
(0.9470,0.9680)
(0.1270,0.9680)

\catcode`@=12
\fi
\endpspicture

\end{figure}
\begin{figure}[p]\caption{Interpolazione dei dati con inverso della pressione in ascissa e volume in ordinata. Temperatura di 15 \celsius, fase di compressione.}\label{15c}
% GNUPLOT: LaTeX picture using PSTRICKS macros
% Define new PST objects, if not already defined
\ifx\PSTloaded\undefined
\def\PSTloaded{t}
\psset{arrowsize=.01 3.2 1.4 .3}
\psset{dotsize=.04}
\catcode`@=11

\newpsobject{PST@Border}{psline}{linewidth=.0015,linestyle=solid}
\newpsobject{PST@Axes}{psline}{linewidth=.0015,linestyle=dotted,dotsep=.004}
\newpsobject{PST@Solid}{psline}{linewidth=.0015,linestyle=solid}
\newpsobject{PST@Dashed}{psline}{linewidth=.0015,linestyle=dashed,dash=.01 .01}
\newpsobject{PST@Dotted}{psline}{linewidth=.0025,linestyle=dotted,dotsep=.008}
\newpsobject{PST@LongDash}{psline}{linewidth=.0015,linestyle=dashed,dash=.02 .01}
\newpsobject{PST@Diamond}{psdots}{linewidth=.001,linestyle=solid,dotstyle=square,dotangle=45}
\newpsobject{PST@Filldiamond}{psdots}{linewidth=.001,linestyle=solid,dotstyle=square*,dotangle=45}
\newpsobject{PST@Cross}{psdots}{linewidth=.001,linestyle=solid,dotstyle=+,dotangle=45}
\newpsobject{PST@Plus}{psdots}{linewidth=.001,linestyle=solid,dotstyle=+}
\newpsobject{PST@Square}{psdots}{linewidth=.001,linestyle=solid,dotstyle=square}
\newpsobject{PST@Circle}{psdots}{linewidth=.001,linestyle=solid,dotstyle=o}
\newpsobject{PST@Triangle}{psdots}{linewidth=.001,linestyle=solid,dotstyle=triangle}
\newpsobject{PST@Pentagon}{psdots}{linewidth=.001,linestyle=solid,dotstyle=pentagon}
\newpsobject{PST@Fillsquare}{psdots}{linewidth=.001,linestyle=solid,dotstyle=square*}
\newpsobject{PST@Fillcircle}{psdots}{linewidth=.001,linestyle=solid,dotstyle=*}
\newpsobject{PST@Filltriangle}{psdots}{linewidth=.001,linestyle=solid,dotstyle=triangle*}
\newpsobject{PST@Fillpentagon}{psdots}{linewidth=.001,linestyle=solid,dotstyle=pentagon*}
\newpsobject{PST@Arrow}{psline}{linewidth=.001,linestyle=solid}
\catcode`@=12

\fi
\psset{unit=5.0in,xunit=5.0in,yunit=3.0in}
\pspicture(0.000000,0.000000)(1.000000,1.000000)
\ifx\nofigs\undefined
\catcode`@=11

\PST@Border(0.1270,0.1260)
(0.1420,0.1260)

\rput[r](0.1110,0.1260){4}
\PST@Border(0.1270,0.2196)
(0.1420,0.2196)

\rput[r](0.1110,0.2196){6}
\PST@Border(0.1270,0.3131)
(0.1420,0.3131)

\rput[r](0.1110,0.3131){8}
\PST@Border(0.1270,0.4067)
(0.1420,0.4067)

\rput[r](0.1110,0.4067){10}
\PST@Border(0.1270,0.5002)
(0.1420,0.5002)

\rput[r](0.1110,0.5002){12}
\PST@Border(0.1270,0.5938)
(0.1420,0.5938)

\rput[r](0.1110,0.5938){14}
\PST@Border(0.1270,0.6873)
(0.1420,0.6873)

\rput[r](0.1110,0.6873){16}
\PST@Border(0.1270,0.7809)
(0.1420,0.7809)

\rput[r](0.1110,0.7809){18}
\PST@Border(0.1270,0.8744)
(0.1420,0.8744)

\rput[r](0.1110,0.8744){20}
\PST@Border(0.1270,0.9680)
(0.1420,0.9680)

\rput[r](0.1110,0.9680){22}
\PST@Border(0.1270,0.1260)
(0.1270,0.1460)

\rput(0.1270,0.0840){2}
\PST@Border(0.2295,0.1260)
(0.2295,0.1460)

\rput(0.2295,0.0840){3}
\PST@Border(0.3320,0.1260)
(0.3320,0.1460)

\rput(0.3320,0.0840){4}
\PST@Border(0.4345,0.1260)
(0.4345,0.1460)

\rput(0.4345,0.0840){5}
\PST@Border(0.5370,0.1260)
(0.5370,0.1460)

\rput(0.5370,0.0840){6}
\PST@Border(0.6395,0.1260)
(0.6395,0.1460)

\rput(0.6395,0.0840){7}
\PST@Border(0.7420,0.1260)
(0.7420,0.1460)

\rput(0.7420,0.0840){8}
\PST@Border(0.8445,0.1260)
(0.8445,0.1460)

\rput(0.8445,0.0840){9}
\PST@Border(0.9470,0.1260)
(0.9470,0.1460)

\rput(0.9470,0.0840){10}
\PST@Border(0.1270,0.9680)
(0.1270,0.1260)
(0.9470,0.1260)
(0.9470,0.9680)
(0.1270,0.9680)

\rput{L}(0.0420,0.5470){$V\ (\unit{cm^3})$}
\rput(0.5370,0.0210){$1/P\ (\unit{Pa^{-1}})$}
\PST@Diamond(0.8658,0.9446)
\PST@Diamond(0.8637,0.9404)
\PST@Diamond(0.8564,0.9339)
\PST@Diamond(0.8501,0.9268)
\PST@Diamond(0.8470,0.9231)
\PST@Diamond(0.8428,0.9175)
\PST@Diamond(0.8375,0.9119)
\PST@Diamond(0.8313,0.9048)
\PST@Diamond(0.8271,0.9006)
\PST@Diamond(0.8240,0.8960)
\PST@Diamond(0.8198,0.8922)
\PST@Diamond(0.8156,0.8871)
\PST@Diamond(0.8104,0.8819)
\PST@Diamond(0.8062,0.8773)
\PST@Diamond(0.8020,0.8726)
\PST@Diamond(0.7989,0.8684)
\PST@Diamond(0.7957,0.8646)
\PST@Diamond(0.7915,0.8599)
\PST@Diamond(0.7883,0.8571)
\PST@Diamond(0.7863,0.8548)
\PST@Diamond(0.7842,0.8525)
\PST@Diamond(0.7811,0.8487)
\PST@Diamond(0.7790,0.8459)
\PST@Diamond(0.7758,0.8431)
\PST@Diamond(0.7738,0.8403)
\PST@Diamond(0.7716,0.8375)
\PST@Diamond(0.7706,0.8366)
\PST@Diamond(0.7684,0.8347)
\PST@Diamond(0.7674,0.8333)
\PST@Diamond(0.7664,0.8323)
\PST@Diamond(0.7654,0.8314)
\PST@Diamond(0.7654,0.8300)
\PST@Diamond(0.7632,0.8286)
\PST@Diamond(0.7622,0.8272)
\PST@Diamond(0.7601,0.8253)
\PST@Diamond(0.7591,0.8235)
\PST@Diamond(0.7580,0.8225)
\PST@Diamond(0.7570,0.8206)
\PST@Diamond(0.7549,0.8192)
\PST@Diamond(0.7539,0.8174)
\PST@Diamond(0.7528,0.8164)
\PST@Diamond(0.7517,0.8150)
\PST@Diamond(0.7497,0.8136)
\PST@Diamond(0.7487,0.8132)
\PST@Diamond(0.7475,0.8122)
\PST@Diamond(0.7475,0.8113)
\PST@Diamond(0.7465,0.8104)
\PST@Diamond(0.7455,0.8094)
\PST@Diamond(0.7445,0.8080)
\PST@Diamond(0.7433,0.8061)
\PST@Diamond(0.7413,0.8047)
\PST@Diamond(0.7403,0.8029)
\PST@Diamond(0.7381,0.8005)
\PST@Diamond(0.7329,0.7954)
\PST@Diamond(0.7236,0.7851)
\PST@Diamond(0.7204,0.7809)
\PST@Diamond(0.7130,0.7734)
\PST@Diamond(0.7089,0.7692)
\PST@Diamond(0.7047,0.7645)
\PST@Diamond(0.6995,0.7584)
\PST@Diamond(0.6963,0.7533)
\PST@Diamond(0.6931,0.7500)
\PST@Diamond(0.6890,0.7463)
\PST@Diamond(0.6848,0.7416)
\PST@Diamond(0.6786,0.7350)
\PST@Diamond(0.6744,0.7299)
\PST@Diamond(0.6691,0.7243)
\PST@Diamond(0.6628,0.7182)
\PST@Diamond(0.6597,0.7135)
\PST@Diamond(0.6545,0.7074)
\PST@Diamond(0.6492,0.7023)
\PST@Diamond(0.6429,0.6962)
\PST@Diamond(0.6366,0.6883)
\PST@Diamond(0.6294,0.6803)
\PST@Diamond(0.6231,0.6738)
\PST@Diamond(0.6157,0.6649)
\PST@Diamond(0.6115,0.6602)
\PST@Diamond(0.6063,0.6555)
\PST@Diamond(0.6042,0.6518)
\PST@Diamond(0.5969,0.6438)
\PST@Diamond(0.5906,0.6359)
\PST@Diamond(0.5844,0.6303)
\PST@Diamond(0.5791,0.6232)
\PST@Diamond(0.5749,0.6176)
\PST@Diamond(0.5697,0.6125)
\PST@Diamond(0.5634,0.6059)
\PST@Diamond(0.5540,0.5952)
\PST@Diamond(0.5446,0.5858)
\PST@Diamond(0.5383,0.5774)
\PST@Diamond(0.5289,0.5680)
\PST@Diamond(0.5215,0.5592)
\PST@Diamond(0.5153,0.5531)
\PST@Diamond(0.5080,0.5442)
\PST@Diamond(0.5006,0.5362)
\PST@Diamond(0.4944,0.5283)
\PST@Diamond(0.4849,0.5189)
\PST@Diamond(0.4787,0.5110)
\PST@Diamond(0.4692,0.5012)
\PST@Diamond(0.4630,0.4946)
\PST@Diamond(0.4566,0.4862)
\PST@Diamond(0.4525,0.4815)
\PST@Diamond(0.4420,0.4712)
\PST@Diamond(0.4315,0.4600)
\PST@Diamond(0.4274,0.4534)
\PST@Diamond(0.4232,0.4492)
\PST@Diamond(0.4180,0.4432)
\PST@Diamond(0.4138,0.4385)
\PST@Diamond(0.4096,0.4338)
\PST@Diamond(0.4054,0.4296)
\PST@Diamond(0.4002,0.4240)
\PST@Diamond(0.3939,0.4160)
\PST@Diamond(0.3865,0.4085)
\PST@Diamond(0.3803,0.4011)
\PST@Diamond(0.3740,0.3938)
\PST@Diamond(0.3688,0.3881)
\PST@Diamond(0.3656,0.3847)
\PST@Diamond(0.3636,0.3822)
\PST@Diamond(0.3583,0.3762)
\PST@Diamond(0.3531,0.3708)
\PST@Diamond(0.3510,0.3676)
\PST@Diamond(0.3469,0.3630)
\PST@Diamond(0.3437,0.3602)
\PST@Diamond(0.3395,0.3544)
\PST@Diamond(0.3311,0.3457)
\PST@Diamond(0.3248,0.3385)
\PST@Diamond(0.3165,0.3287)
\PST@Diamond(0.3081,0.3193)
\PST@Diamond(0.3019,0.3122)
\PST@Diamond(0.2955,0.3047)
\PST@Diamond(0.2893,0.2980)
\PST@Diamond(0.2830,0.2903)
\PST@Diamond(0.2788,0.2852)
\PST@Diamond(0.2736,0.2795)
\PST@Diamond(0.2694,0.2755)
\PST@Diamond(0.2663,0.2718)
\PST@Diamond(0.2610,0.2660)
\PST@Diamond(0.2527,0.2573)
\PST@Diamond(0.2495,0.2527)
\PST@Diamond(0.2443,0.2470)
\PST@Diamond(0.2390,0.2406)
\PST@Diamond(0.2328,0.2345)
\PST@Diamond(0.2254,0.2268)
\PST@Diamond(0.2119,0.2167)
\PST@Dashed(0.2119,0.2123)
(0.2119,0.2123)
(0.2185,0.2197)
(0.2251,0.2271)
(0.2317,0.2344)
(0.2383,0.2418)
(0.2449,0.2492)
(0.2515,0.2566)
(0.2581,0.2640)
(0.2647,0.2714)
(0.2713,0.2788)
(0.2779,0.2862)
(0.2845,0.2936)
(0.2911,0.3010)
(0.2977,0.3083)
(0.3043,0.3157)
(0.3110,0.3231)
(0.3176,0.3305)
(0.3242,0.3379)
(0.3308,0.3453)
(0.3374,0.3527)
(0.3440,0.3601)
(0.3506,0.3675)
(0.3572,0.3749)
(0.3638,0.3822)
(0.3704,0.3896)
(0.3770,0.3970)
(0.3836,0.4044)
(0.3902,0.4118)
(0.3968,0.4192)
(0.4034,0.4266)
(0.4100,0.4340)
(0.4166,0.4414)
(0.4232,0.4488)
(0.4299,0.4561)
(0.4365,0.4635)
(0.4431,0.4709)
(0.4497,0.4783)
(0.4563,0.4857)
(0.4629,0.4931)
(0.4695,0.5005)
(0.4761,0.5079)
(0.4827,0.5153)
(0.4893,0.5226)
(0.4959,0.5300)
(0.5025,0.5374)
(0.5091,0.5448)
(0.5157,0.5522)
(0.5223,0.5596)
(0.5289,0.5670)
(0.5355,0.5744)
(0.5421,0.5818)
(0.5488,0.5892)
(0.5554,0.5965)
(0.5620,0.6039)
(0.5686,0.6113)
(0.5752,0.6187)
(0.5818,0.6261)
(0.5884,0.6335)
(0.5950,0.6409)
(0.6016,0.6483)
(0.6082,0.6557)
(0.6148,0.6631)
(0.6214,0.6704)
(0.6280,0.6778)
(0.6346,0.6852)
(0.6412,0.6926)
(0.6478,0.7000)
(0.6544,0.7074)
(0.6610,0.7148)
(0.6677,0.7222)
(0.6743,0.7296)
(0.6809,0.7370)
(0.6875,0.7443)
(0.6941,0.7517)
(0.7007,0.7591)
(0.7073,0.7665)
(0.7139,0.7739)
(0.7205,0.7813)
(0.7271,0.7887)
(0.7337,0.7961)
(0.7403,0.8035)
(0.7469,0.8109)
(0.7535,0.8182)
(0.7601,0.8256)
(0.7667,0.8330)
(0.7733,0.8404)
(0.7799,0.8478)
(0.7866,0.8552)
(0.7932,0.8626)
(0.7998,0.8700)
(0.8064,0.8774)
(0.8130,0.8847)
(0.8196,0.8921)
(0.8262,0.8995)
(0.8328,0.9069)
(0.8394,0.9143)
(0.8460,0.9217)
(0.8526,0.9291)
(0.8592,0.9365)
(0.8658,0.9439)

\PST@Border(0.1270,0.9680)
(0.1270,0.1260)
(0.9470,0.1260)
(0.9470,0.9680)
(0.1270,0.9680)

\catcode`@=12
\fi
\endpspicture

\end{figure}
\begin{figure}[p]\caption{Interpolazione dei dati con inverso della pressione in ascissa e volume in ordinata. Temperatura di 15 \celsius, fase di espansione.}\label{15e}
% GNUPLOT: LaTeX picture using PSTRICKS macros
% Define new PST objects, if not already defined
\ifx\PSTloaded\undefined
\def\PSTloaded{t}
\psset{arrowsize=.01 3.2 1.4 .3}
\psset{dotsize=.04}
\catcode`@=11

\newpsobject{PST@Border}{psline}{linewidth=.0015,linestyle=solid}
\newpsobject{PST@Axes}{psline}{linewidth=.0015,linestyle=dotted,dotsep=.004}
\newpsobject{PST@Solid}{psline}{linewidth=.0015,linestyle=solid}
\newpsobject{PST@Dashed}{psline}{linewidth=.0015,linestyle=dashed,dash=.01 .01}
\newpsobject{PST@Dotted}{psline}{linewidth=.0025,linestyle=dotted,dotsep=.008}
\newpsobject{PST@LongDash}{psline}{linewidth=.0015,linestyle=dashed,dash=.02 .01}
\newpsobject{PST@Diamond}{psdots}{linewidth=.001,linestyle=solid,dotstyle=square,dotangle=45}
\newpsobject{PST@Filldiamond}{psdots}{linewidth=.001,linestyle=solid,dotstyle=square*,dotangle=45}
\newpsobject{PST@Cross}{psdots}{linewidth=.001,linestyle=solid,dotstyle=+,dotangle=45}
\newpsobject{PST@Plus}{psdots}{linewidth=.001,linestyle=solid,dotstyle=+}
\newpsobject{PST@Square}{psdots}{linewidth=.001,linestyle=solid,dotstyle=square}
\newpsobject{PST@Circle}{psdots}{linewidth=.001,linestyle=solid,dotstyle=o}
\newpsobject{PST@Triangle}{psdots}{linewidth=.001,linestyle=solid,dotstyle=triangle}
\newpsobject{PST@Pentagon}{psdots}{linewidth=.001,linestyle=solid,dotstyle=pentagon}
\newpsobject{PST@Fillsquare}{psdots}{linewidth=.001,linestyle=solid,dotstyle=square*}
\newpsobject{PST@Fillcircle}{psdots}{linewidth=.001,linestyle=solid,dotstyle=*}
\newpsobject{PST@Filltriangle}{psdots}{linewidth=.001,linestyle=solid,dotstyle=triangle*}
\newpsobject{PST@Fillpentagon}{psdots}{linewidth=.001,linestyle=solid,dotstyle=pentagon*}
\newpsobject{PST@Arrow}{psline}{linewidth=.001,linestyle=solid}
\catcode`@=12

\fi
\psset{unit=5.0in,xunit=5.0in,yunit=3.0in}
\pspicture(0.000000,0.000000)(1.000000,1.000000)
\ifx\nofigs\undefined
\catcode`@=11

\PST@Border(0.1270,0.1260)
(0.1420,0.1260)

\rput[r](0.1110,0.1260){4}
\PST@Border(0.1270,0.2196)
(0.1420,0.2196)

\rput[r](0.1110,0.2196){6}
\PST@Border(0.1270,0.3131)
(0.1420,0.3131)

\rput[r](0.1110,0.3131){8}
\PST@Border(0.1270,0.4067)
(0.1420,0.4067)

\rput[r](0.1110,0.4067){10}
\PST@Border(0.1270,0.5002)
(0.1420,0.5002)

\rput[r](0.1110,0.5002){12}
\PST@Border(0.1270,0.5938)
(0.1420,0.5938)

\rput[r](0.1110,0.5938){14}
\PST@Border(0.1270,0.6873)
(0.1420,0.6873)

\rput[r](0.1110,0.6873){16}
\PST@Border(0.1270,0.7809)
(0.1420,0.7809)

\rput[r](0.1110,0.7809){18}
\PST@Border(0.1270,0.8744)
(0.1420,0.8744)

\rput[r](0.1110,0.8744){20}
\PST@Border(0.1270,0.9680)
(0.1420,0.9680)

\rput[r](0.1110,0.9680){22}
\PST@Border(0.1270,0.1260)
(0.1270,0.1460)

\rput(0.1270,0.0840){2}
\PST@Border(0.2295,0.1260)
(0.2295,0.1460)

\rput(0.2295,0.0840){3}
\PST@Border(0.3320,0.1260)
(0.3320,0.1460)

\rput(0.3320,0.0840){4}
\PST@Border(0.4345,0.1260)
(0.4345,0.1460)

\rput(0.4345,0.0840){5}
\PST@Border(0.5370,0.1260)
(0.5370,0.1460)

\rput(0.5370,0.0840){6}
\PST@Border(0.6395,0.1260)
(0.6395,0.1460)

\rput(0.6395,0.0840){7}
\PST@Border(0.7420,0.1260)
(0.7420,0.1460)

\rput(0.7420,0.0840){8}
\PST@Border(0.8445,0.1260)
(0.8445,0.1460)

\rput(0.8445,0.0840){9}
\PST@Border(0.9470,0.1260)
(0.9470,0.1460)

\rput(0.9470,0.0840){10}
\PST@Border(0.1270,0.9680)
(0.1270,0.1260)
(0.9470,0.1260)
(0.9470,0.9680)
(0.1270,0.9680)

\rput{L}(0.0420,0.5470){$V\ (\unit{cm^3})$}
\rput(0.5370,0.0210){$1/P\ (\unit{Pa^{-1}})$}
\PST@Diamond(0.2107,0.2155)
\PST@Diamond(0.2161,0.2171)
\PST@Diamond(0.2234,0.2208)
\PST@Diamond(0.2286,0.2273)
\PST@Diamond(0.2338,0.2331)
\PST@Diamond(0.2390,0.2394)
\PST@Diamond(0.2474,0.2482)
\PST@Diamond(0.2505,0.2523)
\PST@Diamond(0.2527,0.2552)
\PST@Diamond(0.2589,0.2608)
\PST@Diamond(0.2631,0.2661)
\PST@Diamond(0.2683,0.2716)
\PST@Diamond(0.2736,0.2773)
\PST@Diamond(0.2798,0.2841)
\PST@Diamond(0.2840,0.2892)
\PST@Diamond(0.2914,0.2970)
\PST@Diamond(0.2977,0.3036)
\PST@Diamond(0.3007,0.3081)
\PST@Diamond(0.3060,0.3141)
\PST@Diamond(0.3112,0.3195)
\PST@Diamond(0.3175,0.3263)
\PST@Diamond(0.3290,0.3394)
\PST@Diamond(0.3374,0.3485)
\PST@Diamond(0.3489,0.3614)
\PST@Diamond(0.3614,0.3751)
\PST@Diamond(0.3720,0.3877)
\PST@Diamond(0.3761,0.3929)
\PST@Diamond(0.3793,0.3971)
\PST@Diamond(0.3855,0.4033)
\PST@Diamond(0.3929,0.4113)
\PST@Diamond(0.4012,0.4198)
\PST@Diamond(0.4075,0.4277)
\PST@Diamond(0.4138,0.4343)
\PST@Diamond(0.4211,0.4427)
\PST@Diamond(0.4263,0.4488)
\PST@Diamond(0.4327,0.4548)
\PST@Diamond(0.4410,0.4637)
\PST@Diamond(0.4483,0.4731)
\PST@Diamond(0.4556,0.4801)
\PST@Diamond(0.4619,0.4871)
\PST@Diamond(0.4671,0.4937)
\PST@Diamond(0.4745,0.5012)
\PST@Diamond(0.4829,0.5105)
\PST@Diamond(0.4881,0.5166)
\PST@Diamond(0.4964,0.5250)
\PST@Diamond(0.5048,0.5348)
\PST@Diamond(0.5111,0.5423)
\PST@Diamond(0.5195,0.5507)
\PST@Diamond(0.5279,0.5606)
\PST@Diamond(0.5362,0.5690)
\PST@Diamond(0.5414,0.5760)
\PST@Diamond(0.5466,0.5821)
\PST@Diamond(0.5519,0.5877)
\PST@Diamond(0.5592,0.5956)
\PST@Diamond(0.5655,0.6031)
\PST@Diamond(0.5697,0.6087)
\PST@Diamond(0.5781,0.6172)
\PST@Diamond(0.5854,0.6261)
\PST@Diamond(0.5906,0.6321)
\PST@Diamond(0.5980,0.6396)
\PST@Diamond(0.6042,0.6476)
\PST@Diamond(0.6063,0.6499)
\PST@Diamond(0.6137,0.6574)
\PST@Diamond(0.6220,0.6672)
\PST@Diamond(0.6283,0.6747)
\PST@Diamond(0.6408,0.6873)
\PST@Diamond(0.6503,0.6986)
\PST@Diamond(0.6503,0.7000)
\PST@Diamond(0.6503,0.7004)
\PST@Diamond(0.6503,0.7004)
\PST@Diamond(0.6503,0.7009)
\PST@Diamond(0.6503,0.7009)
\PST@Diamond(0.6513,0.7014)
\PST@Diamond(0.6513,0.7018)
\PST@Diamond(0.6513,0.7023)
\PST@Diamond(0.6806,0.7327)
\PST@Diamond(0.6796,0.7327)
\PST@Diamond(0.6796,0.7327)
\PST@Diamond(0.6786,0.7322)
\PST@Diamond(0.6786,0.7327)
\PST@Diamond(0.6786,0.7327)
\PST@Diamond(0.6786,0.7327)
\PST@Diamond(0.6786,0.7327)
\PST@Diamond(0.6796,0.7341)
\PST@Diamond(0.6858,0.7397)
\PST@Diamond(0.6879,0.7430)
\PST@Diamond(0.6931,0.7491)
\PST@Diamond(0.6963,0.7528)
\PST@Diamond(0.7005,0.7575)
\PST@Diamond(0.7037,0.7617)
\PST@Diamond(0.7057,0.7640)
\PST@Diamond(0.7089,0.7678)
\PST@Diamond(0.7130,0.7715)
\PST@Diamond(0.7193,0.7785)
\PST@Diamond(0.7246,0.7842)
\PST@Diamond(0.7277,0.7888)
\PST@Diamond(0.7329,0.7945)
\PST@Diamond(0.7381,0.8005)
\PST@Diamond(0.7423,0.8043)
\PST@Diamond(0.7465,0.8085)
\PST@Diamond(0.7507,0.8136)
\PST@Diamond(0.7559,0.8197)
\PST@Diamond(0.7591,0.8239)
\PST@Diamond(0.7654,0.8291)
\PST@Diamond(0.7674,0.8333)
\PST@Diamond(0.7726,0.8389)
\PST@Diamond(0.7758,0.8431)
\PST@Diamond(0.7800,0.8468)
\PST@Diamond(0.7831,0.8506)
\PST@Diamond(0.7842,0.8525)
\PST@Diamond(0.7873,0.8557)
\PST@Diamond(0.7905,0.8595)
\PST@Diamond(0.7947,0.8637)
\PST@Diamond(0.7999,0.8698)
\PST@Diamond(0.8041,0.8744)
\PST@Diamond(0.8082,0.8787)
\PST@Diamond(0.8114,0.8829)
\PST@Diamond(0.8166,0.8880)
\PST@Diamond(0.8218,0.8941)
\PST@Diamond(0.8250,0.8978)
\PST@Diamond(0.8281,0.9016)
\PST@Diamond(0.8303,0.9034)
\PST@Diamond(0.8313,0.9048)
\PST@Diamond(0.8333,0.9077)
\PST@Diamond(0.8375,0.9128)
\PST@Diamond(0.8397,0.9156)
\PST@Diamond(0.8417,0.9175)
\PST@Diamond(0.8449,0.9208)
\PST@Diamond(0.8491,0.9250)
\PST@Diamond(0.8532,0.9306)
\PST@Diamond(0.8596,0.9371)
\PST@Diamond(0.8626,0.9413)
\PST@Dashed(0.2107,0.2062)
(0.2107,0.2062)
(0.2173,0.2136)
(0.2239,0.2210)
(0.2305,0.2284)
(0.2371,0.2358)
(0.2437,0.2432)
(0.2503,0.2506)
(0.2568,0.2580)
(0.2634,0.2654)
(0.2700,0.2728)
(0.2766,0.2802)
(0.2832,0.2876)
(0.2898,0.2951)
(0.2963,0.3025)
(0.3029,0.3099)
(0.3095,0.3173)
(0.3161,0.3247)
(0.3227,0.3321)
(0.3293,0.3395)
(0.3359,0.3469)
(0.3424,0.3543)
(0.3490,0.3617)
(0.3556,0.3691)
(0.3622,0.3765)
(0.3688,0.3839)
(0.3754,0.3913)
(0.3819,0.3988)
(0.3885,0.4062)
(0.3951,0.4136)
(0.4017,0.4210)
(0.4083,0.4284)
(0.4149,0.4358)
(0.4215,0.4432)
(0.4280,0.4506)
(0.4346,0.4580)
(0.4412,0.4654)
(0.4478,0.4728)
(0.4544,0.4802)
(0.4610,0.4876)
(0.4676,0.4950)
(0.4741,0.5025)
(0.4807,0.5099)
(0.4873,0.5173)
(0.4939,0.5247)
(0.5005,0.5321)
(0.5071,0.5395)
(0.5136,0.5469)
(0.5202,0.5543)
(0.5268,0.5617)
(0.5334,0.5691)
(0.5400,0.5765)
(0.5466,0.5839)
(0.5532,0.5913)
(0.5597,0.5987)
(0.5663,0.6062)
(0.5729,0.6136)
(0.5795,0.6210)
(0.5861,0.6284)
(0.5927,0.6358)
(0.5992,0.6432)
(0.6058,0.6506)
(0.6124,0.6580)
(0.6190,0.6654)
(0.6256,0.6728)
(0.6322,0.6802)
(0.6388,0.6876)
(0.6453,0.6950)
(0.6519,0.7024)
(0.6585,0.7099)
(0.6651,0.7173)
(0.6717,0.7247)
(0.6783,0.7321)
(0.6849,0.7395)
(0.6914,0.7469)
(0.6980,0.7543)
(0.7046,0.7617)
(0.7112,0.7691)
(0.7178,0.7765)
(0.7244,0.7839)
(0.7309,0.7913)
(0.7375,0.7987)
(0.7441,0.8061)
(0.7507,0.8136)
(0.7573,0.8210)
(0.7639,0.8284)
(0.7705,0.8358)
(0.7770,0.8432)
(0.7836,0.8506)
(0.7902,0.8580)
(0.7968,0.8654)
(0.8034,0.8728)
(0.8100,0.8802)
(0.8165,0.8876)
(0.8231,0.8950)
(0.8297,0.9024)
(0.8363,0.9098)
(0.8429,0.9173)
(0.8495,0.9247)
(0.8561,0.9321)
(0.8626,0.9395)

\PST@Border(0.1270,0.9680)
(0.1270,0.1260)
(0.9470,0.1260)
(0.9470,0.9680)
(0.1270,0.9680)

\catcode`@=12
\fi
\endpspicture

\end{figure}
\begin{figure}[p]\caption{Interpolazione dei dati con inverso della pressione in ascissa e volume in ordinata. Temperatura di 25 \celsius, fase di compressione.}\label{25c}
% GNUPLOT: LaTeX picture using PSTRICKS macros
% Define new PST objects, if not already defined
\ifx\PSTloaded\undefined
\def\PSTloaded{t}
\psset{arrowsize=.01 3.2 1.4 .3}
\psset{dotsize=.04}
\catcode`@=11

\newpsobject{PST@Border}{psline}{linewidth=.0015,linestyle=solid}
\newpsobject{PST@Axes}{psline}{linewidth=.0015,linestyle=dotted,dotsep=.004}
\newpsobject{PST@Solid}{psline}{linewidth=.0015,linestyle=solid}
\newpsobject{PST@Dashed}{psline}{linewidth=.0015,linestyle=dashed,dash=.01 .01}
\newpsobject{PST@Dotted}{psline}{linewidth=.0025,linestyle=dotted,dotsep=.008}
\newpsobject{PST@LongDash}{psline}{linewidth=.0015,linestyle=dashed,dash=.02 .01}
\newpsobject{PST@Diamond}{psdots}{linewidth=.001,linestyle=solid,dotstyle=square,dotangle=45}
\newpsobject{PST@Filldiamond}{psdots}{linewidth=.001,linestyle=solid,dotstyle=square*,dotangle=45}
\newpsobject{PST@Cross}{psdots}{linewidth=.001,linestyle=solid,dotstyle=+,dotangle=45}
\newpsobject{PST@Plus}{psdots}{linewidth=.001,linestyle=solid,dotstyle=+}
\newpsobject{PST@Square}{psdots}{linewidth=.001,linestyle=solid,dotstyle=square}
\newpsobject{PST@Circle}{psdots}{linewidth=.001,linestyle=solid,dotstyle=o}
\newpsobject{PST@Triangle}{psdots}{linewidth=.001,linestyle=solid,dotstyle=triangle}
\newpsobject{PST@Pentagon}{psdots}{linewidth=.001,linestyle=solid,dotstyle=pentagon}
\newpsobject{PST@Fillsquare}{psdots}{linewidth=.001,linestyle=solid,dotstyle=square*}
\newpsobject{PST@Fillcircle}{psdots}{linewidth=.001,linestyle=solid,dotstyle=*}
\newpsobject{PST@Filltriangle}{psdots}{linewidth=.001,linestyle=solid,dotstyle=triangle*}
\newpsobject{PST@Fillpentagon}{psdots}{linewidth=.001,linestyle=solid,dotstyle=pentagon*}
\newpsobject{PST@Arrow}{psline}{linewidth=.001,linestyle=solid}
\catcode`@=12

\fi
\psset{unit=5.0in,xunit=5.0in,yunit=3.0in}
\pspicture(0.000000,0.000000)(1.000000,1.000000)
\ifx\nofigs\undefined
\catcode`@=11

\PST@Border(0.1270,0.1260)
(0.1420,0.1260)

\rput[r](0.1110,0.1260){4}
\PST@Border(0.1270,0.2196)
(0.1420,0.2196)

\rput[r](0.1110,0.2196){6}
\PST@Border(0.1270,0.3131)
(0.1420,0.3131)

\rput[r](0.1110,0.3131){8}
\PST@Border(0.1270,0.4067)
(0.1420,0.4067)

\rput[r](0.1110,0.4067){10}
\PST@Border(0.1270,0.5002)
(0.1420,0.5002)

\rput[r](0.1110,0.5002){12}
\PST@Border(0.1270,0.5938)
(0.1420,0.5938)

\rput[r](0.1110,0.5938){14}
\PST@Border(0.1270,0.6873)
(0.1420,0.6873)

\rput[r](0.1110,0.6873){16}
\PST@Border(0.1270,0.7809)
(0.1420,0.7809)

\rput[r](0.1110,0.7809){18}
\PST@Border(0.1270,0.8744)
(0.1420,0.8744)

\rput[r](0.1110,0.8744){20}
\PST@Border(0.1270,0.9680)
(0.1420,0.9680)

\rput[r](0.1110,0.9680){22}
\PST@Border(0.1270,0.1260)
(0.1270,0.1460)

\rput(0.1270,0.0840){2}
\PST@Border(0.2441,0.1260)
(0.2441,0.1460)

\rput(0.2441,0.0840){3}
\PST@Border(0.3613,0.1260)
(0.3613,0.1460)

\rput(0.3613,0.0840){4}
\PST@Border(0.4784,0.1260)
(0.4784,0.1460)

\rput(0.4784,0.0840){5}
\PST@Border(0.5956,0.1260)
(0.5956,0.1460)

\rput(0.5956,0.0840){6}
\PST@Border(0.7127,0.1260)
(0.7127,0.1460)

\rput(0.7127,0.0840){7}
\PST@Border(0.8299,0.1260)
(0.8299,0.1460)

\rput(0.8299,0.0840){8}
\PST@Border(0.9470,0.1260)
(0.9470,0.1460)

\rput(0.9470,0.0840){9}
\PST@Border(0.1270,0.9680)
(0.1270,0.1260)
(0.9470,0.1260)
(0.9470,0.9680)
(0.1270,0.9680)

\rput{L}(0.0420,0.5470){$V\ (\unit{cm^3})$}
\rput(0.5370,0.0210){$1/P\ (\unit{Pa^{-1}})$}
\PST@Diamond(0.9390,0.9441)
\PST@Diamond(0.9223,0.9268)
\PST@Diamond(0.9140,0.9165)
\PST@Diamond(0.8984,0.9016)
\PST@Diamond(0.8924,0.8946)
\PST@Diamond(0.8853,0.8871)
\PST@Diamond(0.8733,0.8763)
\PST@Diamond(0.8662,0.8684)
\PST@Diamond(0.8601,0.8618)
\PST@Diamond(0.8541,0.8562)
\PST@Diamond(0.8446,0.8468)
\PST@Diamond(0.8362,0.8384)
\PST@Diamond(0.8267,0.8291)
\PST@Diamond(0.8183,0.8206)
\PST@Diamond(0.8123,0.8146)
\PST@Diamond(0.8063,0.8085)
\PST@Diamond(0.8003,0.8029)
\PST@Diamond(0.7907,0.7940)
\PST@Diamond(0.7824,0.7856)
\PST@Diamond(0.7776,0.7795)
\PST@Diamond(0.7705,0.7725)
\PST@Diamond(0.7657,0.7673)
\PST@Diamond(0.7620,0.7640)
\PST@Diamond(0.7573,0.7584)
\PST@Diamond(0.7525,0.7538)
\PST@Diamond(0.7489,0.7495)
\PST@Diamond(0.7453,0.7463)
\PST@Diamond(0.7418,0.7430)
\PST@Diamond(0.7381,0.7397)
\PST@Diamond(0.7358,0.7364)
\PST@Diamond(0.7322,0.7322)
\PST@Diamond(0.7286,0.7304)
\PST@Diamond(0.7238,0.7257)
\PST@Diamond(0.7214,0.7219)
\PST@Diamond(0.7154,0.7168)
\PST@Diamond(0.7106,0.7117)
\PST@Diamond(0.7071,0.7089)
\PST@Diamond(0.7035,0.7042)
\PST@Diamond(0.6999,0.7009)
\PST@Diamond(0.6940,0.6962)
\PST@Diamond(0.6867,0.6887)
\PST@Diamond(0.6807,0.6822)
\PST@Diamond(0.6772,0.6789)
\PST@Diamond(0.6712,0.6724)
\PST@Diamond(0.6605,0.6621)
\PST@Diamond(0.6532,0.6551)
\PST@Diamond(0.6461,0.6476)
\PST@Diamond(0.6389,0.6401)
\PST@Diamond(0.6293,0.6312)
\PST@Diamond(0.6222,0.6228)
\PST@Diamond(0.6126,0.6134)
\PST@Diamond(0.6031,0.6031)
\PST@Diamond(0.5923,0.5928)
\PST@Diamond(0.5839,0.5844)
\PST@Diamond(0.5755,0.5760)
\PST@Diamond(0.5671,0.5676)
\PST@Diamond(0.5564,0.5573)
\PST@Diamond(0.5432,0.5433)
\PST@Diamond(0.5349,0.5348)
\PST@Diamond(0.5241,0.5250)
\PST@Diamond(0.5157,0.5171)
\PST@Diamond(0.5122,0.5119)
\PST@Diamond(0.5085,0.5077)
\PST@Diamond(0.5026,0.5021)
\PST@Diamond(0.4978,0.4974)
\PST@Diamond(0.4918,0.4913)
\PST@Diamond(0.4870,0.4867)
\PST@Diamond(0.4787,0.4782)
\PST@Diamond(0.4691,0.4693)
\PST@Diamond(0.4607,0.4600)
\PST@Diamond(0.4559,0.4558)
\PST@Diamond(0.4511,0.4506)
\PST@Diamond(0.4488,0.4478)
\PST@Diamond(0.4416,0.4408)
\PST@Diamond(0.4368,0.4352)
\PST@Diamond(0.4297,0.4287)
\PST@Diamond(0.4224,0.4216)
\PST@Diamond(0.4165,0.4151)
\PST@Diamond(0.4093,0.4085)
\PST@Diamond(0.4010,0.3994)
\PST@Diamond(0.3902,0.3893)
\PST@Diamond(0.3854,0.3838)
\PST@Diamond(0.3770,0.3753)
\PST@Diamond(0.3710,0.3695)
\PST@Diamond(0.3639,0.3622)
\PST@Diamond(0.3555,0.3539)
\PST@Diamond(0.3448,0.3434)
\PST@Diamond(0.3388,0.3365)
\PST@Diamond(0.3280,0.3257)
\PST@Diamond(0.3209,0.3174)
\PST@Diamond(0.3124,0.3085)
\PST@Diamond(0.2969,0.2947)
\PST@Diamond(0.2922,0.2885)
\PST@Diamond(0.2862,0.2824)
\PST@Diamond(0.2826,0.2779)
\PST@Diamond(0.2754,0.2714)
\PST@Diamond(0.2718,0.2670)
\PST@Diamond(0.2670,0.2626)
\PST@Diamond(0.2622,0.2574)
\PST@Diamond(0.2550,0.2509)
\PST@Diamond(0.2491,0.2441)
\PST@Diamond(0.2407,0.2358)
\PST@Diamond(0.2359,0.2306)
\PST@Diamond(0.2300,0.2254)
\PST@Diamond(0.2263,0.2209)
\PST@Diamond(0.2204,0.2179)
\PST@Diamond(0.2132,0.2157)
\PST@Dashed(0.2132,0.2096)
(0.2132,0.2096)
(0.2205,0.2170)
(0.2279,0.2244)
(0.2352,0.2318)
(0.2425,0.2392)
(0.2499,0.2466)
(0.2572,0.2540)
(0.2645,0.2614)
(0.2719,0.2688)
(0.2792,0.2762)
(0.2865,0.2836)
(0.2939,0.2910)
(0.3012,0.2984)
(0.3085,0.3058)
(0.3159,0.3133)
(0.3232,0.3207)
(0.3305,0.3281)
(0.3379,0.3355)
(0.3452,0.3429)
(0.3525,0.3503)
(0.3598,0.3577)
(0.3672,0.3651)
(0.3745,0.3725)
(0.3818,0.3799)
(0.3892,0.3873)
(0.3965,0.3947)
(0.4038,0.4021)
(0.4112,0.4095)
(0.4185,0.4169)
(0.4258,0.4244)
(0.4332,0.4318)
(0.4405,0.4392)
(0.4478,0.4466)
(0.4552,0.4540)
(0.4625,0.4614)
(0.4698,0.4688)
(0.4772,0.4762)
(0.4845,0.4836)
(0.4918,0.4910)
(0.4991,0.4984)
(0.5065,0.5058)
(0.5138,0.5132)
(0.5211,0.5206)
(0.5285,0.5280)
(0.5358,0.5354)
(0.5431,0.5429)
(0.5505,0.5503)
(0.5578,0.5577)
(0.5651,0.5651)
(0.5725,0.5725)
(0.5798,0.5799)
(0.5871,0.5873)
(0.5945,0.5947)
(0.6018,0.6021)
(0.6091,0.6095)
(0.6164,0.6169)
(0.6238,0.6243)
(0.6311,0.6317)
(0.6384,0.6391)
(0.6458,0.6465)
(0.6531,0.6539)
(0.6604,0.6614)
(0.6678,0.6688)
(0.6751,0.6762)
(0.6824,0.6836)
(0.6898,0.6910)
(0.6971,0.6984)
(0.7044,0.7058)
(0.7118,0.7132)
(0.7191,0.7206)
(0.7264,0.7280)
(0.7338,0.7354)
(0.7411,0.7428)
(0.7484,0.7502)
(0.7557,0.7576)
(0.7631,0.7650)
(0.7704,0.7725)
(0.7777,0.7799)
(0.7851,0.7873)
(0.7924,0.7947)
(0.7997,0.8021)
(0.8071,0.8095)
(0.8144,0.8169)
(0.8217,0.8243)
(0.8291,0.8317)
(0.8364,0.8391)
(0.8437,0.8465)
(0.8511,0.8539)
(0.8584,0.8613)
(0.8657,0.8687)
(0.8731,0.8761)
(0.8804,0.8835)
(0.8877,0.8910)
(0.8950,0.8984)
(0.9024,0.9058)
(0.9097,0.9132)
(0.9170,0.9206)
(0.9244,0.9280)
(0.9317,0.9354)
(0.9390,0.9428)

\PST@Border(0.1270,0.9680)
(0.1270,0.1260)
(0.9470,0.1260)
(0.9470,0.9680)
(0.1270,0.9680)

\catcode`@=12
\fi
\endpspicture

\end{figure}
\begin{figure}[p]\caption{Interpolazione dei dati con inverso della pressione in ascissa e volume in ordinata. Temperatura di 25 \celsius, fase di espansione.}\label{25e}
% GNUPLOT: LaTeX picture using PSTRICKS macros
% Define new PST objects, if not already defined
\ifx\PSTloaded\undefined
\def\PSTloaded{t}
\psset{arrowsize=.01 3.2 1.4 .3}
\psset{dotsize=.04}
\catcode`@=11

\newpsobject{PST@Border}{psline}{linewidth=.0015,linestyle=solid}
\newpsobject{PST@Axes}{psline}{linewidth=.0015,linestyle=dotted,dotsep=.004}
\newpsobject{PST@Solid}{psline}{linewidth=.0015,linestyle=solid}
\newpsobject{PST@Dashed}{psline}{linewidth=.0015,linestyle=dashed,dash=.01 .01}
\newpsobject{PST@Dotted}{psline}{linewidth=.0025,linestyle=dotted,dotsep=.008}
\newpsobject{PST@LongDash}{psline}{linewidth=.0015,linestyle=dashed,dash=.02 .01}
\newpsobject{PST@Diamond}{psdots}{linewidth=.001,linestyle=solid,dotstyle=square,dotangle=45}
\newpsobject{PST@Filldiamond}{psdots}{linewidth=.001,linestyle=solid,dotstyle=square*,dotangle=45}
\newpsobject{PST@Cross}{psdots}{linewidth=.001,linestyle=solid,dotstyle=+,dotangle=45}
\newpsobject{PST@Plus}{psdots}{linewidth=.001,linestyle=solid,dotstyle=+}
\newpsobject{PST@Square}{psdots}{linewidth=.001,linestyle=solid,dotstyle=square}
\newpsobject{PST@Circle}{psdots}{linewidth=.001,linestyle=solid,dotstyle=o}
\newpsobject{PST@Triangle}{psdots}{linewidth=.001,linestyle=solid,dotstyle=triangle}
\newpsobject{PST@Pentagon}{psdots}{linewidth=.001,linestyle=solid,dotstyle=pentagon}
\newpsobject{PST@Fillsquare}{psdots}{linewidth=.001,linestyle=solid,dotstyle=square*}
\newpsobject{PST@Fillcircle}{psdots}{linewidth=.001,linestyle=solid,dotstyle=*}
\newpsobject{PST@Filltriangle}{psdots}{linewidth=.001,linestyle=solid,dotstyle=triangle*}
\newpsobject{PST@Fillpentagon}{psdots}{linewidth=.001,linestyle=solid,dotstyle=pentagon*}
\newpsobject{PST@Arrow}{psline}{linewidth=.001,linestyle=solid}
\catcode`@=12

\fi
\psset{unit=5.0in,xunit=5.0in,yunit=3.0in}
\pspicture(0.000000,0.000000)(1.000000,1.000000)
\ifx\nofigs\undefined
\catcode`@=11

\PST@Border(0.1270,0.1260)
(0.1420,0.1260)

\rput[r](0.1110,0.1260){4}
\PST@Border(0.1270,0.2196)
(0.1420,0.2196)

\rput[r](0.1110,0.2196){6}
\PST@Border(0.1270,0.3131)
(0.1420,0.3131)

\rput[r](0.1110,0.3131){8}
\PST@Border(0.1270,0.4067)
(0.1420,0.4067)

\rput[r](0.1110,0.4067){10}
\PST@Border(0.1270,0.5002)
(0.1420,0.5002)

\rput[r](0.1110,0.5002){12}
\PST@Border(0.1270,0.5938)
(0.1420,0.5938)

\rput[r](0.1110,0.5938){14}
\PST@Border(0.1270,0.6873)
(0.1420,0.6873)

\rput[r](0.1110,0.6873){16}
\PST@Border(0.1270,0.7809)
(0.1420,0.7809)

\rput[r](0.1110,0.7809){18}
\PST@Border(0.1270,0.8744)
(0.1420,0.8744)

\rput[r](0.1110,0.8744){20}
\PST@Border(0.1270,0.9680)
(0.1420,0.9680)

\rput[r](0.1110,0.9680){22}
\PST@Border(0.1270,0.1260)
(0.1270,0.1460)

\rput(0.1270,0.0840){2}
\PST@Border(0.2441,0.1260)
(0.2441,0.1460)

\rput(0.2441,0.0840){3}
\PST@Border(0.3613,0.1260)
(0.3613,0.1460)

\rput(0.3613,0.0840){4}
\PST@Border(0.4784,0.1260)
(0.4784,0.1460)

\rput(0.4784,0.0840){5}
\PST@Border(0.5956,0.1260)
(0.5956,0.1460)

\rput(0.5956,0.0840){6}
\PST@Border(0.7127,0.1260)
(0.7127,0.1460)

\rput(0.7127,0.0840){7}
\PST@Border(0.8299,0.1260)
(0.8299,0.1460)

\rput(0.8299,0.0840){8}
\PST@Border(0.9470,0.1260)
(0.9470,0.1460)

\rput(0.9470,0.0840){9}
\PST@Border(0.1270,0.9680)
(0.1270,0.1260)
(0.9470,0.1260)
(0.9470,0.9680)
(0.1270,0.9680)

\rput{L}(0.0420,0.5470){$V\ (\unit{cm^3})$}
\rput(0.5370,0.0210){$1/P\ (\unit{Pa^{-1}})$}
\PST@Diamond(0.2144,0.2153)
\PST@Diamond(0.2180,0.2165)
\PST@Diamond(0.2227,0.2178)
\PST@Diamond(0.2252,0.2190)
\PST@Diamond(0.2288,0.2216)
\PST@Diamond(0.2335,0.2263)
\PST@Diamond(0.2383,0.2314)
\PST@Diamond(0.2431,0.2358)
\PST@Diamond(0.2502,0.2428)
\PST@Diamond(0.2550,0.2485)
\PST@Diamond(0.2598,0.2533)
\PST@Diamond(0.2658,0.2588)
\PST@Diamond(0.2682,0.2607)
\PST@Diamond(0.2718,0.2648)
\PST@Diamond(0.2754,0.2683)
\PST@Diamond(0.2778,0.2707)
\PST@Diamond(0.2801,0.2734)
\PST@Diamond(0.2862,0.2786)
\PST@Diamond(0.2922,0.2849)
\PST@Diamond(0.2969,0.2900)
\PST@Diamond(0.3005,0.2938)
\PST@Diamond(0.3028,0.2973)
\PST@Diamond(0.3076,0.3015)
\PST@Diamond(0.3124,0.3056)
\PST@Diamond(0.3161,0.3103)
\PST@Diamond(0.3209,0.3154)
\PST@Diamond(0.3292,0.3232)
\PST@Diamond(0.3375,0.3316)
\PST@Diamond(0.3448,0.3399)
\PST@Diamond(0.3496,0.3445)
\PST@Diamond(0.3579,0.3519)
\PST@Diamond(0.3627,0.3575)
\PST@Diamond(0.3662,0.3620)
\PST@Diamond(0.3723,0.3674)
\PST@Diamond(0.3783,0.3730)
\PST@Diamond(0.3830,0.3774)
\PST@Diamond(0.3866,0.3827)
\PST@Diamond(0.3926,0.3875)
\PST@Diamond(0.3974,0.3924)
\PST@Diamond(0.4033,0.3981)
\PST@Diamond(0.4117,0.4067)
\PST@Diamond(0.4165,0.4118)
\PST@Diamond(0.4201,0.4160)
\PST@Diamond(0.4261,0.4216)
\PST@Diamond(0.4309,0.4263)
\PST@Diamond(0.4344,0.4301)
\PST@Diamond(0.4416,0.4371)
\PST@Diamond(0.4500,0.4450)
\PST@Diamond(0.4559,0.4511)
\PST@Diamond(0.4644,0.4595)
\PST@Diamond(0.4727,0.4679)
\PST@Diamond(0.4823,0.4773)
\PST@Diamond(0.4870,0.4824)
\PST@Diamond(0.4942,0.4899)
\PST@Diamond(0.5050,0.4998)
\PST@Diamond(0.5145,0.5096)
\PST@Diamond(0.5205,0.5166)
\PST@Diamond(0.5277,0.5227)
\PST@Diamond(0.5324,0.5283)
\PST@Diamond(0.5384,0.5344)
\PST@Diamond(0.5444,0.5405)
\PST@Diamond(0.5528,0.5479)
\PST@Diamond(0.5600,0.5554)
\PST@Diamond(0.5671,0.5624)
\PST@Diamond(0.5707,0.5676)
\PST@Diamond(0.5767,0.5723)
\PST@Diamond(0.5792,0.5755)
\PST@Diamond(0.5851,0.5811)
\PST@Diamond(0.5923,0.5886)
\PST@Diamond(0.5946,0.5919)
\PST@Diamond(0.5994,0.5966)
\PST@Diamond(0.6079,0.6041)
\PST@Diamond(0.6174,0.6134)
\PST@Diamond(0.6245,0.6214)
\PST@Diamond(0.6329,0.6303)
\PST@Diamond(0.6425,0.6396)
\PST@Diamond(0.6520,0.6494)
\PST@Diamond(0.6592,0.6579)
\PST@Diamond(0.6664,0.6658)
\PST@Diamond(0.6724,0.6719)
\PST@Diamond(0.6796,0.6789)
\PST@Diamond(0.6844,0.6836)
\PST@Diamond(0.6867,0.6869)
\PST@Diamond(0.6892,0.6887)
\PST@Diamond(0.6903,0.6906)
\PST@Diamond(0.6927,0.6925)
\PST@Diamond(0.6951,0.6943)
\PST@Diamond(0.6963,0.6967)
\PST@Diamond(0.6975,0.6981)
\PST@Diamond(0.6987,0.6995)
\PST@Diamond(0.7011,0.7009)
\PST@Diamond(0.7023,0.7023)
\PST@Diamond(0.7035,0.7032)
\PST@Diamond(0.7046,0.7046)
\PST@Diamond(0.7059,0.7060)
\PST@Diamond(0.7059,0.7070)
\PST@Diamond(0.7071,0.7084)
\PST@Diamond(0.7083,0.7089)
\PST@Diamond(0.7094,0.7103)
\PST@Diamond(0.7106,0.7112)
\PST@Diamond(0.7119,0.7121)
\PST@Diamond(0.7131,0.7140)
\PST@Diamond(0.7166,0.7163)
\PST@Diamond(0.7190,0.7201)
\PST@Diamond(0.7214,0.7224)
\PST@Diamond(0.7250,0.7257)
\PST@Diamond(0.7274,0.7280)
\PST@Diamond(0.7298,0.7308)
\PST@Diamond(0.7358,0.7360)
\PST@Diamond(0.7514,0.7514)
\PST@Diamond(0.7514,0.7519)
\PST@Diamond(0.7525,0.7533)
\PST@Diamond(0.7597,0.7603)
\PST@Diamond(0.7668,0.7673)
\PST@Diamond(0.7728,0.7739)
\PST@Diamond(0.7788,0.7804)
\PST@Diamond(0.7824,0.7837)
\PST@Diamond(0.7896,0.7907)
\PST@Diamond(0.7920,0.7935)
\PST@Diamond(0.7944,0.7963)
\PST@Diamond(0.7992,0.8005)
\PST@Diamond(0.8027,0.8047)
\PST@Diamond(0.8111,0.8118)
\PST@Diamond(0.8159,0.8169)
\PST@Diamond(0.8207,0.8225)
\PST@Diamond(0.8242,0.8263)
\PST@Diamond(0.8279,0.8300)
\PST@Diamond(0.8338,0.8352)
\PST@Diamond(0.8386,0.8408)
\PST@Diamond(0.8434,0.8459)
\PST@Diamond(0.8518,0.8534)
\PST@Diamond(0.8566,0.8585)
\PST@Diamond(0.8662,0.8679)
\PST@Diamond(0.8721,0.8740)
\PST@Diamond(0.8768,0.8791)
\PST@Diamond(0.8816,0.8843)
\PST@Diamond(0.8864,0.8885)
\PST@Diamond(0.8912,0.8941)
\PST@Diamond(0.8960,0.8988)
\PST@Diamond(0.9008,0.9030)
\PST@Diamond(0.9044,0.9067)
\PST@Diamond(0.9092,0.9119)
\PST@Diamond(0.9128,0.9156)
\PST@Diamond(0.9151,0.9189)
\PST@Diamond(0.9188,0.9217)
\PST@Diamond(0.9223,0.9254)
\PST@Diamond(0.9247,0.9287)
\PST@Diamond(0.9307,0.9339)
\PST@Diamond(0.9331,0.9367)
\PST@Diamond(0.9379,0.9409)
\PST@Diamond(0.9390,0.9437)
\PST@Dashed(0.2144,0.2066)
(0.2144,0.2066)
(0.2217,0.2140)
(0.2290,0.2214)
(0.2363,0.2289)
(0.2437,0.2363)
(0.2510,0.2437)
(0.2583,0.2512)
(0.2656,0.2586)
(0.2729,0.2660)
(0.2803,0.2735)
(0.2876,0.2809)
(0.2949,0.2883)
(0.3022,0.2958)
(0.3095,0.3032)
(0.3169,0.3106)
(0.3242,0.3180)
(0.3315,0.3255)
(0.3388,0.3329)
(0.3461,0.3403)
(0.3535,0.3478)
(0.3608,0.3552)
(0.3681,0.3626)
(0.3754,0.3701)
(0.3827,0.3775)
(0.3901,0.3849)
(0.3974,0.3924)
(0.4047,0.3998)
(0.4120,0.4072)
(0.4193,0.4147)
(0.4267,0.4221)
(0.4340,0.4295)
(0.4413,0.4370)
(0.4486,0.4444)
(0.4559,0.4518)
(0.4633,0.4593)
(0.4706,0.4667)
(0.4779,0.4741)
(0.4852,0.4815)
(0.4925,0.4890)
(0.4999,0.4964)
(0.5072,0.5038)
(0.5145,0.5113)
(0.5218,0.5187)
(0.5291,0.5261)
(0.5365,0.5336)
(0.5438,0.5410)
(0.5511,0.5484)
(0.5584,0.5559)
(0.5657,0.5633)
(0.5731,0.5707)
(0.5804,0.5782)
(0.5877,0.5856)
(0.5950,0.5930)
(0.6023,0.6005)
(0.6096,0.6079)
(0.6170,0.6153)
(0.6243,0.6227)
(0.6316,0.6302)
(0.6389,0.6376)
(0.6462,0.6450)
(0.6536,0.6525)
(0.6609,0.6599)
(0.6682,0.6673)
(0.6755,0.6748)
(0.6828,0.6822)
(0.6902,0.6896)
(0.6975,0.6971)
(0.7048,0.7045)
(0.7121,0.7119)
(0.7194,0.7194)
(0.7268,0.7268)
(0.7341,0.7342)
(0.7414,0.7417)
(0.7487,0.7491)
(0.7560,0.7565)
(0.7634,0.7640)
(0.7707,0.7714)
(0.7780,0.7788)
(0.7853,0.7862)
(0.7926,0.7937)
(0.8000,0.8011)
(0.8073,0.8085)
(0.8146,0.8160)
(0.8219,0.8234)
(0.8292,0.8308)
(0.8366,0.8383)
(0.8439,0.8457)
(0.8512,0.8531)
(0.8585,0.8606)
(0.8658,0.8680)
(0.8732,0.8754)
(0.8805,0.8829)
(0.8878,0.8903)
(0.8951,0.8977)
(0.9024,0.9052)
(0.9098,0.9126)
(0.9171,0.9200)
(0.9244,0.9274)
(0.9317,0.9349)
(0.9390,0.9423)

\PST@Border(0.1270,0.9680)
(0.1270,0.1260)
(0.9470,0.1260)
(0.9470,0.9680)
(0.1270,0.9680)

\catcode`@=12
\fi
\endpspicture

\end{figure}
\begin{figure}[p]\caption{Interpolazione dei dati con inverso della pressione in ascissa e volume in ordinata. Temperatura di 35 \celsius, fase di compressione.}\label{35c}
% GNUPLOT: LaTeX picture using PSTRICKS macros
% Define new PST objects, if not already defined
\ifx\PSTloaded\undefined
\def\PSTloaded{t}
\psset{arrowsize=.01 3.2 1.4 .3}
\psset{dotsize=.04}
\catcode`@=11

\newpsobject{PST@Border}{psline}{linewidth=.0015,linestyle=solid}
\newpsobject{PST@Axes}{psline}{linewidth=.0015,linestyle=dotted,dotsep=.004}
\newpsobject{PST@Solid}{psline}{linewidth=.0015,linestyle=solid}
\newpsobject{PST@Dashed}{psline}{linewidth=.0015,linestyle=dashed,dash=.01 .01}
\newpsobject{PST@Dotted}{psline}{linewidth=.0025,linestyle=dotted,dotsep=.008}
\newpsobject{PST@LongDash}{psline}{linewidth=.0015,linestyle=dashed,dash=.02 .01}
\newpsobject{PST@Diamond}{psdots}{linewidth=.001,linestyle=solid,dotstyle=square,dotangle=45}
\newpsobject{PST@Filldiamond}{psdots}{linewidth=.001,linestyle=solid,dotstyle=square*,dotangle=45}
\newpsobject{PST@Cross}{psdots}{linewidth=.001,linestyle=solid,dotstyle=+,dotangle=45}
\newpsobject{PST@Plus}{psdots}{linewidth=.001,linestyle=solid,dotstyle=+}
\newpsobject{PST@Square}{psdots}{linewidth=.001,linestyle=solid,dotstyle=square}
\newpsobject{PST@Circle}{psdots}{linewidth=.001,linestyle=solid,dotstyle=o}
\newpsobject{PST@Triangle}{psdots}{linewidth=.001,linestyle=solid,dotstyle=triangle}
\newpsobject{PST@Pentagon}{psdots}{linewidth=.001,linestyle=solid,dotstyle=pentagon}
\newpsobject{PST@Fillsquare}{psdots}{linewidth=.001,linestyle=solid,dotstyle=square*}
\newpsobject{PST@Fillcircle}{psdots}{linewidth=.001,linestyle=solid,dotstyle=*}
\newpsobject{PST@Filltriangle}{psdots}{linewidth=.001,linestyle=solid,dotstyle=triangle*}
\newpsobject{PST@Fillpentagon}{psdots}{linewidth=.001,linestyle=solid,dotstyle=pentagon*}
\newpsobject{PST@Arrow}{psline}{linewidth=.001,linestyle=solid}
\catcode`@=12

\fi
\psset{unit=5.0in,xunit=5.0in,yunit=3.0in}
\pspicture(0.000000,0.000000)(1.000000,1.000000)
\ifx\nofigs\undefined
\catcode`@=11

\PST@Border(0.1270,0.1260)
(0.1420,0.1260)

\rput[r](0.1110,0.1260){4}
\PST@Border(0.1270,0.2196)
(0.1420,0.2196)

\rput[r](0.1110,0.2196){6}
\PST@Border(0.1270,0.3131)
(0.1420,0.3131)

\rput[r](0.1110,0.3131){8}
\PST@Border(0.1270,0.4067)
(0.1420,0.4067)

\rput[r](0.1110,0.4067){10}
\PST@Border(0.1270,0.5002)
(0.1420,0.5002)

\rput[r](0.1110,0.5002){12}
\PST@Border(0.1270,0.5938)
(0.1420,0.5938)

\rput[r](0.1110,0.5938){14}
\PST@Border(0.1270,0.6873)
(0.1420,0.6873)

\rput[r](0.1110,0.6873){16}
\PST@Border(0.1270,0.7809)
(0.1420,0.7809)

\rput[r](0.1110,0.7809){18}
\PST@Border(0.1270,0.8744)
(0.1420,0.8744)

\rput[r](0.1110,0.8744){20}
\PST@Border(0.1270,0.9680)
(0.1420,0.9680)

\rput[r](0.1110,0.9680){22}
\PST@Border(0.1270,0.1260)
(0.1270,0.1460)

\rput(0.1270,0.0840){2}
\PST@Border(0.2441,0.1260)
(0.2441,0.1460)

\rput(0.2441,0.0840){3}
\PST@Border(0.3613,0.1260)
(0.3613,0.1460)

\rput(0.3613,0.0840){4}
\PST@Border(0.4784,0.1260)
(0.4784,0.1460)

\rput(0.4784,0.0840){5}
\PST@Border(0.5956,0.1260)
(0.5956,0.1460)

\rput(0.5956,0.0840){6}
\PST@Border(0.7127,0.1260)
(0.7127,0.1460)

\rput(0.7127,0.0840){7}
\PST@Border(0.8299,0.1260)
(0.8299,0.1460)

\rput(0.8299,0.0840){8}
\PST@Border(0.9470,0.1260)
(0.9470,0.1460)

\rput(0.9470,0.0840){9}
\PST@Border(0.1270,0.9680)
(0.1270,0.1260)
(0.9470,0.1260)
(0.9470,0.9680)
(0.1270,0.9680)

\rput{L}(0.0420,0.5470){$V\ (\unit{cm^3})$}
\rput(0.5370,0.0210){$1/P\ (\unit{Pa^{-1}})$}
\PST@Diamond(0.9128,0.9446)
\PST@Diamond(0.9080,0.9385)
\PST@Diamond(0.8972,0.9282)
\PST@Diamond(0.8888,0.9203)
\PST@Diamond(0.8828,0.9128)
\PST@Diamond(0.8721,0.9025)
\PST@Diamond(0.8649,0.8936)
\PST@Diamond(0.8566,0.8852)
\PST@Diamond(0.8398,0.8684)
\PST@Diamond(0.8254,0.8543)
\PST@Diamond(0.8219,0.8501)
\PST@Diamond(0.8135,0.8412)
\PST@Diamond(0.8063,0.8333)
\PST@Diamond(0.7907,0.8178)
\PST@Diamond(0.7848,0.8118)
\PST@Diamond(0.7812,0.8076)
\PST@Diamond(0.7728,0.7996)
\PST@Diamond(0.7620,0.7893)
\PST@Diamond(0.7549,0.7809)
\PST@Diamond(0.7489,0.7739)
\PST@Diamond(0.7429,0.7678)
\PST@Diamond(0.7358,0.7603)
\PST@Diamond(0.7286,0.7528)
\PST@Diamond(0.7238,0.7477)
\PST@Diamond(0.7166,0.7407)
\PST@Diamond(0.7119,0.7355)
\PST@Diamond(0.7083,0.7313)
\PST@Diamond(0.7035,0.7271)
\PST@Diamond(0.6987,0.7219)
\PST@Diamond(0.6927,0.7163)
\PST@Diamond(0.6867,0.7093)
\PST@Diamond(0.6819,0.7051)
\PST@Diamond(0.6772,0.7000)
\PST@Diamond(0.6712,0.6943)
\PST@Diamond(0.6640,0.6859)
\PST@Diamond(0.6532,0.6752)
\PST@Diamond(0.6449,0.6663)
\PST@Diamond(0.6389,0.6602)
\PST@Diamond(0.6329,0.6541)
\PST@Diamond(0.6281,0.6494)
\PST@Diamond(0.6245,0.6448)
\PST@Diamond(0.6210,0.6410)
\PST@Diamond(0.6174,0.6382)
\PST@Diamond(0.6138,0.6340)
\PST@Diamond(0.6114,0.6312)
\PST@Diamond(0.6066,0.6261)
\PST@Diamond(0.6031,0.6214)
\PST@Diamond(0.5958,0.6153)
\PST@Diamond(0.5911,0.6097)
\PST@Diamond(0.5851,0.6036)
\PST@Diamond(0.5755,0.5942)
\PST@Diamond(0.5707,0.5891)
\PST@Diamond(0.5659,0.5840)
\PST@Diamond(0.5611,0.5779)
\PST@Diamond(0.5540,0.5713)
\PST@Diamond(0.5492,0.5662)
\PST@Diamond(0.5397,0.5568)
\PST@Diamond(0.5337,0.5503)
\PST@Diamond(0.5229,0.5395)
\PST@Diamond(0.5145,0.5311)
\PST@Diamond(0.5062,0.5222)
\PST@Diamond(0.5026,0.5180)
\PST@Diamond(0.4942,0.5091)
\PST@Diamond(0.4870,0.5016)
\PST@Diamond(0.4787,0.4941)
\PST@Diamond(0.4727,0.4871)
\PST@Diamond(0.4644,0.4787)
\PST@Diamond(0.4583,0.4731)
\PST@Diamond(0.4536,0.4670)
\PST@Diamond(0.4463,0.4600)
\PST@Diamond(0.4392,0.4534)
\PST@Diamond(0.4332,0.4460)
\PST@Diamond(0.4284,0.4403)
\PST@Diamond(0.4165,0.4291)
\PST@Diamond(0.4093,0.4212)
\PST@Diamond(0.3997,0.4127)
\PST@Diamond(0.3914,0.4040)
\PST@Diamond(0.3842,0.3952)
\PST@Diamond(0.3735,0.3851)
\PST@Diamond(0.3662,0.3765)
\PST@Diamond(0.3567,0.3675)
\PST@Diamond(0.3459,0.3564)
\PST@Diamond(0.3400,0.3496)
\PST@Diamond(0.3280,0.3375)
\PST@Diamond(0.3220,0.3300)
\PST@Diamond(0.3101,0.3185)
\PST@Diamond(0.2993,0.3056)
\PST@Diamond(0.2874,0.2940)
\PST@Diamond(0.2801,0.2851)
\PST@Diamond(0.2658,0.2719)
\PST@Diamond(0.2562,0.2606)
\PST@Diamond(0.2467,0.2516)
\PST@Diamond(0.2371,0.2412)
\PST@Diamond(0.2311,0.2342)
\PST@Diamond(0.2215,0.2253)
\PST@Diamond(0.2132,0.2181)
\PST@Diamond(0.2108,0.2155)
\PST@Dashed(0.2108,0.2148)
(0.2108,0.2148)
(0.2178,0.2222)
(0.2249,0.2296)
(0.2320,0.2369)
(0.2391,0.2443)
(0.2462,0.2517)
(0.2533,0.2591)
(0.2604,0.2664)
(0.2675,0.2738)
(0.2746,0.2812)
(0.2817,0.2885)
(0.2888,0.2959)
(0.2959,0.3033)
(0.3029,0.3106)
(0.3100,0.3180)
(0.3171,0.3254)
(0.3242,0.3328)
(0.3313,0.3401)
(0.3384,0.3475)
(0.3455,0.3549)
(0.3526,0.3622)
(0.3597,0.3696)
(0.3668,0.3770)
(0.3739,0.3843)
(0.3809,0.3917)
(0.3880,0.3991)
(0.3951,0.4065)
(0.4022,0.4138)
(0.4093,0.4212)
(0.4164,0.4286)
(0.4235,0.4359)
(0.4306,0.4433)
(0.4377,0.4507)
(0.4448,0.4581)
(0.4519,0.4654)
(0.4590,0.4728)
(0.4660,0.4802)
(0.4731,0.4875)
(0.4802,0.4949)
(0.4873,0.5023)
(0.4944,0.5096)
(0.5015,0.5170)
(0.5086,0.5244)
(0.5157,0.5318)
(0.5228,0.5391)
(0.5299,0.5465)
(0.5370,0.5539)
(0.5440,0.5612)
(0.5511,0.5686)
(0.5582,0.5760)
(0.5653,0.5833)
(0.5724,0.5907)
(0.5795,0.5981)
(0.5866,0.6055)
(0.5937,0.6128)
(0.6008,0.6202)
(0.6079,0.6276)
(0.6150,0.6349)
(0.6221,0.6423)
(0.6291,0.6497)
(0.6362,0.6571)
(0.6433,0.6644)
(0.6504,0.6718)
(0.6575,0.6792)
(0.6646,0.6865)
(0.6717,0.6939)
(0.6788,0.7013)
(0.6859,0.7086)
(0.6930,0.7160)
(0.7001,0.7234)
(0.7071,0.7308)
(0.7142,0.7381)
(0.7213,0.7455)
(0.7284,0.7529)
(0.7355,0.7602)
(0.7426,0.7676)
(0.7497,0.7750)
(0.7568,0.7824)
(0.7639,0.7897)
(0.7710,0.7971)
(0.7781,0.8045)
(0.7852,0.8118)
(0.7922,0.8192)
(0.7993,0.8266)
(0.8064,0.8339)
(0.8135,0.8413)
(0.8206,0.8487)
(0.8277,0.8561)
(0.8348,0.8634)
(0.8419,0.8708)
(0.8490,0.8782)
(0.8561,0.8855)
(0.8632,0.8929)
(0.8702,0.9003)
(0.8773,0.9076)
(0.8844,0.9150)
(0.8915,0.9224)
(0.8986,0.9298)
(0.9057,0.9371)
(0.9128,0.9445)

\PST@Border(0.1270,0.9680)
(0.1270,0.1260)
(0.9470,0.1260)
(0.9470,0.9680)
(0.1270,0.9680)

\catcode`@=12
\fi
\endpspicture

\end{figure}
\begin{figure}[p]\caption{Interpolazione dei dati con inverso della pressione in ascissa e volume in ordinata. Temperatura di 35 \celsius, fase di espansione.}\label{35e}
% GNUPLOT: LaTeX picture using PSTRICKS macros
% Define new PST objects, if not already defined
\ifx\PSTloaded\undefined
\def\PSTloaded{t}
\psset{arrowsize=.01 3.2 1.4 .3}
\psset{dotsize=.04}
\catcode`@=11

\newpsobject{PST@Border}{psline}{linewidth=.0015,linestyle=solid}
\newpsobject{PST@Axes}{psline}{linewidth=.0015,linestyle=dotted,dotsep=.004}
\newpsobject{PST@Solid}{psline}{linewidth=.0015,linestyle=solid}
\newpsobject{PST@Dashed}{psline}{linewidth=.0015,linestyle=dashed,dash=.01 .01}
\newpsobject{PST@Dotted}{psline}{linewidth=.0025,linestyle=dotted,dotsep=.008}
\newpsobject{PST@LongDash}{psline}{linewidth=.0015,linestyle=dashed,dash=.02 .01}
\newpsobject{PST@Diamond}{psdots}{linewidth=.001,linestyle=solid,dotstyle=square,dotangle=45}
\newpsobject{PST@Filldiamond}{psdots}{linewidth=.001,linestyle=solid,dotstyle=square*,dotangle=45}
\newpsobject{PST@Cross}{psdots}{linewidth=.001,linestyle=solid,dotstyle=+,dotangle=45}
\newpsobject{PST@Plus}{psdots}{linewidth=.001,linestyle=solid,dotstyle=+}
\newpsobject{PST@Square}{psdots}{linewidth=.001,linestyle=solid,dotstyle=square}
\newpsobject{PST@Circle}{psdots}{linewidth=.001,linestyle=solid,dotstyle=o}
\newpsobject{PST@Triangle}{psdots}{linewidth=.001,linestyle=solid,dotstyle=triangle}
\newpsobject{PST@Pentagon}{psdots}{linewidth=.001,linestyle=solid,dotstyle=pentagon}
\newpsobject{PST@Fillsquare}{psdots}{linewidth=.001,linestyle=solid,dotstyle=square*}
\newpsobject{PST@Fillcircle}{psdots}{linewidth=.001,linestyle=solid,dotstyle=*}
\newpsobject{PST@Filltriangle}{psdots}{linewidth=.001,linestyle=solid,dotstyle=triangle*}
\newpsobject{PST@Fillpentagon}{psdots}{linewidth=.001,linestyle=solid,dotstyle=pentagon*}
\newpsobject{PST@Arrow}{psline}{linewidth=.001,linestyle=solid}
\catcode`@=12

\fi
\psset{unit=5.0in,xunit=5.0in,yunit=3.0in}
\pspicture(0.000000,0.000000)(1.000000,1.000000)
\ifx\nofigs\undefined
\catcode`@=11

\PST@Border(0.1270,0.1260)
(0.1420,0.1260)

\rput[r](0.1110,0.1260){4}
\PST@Border(0.1270,0.2196)
(0.1420,0.2196)

\rput[r](0.1110,0.2196){6}
\PST@Border(0.1270,0.3131)
(0.1420,0.3131)

\rput[r](0.1110,0.3131){8}
\PST@Border(0.1270,0.4067)
(0.1420,0.4067)

\rput[r](0.1110,0.4067){10}
\PST@Border(0.1270,0.5002)
(0.1420,0.5002)

\rput[r](0.1110,0.5002){12}
\PST@Border(0.1270,0.5938)
(0.1420,0.5938)

\rput[r](0.1110,0.5938){14}
\PST@Border(0.1270,0.6873)
(0.1420,0.6873)

\rput[r](0.1110,0.6873){16}
\PST@Border(0.1270,0.7809)
(0.1420,0.7809)

\rput[r](0.1110,0.7809){18}
\PST@Border(0.1270,0.8744)
(0.1420,0.8744)

\rput[r](0.1110,0.8744){20}
\PST@Border(0.1270,0.9680)
(0.1420,0.9680)

\rput[r](0.1110,0.9680){22}
\PST@Border(0.1270,0.1260)
(0.1270,0.1460)

\rput(0.1270,0.0840){2}
\PST@Border(0.2441,0.1260)
(0.2441,0.1460)

\rput(0.2441,0.0840){3}
\PST@Border(0.3613,0.1260)
(0.3613,0.1460)

\rput(0.3613,0.0840){4}
\PST@Border(0.4784,0.1260)
(0.4784,0.1460)

\rput(0.4784,0.0840){5}
\PST@Border(0.5956,0.1260)
(0.5956,0.1460)

\rput(0.5956,0.0840){6}
\PST@Border(0.7127,0.1260)
(0.7127,0.1460)

\rput(0.7127,0.0840){7}
\PST@Border(0.8299,0.1260)
(0.8299,0.1460)

\rput(0.8299,0.0840){8}
\PST@Border(0.9470,0.1260)
(0.9470,0.1460)

\rput(0.9470,0.0840){9}
\PST@Border(0.1270,0.9680)
(0.1270,0.1260)
(0.9470,0.1260)
(0.9470,0.9680)
(0.1270,0.9680)

\rput{L}(0.0420,0.5470){$V\ (\unit{cm^3})$}
\rput(0.5370,0.0210){$1/P\ (\unit{Pa^{-1}})$}
\PST@Diamond(0.2108,0.2154)
\PST@Diamond(0.2227,0.2225)
\PST@Diamond(0.2359,0.2363)
\PST@Diamond(0.2491,0.2498)
\PST@Diamond(0.2550,0.2561)
\PST@Diamond(0.2610,0.2622)
\PST@Diamond(0.2682,0.2702)
\PST@Diamond(0.2766,0.2784)
\PST@Diamond(0.2837,0.2865)
\PST@Diamond(0.2945,0.2979)
\PST@Diamond(0.3053,0.3087)
\PST@Diamond(0.3161,0.3207)
\PST@Diamond(0.3256,0.3300)
\PST@Diamond(0.3328,0.3391)
\PST@Diamond(0.3400,0.3467)
\PST@Diamond(0.3471,0.3538)
\PST@Diamond(0.3519,0.3589)
\PST@Diamond(0.3567,0.3642)
\PST@Diamond(0.3627,0.3705)
\PST@Diamond(0.3675,0.3754)
\PST@Diamond(0.3735,0.3816)
\PST@Diamond(0.3830,0.3906)
\PST@Diamond(0.3914,0.3994)
\PST@Diamond(0.3997,0.4090)
\PST@Diamond(0.4057,0.4151)
\PST@Diamond(0.4141,0.4235)
\PST@Diamond(0.4249,0.4338)
\PST@Diamond(0.4380,0.4469)
\PST@Diamond(0.4463,0.4567)
\PST@Diamond(0.4571,0.4675)
\PST@Diamond(0.4644,0.4754)
\PST@Diamond(0.4727,0.4838)
\PST@Diamond(0.4835,0.4941)
\PST@Diamond(0.4918,0.5035)
\PST@Diamond(0.5074,0.5185)
\PST@Diamond(0.5157,0.5278)
\PST@Diamond(0.5277,0.5409)
\PST@Diamond(0.5372,0.5507)
\PST@Diamond(0.5516,0.5648)
\PST@Diamond(0.5576,0.5718)
\PST@Diamond(0.5659,0.5811)
\PST@Diamond(0.5755,0.5900)
\PST@Diamond(0.5863,0.6013)
\PST@Diamond(0.5935,0.6092)
\PST@Diamond(0.6018,0.6176)
\PST@Diamond(0.6102,0.6270)
\PST@Diamond(0.6185,0.6359)
\PST@Diamond(0.6258,0.6434)
\PST@Diamond(0.6366,0.6546)
\PST@Diamond(0.6437,0.6630)
\PST@Diamond(0.6605,0.6794)
\PST@Diamond(0.6664,0.6869)
\PST@Diamond(0.6748,0.6953)
\PST@Diamond(0.6855,0.7070)
\PST@Diamond(0.6915,0.7131)
\PST@Diamond(0.6963,0.7173)
\PST@Diamond(0.6999,0.7215)
\PST@Diamond(0.7023,0.7248)
\PST@Diamond(0.7059,0.7276)
\PST@Diamond(0.7083,0.7304)
\PST@Diamond(0.7106,0.7327)
\PST@Diamond(0.7131,0.7350)
\PST@Diamond(0.7142,0.7374)
\PST@Diamond(0.7166,0.7393)
\PST@Diamond(0.7179,0.7407)
\PST@Diamond(0.7190,0.7416)
\PST@Diamond(0.7381,0.7598)
\PST@Diamond(0.7370,0.7612)
\PST@Diamond(0.7489,0.7720)
\PST@Diamond(0.7561,0.7814)
\PST@Diamond(0.7657,0.7907)
\PST@Diamond(0.7728,0.7991)
\PST@Diamond(0.7812,0.8066)
\PST@Diamond(0.7872,0.8136)
\PST@Diamond(0.7967,0.8230)
\PST@Diamond(0.8051,0.8314)
\PST@Diamond(0.8183,0.8459)
\PST@Diamond(0.8231,0.8515)
\PST@Diamond(0.8375,0.8651)
\PST@Diamond(0.8481,0.8754)
\PST@Diamond(0.8577,0.8861)
\PST@Diamond(0.8673,0.8960)
\PST@Diamond(0.8768,0.9063)
\PST@Diamond(0.8936,0.9245)
\PST@Diamond(0.9080,0.9390)
\PST@Dashed(0.2108,0.2105)
(0.2108,0.2105)
(0.2178,0.2178)
(0.2248,0.2252)
(0.2319,0.2326)
(0.2389,0.2399)
(0.2460,0.2473)
(0.2530,0.2546)
(0.2601,0.2620)
(0.2671,0.2693)
(0.2741,0.2767)
(0.2812,0.2841)
(0.2882,0.2914)
(0.2953,0.2988)
(0.3023,0.3061)
(0.3094,0.3135)
(0.3164,0.3208)
(0.3234,0.3282)
(0.3305,0.3356)
(0.3375,0.3429)
(0.3446,0.3503)
(0.3516,0.3576)
(0.3587,0.3650)
(0.3657,0.3724)
(0.3727,0.3797)
(0.3798,0.3871)
(0.3868,0.3944)
(0.3939,0.4018)
(0.4009,0.4091)
(0.4080,0.4165)
(0.4150,0.4239)
(0.4220,0.4312)
(0.4291,0.4386)
(0.4361,0.4459)
(0.4432,0.4533)
(0.4502,0.4606)
(0.4573,0.4680)
(0.4643,0.4754)
(0.4713,0.4827)
(0.4784,0.4901)
(0.4854,0.4974)
(0.4925,0.5048)
(0.4995,0.5121)
(0.5066,0.5195)
(0.5136,0.5269)
(0.5206,0.5342)
(0.5277,0.5416)
(0.5347,0.5489)
(0.5418,0.5563)
(0.5488,0.5636)
(0.5559,0.5710)
(0.5629,0.5784)
(0.5699,0.5857)
(0.5770,0.5931)
(0.5840,0.6004)
(0.5911,0.6078)
(0.5981,0.6151)
(0.6052,0.6225)
(0.6122,0.6299)
(0.6192,0.6372)
(0.6263,0.6446)
(0.6333,0.6519)
(0.6404,0.6593)
(0.6474,0.6666)
(0.6545,0.6740)
(0.6615,0.6814)
(0.6685,0.6887)
(0.6756,0.6961)
(0.6826,0.7034)
(0.6897,0.7108)
(0.6967,0.7181)
(0.7038,0.7255)
(0.7108,0.7329)
(0.7178,0.7402)
(0.7249,0.7476)
(0.7319,0.7549)
(0.7390,0.7623)
(0.7460,0.7697)
(0.7531,0.7770)
(0.7601,0.7844)
(0.7671,0.7917)
(0.7742,0.7991)
(0.7812,0.8064)
(0.7883,0.8138)
(0.7953,0.8212)
(0.8023,0.8285)
(0.8094,0.8359)
(0.8164,0.8432)
(0.8235,0.8506)
(0.8305,0.8579)
(0.8376,0.8653)
(0.8446,0.8727)
(0.8516,0.8800)
(0.8587,0.8874)
(0.8657,0.8947)
(0.8728,0.9021)
(0.8798,0.9094)
(0.8869,0.9168)
(0.8939,0.9242)
(0.9009,0.9315)
(0.9080,0.9389)

\PST@Border(0.1270,0.9680)
(0.1270,0.1260)
(0.9470,0.1260)
(0.9470,0.9680)
(0.1270,0.9680)

\catcode`@=12
\fi
\endpspicture

\end{figure}
\begin{figure}[p]\caption{Interpolazione dei dati con inverso della pressione in ascissa e volume in ordinata. Temperatura di 45 \celsius, fase di compressione.}\label{45c}
% GNUPLOT: LaTeX picture using PSTRICKS macros
% Define new PST objects, if not already defined
\ifx\PSTloaded\undefined
\def\PSTloaded{t}
\psset{arrowsize=.01 3.2 1.4 .3}
\psset{dotsize=.04}
\catcode`@=11

\newpsobject{PST@Border}{psline}{linewidth=.0015,linestyle=solid}
\newpsobject{PST@Axes}{psline}{linewidth=.0015,linestyle=dotted,dotsep=.004}
\newpsobject{PST@Solid}{psline}{linewidth=.0015,linestyle=solid}
\newpsobject{PST@Dashed}{psline}{linewidth=.0015,linestyle=dashed,dash=.01 .01}
\newpsobject{PST@Dotted}{psline}{linewidth=.0025,linestyle=dotted,dotsep=.008}
\newpsobject{PST@LongDash}{psline}{linewidth=.0015,linestyle=dashed,dash=.02 .01}
\newpsobject{PST@Diamond}{psdots}{linewidth=.001,linestyle=solid,dotstyle=square,dotangle=45}
\newpsobject{PST@Filldiamond}{psdots}{linewidth=.001,linestyle=solid,dotstyle=square*,dotangle=45}
\newpsobject{PST@Cross}{psdots}{linewidth=.001,linestyle=solid,dotstyle=+,dotangle=45}
\newpsobject{PST@Plus}{psdots}{linewidth=.001,linestyle=solid,dotstyle=+}
\newpsobject{PST@Square}{psdots}{linewidth=.001,linestyle=solid,dotstyle=square}
\newpsobject{PST@Circle}{psdots}{linewidth=.001,linestyle=solid,dotstyle=o}
\newpsobject{PST@Triangle}{psdots}{linewidth=.001,linestyle=solid,dotstyle=triangle}
\newpsobject{PST@Pentagon}{psdots}{linewidth=.001,linestyle=solid,dotstyle=pentagon}
\newpsobject{PST@Fillsquare}{psdots}{linewidth=.001,linestyle=solid,dotstyle=square*}
\newpsobject{PST@Fillcircle}{psdots}{linewidth=.001,linestyle=solid,dotstyle=*}
\newpsobject{PST@Filltriangle}{psdots}{linewidth=.001,linestyle=solid,dotstyle=triangle*}
\newpsobject{PST@Fillpentagon}{psdots}{linewidth=.001,linestyle=solid,dotstyle=pentagon*}
\newpsobject{PST@Arrow}{psline}{linewidth=.001,linestyle=solid}
\catcode`@=12

\fi
\psset{unit=5.0in,xunit=5.0in,yunit=3.0in}
\pspicture(0.000000,0.000000)(1.000000,1.000000)
\ifx\nofigs\undefined
\catcode`@=11

\PST@Border(0.1270,0.1260)
(0.1420,0.1260)

\rput[r](0.1110,0.1260){4}
\PST@Border(0.1270,0.2196)
(0.1420,0.2196)

\rput[r](0.1110,0.2196){6}
\PST@Border(0.1270,0.3131)
(0.1420,0.3131)

\rput[r](0.1110,0.3131){8}
\PST@Border(0.1270,0.4067)
(0.1420,0.4067)

\rput[r](0.1110,0.4067){10}
\PST@Border(0.1270,0.5002)
(0.1420,0.5002)

\rput[r](0.1110,0.5002){12}
\PST@Border(0.1270,0.5938)
(0.1420,0.5938)

\rput[r](0.1110,0.5938){14}
\PST@Border(0.1270,0.6873)
(0.1420,0.6873)

\rput[r](0.1110,0.6873){16}
\PST@Border(0.1270,0.7809)
(0.1420,0.7809)

\rput[r](0.1110,0.7809){18}
\PST@Border(0.1270,0.8744)
(0.1420,0.8744)

\rput[r](0.1110,0.8744){20}
\PST@Border(0.1270,0.9680)
(0.1420,0.9680)

\rput[r](0.1110,0.9680){22}
\PST@Border(0.1270,0.1260)
(0.1270,0.1460)

\rput(0.1270,0.0840){2}
\PST@Border(0.2441,0.1260)
(0.2441,0.1460)

\rput(0.2441,0.0840){3}
\PST@Border(0.3613,0.1260)
(0.3613,0.1460)

\rput(0.3613,0.0840){4}
\PST@Border(0.4784,0.1260)
(0.4784,0.1460)

\rput(0.4784,0.0840){5}
\PST@Border(0.5956,0.1260)
(0.5956,0.1460)

\rput(0.5956,0.0840){6}
\PST@Border(0.7127,0.1260)
(0.7127,0.1460)

\rput(0.7127,0.0840){7}
\PST@Border(0.8299,0.1260)
(0.8299,0.1460)

\rput(0.8299,0.0840){8}
\PST@Border(0.9470,0.1260)
(0.9470,0.1460)

\rput(0.9470,0.0840){9}
\PST@Border(0.1270,0.9680)
(0.1270,0.1260)
(0.9470,0.1260)
(0.9470,0.9680)
(0.1270,0.9680)

\rput{L}(0.0420,0.5470){$V\ (\unit{cm^3})$}
\rput(0.5370,0.0210){$1/P\ (\unit{Pa^{-1}})$}
\PST@Diamond(0.8841,0.9446)
\PST@Diamond(0.8745,0.9339)
\PST@Diamond(0.8625,0.9212)
\PST@Diamond(0.8529,0.9100)
\PST@Diamond(0.8422,0.8983)
\PST@Diamond(0.8327,0.8889)
\PST@Diamond(0.8267,0.8815)
\PST@Diamond(0.8183,0.8735)
\PST@Diamond(0.8088,0.8632)
\PST@Diamond(0.8003,0.8543)
\PST@Diamond(0.7944,0.8473)
\PST@Diamond(0.7872,0.8403)
\PST@Diamond(0.7824,0.8352)
\PST@Diamond(0.7740,0.8267)
\PST@Diamond(0.7645,0.8155)
\PST@Diamond(0.7549,0.8057)
\PST@Diamond(0.7489,0.7991)
\PST@Diamond(0.7250,0.7725)
\PST@Diamond(0.7179,0.7645)
\PST@Diamond(0.7094,0.7570)
\PST@Diamond(0.7023,0.7486)
\PST@Diamond(0.6987,0.7444)
\PST@Diamond(0.6951,0.7411)
\PST@Diamond(0.6892,0.7341)
\PST@Diamond(0.6807,0.7243)
\PST@Diamond(0.6759,0.7210)
\PST@Diamond(0.6724,0.7107)
\PST@Diamond(0.6664,0.7089)
\PST@Diamond(0.6605,0.7042)
\PST@Diamond(0.6580,0.7014)
\PST@Diamond(0.6545,0.6976)
\PST@Diamond(0.6509,0.6939)
\PST@Diamond(0.6449,0.6873)
\PST@Diamond(0.6389,0.6813)
\PST@Diamond(0.6245,0.6644)
\PST@Diamond(0.6174,0.6588)
\PST@Diamond(0.6138,0.6551)
\PST@Diamond(0.6114,0.6513)
\PST@Diamond(0.6066,0.6466)
\PST@Diamond(0.6006,0.6392)
\PST@Diamond(0.5958,0.6321)
\PST@Diamond(0.5935,0.6312)
\PST@Diamond(0.5875,0.6256)
\PST@Diamond(0.5827,0.6200)
\PST@Diamond(0.5779,0.6153)
\PST@Diamond(0.5731,0.6106)
\PST@Diamond(0.5696,0.6064)
\PST@Diamond(0.5624,0.5980)
\PST@Diamond(0.5505,0.5863)
\PST@Diamond(0.5444,0.5788)
\PST@Diamond(0.5384,0.5727)
\PST@Diamond(0.5337,0.5676)
\PST@Diamond(0.5265,0.5606)
\PST@Diamond(0.5218,0.5540)
\PST@Diamond(0.5157,0.5493)
\PST@Diamond(0.5122,0.5442)
\PST@Diamond(0.5062,0.5376)
\PST@Diamond(0.4990,0.5311)
\PST@Diamond(0.4966,0.5274)
\PST@Diamond(0.4918,0.5231)
\PST@Diamond(0.4846,0.5152)
\PST@Diamond(0.4787,0.5091)
\PST@Diamond(0.4750,0.5044)
\PST@Diamond(0.4703,0.5002)
\PST@Diamond(0.4631,0.4927)
\PST@Diamond(0.4596,0.4876)
\PST@Diamond(0.4536,0.4820)
\PST@Diamond(0.4488,0.4773)
\PST@Diamond(0.4452,0.4736)
\PST@Diamond(0.4380,0.4651)
\PST@Diamond(0.4320,0.4577)
\PST@Diamond(0.4224,0.4502)
\PST@Diamond(0.4165,0.4422)
\PST@Diamond(0.4105,0.4361)
\PST@Diamond(0.4057,0.4310)
\PST@Diamond(0.4010,0.4254)
\PST@Diamond(0.3949,0.4193)
\PST@Diamond(0.3889,0.4127)
\PST@Diamond(0.3830,0.4065)
\PST@Diamond(0.3794,0.4027)
\PST@Diamond(0.3735,0.3959)
\PST@Diamond(0.3639,0.3872)
\PST@Diamond(0.3579,0.3796)
\PST@Diamond(0.3519,0.3734)
\PST@Diamond(0.3436,0.3650)
\PST@Diamond(0.3363,0.3569)
\PST@Diamond(0.3292,0.3492)
\PST@Diamond(0.3232,0.3427)
\PST@Diamond(0.3184,0.3372)
\PST@Diamond(0.3124,0.3302)
\PST@Diamond(0.3053,0.3230)
\PST@Diamond(0.2993,0.3158)
\PST@Diamond(0.2945,0.3105)
\PST@Diamond(0.2897,0.3051)
\PST@Diamond(0.2849,0.3002)
\PST@Diamond(0.2789,0.2945)
\PST@Diamond(0.2754,0.2895)
\PST@Diamond(0.2706,0.2839)
\PST@Diamond(0.2646,0.2779)
\PST@Diamond(0.2598,0.2735)
\PST@Diamond(0.2550,0.2683)
\PST@Diamond(0.2514,0.2641)
\PST@Diamond(0.2467,0.2585)
\PST@Diamond(0.2419,0.2536)
\PST@Diamond(0.2371,0.2483)
\PST@Diamond(0.2323,0.2430)
\PST@Diamond(0.2252,0.2358)
\PST@Diamond(0.2167,0.2276)
\PST@Diamond(0.2108,0.2189)
\PST@Diamond(0.2108,0.2161)
\PST@Dashed(0.2108,0.2209)
(0.2108,0.2209)
(0.2176,0.2282)
(0.2244,0.2355)
(0.2312,0.2428)
(0.2380,0.2501)
(0.2448,0.2574)
(0.2516,0.2647)
(0.2584,0.2720)
(0.2652,0.2793)
(0.2720,0.2866)
(0.2788,0.2939)
(0.2856,0.3012)
(0.2924,0.3086)
(0.2992,0.3159)
(0.3060,0.3232)
(0.3128,0.3305)
(0.3196,0.3378)
(0.3264,0.3451)
(0.3332,0.3524)
(0.3400,0.3597)
(0.3468,0.3670)
(0.3536,0.3743)
(0.3604,0.3816)
(0.3672,0.3889)
(0.3740,0.3962)
(0.3808,0.4035)
(0.3876,0.4108)
(0.3944,0.4181)
(0.4012,0.4254)
(0.4080,0.4327)
(0.4148,0.4400)
(0.4216,0.4473)
(0.4284,0.4546)
(0.4352,0.4620)
(0.4420,0.4693)
(0.4488,0.4766)
(0.4556,0.4839)
(0.4624,0.4912)
(0.4692,0.4985)
(0.4760,0.5058)
(0.4828,0.5131)
(0.4896,0.5204)
(0.4964,0.5277)
(0.5032,0.5350)
(0.5100,0.5423)
(0.5168,0.5496)
(0.5236,0.5569)
(0.5304,0.5642)
(0.5372,0.5715)
(0.5440,0.5788)
(0.5508,0.5861)
(0.5576,0.5934)
(0.5644,0.6007)
(0.5712,0.6081)
(0.5780,0.6154)
(0.5848,0.6227)
(0.5916,0.6300)
(0.5984,0.6373)
(0.6052,0.6446)
(0.6120,0.6519)
(0.6188,0.6592)
(0.6256,0.6665)
(0.6324,0.6738)
(0.6392,0.6811)
(0.6460,0.6884)
(0.6528,0.6957)
(0.6596,0.7030)
(0.6664,0.7103)
(0.6733,0.7176)
(0.6801,0.7249)
(0.6869,0.7322)
(0.6937,0.7395)
(0.7005,0.7468)
(0.7073,0.7542)
(0.7141,0.7615)
(0.7209,0.7688)
(0.7277,0.7761)
(0.7345,0.7834)
(0.7413,0.7907)
(0.7481,0.7980)
(0.7549,0.8053)
(0.7617,0.8126)
(0.7685,0.8199)
(0.7753,0.8272)
(0.7821,0.8345)
(0.7889,0.8418)
(0.7957,0.8491)
(0.8025,0.8564)
(0.8093,0.8637)
(0.8161,0.8710)
(0.8229,0.8783)
(0.8297,0.8856)
(0.8365,0.8929)
(0.8433,0.9002)
(0.8501,0.9076)
(0.8569,0.9149)
(0.8637,0.9222)
(0.8705,0.9295)
(0.8773,0.9368)
(0.8841,0.9441)

\PST@Border(0.1270,0.9680)
(0.1270,0.1260)
(0.9470,0.1260)
(0.9470,0.9680)
(0.1270,0.9680)

\catcode`@=12
\fi
\endpspicture

\end{figure}
\begin{figure}[p]\caption{Interpolazione dei dati con inverso della pressione in ascissa e volume in ordinata. Temperatura di 45 \celsius, fase di espansione.}\label{45e}
% GNUPLOT: LaTeX picture using PSTRICKS macros
% Define new PST objects, if not already defined
\ifx\PSTloaded\undefined
\def\PSTloaded{t}
\psset{arrowsize=.01 3.2 1.4 .3}
\psset{dotsize=.04}
\catcode`@=11

\newpsobject{PST@Border}{psline}{linewidth=.0015,linestyle=solid}
\newpsobject{PST@Axes}{psline}{linewidth=.0015,linestyle=dotted,dotsep=.004}
\newpsobject{PST@Solid}{psline}{linewidth=.0015,linestyle=solid}
\newpsobject{PST@Dashed}{psline}{linewidth=.0015,linestyle=dashed,dash=.01 .01}
\newpsobject{PST@Dotted}{psline}{linewidth=.0025,linestyle=dotted,dotsep=.008}
\newpsobject{PST@LongDash}{psline}{linewidth=.0015,linestyle=dashed,dash=.02 .01}
\newpsobject{PST@Diamond}{psdots}{linewidth=.001,linestyle=solid,dotstyle=square,dotangle=45}
\newpsobject{PST@Filldiamond}{psdots}{linewidth=.001,linestyle=solid,dotstyle=square*,dotangle=45}
\newpsobject{PST@Cross}{psdots}{linewidth=.001,linestyle=solid,dotstyle=+,dotangle=45}
\newpsobject{PST@Plus}{psdots}{linewidth=.001,linestyle=solid,dotstyle=+}
\newpsobject{PST@Square}{psdots}{linewidth=.001,linestyle=solid,dotstyle=square}
\newpsobject{PST@Circle}{psdots}{linewidth=.001,linestyle=solid,dotstyle=o}
\newpsobject{PST@Triangle}{psdots}{linewidth=.001,linestyle=solid,dotstyle=triangle}
\newpsobject{PST@Pentagon}{psdots}{linewidth=.001,linestyle=solid,dotstyle=pentagon}
\newpsobject{PST@Fillsquare}{psdots}{linewidth=.001,linestyle=solid,dotstyle=square*}
\newpsobject{PST@Fillcircle}{psdots}{linewidth=.001,linestyle=solid,dotstyle=*}
\newpsobject{PST@Filltriangle}{psdots}{linewidth=.001,linestyle=solid,dotstyle=triangle*}
\newpsobject{PST@Fillpentagon}{psdots}{linewidth=.001,linestyle=solid,dotstyle=pentagon*}
\newpsobject{PST@Arrow}{psline}{linewidth=.001,linestyle=solid}
\catcode`@=12

\fi
\psset{unit=5.0in,xunit=5.0in,yunit=3.0in}
\pspicture(0.000000,0.000000)(1.000000,1.000000)
\ifx\nofigs\undefined
\catcode`@=11

\PST@Border(0.1270,0.1260)
(0.1420,0.1260)

\rput[r](0.1110,0.1260){4}
\PST@Border(0.1270,0.2196)
(0.1420,0.2196)

\rput[r](0.1110,0.2196){6}
\PST@Border(0.1270,0.3131)
(0.1420,0.3131)

\rput[r](0.1110,0.3131){8}
\PST@Border(0.1270,0.4067)
(0.1420,0.4067)

\rput[r](0.1110,0.4067){10}
\PST@Border(0.1270,0.5002)
(0.1420,0.5002)

\rput[r](0.1110,0.5002){12}
\PST@Border(0.1270,0.5938)
(0.1420,0.5938)

\rput[r](0.1110,0.5938){14}
\PST@Border(0.1270,0.6873)
(0.1420,0.6873)

\rput[r](0.1110,0.6873){16}
\PST@Border(0.1270,0.7809)
(0.1420,0.7809)

\rput[r](0.1110,0.7809){18}
\PST@Border(0.1270,0.8744)
(0.1420,0.8744)

\rput[r](0.1110,0.8744){20}
\PST@Border(0.1270,0.9680)
(0.1420,0.9680)

\rput[r](0.1110,0.9680){22}
\PST@Border(0.1270,0.1260)
(0.1270,0.1460)

\rput(0.1270,0.0840){2}
\PST@Border(0.2441,0.1260)
(0.2441,0.1460)

\rput(0.2441,0.0840){3}
\PST@Border(0.3613,0.1260)
(0.3613,0.1460)

\rput(0.3613,0.0840){4}
\PST@Border(0.4784,0.1260)
(0.4784,0.1460)

\rput(0.4784,0.0840){5}
\PST@Border(0.5956,0.1260)
(0.5956,0.1460)

\rput(0.5956,0.0840){6}
\PST@Border(0.7127,0.1260)
(0.7127,0.1460)

\rput(0.7127,0.0840){7}
\PST@Border(0.8299,0.1260)
(0.8299,0.1460)

\rput(0.8299,0.0840){8}
\PST@Border(0.9470,0.1260)
(0.9470,0.1460)

\rput(0.9470,0.0840){9}
\PST@Border(0.1270,0.9680)
(0.1270,0.1260)
(0.9470,0.1260)
(0.9470,0.9680)
(0.1270,0.9680)

\rput{L}(0.0420,0.5470){$V\ (\unit{cm^3})$}
\rput(0.5370,0.0210){$1/P\ (\unit{Pa^{-1}})$}
\PST@Diamond(0.2108,0.2154)
\PST@Diamond(0.2108,0.2175)
\PST@Diamond(0.2132,0.2208)
\PST@Diamond(0.2180,0.2267)
\PST@Diamond(0.2240,0.2319)
\PST@Diamond(0.2288,0.2379)
\PST@Diamond(0.2335,0.2427)
\PST@Diamond(0.2371,0.2468)
\PST@Diamond(0.2419,0.2519)
\PST@Diamond(0.2454,0.2559)
\PST@Diamond(0.2491,0.2599)
\PST@Diamond(0.2527,0.2630)
\PST@Diamond(0.2550,0.2664)
\PST@Diamond(0.2587,0.2697)
\PST@Diamond(0.2610,0.2722)
\PST@Diamond(0.2658,0.2775)
\PST@Diamond(0.2682,0.2805)
\PST@Diamond(0.2741,0.2867)
\PST@Diamond(0.2789,0.2917)
\PST@Diamond(0.2826,0.2951)
\PST@Diamond(0.2874,0.3001)
\PST@Diamond(0.2922,0.3056)
\PST@Diamond(0.3005,0.3140)
\PST@Diamond(0.3065,0.3209)
\PST@Diamond(0.3136,0.3284)
\PST@Diamond(0.3220,0.3383)
\PST@Diamond(0.3292,0.3460)
\PST@Diamond(0.3375,0.3554)
\PST@Diamond(0.3423,0.3602)
\PST@Diamond(0.3471,0.3655)
\PST@Diamond(0.3531,0.3710)
\PST@Diamond(0.3627,0.3810)
\PST@Diamond(0.3675,0.3867)
\PST@Diamond(0.3698,0.3898)
\PST@Diamond(0.3735,0.3938)
\PST@Diamond(0.3794,0.3996)
\PST@Diamond(0.3842,0.4048)
\PST@Diamond(0.3878,0.4085)
\PST@Diamond(0.3926,0.4137)
\PST@Diamond(0.3962,0.4179)
\PST@Diamond(0.4022,0.4244)
\PST@Diamond(0.4093,0.4319)
\PST@Diamond(0.4153,0.4380)
\PST@Diamond(0.4224,0.4450)
\PST@Diamond(0.4320,0.4553)
\PST@Diamond(0.4392,0.4633)
\PST@Diamond(0.4500,0.4745)
\PST@Diamond(0.4548,0.4796)
\PST@Diamond(0.4619,0.4876)
\PST@Diamond(0.4679,0.4941)
\PST@Diamond(0.4727,0.4998)
\PST@Diamond(0.4775,0.5049)
\PST@Diamond(0.4858,0.5129)
\PST@Diamond(0.4906,0.5180)
\PST@Diamond(0.4931,0.5217)
\PST@Diamond(0.4990,0.5269)
\PST@Diamond(0.5062,0.5348)
\PST@Diamond(0.5122,0.5405)
\PST@Diamond(0.5205,0.5498)
\PST@Diamond(0.5265,0.5573)
\PST@Diamond(0.5324,0.5634)
\PST@Diamond(0.5361,0.5671)
\PST@Diamond(0.5397,0.5709)
\PST@Diamond(0.5468,0.5788)
\PST@Diamond(0.5564,0.5882)
\PST@Diamond(0.5611,0.5942)
\PST@Diamond(0.5671,0.6003)
\PST@Diamond(0.5755,0.6087)
\PST@Diamond(0.5803,0.6148)
\PST@Diamond(0.5839,0.6195)
\PST@Diamond(0.5911,0.6261)
\PST@Diamond(0.5958,0.6312)
\PST@Diamond(0.6006,0.6359)
\PST@Diamond(0.6031,0.6401)
\PST@Diamond(0.6102,0.6476)
\PST@Diamond(0.6185,0.6565)
\PST@Diamond(0.6222,0.6611)
\PST@Diamond(0.6281,0.6663)
\PST@Diamond(0.6305,0.6700)
\PST@Diamond(0.6389,0.6780)
\PST@Diamond(0.6413,0.6817)
\PST@Diamond(0.6449,0.6855)
\PST@Diamond(0.6520,0.6929)
\PST@Diamond(0.6605,0.7009)
\PST@Diamond(0.6688,0.7107)
\PST@Diamond(0.6748,0.7173)
\PST@Diamond(0.6819,0.7252)
\PST@Diamond(0.6879,0.7318)
\PST@Diamond(0.6963,0.7393)
\PST@Diamond(0.7023,0.7463)
\PST@Diamond(0.7059,0.7500)
\PST@Diamond(0.7071,0.7514)
\PST@Diamond(0.7059,0.7514)
\PST@Diamond(0.7059,0.7519)
\PST@Diamond(0.7059,0.7519)
\PST@Diamond(0.7059,0.7519)
\PST@Diamond(0.7059,0.7519)
\PST@Diamond(0.7059,0.7519)
\PST@Diamond(0.7059,0.7519)
\PST@Diamond(0.7059,0.7519)
\PST@Diamond(0.7059,0.7519)
\PST@Diamond(0.7059,0.7519)
\PST@Diamond(0.7227,0.7673)
\PST@Diamond(0.7214,0.7678)
\PST@Diamond(0.7227,0.7687)
\PST@Diamond(0.7286,0.7762)
\PST@Diamond(0.7346,0.7818)
\PST@Diamond(0.7393,0.7870)
\PST@Diamond(0.7477,0.7963)
\PST@Diamond(0.7561,0.8052)
\PST@Diamond(0.7645,0.8136)
\PST@Diamond(0.7740,0.8244)
\PST@Diamond(0.7812,0.8319)
\PST@Diamond(0.7920,0.8431)
\PST@Diamond(0.8003,0.8529)
\PST@Diamond(0.8088,0.8613)
\PST@Diamond(0.8171,0.8698)
\PST@Diamond(0.8231,0.8773)
\PST@Diamond(0.8290,0.8838)
\PST@Diamond(0.8375,0.8922)
\PST@Diamond(0.8434,0.8992)
\PST@Diamond(0.8494,0.9053)
\PST@Diamond(0.8577,0.9142)
\PST@Diamond(0.8662,0.9240)
\PST@Diamond(0.8733,0.9310)
\PST@Diamond(0.8793,0.9390)
\PST@Dashed(0.2108,0.2178)
(0.2108,0.2178)
(0.2175,0.2251)
(0.2243,0.2324)
(0.2310,0.2396)
(0.2378,0.2469)
(0.2445,0.2542)
(0.2513,0.2614)
(0.2580,0.2687)
(0.2648,0.2760)
(0.2715,0.2832)
(0.2783,0.2905)
(0.2850,0.2978)
(0.2918,0.3050)
(0.2985,0.3123)
(0.3053,0.3196)
(0.3121,0.3269)
(0.3188,0.3341)
(0.3256,0.3414)
(0.3323,0.3487)
(0.3391,0.3559)
(0.3458,0.3632)
(0.3526,0.3705)
(0.3593,0.3777)
(0.3661,0.3850)
(0.3728,0.3923)
(0.3796,0.3995)
(0.3863,0.4068)
(0.3931,0.4141)
(0.3998,0.4213)
(0.4066,0.4286)
(0.4133,0.4359)
(0.4201,0.4431)
(0.4268,0.4504)
(0.4336,0.4577)
(0.4404,0.4650)
(0.4471,0.4722)
(0.4539,0.4795)
(0.4606,0.4868)
(0.4674,0.4940)
(0.4741,0.5013)
(0.4809,0.5086)
(0.4876,0.5158)
(0.4944,0.5231)
(0.5011,0.5304)
(0.5079,0.5376)
(0.5146,0.5449)
(0.5214,0.5522)
(0.5281,0.5594)
(0.5349,0.5667)
(0.5416,0.5740)
(0.5484,0.5812)
(0.5552,0.5885)
(0.5619,0.5958)
(0.5687,0.6031)
(0.5754,0.6103)
(0.5822,0.6176)
(0.5889,0.6249)
(0.5957,0.6321)
(0.6024,0.6394)
(0.6092,0.6467)
(0.6159,0.6539)
(0.6227,0.6612)
(0.6294,0.6685)
(0.6362,0.6757)
(0.6429,0.6830)
(0.6497,0.6903)
(0.6564,0.6975)
(0.6632,0.7048)
(0.6700,0.7121)
(0.6767,0.7193)
(0.6835,0.7266)
(0.6902,0.7339)
(0.6970,0.7412)
(0.7037,0.7484)
(0.7105,0.7557)
(0.7172,0.7630)
(0.7240,0.7702)
(0.7307,0.7775)
(0.7375,0.7848)
(0.7442,0.7920)
(0.7510,0.7993)
(0.7577,0.8066)
(0.7645,0.8138)
(0.7712,0.8211)
(0.7780,0.8284)
(0.7848,0.8356)
(0.7915,0.8429)
(0.7983,0.8502)
(0.8050,0.8574)
(0.8118,0.8647)
(0.8185,0.8720)
(0.8253,0.8793)
(0.8320,0.8865)
(0.8388,0.8938)
(0.8455,0.9011)
(0.8523,0.9083)
(0.8590,0.9156)
(0.8658,0.9229)
(0.8725,0.9301)
(0.8793,0.9374)

\PST@Border(0.1270,0.9680)
(0.1270,0.1260)
(0.9470,0.1260)
(0.9470,0.9680)
(0.1270,0.9680)

\catcode`@=12
\fi
\endpspicture

\end{figure}
\begin{figure}[p]\caption{Interpolazione dei dati con inverso della pressione in ascissa e volume in ordinata. Temperatura di 55 \celsius, fase di compressione.}\label{55c}
% GNUPLOT: LaTeX picture using PSTRICKS macros
% Define new PST objects, if not already defined
\ifx\PSTloaded\undefined
\def\PSTloaded{t}
\psset{arrowsize=.01 3.2 1.4 .3}
\psset{dotsize=.04}
\catcode`@=11

\newpsobject{PST@Border}{psline}{linewidth=.0015,linestyle=solid}
\newpsobject{PST@Axes}{psline}{linewidth=.0015,linestyle=dotted,dotsep=.004}
\newpsobject{PST@Solid}{psline}{linewidth=.0015,linestyle=solid}
\newpsobject{PST@Dashed}{psline}{linewidth=.0015,linestyle=dashed,dash=.01 .01}
\newpsobject{PST@Dotted}{psline}{linewidth=.0025,linestyle=dotted,dotsep=.008}
\newpsobject{PST@LongDash}{psline}{linewidth=.0015,linestyle=dashed,dash=.02 .01}
\newpsobject{PST@Diamond}{psdots}{linewidth=.001,linestyle=solid,dotstyle=square,dotangle=45}
\newpsobject{PST@Filldiamond}{psdots}{linewidth=.001,linestyle=solid,dotstyle=square*,dotangle=45}
\newpsobject{PST@Cross}{psdots}{linewidth=.001,linestyle=solid,dotstyle=+,dotangle=45}
\newpsobject{PST@Plus}{psdots}{linewidth=.001,linestyle=solid,dotstyle=+}
\newpsobject{PST@Square}{psdots}{linewidth=.001,linestyle=solid,dotstyle=square}
\newpsobject{PST@Circle}{psdots}{linewidth=.001,linestyle=solid,dotstyle=o}
\newpsobject{PST@Triangle}{psdots}{linewidth=.001,linestyle=solid,dotstyle=triangle}
\newpsobject{PST@Pentagon}{psdots}{linewidth=.001,linestyle=solid,dotstyle=pentagon}
\newpsobject{PST@Fillsquare}{psdots}{linewidth=.001,linestyle=solid,dotstyle=square*}
\newpsobject{PST@Fillcircle}{psdots}{linewidth=.001,linestyle=solid,dotstyle=*}
\newpsobject{PST@Filltriangle}{psdots}{linewidth=.001,linestyle=solid,dotstyle=triangle*}
\newpsobject{PST@Fillpentagon}{psdots}{linewidth=.001,linestyle=solid,dotstyle=pentagon*}
\newpsobject{PST@Arrow}{psline}{linewidth=.001,linestyle=solid}
\catcode`@=12

\fi
\psset{unit=5.0in,xunit=5.0in,yunit=3.0in}
\pspicture(0.000000,0.000000)(1.000000,1.000000)
\ifx\nofigs\undefined
\catcode`@=11

\PST@Border(0.1270,0.1260)
(0.1420,0.1260)

\rput[r](0.1110,0.1260){4}
\PST@Border(0.1270,0.2196)
(0.1420,0.2196)

\rput[r](0.1110,0.2196){6}
\PST@Border(0.1270,0.3131)
(0.1420,0.3131)

\rput[r](0.1110,0.3131){8}
\PST@Border(0.1270,0.4067)
(0.1420,0.4067)

\rput[r](0.1110,0.4067){10}
\PST@Border(0.1270,0.5002)
(0.1420,0.5002)

\rput[r](0.1110,0.5002){12}
\PST@Border(0.1270,0.5938)
(0.1420,0.5938)

\rput[r](0.1110,0.5938){14}
\PST@Border(0.1270,0.6873)
(0.1420,0.6873)

\rput[r](0.1110,0.6873){16}
\PST@Border(0.1270,0.7809)
(0.1420,0.7809)

\rput[r](0.1110,0.7809){18}
\PST@Border(0.1270,0.8744)
(0.1420,0.8744)

\rput[r](0.1110,0.8744){20}
\PST@Border(0.1270,0.9680)
(0.1420,0.9680)

\rput[r](0.1110,0.9680){22}
\PST@Border(0.1270,0.1260)
(0.1270,0.1460)

\rput(0.1270,0.0840){2}
\PST@Border(0.2441,0.1260)
(0.2441,0.1460)

\rput(0.2441,0.0840){3}
\PST@Border(0.3613,0.1260)
(0.3613,0.1460)

\rput(0.3613,0.0840){4}
\PST@Border(0.4784,0.1260)
(0.4784,0.1460)

\rput(0.4784,0.0840){5}
\PST@Border(0.5956,0.1260)
(0.5956,0.1460)

\rput(0.5956,0.0840){6}
\PST@Border(0.7127,0.1260)
(0.7127,0.1460)

\rput(0.7127,0.0840){7}
\PST@Border(0.8299,0.1260)
(0.8299,0.1460)

\rput(0.8299,0.0840){8}
\PST@Border(0.9470,0.1260)
(0.9470,0.1460)

\rput(0.9470,0.0840){9}
\PST@Border(0.1270,0.9680)
(0.1270,0.1260)
(0.9470,0.1260)
(0.9470,0.9680)
(0.1270,0.9680)

\rput{L}(0.0420,0.5470){$V\ (\unit{cm^3})$}
\rput(0.5370,0.0210){$1/P\ (\unit{Pa^{-1}})$}
\PST@Diamond(0.2108,0.2153)
\PST@Diamond(0.2108,0.2173)
\PST@Diamond(0.2108,0.2217)
\PST@Diamond(0.2156,0.2305)
\PST@Diamond(0.2227,0.2384)
\PST@Diamond(0.2300,0.2465)
\PST@Diamond(0.2371,0.2541)
\PST@Diamond(0.2431,0.2607)
\PST@Diamond(0.2479,0.2669)
\PST@Diamond(0.2502,0.2698)
\PST@Diamond(0.2562,0.2754)
\PST@Diamond(0.2610,0.2811)
\PST@Diamond(0.2670,0.2874)
\PST@Diamond(0.2730,0.2946)
\PST@Diamond(0.2789,0.3003)
\PST@Diamond(0.2862,0.3081)
\PST@Diamond(0.2909,0.3143)
\PST@Diamond(0.3017,0.3251)
\PST@Diamond(0.3088,0.3337)
\PST@Diamond(0.3161,0.3423)
\PST@Diamond(0.3244,0.3513)
\PST@Diamond(0.3292,0.3574)
\PST@Diamond(0.3352,0.3639)
\PST@Diamond(0.3388,0.3677)
\PST@Diamond(0.3448,0.3741)
\PST@Diamond(0.3496,0.3793)
\PST@Diamond(0.3567,0.3872)
\PST@Diamond(0.3627,0.3931)
\PST@Diamond(0.3698,0.4022)
\PST@Diamond(0.3758,0.4081)
\PST@Diamond(0.3783,0.4109)
\PST@Diamond(0.3806,0.4137)
\PST@Diamond(0.3854,0.4193)
\PST@Diamond(0.3889,0.4240)
\PST@Diamond(0.3949,0.4296)
\PST@Diamond(0.3985,0.4338)
\PST@Diamond(0.4045,0.4399)
\PST@Diamond(0.4081,0.4441)
\PST@Diamond(0.4105,0.4474)
\PST@Diamond(0.4153,0.4520)
\PST@Diamond(0.4189,0.4558)
\PST@Diamond(0.4224,0.4600)
\PST@Diamond(0.4261,0.4647)
\PST@Diamond(0.4297,0.4679)
\PST@Diamond(0.4332,0.4722)
\PST@Diamond(0.4368,0.4764)
\PST@Diamond(0.4416,0.4815)
\PST@Diamond(0.4500,0.4899)
\PST@Diamond(0.4559,0.4969)
\PST@Diamond(0.4619,0.5030)
\PST@Diamond(0.4703,0.5124)
\PST@Diamond(0.4787,0.5213)
\PST@Diamond(0.4883,0.5316)
\PST@Diamond(0.4942,0.5390)
\PST@Diamond(0.5002,0.5451)
\PST@Diamond(0.5085,0.5540)
\PST@Diamond(0.5145,0.5606)
\PST@Diamond(0.5205,0.5671)
\PST@Diamond(0.5277,0.5751)
\PST@Diamond(0.5313,0.5802)
\PST@Diamond(0.5349,0.5840)
\PST@Diamond(0.5397,0.5891)
\PST@Diamond(0.5444,0.5947)
\PST@Diamond(0.5492,0.6003)
\PST@Diamond(0.5576,0.6092)
\PST@Diamond(0.5624,0.6148)
\PST@Diamond(0.5684,0.6209)
\PST@Diamond(0.5719,0.6256)
\PST@Diamond(0.5792,0.6335)
\PST@Diamond(0.5875,0.6424)
\PST@Diamond(0.5935,0.6490)
\PST@Diamond(0.5994,0.6560)
\PST@Diamond(0.6079,0.6649)
\PST@Diamond(0.6174,0.6752)
\PST@Diamond(0.6245,0.6836)
\PST@Diamond(0.6318,0.6920)
\PST@Diamond(0.6413,0.7028)
\PST@Diamond(0.6497,0.7117)
\PST@Diamond(0.6557,0.7187)
\PST@Diamond(0.6628,0.7257)
\PST@Diamond(0.6736,0.7379)
\PST@Diamond(0.6796,0.7439)
\PST@Diamond(0.6879,0.7538)
\PST@Diamond(0.6975,0.7640)
\PST@Diamond(0.7083,0.7739)
\PST@Diamond(0.7166,0.7865)
\PST@Diamond(0.7166,0.7870)
\PST@Diamond(0.7166,0.7870)
\PST@Diamond(0.7166,0.7870)
\PST@Diamond(0.7166,0.7870)
\PST@Diamond(0.7166,0.7874)
\PST@Diamond(0.7166,0.7874)
\PST@Diamond(0.7166,0.7874)
\PST@Diamond(0.7166,0.7870)
\PST@Diamond(0.7166,0.7874)
\PST@Diamond(0.7346,0.8052)
\PST@Diamond(0.7333,0.8057)
\PST@Diamond(0.7358,0.8080)
\PST@Diamond(0.7429,0.8160)
\PST@Diamond(0.7501,0.8249)
\PST@Diamond(0.7573,0.8328)
\PST@Diamond(0.7620,0.8370)
\PST@Diamond(0.7645,0.8403)
\PST@Diamond(0.7705,0.8468)
\PST@Diamond(0.7728,0.8506)
\PST@Diamond(0.7776,0.8557)
\PST@Diamond(0.7836,0.8618)
\PST@Diamond(0.7932,0.8721)
\PST@Diamond(0.8015,0.8815)
\PST@Diamond(0.8088,0.8894)
\PST@Diamond(0.8135,0.8955)
\PST@Diamond(0.8183,0.9006)
\PST@Diamond(0.8242,0.9072)
\PST@Diamond(0.8314,0.9156)
\PST@Diamond(0.8422,0.9273)
\PST@Diamond(0.8506,0.9357)
\PST@Dashed(0.2108,0.2241)
(0.2108,0.2241)
(0.2172,0.2313)
(0.2237,0.2385)
(0.2301,0.2457)
(0.2366,0.2529)
(0.2431,0.2600)
(0.2495,0.2672)
(0.2560,0.2744)
(0.2625,0.2816)
(0.2689,0.2888)
(0.2754,0.2960)
(0.2818,0.3032)
(0.2883,0.3104)
(0.2948,0.3176)
(0.3012,0.3247)
(0.3077,0.3319)
(0.3142,0.3391)
(0.3206,0.3463)
(0.3271,0.3535)
(0.3336,0.3607)
(0.3400,0.3679)
(0.3465,0.3751)
(0.3529,0.3823)
(0.3594,0.3894)
(0.3659,0.3966)
(0.3723,0.4038)
(0.3788,0.4110)
(0.3853,0.4182)
(0.3917,0.4254)
(0.3982,0.4326)
(0.4046,0.4398)
(0.4111,0.4470)
(0.4176,0.4541)
(0.4240,0.4613)
(0.4305,0.4685)
(0.4370,0.4757)
(0.4434,0.4829)
(0.4499,0.4901)
(0.4564,0.4973)
(0.4628,0.5045)
(0.4693,0.5117)
(0.4757,0.5188)
(0.4822,0.5260)
(0.4887,0.5332)
(0.4951,0.5404)
(0.5016,0.5476)
(0.5081,0.5548)
(0.5145,0.5620)
(0.5210,0.5692)
(0.5274,0.5764)
(0.5339,0.5835)
(0.5404,0.5907)
(0.5468,0.5979)
(0.5533,0.6051)
(0.5598,0.6123)
(0.5662,0.6195)
(0.5727,0.6267)
(0.5791,0.6339)
(0.5856,0.6411)
(0.5921,0.6482)
(0.5985,0.6554)
(0.6050,0.6626)
(0.6115,0.6698)
(0.6179,0.6770)
(0.6244,0.6842)
(0.6309,0.6914)
(0.6373,0.6986)
(0.6438,0.7058)
(0.6502,0.7129)
(0.6567,0.7201)
(0.6632,0.7273)
(0.6696,0.7345)
(0.6761,0.7417)
(0.6826,0.7489)
(0.6890,0.7561)
(0.6955,0.7633)
(0.7019,0.7705)
(0.7084,0.7777)
(0.7149,0.7848)
(0.7213,0.7920)
(0.7278,0.7992)
(0.7343,0.8064)
(0.7407,0.8136)
(0.7472,0.8208)
(0.7536,0.8280)
(0.7601,0.8352)
(0.7666,0.8424)
(0.7730,0.8495)
(0.7795,0.8567)
(0.7860,0.8639)
(0.7924,0.8711)
(0.7989,0.8783)
(0.8054,0.8855)
(0.8118,0.8927)
(0.8183,0.8999)
(0.8247,0.9071)
(0.8312,0.9142)
(0.8377,0.9214)
(0.8441,0.9286)
(0.8506,0.9358)

\PST@Border(0.1270,0.9680)
(0.1270,0.1260)
(0.9470,0.1260)
(0.9470,0.9680)
(0.1270,0.9680)

\catcode`@=12
\fi
\endpspicture

\end{figure}
\begin{figure}[p]\caption{Interpolazione dei dati con inverso della pressione in ascissa e volume in ordinata. Temperatura di 55 \celsius, fase di espansione.}\label{55e}
% GNUPLOT: LaTeX picture using PSTRICKS macros
% Define new PST objects, if not already defined
\ifx\PSTloaded\undefined
\def\PSTloaded{t}
\psset{arrowsize=.01 3.2 1.4 .3}
\psset{dotsize=.04}
\catcode`@=11

\newpsobject{PST@Border}{psline}{linewidth=.0015,linestyle=solid}
\newpsobject{PST@Axes}{psline}{linewidth=.0015,linestyle=dotted,dotsep=.004}
\newpsobject{PST@Solid}{psline}{linewidth=.0015,linestyle=solid}
\newpsobject{PST@Dashed}{psline}{linewidth=.0015,linestyle=dashed,dash=.01 .01}
\newpsobject{PST@Dotted}{psline}{linewidth=.0025,linestyle=dotted,dotsep=.008}
\newpsobject{PST@LongDash}{psline}{linewidth=.0015,linestyle=dashed,dash=.02 .01}
\newpsobject{PST@Diamond}{psdots}{linewidth=.001,linestyle=solid,dotstyle=square,dotangle=45}
\newpsobject{PST@Filldiamond}{psdots}{linewidth=.001,linestyle=solid,dotstyle=square*,dotangle=45}
\newpsobject{PST@Cross}{psdots}{linewidth=.001,linestyle=solid,dotstyle=+,dotangle=45}
\newpsobject{PST@Plus}{psdots}{linewidth=.001,linestyle=solid,dotstyle=+}
\newpsobject{PST@Square}{psdots}{linewidth=.001,linestyle=solid,dotstyle=square}
\newpsobject{PST@Circle}{psdots}{linewidth=.001,linestyle=solid,dotstyle=o}
\newpsobject{PST@Triangle}{psdots}{linewidth=.001,linestyle=solid,dotstyle=triangle}
\newpsobject{PST@Pentagon}{psdots}{linewidth=.001,linestyle=solid,dotstyle=pentagon}
\newpsobject{PST@Fillsquare}{psdots}{linewidth=.001,linestyle=solid,dotstyle=square*}
\newpsobject{PST@Fillcircle}{psdots}{linewidth=.001,linestyle=solid,dotstyle=*}
\newpsobject{PST@Filltriangle}{psdots}{linewidth=.001,linestyle=solid,dotstyle=triangle*}
\newpsobject{PST@Fillpentagon}{psdots}{linewidth=.001,linestyle=solid,dotstyle=pentagon*}
\newpsobject{PST@Arrow}{psline}{linewidth=.001,linestyle=solid}
\catcode`@=12

\fi
\psset{unit=5.0in,xunit=5.0in,yunit=3.0in}
\pspicture(0.000000,0.000000)(1.000000,1.000000)
\ifx\nofigs\undefined
\catcode`@=11

\PST@Border(0.1270,0.1260)
(0.1420,0.1260)

\rput[r](0.1110,0.1260){4}
\PST@Border(0.1270,0.2196)
(0.1420,0.2196)

\rput[r](0.1110,0.2196){6}
\PST@Border(0.1270,0.3131)
(0.1420,0.3131)

\rput[r](0.1110,0.3131){8}
\PST@Border(0.1270,0.4067)
(0.1420,0.4067)

\rput[r](0.1110,0.4067){10}
\PST@Border(0.1270,0.5002)
(0.1420,0.5002)

\rput[r](0.1110,0.5002){12}
\PST@Border(0.1270,0.5938)
(0.1420,0.5938)

\rput[r](0.1110,0.5938){14}
\PST@Border(0.1270,0.6873)
(0.1420,0.6873)

\rput[r](0.1110,0.6873){16}
\PST@Border(0.1270,0.7809)
(0.1420,0.7809)

\rput[r](0.1110,0.7809){18}
\PST@Border(0.1270,0.8744)
(0.1420,0.8744)

\rput[r](0.1110,0.8744){20}
\PST@Border(0.1270,0.9680)
(0.1420,0.9680)

\rput[r](0.1110,0.9680){22}
\PST@Border(0.1270,0.1260)
(0.1270,0.1460)

\rput(0.1270,0.0840){2}
\PST@Border(0.2441,0.1260)
(0.2441,0.1460)

\rput(0.2441,0.0840){3}
\PST@Border(0.3613,0.1260)
(0.3613,0.1460)

\rput(0.3613,0.0840){4}
\PST@Border(0.4784,0.1260)
(0.4784,0.1460)

\rput(0.4784,0.0840){5}
\PST@Border(0.5956,0.1260)
(0.5956,0.1460)

\rput(0.5956,0.0840){6}
\PST@Border(0.7127,0.1260)
(0.7127,0.1460)

\rput(0.7127,0.0840){7}
\PST@Border(0.8299,0.1260)
(0.8299,0.1460)

\rput(0.8299,0.0840){8}
\PST@Border(0.9470,0.1260)
(0.9470,0.1460)

\rput(0.9470,0.0840){9}
\PST@Border(0.1270,0.9680)
(0.1270,0.1260)
(0.9470,0.1260)
(0.9470,0.9680)
(0.1270,0.9680)

\rput{L}(0.0420,0.5470){$V\ (\unit{cm^3})$}
\rput(0.5370,0.0210){$1/P\ (\unit{Pa^{-1}})$}
\PST@Diamond(0.8577,0.9446)
\PST@Diamond(0.8470,0.9329)
\PST@Diamond(0.8362,0.9212)
\PST@Diamond(0.8254,0.9086)
\PST@Diamond(0.8171,0.9002)
\PST@Diamond(0.8123,0.8941)
\PST@Diamond(0.8015,0.8815)
\PST@Diamond(0.7920,0.8707)
\PST@Diamond(0.7812,0.8595)
\PST@Diamond(0.7740,0.8506)
\PST@Diamond(0.7645,0.8408)
\PST@Diamond(0.7573,0.8333)
\PST@Diamond(0.7514,0.8258)
\PST@Diamond(0.7418,0.8160)
\PST@Diamond(0.7358,0.8085)
\PST@Diamond(0.7262,0.7982)
\PST@Diamond(0.7190,0.7912)
\PST@Diamond(0.7142,0.7856)
\PST@Diamond(0.7083,0.7785)
\PST@Diamond(0.7035,0.7734)
\PST@Diamond(0.6963,0.7655)
\PST@Diamond(0.6892,0.7570)
\PST@Diamond(0.6832,0.7505)
\PST@Diamond(0.6784,0.7458)
\PST@Diamond(0.6748,0.7416)
\PST@Diamond(0.6688,0.7350)
\PST@Diamond(0.6605,0.7257)
\PST@Diamond(0.6568,0.7210)
\PST@Diamond(0.6545,0.7182)
\PST@Diamond(0.6520,0.7159)
\PST@Diamond(0.6497,0.7126)
\PST@Diamond(0.6449,0.7089)
\PST@Diamond(0.6401,0.7032)
\PST@Diamond(0.6353,0.6981)
\PST@Diamond(0.6305,0.6925)
\PST@Diamond(0.6198,0.6817)
\PST@Diamond(0.6150,0.6756)
\PST@Diamond(0.6079,0.6677)
\PST@Diamond(0.6018,0.6611)
\PST@Diamond(0.5958,0.6546)
\PST@Diamond(0.5898,0.6471)
\PST@Diamond(0.5851,0.6424)
\PST@Diamond(0.5792,0.6359)
\PST@Diamond(0.5719,0.6275)
\PST@Diamond(0.5648,0.6200)
\PST@Diamond(0.5588,0.6130)
\PST@Diamond(0.5505,0.6041)
\PST@Diamond(0.5444,0.5961)
\PST@Diamond(0.5384,0.5905)
\PST@Diamond(0.5324,0.5840)
\PST@Diamond(0.5289,0.5793)
\PST@Diamond(0.5241,0.5746)
\PST@Diamond(0.5181,0.5680)
\PST@Diamond(0.5133,0.5634)
\PST@Diamond(0.5050,0.5540)
\PST@Diamond(0.4990,0.5470)
\PST@Diamond(0.4942,0.5414)
\PST@Diamond(0.4823,0.5292)
\PST@Diamond(0.4763,0.5227)
\PST@Diamond(0.4667,0.5119)
\PST@Diamond(0.4571,0.5016)
\PST@Diamond(0.4488,0.4932)
\PST@Diamond(0.4416,0.4843)
\PST@Diamond(0.4357,0.4778)
\PST@Diamond(0.4297,0.4717)
\PST@Diamond(0.4236,0.4642)
\PST@Diamond(0.4189,0.4591)
\PST@Diamond(0.4117,0.4511)
\PST@Diamond(0.4022,0.4394)
\PST@Diamond(0.3985,0.4361)
\PST@Diamond(0.3914,0.4291)
\PST@Diamond(0.3830,0.4188)
\PST@Diamond(0.3783,0.4137)
\PST@Diamond(0.3723,0.4081)
\PST@Diamond(0.3650,0.3997)
\PST@Diamond(0.3555,0.3896)
\PST@Diamond(0.3483,0.3815)
\PST@Diamond(0.3436,0.3755)
\PST@Diamond(0.3388,0.3699)
\PST@Diamond(0.3340,0.3654)
\PST@Diamond(0.3292,0.3605)
\PST@Diamond(0.3256,0.3550)
\PST@Diamond(0.3209,0.3495)
\PST@Diamond(0.3172,0.3458)
\PST@Diamond(0.3113,0.3401)
\PST@Diamond(0.3053,0.3327)
\PST@Diamond(0.3005,0.3264)
\PST@Diamond(0.2957,0.3219)
\PST@Diamond(0.2897,0.3144)
\PST@Diamond(0.2826,0.3067)
\PST@Diamond(0.2766,0.3007)
\PST@Diamond(0.2718,0.2950)
\PST@Diamond(0.2670,0.2902)
\PST@Diamond(0.2622,0.2841)
\PST@Diamond(0.2550,0.2766)
\PST@Diamond(0.2514,0.2717)
\PST@Diamond(0.2431,0.2629)
\PST@Diamond(0.2348,0.2543)
\PST@Diamond(0.2311,0.2498)
\PST@Diamond(0.2252,0.2429)
\PST@Diamond(0.2167,0.2342)
\PST@Diamond(0.2120,0.2284)
\PST@Diamond(0.2108,0.2255)
\PST@Diamond(0.2108,0.2198)
\PST@Diamond(0.2108,0.2166)
\PST@Dashed(0.2108,0.2273)
(0.2108,0.2273)
(0.2173,0.2345)
(0.2238,0.2417)
(0.2304,0.2490)
(0.2369,0.2562)
(0.2434,0.2635)
(0.2500,0.2707)
(0.2565,0.2780)
(0.2630,0.2852)
(0.2696,0.2925)
(0.2761,0.2997)
(0.2826,0.3070)
(0.2892,0.3142)
(0.2957,0.3215)
(0.3022,0.3287)
(0.3088,0.3360)
(0.3153,0.3432)
(0.3219,0.3504)
(0.3284,0.3577)
(0.3349,0.3649)
(0.3415,0.3722)
(0.3480,0.3794)
(0.3545,0.3867)
(0.3611,0.3939)
(0.3676,0.4012)
(0.3741,0.4084)
(0.3807,0.4157)
(0.3872,0.4229)
(0.3937,0.4302)
(0.4003,0.4374)
(0.4068,0.4447)
(0.4133,0.4519)
(0.4199,0.4591)
(0.4264,0.4664)
(0.4330,0.4736)
(0.4395,0.4809)
(0.4460,0.4881)
(0.4526,0.4954)
(0.4591,0.5026)
(0.4656,0.5099)
(0.4722,0.5171)
(0.4787,0.5244)
(0.4852,0.5316)
(0.4918,0.5389)
(0.4983,0.5461)
(0.5048,0.5534)
(0.5114,0.5606)
(0.5179,0.5678)
(0.5244,0.5751)
(0.5310,0.5823)
(0.5375,0.5896)
(0.5440,0.5968)
(0.5506,0.6041)
(0.5571,0.6113)
(0.5637,0.6186)
(0.5702,0.6258)
(0.5767,0.6331)
(0.5833,0.6403)
(0.5898,0.6476)
(0.5963,0.6548)
(0.6029,0.6621)
(0.6094,0.6693)
(0.6159,0.6766)
(0.6225,0.6838)
(0.6290,0.6910)
(0.6355,0.6983)
(0.6421,0.7055)
(0.6486,0.7128)
(0.6551,0.7200)
(0.6617,0.7273)
(0.6682,0.7345)
(0.6748,0.7418)
(0.6813,0.7490)
(0.6878,0.7563)
(0.6944,0.7635)
(0.7009,0.7708)
(0.7074,0.7780)
(0.7140,0.7853)
(0.7205,0.7925)
(0.7270,0.7997)
(0.7336,0.8070)
(0.7401,0.8142)
(0.7466,0.8215)
(0.7532,0.8287)
(0.7597,0.8360)
(0.7662,0.8432)
(0.7728,0.8505)
(0.7793,0.8577)
(0.7859,0.8650)
(0.7924,0.8722)
(0.7989,0.8795)
(0.8055,0.8867)
(0.8120,0.8940)
(0.8185,0.9012)
(0.8251,0.9084)
(0.8316,0.9157)
(0.8381,0.9229)
(0.8447,0.9302)
(0.8512,0.9374)
(0.8577,0.9447)

\PST@Border(0.1270,0.9680)
(0.1270,0.1260)
(0.9470,0.1260)
(0.9470,0.9680)
(0.1270,0.9680)

\catcode`@=12
\fi
\endpspicture

\end{figure}
\begin{figure}[p]\caption{Interpolazione dei dati con temperatura media in ascissa e pressione minima in ordinata. Fase di compressione.}\label{t-pmincomp}
% GNUPLOT: LaTeX picture using PSTRICKS macros
% Define new PST objects, if not already defined
\ifx\PSTloaded\undefined
\def\PSTloaded{t}
\psset{arrowsize=.01 3.2 1.4 .3}
\psset{dotsize=.08}
\catcode`@=11

\newpsobject{PST@Border}{psline}{linewidth=.0015,linestyle=solid}
\newpsobject{PST@Axes}{psline}{linewidth=.0015,linestyle=dotted,dotsep=.004}
\newpsobject{PST@Solid}{psline}{linewidth=.0015,linestyle=solid}
\newpsobject{PST@Dashed}{psline}{linewidth=.0015,linestyle=dashed,dash=.01 .01}
\newpsobject{PST@Dotted}{psline}{linewidth=.0025,linestyle=dotted,dotsep=.008}
\newpsobject{PST@LongDash}{psline}{linewidth=.0015,linestyle=dashed,dash=.02 .01}
\newpsobject{PST@Diamond}{psdots}{linewidth=.001,linestyle=solid,dotstyle=*}
\newpsobject{PST@Filldiamond}{psdots}{linewidth=.001,linestyle=solid,dotstyle=square*,dotangle=45}
\newpsobject{PST@Cross}{psdots}{linewidth=.001,linestyle=solid,dotstyle=+,dotangle=45}
\newpsobject{PST@Plus}{psdots}{linewidth=.001,linestyle=solid,dotstyle=+}
\newpsobject{PST@Square}{psdots}{linewidth=.001,linestyle=solid,dotstyle=square}
\newpsobject{PST@Circle}{psdots}{linewidth=.001,linestyle=solid,dotstyle=o}
\newpsobject{PST@Triangle}{psdots}{linewidth=.001,linestyle=solid,dotstyle=triangle}
\newpsobject{PST@Pentagon}{psdots}{linewidth=.001,linestyle=solid,dotstyle=pentagon}
\newpsobject{PST@Fillsquare}{psdots}{linewidth=.001,linestyle=solid,dotstyle=square*}
\newpsobject{PST@Fillcircle}{psdots}{linewidth=.001,linestyle=solid,dotstyle=*}
\newpsobject{PST@Filltriangle}{psdots}{linewidth=.001,linestyle=solid,dotstyle=triangle*}
\newpsobject{PST@Fillpentagon}{psdots}{linewidth=.001,linestyle=solid,dotstyle=pentagon*}
\newpsobject{PST@Arrow}{psline}{linewidth=.001,linestyle=solid}
\catcode`@=12

\fi
\psset{unit=5.0in,xunit=5.0in,yunit=3.0in}
\pspicture(0.000000,0.000000)(1.000000,1.000000)
\ifx\nofigs\undefined
\catcode`@=11

\PST@Border(0.1430,0.1260)
(0.1580,0.1260)

\rput[r](0.1270,0.1260){102}
\PST@Border(0.1430,0.2102)
(0.1580,0.2102)

\rput[r](0.1270,0.2102){104}
\PST@Border(0.1430,0.2944)
(0.1580,0.2944)

\rput[r](0.1270,0.2944){106}
\PST@Border(0.1430,0.3786)
(0.1580,0.3786)

\rput[r](0.1270,0.3786){108}
\PST@Border(0.1430,0.4628)
(0.1580,0.4628)

\rput[r](0.1270,0.4628){110}
\PST@Border(0.1430,0.5470)
(0.1580,0.5470)

\rput[r](0.1270,0.5470){112}
\PST@Border(0.1430,0.6312)
(0.1580,0.6312)

\rput[r](0.1270,0.6312){114}
\PST@Border(0.1430,0.7154)
(0.1580,0.7154)

\rput[r](0.1270,0.7154){116}
\PST@Border(0.1430,0.7996)
(0.1580,0.7996)

\rput[r](0.1270,0.7996){118}
\PST@Border(0.1430,0.8838)
(0.1580,0.8838)

\rput[r](0.1270,0.8838){120}
\PST@Border(0.1430,0.9680)
(0.1580,0.9680)

\rput[r](0.1270,0.9680){122}
\PST@Border(0.1430,0.1260)
(0.1430,0.1460)

\rput(0.1430,0.0840){0}
\PST@Border(0.2770,0.1260)
(0.2770,0.1460)

\rput(0.2770,0.0840){10}
\PST@Border(0.4110,0.1260)
(0.4110,0.1460)

\rput(0.4110,0.0840){20}
\PST@Border(0.5450,0.1260)
(0.5450,0.1460)

\rput(0.5450,0.0840){30}
\PST@Border(0.6790,0.1260)
(0.6790,0.1460)

\rput(0.6790,0.0840){40}
\PST@Border(0.8130,0.1260)
(0.8130,0.1460)

\rput(0.8130,0.0840){50}
\PST@Border(0.9470,0.1260)
(0.9470,0.1460)

\rput(0.9470,0.0840){60}
\PST@Border(0.1430,0.9680)
(0.1430,0.1260)
(0.9470,0.1260)
(0.9470,0.9680)
(0.1430,0.9680)

\rput{L}(0.0420,0.5470){$P_{min}\ (\unit{kPa})$}
\rput(0.5450,0.0210){T\ (\unit{\celsius})}
\PST@Diamond(0.1479,0.1546)
\PST@Diamond(0.3569,0.4039)
\PST@Diamond(0.4848,0.5399)
\PST@Diamond(0.6176,0.6609)
\PST@Diamond(0.7501,0.8005)
\PST@Diamond(0.8831,0.9423)
\PST@Dashed(0.1479,0.1697)
(0.1479,0.1697)
(0.1553,0.1775)
(0.1627,0.1853)
(0.1701,0.1932)
(0.1776,0.2010)
(0.1850,0.2089)
(0.1924,0.2167)
(0.1999,0.2245)
(0.2073,0.2324)
(0.2147,0.2402)
(0.2221,0.2481)
(0.2296,0.2559)
(0.2370,0.2637)
(0.2444,0.2716)
(0.2518,0.2794)
(0.2593,0.2873)
(0.2667,0.2951)
(0.2741,0.3029)
(0.2815,0.3108)
(0.2890,0.3186)
(0.2964,0.3265)
(0.3038,0.3343)
(0.3113,0.3421)
(0.3187,0.3500)
(0.3261,0.3578)
(0.3335,0.3657)
(0.3410,0.3735)
(0.3484,0.3813)
(0.3558,0.3892)
(0.3632,0.3970)
(0.3707,0.4049)
(0.3781,0.4127)
(0.3855,0.4205)
(0.3929,0.4284)
(0.4004,0.4362)
(0.4078,0.4441)
(0.4152,0.4519)
(0.4227,0.4597)
(0.4301,0.4676)
(0.4375,0.4754)
(0.4449,0.4832)
(0.4524,0.4911)
(0.4598,0.4989)
(0.4672,0.5068)
(0.4746,0.5146)
(0.4821,0.5224)
(0.4895,0.5303)
(0.4969,0.5381)
(0.5043,0.5460)
(0.5118,0.5538)
(0.5192,0.5616)
(0.5266,0.5695)
(0.5341,0.5773)
(0.5415,0.5852)
(0.5489,0.5930)
(0.5563,0.6008)
(0.5638,0.6087)
(0.5712,0.6165)
(0.5786,0.6244)
(0.5860,0.6322)
(0.5935,0.6400)
(0.6009,0.6479)
(0.6083,0.6557)
(0.6157,0.6636)
(0.6232,0.6714)
(0.6306,0.6792)
(0.6380,0.6871)
(0.6455,0.6949)
(0.6529,0.7028)
(0.6603,0.7106)
(0.6677,0.7184)
(0.6752,0.7263)
(0.6826,0.7341)
(0.6900,0.7420)
(0.6974,0.7498)
(0.7049,0.7576)
(0.7123,0.7655)
(0.7197,0.7733)
(0.7271,0.7812)
(0.7346,0.7890)
(0.7420,0.7968)
(0.7494,0.8047)
(0.7569,0.8125)
(0.7643,0.8204)
(0.7717,0.8282)
(0.7791,0.8360)
(0.7866,0.8439)
(0.7940,0.8517)
(0.8014,0.8596)
(0.8088,0.8674)
(0.8163,0.8752)
(0.8237,0.8831)
(0.8311,0.8909)
(0.8385,0.8988)
(0.8460,0.9066)
(0.8534,0.9144)
(0.8608,0.9223)
(0.8683,0.9301)
(0.8757,0.9380)
(0.8831,0.9458)

\PST@Border(0.1430,0.9680)
(0.1430,0.1260)
(0.9470,0.1260)
(0.9470,0.9680)
(0.1430,0.9680)

\catcode`@=12
\fi
\endpspicture

\end{figure}
\begin{figure}[p]\caption{Interpolazione dei dati con temperatura media in ascissa e pressione minima in ordinata. Fase di espansione.}\label{t-pminesp}
% GNUPLOT: LaTeX picture using PSTRICKS macros
% Define new PST objects, if not already defined
\ifx\PSTloaded\undefined
\def\PSTloaded{t}
\psset{arrowsize=.01 3.2 1.4 .3}
\psset{dotsize=.08}
\catcode`@=11

\newpsobject{PST@Border}{psline}{linewidth=.0015,linestyle=solid}
\newpsobject{PST@Axes}{psline}{linewidth=.0015,linestyle=dotted,dotsep=.004}
\newpsobject{PST@Solid}{psline}{linewidth=.0015,linestyle=solid}
\newpsobject{PST@Dashed}{psline}{linewidth=.0015,linestyle=dashed,dash=.01 .01}
\newpsobject{PST@Dotted}{psline}{linewidth=.0025,linestyle=dotted,dotsep=.008}
\newpsobject{PST@LongDash}{psline}{linewidth=.0015,linestyle=dashed,dash=.02 .01}
\newpsobject{PST@Diamond}{psdots}{linewidth=.001,linestyle=solid,dotstyle=*}
\newpsobject{PST@Filldiamond}{psdots}{linewidth=.001,linestyle=solid,dotstyle=square*,dotangle=45}
\newpsobject{PST@Cross}{psdots}{linewidth=.001,linestyle=solid,dotstyle=+,dotangle=45}
\newpsobject{PST@Plus}{psdots}{linewidth=.001,linestyle=solid,dotstyle=+}
\newpsobject{PST@Square}{psdots}{linewidth=.001,linestyle=solid,dotstyle=square}
\newpsobject{PST@Circle}{psdots}{linewidth=.001,linestyle=solid,dotstyle=o}
\newpsobject{PST@Triangle}{psdots}{linewidth=.001,linestyle=solid,dotstyle=triangle}
\newpsobject{PST@Pentagon}{psdots}{linewidth=.001,linestyle=solid,dotstyle=pentagon}
\newpsobject{PST@Fillsquare}{psdots}{linewidth=.001,linestyle=solid,dotstyle=square*}
\newpsobject{PST@Fillcircle}{psdots}{linewidth=.001,linestyle=solid,dotstyle=*}
\newpsobject{PST@Filltriangle}{psdots}{linewidth=.001,linestyle=solid,dotstyle=triangle*}
\newpsobject{PST@Fillpentagon}{psdots}{linewidth=.001,linestyle=solid,dotstyle=pentagon*}
\newpsobject{PST@Arrow}{psline}{linewidth=.001,linestyle=solid}
\catcode`@=12

\fi
\psset{unit=5.0in,xunit=5.0in,yunit=3.0in}
\pspicture(0.000000,0.000000)(1.000000,1.000000)
\ifx\nofigs\undefined
\catcode`@=11

\PST@Border(0.1430,0.1260)
(0.1580,0.1260)

\rput[r](0.1270,0.1260){102}
\PST@Border(0.1430,0.2102)
(0.1580,0.2102)

\rput[r](0.1270,0.2102){104}
\PST@Border(0.1430,0.2944)
(0.1580,0.2944)

\rput[r](0.1270,0.2944){106}
\PST@Border(0.1430,0.3786)
(0.1580,0.3786)

\rput[r](0.1270,0.3786){108}
\PST@Border(0.1430,0.4628)
(0.1580,0.4628)

\rput[r](0.1270,0.4628){110}
\PST@Border(0.1430,0.5470)
(0.1580,0.5470)

\rput[r](0.1270,0.5470){112}
\PST@Border(0.1430,0.6312)
(0.1580,0.6312)

\rput[r](0.1270,0.6312){114}
\PST@Border(0.1430,0.7154)
(0.1580,0.7154)

\rput[r](0.1270,0.7154){116}
\PST@Border(0.1430,0.7996)
(0.1580,0.7996)

\rput[r](0.1270,0.7996){118}
\PST@Border(0.1430,0.8838)
(0.1580,0.8838)

\rput[r](0.1270,0.8838){120}
\PST@Border(0.1430,0.9680)
(0.1580,0.9680)

\rput[r](0.1270,0.9680){122}
\PST@Border(0.1430,0.1260)
(0.1430,0.1460)

\rput(0.1430,0.0840){0}
\PST@Border(0.2770,0.1260)
(0.2770,0.1460)

\rput(0.2770,0.0840){10}
\PST@Border(0.4110,0.1260)
(0.4110,0.1460)

\rput(0.4110,0.0840){20}
\PST@Border(0.5450,0.1260)
(0.5450,0.1460)

\rput(0.5450,0.0840){30}
\PST@Border(0.6790,0.1260)
(0.6790,0.1460)

\rput(0.6790,0.0840){40}
\PST@Border(0.8130,0.1260)
(0.8130,0.1460)

\rput(0.8130,0.0840){50}
\PST@Border(0.9470,0.1260)
(0.9470,0.1460)

\rput(0.9470,0.0840){60}
\PST@Border(0.1430,0.9680)
(0.1430,0.1260)
(0.9470,0.1260)
(0.9470,0.9680)
(0.1430,0.9680)

\rput{L}(0.0420,0.5470){$P_{min}\ (\unit{kPa})$}
\rput(0.5450,0.0210){T\ (\unit{\celsius})}
\PST@Diamond(0.1479,0.1546)
\PST@Diamond(0.3569,0.4039)
\PST@Diamond(0.4848,0.5452)
\PST@Diamond(0.6176,0.6664)
\PST@Diamond(0.7501,0.8064)
\PST@Diamond(0.8831,0.9423)
\PST@Dashed(0.1479,0.1709)
(0.1479,0.1709)
(0.1553,0.1788)
(0.1627,0.1867)
(0.1701,0.1945)
(0.1776,0.2024)
(0.1850,0.2103)
(0.1924,0.2181)
(0.1999,0.2260)
(0.2073,0.2339)
(0.2147,0.2418)
(0.2221,0.2496)
(0.2296,0.2575)
(0.2370,0.2654)
(0.2444,0.2732)
(0.2518,0.2811)
(0.2593,0.2890)
(0.2667,0.2968)
(0.2741,0.3047)
(0.2815,0.3126)
(0.2890,0.3204)
(0.2964,0.3283)
(0.3038,0.3362)
(0.3113,0.3440)
(0.3187,0.3519)
(0.3261,0.3598)
(0.3335,0.3676)
(0.3410,0.3755)
(0.3484,0.3834)
(0.3558,0.3912)
(0.3632,0.3991)
(0.3707,0.4070)
(0.3781,0.4149)
(0.3855,0.4227)
(0.3929,0.4306)
(0.4004,0.4385)
(0.4078,0.4463)
(0.4152,0.4542)
(0.4227,0.4621)
(0.4301,0.4699)
(0.4375,0.4778)
(0.4449,0.4857)
(0.4524,0.4935)
(0.4598,0.5014)
(0.4672,0.5093)
(0.4746,0.5171)
(0.4821,0.5250)
(0.4895,0.5329)
(0.4969,0.5407)
(0.5043,0.5486)
(0.5118,0.5565)
(0.5192,0.5643)
(0.5266,0.5722)
(0.5341,0.5801)
(0.5415,0.5880)
(0.5489,0.5958)
(0.5563,0.6037)
(0.5638,0.6116)
(0.5712,0.6194)
(0.5786,0.6273)
(0.5860,0.6352)
(0.5935,0.6430)
(0.6009,0.6509)
(0.6083,0.6588)
(0.6157,0.6666)
(0.6232,0.6745)
(0.6306,0.6824)
(0.6380,0.6902)
(0.6455,0.6981)
(0.6529,0.7060)
(0.6603,0.7138)
(0.6677,0.7217)
(0.6752,0.7296)
(0.6826,0.7374)
(0.6900,0.7453)
(0.6974,0.7532)
(0.7049,0.7611)
(0.7123,0.7689)
(0.7197,0.7768)
(0.7271,0.7847)
(0.7346,0.7925)
(0.7420,0.8004)
(0.7494,0.8083)
(0.7569,0.8161)
(0.7643,0.8240)
(0.7717,0.8319)
(0.7791,0.8397)
(0.7866,0.8476)
(0.7940,0.8555)
(0.8014,0.8633)
(0.8088,0.8712)
(0.8163,0.8791)
(0.8237,0.8869)
(0.8311,0.8948)
(0.8385,0.9027)
(0.8460,0.9105)
(0.8534,0.9184)
(0.8608,0.9263)
(0.8683,0.9342)
(0.8757,0.9420)
(0.8831,0.9499)

\PST@Border(0.1430,0.9680)
(0.1430,0.1260)
(0.9470,0.1260)
(0.9470,0.9680)
(0.1430,0.9680)

\catcode`@=12
\fi
\endpspicture

\end{figure}
\begin{figure}[p]\caption{Interpolazione dei dati con $nRT$ in ascissa e temperatura media in ordinata.}\label{t-pv}
% GNUPLOT: LaTeX picture using PSTRICKS macros
% Define new PST objects, if not already defined
\ifx\PSTloaded\undefined
\def\PSTloaded{t}
\psset{arrowsize=.01 3.2 1.4 .3}
\psset{dotsize=.08}
\catcode`@=11

\newpsobject{PST@Border}{psline}{linewidth=.0015,linestyle=solid}
\newpsobject{PST@Axes}{psline}{linewidth=.0015,linestyle=dotted,dotsep=.004}
\newpsobject{PST@Solid}{psline}{linewidth=.0015,linestyle=solid}
\newpsobject{PST@Dashed}{psline}{linewidth=.0015,linestyle=dashed,dash=.01 .01}
\newpsobject{PST@Dotted}{psline}{linewidth=.0025,linestyle=dotted,dotsep=.008}
\newpsobject{PST@LongDash}{psline}{linewidth=.0015,linestyle=dashed,dash=.02 .01}
\newpsobject{PST@Diamond}{psdots}{linewidth=.001,linestyle=solid,dotstyle=*}
\newpsobject{PST@Filldiamond}{psdots}{linewidth=.001,linestyle=solid,dotstyle=square*,dotangle=45}
\newpsobject{PST@Cross}{psdots}{linewidth=.001,linestyle=solid,dotstyle=+,dotangle=45}
\newpsobject{PST@Plus}{psdots}{linewidth=.001,linestyle=solid,dotstyle=+}
\newpsobject{PST@Square}{psdots}{linewidth=.001,linestyle=solid,dotstyle=square}
\newpsobject{PST@Circle}{psdots}{linewidth=.001,linestyle=solid,dotstyle=o}
\newpsobject{PST@Triangle}{psdots}{linewidth=.001,linestyle=solid,dotstyle=triangle}
\newpsobject{PST@Pentagon}{psdots}{linewidth=.001,linestyle=solid,dotstyle=pentagon}
\newpsobject{PST@Fillsquare}{psdots}{linewidth=.001,linestyle=solid,dotstyle=square*}
\newpsobject{PST@Fillcircle}{psdots}{linewidth=.001,linestyle=solid,dotstyle=*}
\newpsobject{PST@Filltriangle}{psdots}{linewidth=.001,linestyle=solid,dotstyle=triangle*}
\newpsobject{PST@Fillpentagon}{psdots}{linewidth=.001,linestyle=solid,dotstyle=pentagon*}
\newpsobject{PST@Arrow}{psline}{linewidth=.001,linestyle=solid}
\catcode`@=12

\fi
\psset{unit=5.0in,xunit=5.0in,yunit=3.0in}
\pspicture(0.000000,0.000000)(1.000000,1.000000)
\ifx\nofigs\undefined
\catcode`@=11

\PST@Border(0.1430,0.1260)
(0.1580,0.1260)

\rput[r](0.1270,0.1260){-10}
\PST@Border(0.1430,0.2463)
(0.1580,0.2463)

\rput[r](0.1270,0.2463){0}
\PST@Border(0.1430,0.3666)
(0.1580,0.3666)

\rput[r](0.1270,0.3666){10}
\PST@Border(0.1430,0.4869)
(0.1580,0.4869)

\rput[r](0.1270,0.4869){20}
\PST@Border(0.1430,0.6071)
(0.1580,0.6071)

\rput[r](0.1270,0.6071){30}
\PST@Border(0.1430,0.7274)
(0.1580,0.7274)

\rput[r](0.1270,0.7274){40}
\PST@Border(0.1430,0.8477)
(0.1580,0.8477)

\rput[r](0.1270,0.8477){50}
\PST@Border(0.1430,0.9680)
(0.1580,0.9680)

\rput[r](0.1270,0.9680){60}
\PST@Border(0.1430,0.1260)
(0.1430,0.1460)

\rput(0.1430,0.0840){2.30}
\PST@Border(0.2234,0.1260)
(0.2234,0.1460)

\rput(0.2234,0.0840){2.35}
\PST@Border(0.3038,0.1260)
(0.3038,0.1460)

\rput(0.3038,0.0840){2.40}
\PST@Border(0.3842,0.1260)
(0.3842,0.1460)

\rput(0.3842,0.0840){2.45}
\PST@Border(0.4646,0.1260)
(0.4646,0.1460)

\rput(0.4646,0.0840){2.50}
\PST@Border(0.5450,0.1260)
(0.5450,0.1460)

\rput(0.5450,0.0840){2.55}
\PST@Border(0.6254,0.1260)
(0.6254,0.1460)

\rput(0.6254,0.0840){2.60}
\PST@Border(0.7058,0.1260)
(0.7058,0.1460)

\rput(0.7058,0.0840){2.65}
\PST@Border(0.7862,0.1260)
(0.7862,0.1460)

\rput(0.7862,0.0840){2.70}
\PST@Border(0.8666,0.1260)
(0.8666,0.1460)

\rput(0.8666,0.0840){2.75}
\PST@Border(0.9470,0.1260)
(0.9470,0.1460)

\rput(0.9470,0.0840){2.80}
\PST@Border(0.1430,0.9680)
(0.1430,0.1260)
(0.9470,0.1260)
(0.9470,0.9680)
(0.1430,0.9680)

\rput{L}(0.0420,0.5470){$T\ (\unit{\celsius})$}
\rput(0.5450,0.0210){nRT\ (\unit{J})}
\PST@Diamond(0.1686,0.2507)
\PST@Diamond(0.3921,0.4383)
\PST@Diamond(0.5234,0.5531)
\PST@Diamond(0.6378,0.6723)
\PST@Diamond(0.7756,0.7913)
\PST@Diamond(0.9186,0.9106)
\PST@Dashed(0.1686,0.2457)
(0.1686,0.2457)
(0.1762,0.2524)
(0.1838,0.2592)
(0.1913,0.2659)
(0.1989,0.2727)
(0.2065,0.2794)
(0.2141,0.2862)
(0.2216,0.2929)
(0.2292,0.2997)
(0.2368,0.3064)
(0.2444,0.3132)
(0.2519,0.3199)
(0.2595,0.3267)
(0.2671,0.3334)
(0.2747,0.3402)
(0.2822,0.3469)
(0.2898,0.3537)
(0.2974,0.3604)
(0.3050,0.3672)
(0.3125,0.3739)
(0.3201,0.3807)
(0.3277,0.3874)
(0.3353,0.3942)
(0.3429,0.4009)
(0.3504,0.4077)
(0.3580,0.4144)
(0.3656,0.4212)
(0.3732,0.4279)
(0.3807,0.4347)
(0.3883,0.4414)
(0.3959,0.4482)
(0.4035,0.4549)
(0.4110,0.4617)
(0.4186,0.4684)
(0.4262,0.4752)
(0.4338,0.4819)
(0.4413,0.4887)
(0.4489,0.4954)
(0.4565,0.5022)
(0.4641,0.5089)
(0.4716,0.5157)
(0.4792,0.5224)
(0.4868,0.5292)
(0.4944,0.5359)
(0.5019,0.5427)
(0.5095,0.5494)
(0.5171,0.5562)
(0.5247,0.5629)
(0.5322,0.5697)
(0.5398,0.5764)
(0.5474,0.5832)
(0.5550,0.5899)
(0.5625,0.5967)
(0.5701,0.6034)
(0.5777,0.6102)
(0.5853,0.6169)
(0.5929,0.6237)
(0.6004,0.6304)
(0.6080,0.6372)
(0.6156,0.6439)
(0.6232,0.6507)
(0.6307,0.6574)
(0.6383,0.6642)
(0.6459,0.6709)
(0.6535,0.6777)
(0.6610,0.6844)
(0.6686,0.6912)
(0.6762,0.6979)
(0.6838,0.7047)
(0.6913,0.7114)
(0.6989,0.7182)
(0.7065,0.7249)
(0.7141,0.7317)
(0.7216,0.7384)
(0.7292,0.7452)
(0.7368,0.7519)
(0.7444,0.7587)
(0.7519,0.7654)
(0.7595,0.7722)
(0.7671,0.7789)
(0.7747,0.7857)
(0.7822,0.7924)
(0.7898,0.7992)
(0.7974,0.8059)
(0.8050,0.8127)
(0.8126,0.8194)
(0.8201,0.8262)
(0.8277,0.8329)
(0.8353,0.8397)
(0.8429,0.8464)
(0.8504,0.8532)
(0.8580,0.8599)
(0.8656,0.8667)
(0.8732,0.8734)
(0.8807,0.8802)
(0.8883,0.8869)
(0.8959,0.8937)
(0.9035,0.9004)
(0.9110,0.9072)
(0.9186,0.9139)

\PST@Border(0.1430,0.9680)
(0.1430,0.1260)
(0.9470,0.1260)
(0.9470,0.9680)
(0.1430,0.9680)

\catcode`@=12
\fi
\endpspicture

\end{figure}
\begin{figure}[p]\caption{Valori di $V_0$ (medie pesate in espansione e compressione) rispetto alla temperatura media. Si vede che il volume dei tubicini aumenta sensibilmente con la temperatura.}\label{t-v0}
% GNUPLOT: LaTeX picture using PSTRICKS macros
% Define new PST objects, if not already defined
\ifx\PSTloaded\undefined
\def\PSTloaded{t}
\psset{arrowsize=.01 3.2 1.4 .3}
\psset{dotsize=.08}
\catcode`@=11

\newpsobject{PST@Border}{psline}{linewidth=.0015,linestyle=solid}
\newpsobject{PST@Axes}{psline}{linewidth=.0015,linestyle=dotted,dotsep=.004}
\newpsobject{PST@Solid}{psline}{linewidth=.0015,linestyle=solid}
\newpsobject{PST@Dashed}{psline}{linewidth=.0015,linestyle=dashed,dash=.01 .01}
\newpsobject{PST@Dotted}{psline}{linewidth=.0025,linestyle=dotted,dotsep=.008}
\newpsobject{PST@LongDash}{psline}{linewidth=.0015,linestyle=dashed,dash=.02 .01}
\newpsobject{PST@Diamond}{psdots}{linewidth=.001,linestyle=solid,dotstyle=*}
\newpsobject{PST@Filldiamond}{psdots}{linewidth=.001,linestyle=solid,dotstyle=square*,dotangle=45}
\newpsobject{PST@Cross}{psdots}{linewidth=.001,linestyle=solid,dotstyle=+,dotangle=45}
\newpsobject{PST@Plus}{psdots}{linewidth=.001,linestyle=solid,dotstyle=+}
\newpsobject{PST@Square}{psdots}{linewidth=.001,linestyle=solid,dotstyle=square}
\newpsobject{PST@Circle}{psdots}{linewidth=.001,linestyle=solid,dotstyle=o}
\newpsobject{PST@Triangle}{psdots}{linewidth=.001,linestyle=solid,dotstyle=triangle}
\newpsobject{PST@Pentagon}{psdots}{linewidth=.001,linestyle=solid,dotstyle=pentagon}
\newpsobject{PST@Fillsquare}{psdots}{linewidth=.001,linestyle=solid,dotstyle=square*}
\newpsobject{PST@Fillcircle}{psdots}{linewidth=.001,linestyle=solid,dotstyle=*}
\newpsobject{PST@Filltriangle}{psdots}{linewidth=.001,linestyle=solid,dotstyle=triangle*}
\newpsobject{PST@Fillpentagon}{psdots}{linewidth=.001,linestyle=solid,dotstyle=pentagon*}
\newpsobject{PST@Arrow}{psline}{linewidth=.001,linestyle=solid}
\catcode`@=12

\fi
\psset{unit=5.0in,xunit=5.0in,yunit=3.0in}
\pspicture(0.000000,0.000000)(1.000000,1.000000)
\ifx\nofigs\undefined
\catcode`@=11

\PST@Border(0.1590,0.1260)
(0.1740,0.1260)

\rput[r](0.1430,0.1260){1.00}
\PST@Border(0.1590,0.2196)
(0.1740,0.2196)

\rput[r](0.1430,0.2196){1.05}
\PST@Border(0.1590,0.3131)
(0.1740,0.3131)

\rput[r](0.1430,0.3131){1.10}
\PST@Border(0.1590,0.4067)
(0.1740,0.4067)

\rput[r](0.1430,0.4067){1.15}
\PST@Border(0.1590,0.5002)
(0.1740,0.5002)

\rput[r](0.1430,0.5002){1.20}
\PST@Border(0.1590,0.5938)
(0.1740,0.5938)

\rput[r](0.1430,0.5938){1.25}
\PST@Border(0.1590,0.6873)
(0.1740,0.6873)

\rput[r](0.1430,0.6873){1.30}
\PST@Border(0.1590,0.7809)
(0.1740,0.7809)

\rput[r](0.1430,0.7809){1.35}
\PST@Border(0.1590,0.8744)
(0.1740,0.8744)

\rput[r](0.1430,0.8744){1.40}
\PST@Border(0.1590,0.9680)
(0.1740,0.9680)

\rput[r](0.1430,0.9680){1.45}
\PST@Border(0.1590,0.1260)
(0.1590,0.1460)

\rput(0.1590,0.0840){0}
\PST@Border(0.2903,0.1260)
(0.2903,0.1460)

\rput(0.2903,0.0840){10}
\PST@Border(0.4217,0.1260)
(0.4217,0.1460)

\rput(0.4217,0.0840){20}
\PST@Border(0.5530,0.1260)
(0.5530,0.1460)

\rput(0.5530,0.0840){30}
\PST@Border(0.6843,0.1260)
(0.6843,0.1460)

\rput(0.6843,0.0840){40}
\PST@Border(0.8157,0.1260)
(0.8157,0.1460)

\rput(0.8157,0.0840){50}
\PST@Border(0.9470,0.1260)
(0.9470,0.1460)

\rput(0.9470,0.0840){60}
\PST@Border(0.1590,0.9680)
(0.1590,0.1260)
(0.9470,0.1260)
(0.9470,0.9680)
(0.1590,0.9680)

\rput{L}(0.0420,0.5470){$V_0\ (\unit{cm^3})$}
\rput(0.5530,0.0210){T (\celsius)}
\PST@Diamond(0.1638,0.1990)
\PST@Diamond(0.3687,0.3612)
\PST@Diamond(0.4940,0.4970)
\PST@Diamond(0.6242,0.5289)
\PST@Diamond(0.7541,0.7354)
\PST@Diamond(0.8844,0.9319)
\PST@Border(0.1590,0.9680)
(0.1590,0.1260)
(0.9470,0.1260)
(0.9470,0.9680)
(0.1590,0.9680)

\catcode`@=12
\fi
\endpspicture

\end{figure}
\end{document}
