\documentclass[italian,a4paper]{article}
\usepackage[tight,nice]{units}
\usepackage{babel,amsmath,amssymb,amsthm,graphicx,url,gensymb}
\usepackage[text={5.5in,9in},centering]{geometry}
\usepackage[utf8x]{inputenc}
\usepackage[T1]{fontenc}
\usepackage{ae,aecompl}
\usepackage[footnotesize,bf]{caption}
\usepackage[usenames]{color}
\include{pstricks}
\frenchspacing
\pagestyle{plain}
%------------- eliminare prime e ultime linee isolate
\clubpenalty=9999%
\widowpenalty=9999
%--- definizione numerazioni
\renewcommand{\theequation}{\thesection.\arabic{equation}}
\renewcommand{\thefigure}{\arabic{figure}}
\renewcommand{\thetable}{\thesection.\arabic{table}}
\addto\captionsitalian{%
  \renewcommand{\figurename}%
{Grafico}%
}
%
%------------- ridefinizione simbolo per elenchi puntati: en dash
%\renewcommand{\labelitemi}{\textbf{--}}
\renewcommand{\labelenumi}{\textbf{\arabic{enumi}.}}
\setlength{\abovecaptionskip}{\baselineskip}   % 0.5cm as an example
\setlength{\floatsep}{2\baselineskip}
\setlength{\belowcaptionskip}{\baselineskip}   % 0.5cm as an example
%--------- comandi insiemi numeri complessi, naturali, reali e altre abbreviazioni
\renewcommand{\leq}{\leqslant}
%--------- porzione dedicata ai float in una pagina:
\renewcommand{\textfraction}{0.05}
\renewcommand{\topfraction}{0.95}
\renewcommand{\bottomfraction}{0.95}
\renewcommand{\floatpagefraction}{0.35}
\setcounter{totalnumber}{5}
%---------
%
%---------
\begin{document}
\title{Relazione di laboratorio: il calorimetro}
\author{\normalsize Ilaria Brivio (582116)\\%
\normalsize \url{brivio.ilaria@tiscali.it}%
\and %
\normalsize Matteo Abis (584206)\\ %
\normalsize \url{webmaster@latinblog.org}}
\date{\today}
\maketitle
%------------------
\section{Obiettivo dell'esperienza}
Obiettivo dell'esperienza è misurare il calore specifico di un corpo di materiale sconosciuto.
\section{Descrizione dell'apparato strumentale}
Lo strumento impiegato è un calorimetro delle mescolanze di Regnault. Il contenitore esterno isola un cilindro interno in rame stagnato (calore specifico $c_r = \unitfrac[0.093\pm0.001]{cal}{g\celsius}$) che appoggia sul fondo sostenuto da punte di plastica. Nel cilindro si versa poi acqua distillata ($c_a = \unitfrac[1]{cal}{g\celsius}$ per definizione). Attraverso il coperchio si possono introdurre un termometro con sensibilità $\unit[10]{\celsius^{-1}}$ e un agitatore di rame stagnato che permette di mescolare l'acqua all'interno per rendere il più possibile omogenea la temperatura. Nel calorimetro si introduce infine un cilindro di metallo precedentemente riscaldato a una temperatura  di $\unit[100.0\pm0.1]{\celsius}$ con una resistenza. Lo scambio termico avviene tra il sistema campione e l'ambiente, dove per ambiente si intende l'acqua, il contenitore dell'acqua, l'agitatore e il termometro. Il termometro ha capacità termica $k=\unitfrac[2.20\pm0.05]{cal}{\celsius}$. Per pesare il corpo e l'acqua introdotta nel contenitore si usa una bilancia elettronica di sensibilità $\unit[100]{g^{-1}}$, come errore statistico si usa un terzo di questo valore, così come per il termometro.
\section{Descrizione della metodologia di misura}
Si stimano innanzitutto le masse degli strumenti impiegati con una pesata semplice di contenitore e agitatore prima ($m_r$), poi del contenitore e agitatore con acqua ($m_t$) e infine del cilindro campione ($m_x$). La massa dell'acqua $m_a$ risulta per differenza e l'errore è associato per propagazione. Dalle misure, in grammi, risultano:
\begin{table}[h]\centering
\begin{tabular}{r@{$=$}r@{$\pm$}ll}
 $m_r$ & 60.490 &0.003\\
 $m_t$ & 190.350&0.003\\
 $m_x$ & 81.500&0.003\\
 $m_a$ & 129.860&0.005
\end{tabular}
\end{table}\\
Il campione è stato riscaldato fino a una temperatura di $\unit[100.0\pm0.1]{\celsius}$ e poi trasferito rapidamente nel calorimetro. Tuttavia si stima che nel tragitto la sua temperatura sia diminuita di $\unit[1.5\pm1.0]{\celsius}$. Come temperatura iniziale si considera perciò $T_{c,i} = \unit[98.5\pm1.0]{\celsius}$. Prima di immergere il cilindro nell'acqua, si è misurata la temperatura del sistema $T_A = \unit[20.40\pm0.03]{\celsius}$.
Dall'inserimento del campione, è stata registrata la temperatura dell'acqua a intervalli di quindici secondi fino al raggiungimento di un massimo e ogni trenta secondi dopo il massimo fino a una differenza di temperatura di \unit[0.2]{\celsius}.
\section{Risultati sperimentali ed elaborazione dati}
\begin{table}[t]\centering
\begin{tabular}{rlrl }
15 &20.60 &30 &21.45\\
45 &22.45 &60 &23.40\\
75 &24.90 &90 &25.30\\
105 &25.40 &120 &25.50\\
135 &25.50 &150 &25.50\\
165 &25.48 &180 &25.45\\
195 &25.44 &225 &25.40\\
255 &25.40 &285 &25.40\\
315 &25.38 &345 &25.35\\
375 &25.32 &405 &25.30\\
435 &25.30 &465 &25.29\\
495 &25.28 &525 &25.26\\
555 &25.24 &585 &25.22
\end{tabular}
\end{table}
Disponendo la temperatura letta dal termometro in funzione del tempo (grafico~\ref{temp}) si vede che questa ha un transiente iniziale di rapida crescita e poi una diminuzione pressoché lineare dovuta alla dispersione di calore verso l'esterno. Interpolando i dati dopo il massimo con una retta si può ricavare un'intercetta che sarà la miglior stima della temperatura di equilibrio $T_f$. Tale intercetta risulta $T_f=\unit[25.564\pm0.009]{\celsius}$.Come temperatura di equilibrio $T_f$ si è usata la stima fornita dall'interpolazione piuttosto che il massimo valore osservato perché quest'ultimo è affetto da un errore maggiore a causa della dispersione del calore e delle disomogeneità nella temperatura del liquido.
\begin{figure}[t]\caption{Temperatura rilevata in ordinata, tempo in ascissa.}\label{temp}
\centering 
% GNUPLOT: LaTeX picture using PSTRICKS macros
% Define new PST objects, if not already defined
\ifx\PSTloaded\undefined
\def\PSTloaded{t}
\psset{arrowsize=.01 3.2 1.4 .3}
\psset{dotsize=.08}
\catcode`@=11

\newpsobject{PST@Border}{psline}{linewidth=.0015,linestyle=solid}
\newpsobject{PST@Axes}{psline}{linewidth=.0015,linestyle=dotted,dotsep=.004}
\newpsobject{PST@Solid}{psline}{linewidth=.0015,linestyle=solid}
\newpsobject{PST@Dashed}{psline}{linewidth=.0015,linestyle=dashed,dash=.01 .01}
\newpsobject{PST@Dotted}{psline}{linewidth=.0025,linestyle=dotted,dotsep=.008}
\newpsobject{PST@LongDash}{psline}{linewidth=.0015,linestyle=dashed,dash=.02 .01}
\newpsobject{PST@Diamond}{psdots}{linewidth=.001,linestyle=solid,dotstyle=square,dotangle=45}
\newpsobject{PST@Filldiamond}{psdots}{linewidth=.001,linestyle=solid,dotstyle=square*,dotangle=45}
\newpsobject{PST@Cross}{psdots}{linewidth=.001,linestyle=solid,dotstyle=+,dotangle=45}
\newpsobject{PST@Plus}{psdots}{linewidth=.001,linestyle=solid,dotstyle=+}
\newpsobject{PST@Square}{psdots}{linewidth=.001,linestyle=solid,dotstyle=square}
\newpsobject{PST@Circle}{psdots}{linewidth=.001,linestyle=solid,dotstyle=o}
\newpsobject{PST@Triangle}{psdots}{linewidth=.001,linestyle=solid,dotstyle=triangle}
\newpsobject{PST@Pentagon}{psdots}{linewidth=.001,linestyle=solid,dotstyle=pentagon}
\newpsobject{PST@Fillsquare}{psdots}{linewidth=.001,linestyle=solid,dotstyle=square*}
\newpsobject{PST@Fillcircle}{psdots}{linewidth=.001,linestyle=solid,dotstyle=*}
\newpsobject{PST@Filltriangle}{psdots}{linewidth=.001,linestyle=solid,dotstyle=triangle*}
\newpsobject{PST@Fillpentagon}{psdots}{linewidth=.001,linestyle=solid,dotstyle=pentagon*}
\newpsobject{PST@Arrow}{psline}{linewidth=.001,linestyle=solid}
\catcode`@=12

\fi
\psset{unit=5.0in,xunit=5.0in,yunit=3.0in}
\pspicture(0.000000,0.000000)(1.000000,1.000000)
\ifx\nofigs\undefined
\catcode`@=11

\PST@Border(0.1170,0.0840)
(0.1320,0.0840)

\rput[r](0.1010,0.0840){0.00}
\PST@Border(0.1170,0.1724)
(0.1320,0.1724)

\rput[r](0.1010,0.1724){0.05}
\PST@Border(0.1170,0.2608)
(0.1320,0.2608)

\rput[r](0.1010,0.2608){0.10}
\PST@Border(0.1170,0.3492)
(0.1320,0.3492)

\rput[r](0.1010,0.3492){0.15}
\PST@Border(0.1170,0.4376)
(0.1320,0.4376)

\rput[r](0.1010,0.4376){0.20}
\PST@Border(0.1170,0.5260)
(0.1320,0.5260)

\rput[r](0.1010,0.5260){0.25}
\PST@Border(0.1170,0.6144)
(0.1320,0.6144)

\rput[r](0.1010,0.6144){0.30}
\PST@Border(0.1170,0.7028)
(0.1320,0.7028)

\rput[r](0.1010,0.7028){0.35}
\PST@Border(0.1170,0.7912)
(0.1320,0.7912)

\rput[r](0.1010,0.7912){0.40}
\PST@Border(0.1170,0.8796)
(0.1320,0.8796)

\rput[r](0.1010,0.8796){0.45}
\PST@Border(0.1170,0.9680)
(0.1320,0.9680)

\rput[r](0.1010,0.9680){0.50}
\PST@Border(0.1170,0.0840)
(0.1170,0.1040)

\rput(0.1170,0.0420){0.00}
\PST@Border(0.2104,0.0840)
(0.2104,0.1040)

\rput(0.2104,0.0420){0.05}
\PST@Border(0.3039,0.0840)
(0.3039,0.1040)

\rput(0.3039,0.0420){0.10}
\PST@Border(0.3973,0.0840)
(0.3973,0.1040)

\rput(0.3973,0.0420){0.15}
\PST@Border(0.4908,0.0840)
(0.4908,0.1040)

\rput(0.4908,0.0420){0.20}
\PST@Border(0.5842,0.0840)
(0.5842,0.1040)

\rput(0.5842,0.0420){0.25}
\PST@Border(0.6777,0.0840)
(0.6777,0.1040)

\rput(0.6777,0.0420){0.30}
\PST@Border(0.7711,0.0840)
(0.7711,0.1040)

\rput(0.7711,0.0420){0.35}
\PST@Border(0.8646,0.0840)
(0.8646,0.1040)

\rput(0.8646,0.0420){0.40}
\PST@Border(0.9580,0.0840)
(0.9580,0.1040)

\rput(0.9580,0.0420){0.45}
\PST@Border(0.1170,0.9680)
(0.1170,0.0840)
(0.9580,0.0840)
(0.9580,0.9680)
(0.1170,0.9680)

\PST@Solid(0.1357,0.0985)
(0.1357,0.0985)
(0.1440,0.1064)
(0.1523,0.1143)
(0.1606,0.1222)
(0.1689,0.1302)
(0.1772,0.1381)
(0.1855,0.1460)
(0.1938,0.1539)
(0.2021,0.1619)
(0.2104,0.1698)
(0.2188,0.1777)
(0.2271,0.1856)
(0.2354,0.1936)
(0.2437,0.2015)
(0.2520,0.2094)
(0.2603,0.2173)
(0.2686,0.2253)
(0.2769,0.2332)
(0.2852,0.2411)
(0.2935,0.2490)
(0.3018,0.2570)
(0.3101,0.2649)
(0.3184,0.2728)
(0.3267,0.2807)
(0.3350,0.2887)
(0.3433,0.2966)
(0.3516,0.3045)
(0.3600,0.3124)
(0.3683,0.3204)
(0.3766,0.3283)
(0.3849,0.3362)
(0.3932,0.3441)
(0.4015,0.3521)
(0.4098,0.3600)
(0.4181,0.3679)
(0.4264,0.3758)
(0.4347,0.3837)
(0.4430,0.3917)
(0.4513,0.3996)
(0.4596,0.4075)
(0.4679,0.4154)
(0.4762,0.4234)
(0.4845,0.4313)
(0.4929,0.4392)
(0.5012,0.4471)
(0.5095,0.4551)
(0.5178,0.4630)
(0.5261,0.4709)
(0.5344,0.4788)
(0.5427,0.4868)
(0.5510,0.4947)
(0.5593,0.5026)
(0.5676,0.5105)
(0.5759,0.5185)
(0.5842,0.5264)
(0.5925,0.5343)
(0.6008,0.5422)
(0.6091,0.5502)
(0.6174,0.5581)
(0.6258,0.5660)
(0.6341,0.5739)
(0.6424,0.5819)
(0.6507,0.5898)
(0.6590,0.5977)
(0.6673,0.6056)
(0.6756,0.6136)
(0.6839,0.6215)
(0.6922,0.6294)
(0.7005,0.6373)
(0.7088,0.6453)
(0.7171,0.6532)
(0.7254,0.6611)
(0.7337,0.6690)
(0.7420,0.6770)
(0.7503,0.6849)
(0.7587,0.6928)
(0.7670,0.7007)
(0.7753,0.7087)
(0.7836,0.7166)
(0.7919,0.7245)
(0.8002,0.7324)
(0.8085,0.7403)
(0.8168,0.7483)
(0.8251,0.7562)
(0.8334,0.7641)
(0.8417,0.7720)
(0.8500,0.7800)
(0.8583,0.7879)
(0.8666,0.7958)
(0.8749,0.8037)
(0.8832,0.8117)
(0.8916,0.8196)
(0.8999,0.8275)
(0.9082,0.8354)
(0.9165,0.8434)
(0.9248,0.8513)
(0.9331,0.8592)
(0.9414,0.8671)
(0.9497,0.8751)
(0.9580,0.8830)

\PST@Dashed(0.9580,0.8851)
(0.9580,0.8883)

\PST@Dashed(0.9505,0.8851)
(0.9655,0.8851)

\PST@Dashed(0.9505,0.8883)
(0.9655,0.8883)

\PST@Dashed(0.9393,0.8674)
(0.9393,0.8706)

\PST@Dashed(0.9318,0.8674)
(0.9468,0.8674)

\PST@Dashed(0.9318,0.8706)
(0.9468,0.8706)

\PST@Dashed(0.9038,0.8304)
(0.9038,0.8333)

\PST@Dashed(0.8963,0.8304)
(0.9113,0.8304)

\PST@Dashed(0.8963,0.8333)
(0.9113,0.8333)

\PST@Dashed(0.8664,0.7951)
(0.8664,0.7979)

\PST@Dashed(0.8589,0.7951)
(0.8739,0.7951)

\PST@Dashed(0.8589,0.7979)
(0.8739,0.7979)

\PST@Dashed(0.8272,0.7562)
(0.8272,0.7590)

\PST@Dashed(0.8197,0.7562)
(0.8347,0.7562)

\PST@Dashed(0.8197,0.7590)
(0.8347,0.7590)

\PST@Dashed(0.7898,0.7208)
(0.7898,0.7237)

\PST@Dashed(0.7823,0.7208)
(0.7973,0.7208)

\PST@Dashed(0.7823,0.7237)
(0.7973,0.7237)

\PST@Dashed(0.7543,0.6855)
(0.7543,0.6883)

\PST@Dashed(0.7468,0.6855)
(0.7618,0.6855)

\PST@Dashed(0.7468,0.6883)
(0.7618,0.6883)

\PST@Dashed(0.7150,0.6485)
(0.7150,0.6510)

\PST@Dashed(0.7075,0.6485)
(0.7225,0.6485)

\PST@Dashed(0.7075,0.6510)
(0.7225,0.6510)

\PST@Dashed(0.6777,0.6132)
(0.6777,0.6156)

\PST@Dashed(0.6702,0.6132)
(0.6852,0.6132)

\PST@Dashed(0.6702,0.6156)
(0.6852,0.6156)

\PST@Dashed(0.6384,0.5743)
(0.6384,0.5767)

\PST@Dashed(0.6309,0.5743)
(0.6459,0.5743)

\PST@Dashed(0.6309,0.5767)
(0.6459,0.5767)

\PST@Dashed(0.6029,0.5407)
(0.6029,0.5431)

\PST@Dashed(0.5954,0.5407)
(0.6104,0.5407)

\PST@Dashed(0.5954,0.5431)
(0.6104,0.5431)

\PST@Dashed(0.5655,0.5053)
(0.5655,0.5078)

\PST@Dashed(0.5580,0.5053)
(0.5730,0.5053)

\PST@Dashed(0.5580,0.5078)
(0.5730,0.5078)

\PST@Dashed(0.5282,0.4717)
(0.5282,0.4742)

\PST@Dashed(0.5207,0.4717)
(0.5357,0.4717)

\PST@Dashed(0.5207,0.4742)
(0.5357,0.4742)

\PST@Dashed(0.4908,0.4348)
(0.4908,0.4369)

\PST@Dashed(0.4833,0.4348)
(0.4983,0.4348)

\PST@Dashed(0.4833,0.4369)
(0.4983,0.4369)

\PST@Dashed(0.4534,0.3994)
(0.4534,0.4015)

\PST@Dashed(0.4459,0.3994)
(0.4609,0.3994)

\PST@Dashed(0.4459,0.4015)
(0.4609,0.4015)

\PST@Dashed(0.4160,0.3641)
(0.4160,0.3662)

\PST@Dashed(0.4085,0.3641)
(0.4235,0.3641)

\PST@Dashed(0.4085,0.3662)
(0.4235,0.3662)

\PST@Dashed(0.3786,0.3285)
(0.3786,0.3292)

\PST@Dashed(0.3711,0.3285)
(0.3861,0.3285)

\PST@Dashed(0.3711,0.3292)
(0.3861,0.3292)

\PST@Dashed(0.3600,0.3114)
(0.3600,0.3121)

\PST@Dashed(0.3525,0.3114)
(0.3675,0.3114)

\PST@Dashed(0.3525,0.3121)
(0.3675,0.3121)

\PST@Dashed(0.3413,0.2939)
(0.3413,0.2946)

\PST@Dashed(0.3338,0.2939)
(0.3488,0.2939)

\PST@Dashed(0.3338,0.2946)
(0.3488,0.2946)

\PST@Dashed(0.3226,0.2769)
(0.3226,0.2776)

\PST@Dashed(0.3151,0.2769)
(0.3301,0.2769)

\PST@Dashed(0.3151,0.2776)
(0.3301,0.2776)

\PST@Dashed(0.3002,0.2566)
(0.3002,0.2573)

\PST@Dashed(0.2927,0.2566)
(0.3077,0.2566)

\PST@Dashed(0.2927,0.2573)
(0.3077,0.2573)

\PST@Dashed(0.2684,0.2254)
(0.2684,0.2258)

\PST@Dashed(0.2609,0.2254)
(0.2759,0.2254)

\PST@Dashed(0.2609,0.2258)
(0.2759,0.2258)

\PST@Dashed(0.2291,0.1890)
(0.2291,0.1894)

\PST@Dashed(0.2216,0.1890)
(0.2366,0.1890)

\PST@Dashed(0.2216,0.1894)
(0.2366,0.1894)

\PST@Dashed(0.1918,0.1533)
(0.1918,0.1537)

\PST@Dashed(0.1843,0.1533)
(0.1993,0.1533)

\PST@Dashed(0.1843,0.1537)
(0.1993,0.1537)

\PST@Dashed(0.1731,0.1363)
(0.1731,0.1367)

\PST@Dashed(0.1656,0.1363)
(0.1806,0.1363)

\PST@Dashed(0.1656,0.1367)
(0.1806,0.1367)

\PST@Dashed(0.1562,0.1199)
(0.1562,0.1202)

\PST@Dashed(0.1487,0.1199)
(0.1637,0.1199)

\PST@Dashed(0.1487,0.1202)
(0.1637,0.1202)

\PST@Dashed(0.1357,0.0971)
(0.1357,0.0974)

\PST@Dashed(0.1282,0.0971)
(0.1432,0.0971)

\PST@Dashed(0.1282,0.0974)
(0.1432,0.0974)

\PST@Diamond(0.9580,0.8867)
\PST@Diamond(0.9393,0.8690)
\PST@Diamond(0.9038,0.8319)
\PST@Diamond(0.8664,0.7965)
\PST@Diamond(0.8272,0.7576)
\PST@Diamond(0.7898,0.7222)
\PST@Diamond(0.7543,0.6869)
\PST@Diamond(0.7150,0.6498)
\PST@Diamond(0.6777,0.6144)
\PST@Diamond(0.6384,0.5755)
\PST@Diamond(0.6029,0.5419)
\PST@Diamond(0.5655,0.5066)
\PST@Diamond(0.5282,0.4730)
\PST@Diamond(0.4908,0.4358)
\PST@Diamond(0.4534,0.4005)
\PST@Diamond(0.4160,0.3651)
\PST@Diamond(0.3786,0.3289)
\PST@Diamond(0.3600,0.3117)
\PST@Diamond(0.3413,0.2942)
\PST@Diamond(0.3226,0.2772)
\PST@Diamond(0.3002,0.2569)
\PST@Diamond(0.2684,0.2256)
\PST@Diamond(0.2291,0.1892)
\PST@Diamond(0.1918,0.1535)
\PST@Diamond(0.1731,0.1365)
\PST@Diamond(0.1562,0.1201)
\PST@Diamond(0.1357,0.0973)
\PST@Border(0.1170,0.9680)
(0.1170,0.0840)
(0.9580,0.0840)
(0.9580,0.9680)
(0.1170,0.9680)

\catcode`@=12
\fi
\endpspicture

\end{figure}
Si stima infine il calore specifico, con errore per propagazione, del campione:
\begin{equation*}
 c_x =\dfrac{(m_a c_a + m_r c_r +k)(T_f-T_A)}{m_x(T_{c,i}-T_f)} = \unitfrac[0.120\pm0.002]{cal}{g\celsius}
\end{equation*}
\section{Conclusioni}

\end{document}
