\documentclass[italian,a4paper]{article}
\usepackage[tight,nice]{units}
\usepackage{babel,amsmath,amssymb,amsthm,graphicx,url,gensymb}
\usepackage[text={5.5in,9in},centering]{geometry}
\usepackage[utf8x]{inputenc}
%\usepackage[T1]{fontenc}
\usepackage{ae,aecompl}
\usepackage[footnotesize,bf]{caption}
\usepackage[usenames]{color}
\include{pstricks}
\frenchspacing
\pagestyle{plain}
%------------- eliminare prime e ultime linee isolate
\clubpenalty=9999%
\widowpenalty=9999
%--- definizione numerazioni
\renewcommand{\theequation}{\thesection.\arabic{equation}}
\renewcommand{\thefigure}{\arabic{figure}}
\renewcommand{\thetable}{\arabic{table}}
\addto\captionsitalian{%
  \renewcommand{\figurename}%
{Grafico}%
}
%
%------------- ridefinizione simbolo per elenchi puntati: en dash
%\renewcommand{\labelitemi}{\textbf{--}}
\renewcommand{\labelenumi}{\textbf{\arabic{enumi}.}}
\setlength{\abovecaptionskip}{\baselineskip}   % 0.5cm as an example
\setlength{\floatsep}{2\baselineskip}
\setlength{\belowcaptionskip}{\baselineskip}   % 0.5cm as an example
%--------- comandi insiemi numeri complessi, naturali, reali e altre abbreviazioni
\renewcommand{\leq}{\leqslant}
%--------- porzione dedicata ai float in una pagina:
\renewcommand{\textfraction}{0.05}
\renewcommand{\topfraction}{0.95}
\renewcommand{\bottomfraction}{0.95}
\renewcommand{\floatpagefraction}{0.35}
\setcounter{totalnumber}{5}
%---------
%
%---------
\begin{document}
\title{Relazione di laboratorio: il pendolo di torsione}
\author{\normalsize Ilaria Brivio (582116)\\%
\normalsize \url{brivio.ilaria@tiscali.it}%
\and %
\normalsize Matteo Abis (584206)\\ %
\normalsize \url{webmaster@latinblog.org}}
\date{\today}
\maketitle
\begin{figure}[p]\caption{}\label{ampiezzegraf}
\centering
% GNUPLOT: LaTeX picture using PSTRICKS macros
% Define new PST objects, if not already defined
\ifx\PSTloaded\undefined
\def\PSTloaded{t}
\psset{arrowsize=.01 3.2 1.4 .3}
\psset{dotsize=.08}
\catcode`@=11

\newpsobject{PST@Border}{psline}{linewidth=.0015,linestyle=solid}
\newpsobject{PST@Axes}{psline}{linewidth=.0015,linestyle=dotted,dotsep=.004}
\newpsobject{PST@Solid}{psline}{linewidth=.0015,linestyle=solid}
\newpsobject{PST@Dashed}{psline}{linewidth=.0015,linestyle=dashed,dash=.01 .01}
\newpsobject{PST@Dotted}{psline}{linewidth=.0025,linestyle=dotted,dotsep=.008}
\newpsobject{PST@LongDash}{psline}{linewidth=.0015,linestyle=dashed,dash=.02 .01}
\newpsobject{PST@Diamond}{psdots}{linewidth=.001,linestyle=solid,dotstyle=square,dotangle=45}
\newpsobject{PST@Filldiamond}{psdots}{linewidth=.001,linestyle=solid,dotstyle=square*,dotangle=45}
\newpsobject{PST@Cross}{psdots}{linewidth=.001,linestyle=solid,dotstyle=+,dotangle=45}
\newpsobject{PST@Plus}{psdots}{linewidth=.001,linestyle=solid,dotstyle=+}
\newpsobject{PST@Square}{psdots}{linewidth=.001,linestyle=solid,dotstyle=square}
\newpsobject{PST@Circle}{psdots}{linewidth=.001,linestyle=solid,dotstyle=o}
\newpsobject{PST@Triangle}{psdots}{linewidth=.001,linestyle=solid,dotstyle=triangle}
\newpsobject{PST@Pentagon}{psdots}{linewidth=.001,linestyle=solid,dotstyle=pentagon}
\newpsobject{PST@Fillsquare}{psdots}{linewidth=.001,linestyle=solid,dotstyle=square*}
\newpsobject{PST@Fillcircle}{psdots}{linewidth=.001,linestyle=solid,dotstyle=*}
\newpsobject{PST@Filltriangle}{psdots}{linewidth=.001,linestyle=solid,dotstyle=triangle*}
\newpsobject{PST@Fillpentagon}{psdots}{linewidth=.001,linestyle=solid,dotstyle=pentagon*}
\newpsobject{PST@Arrow}{psline}{linewidth=.001,linestyle=solid}
\catcode`@=12

\fi
\psset{unit=5.0in,xunit=5.0in,yunit=3.0in}
\pspicture(0.000000,0.000000)(1.000000,1.000000)
\ifx\nofigs\undefined
\catcode`@=11

\PST@Border(0.1010,0.0840)
(0.1160,0.0840)

\rput[r](0.0850,0.0840){1.5}
\PST@Border(0.1010,0.2608)
(0.1160,0.2608)

\rput[r](0.0850,0.2608){2.0}
\PST@Border(0.1010,0.4376)
(0.1160,0.4376)

\rput[r](0.0850,0.4376){2.5}
\PST@Border(0.1010,0.6144)
(0.1160,0.6144)

\rput[r](0.0850,0.6144){3.0}
\PST@Border(0.1010,0.7912)
(0.1160,0.7912)

\rput[r](0.0850,0.7912){3.5}
\PST@Border(0.1010,0.9680)
(0.1160,0.9680)

\rput[r](0.0850,0.9680){4.0}
\PST@Border(0.1010,0.0840)
(0.1010,0.1040)

\rput(0.1010,0.0420){0.25}
\PST@Border(0.2438,0.0840)
(0.2438,0.1040)

\rput(0.2438,0.0420){0.30}
\PST@Border(0.3867,0.0840)
(0.3867,0.1040)

\rput(0.3867,0.0420){0.35}
\PST@Border(0.5295,0.0840)
(0.5295,0.1040)

\rput(0.5295,0.0420){0.40}
\PST@Border(0.6723,0.0840)
(0.6723,0.1040)

\rput(0.6723,0.0420){0.45}
\PST@Border(0.8152,0.0840)
(0.8152,0.1040)

\rput(0.8152,0.0420){0.50}
\PST@Border(0.9580,0.0840)
(0.9580,0.1040)

\rput(0.9580,0.0420){0.55}
\PST@Border(0.1010,0.9680)
(0.1010,0.0840)
(0.9580,0.0840)
(0.9580,0.9680)
(0.1010,0.9680)

\PST@Solid(0.1010,0.1561)
(0.1010,0.1561)
(0.1097,0.1634)
(0.1183,0.1707)
(0.1270,0.1780)
(0.1356,0.1853)
(0.1443,0.1926)
(0.1529,0.2000)
(0.1616,0.2073)
(0.1703,0.2146)
(0.1789,0.2219)
(0.1876,0.2292)
(0.1962,0.2365)
(0.2049,0.2438)
(0.2135,0.2512)
(0.2222,0.2585)
(0.2308,0.2658)
(0.2395,0.2731)
(0.2482,0.2804)
(0.2568,0.2877)
(0.2655,0.2950)
(0.2741,0.3024)
(0.2828,0.3097)
(0.2914,0.3170)
(0.3001,0.3243)
(0.3088,0.3316)
(0.3174,0.3389)
(0.3261,0.3462)
(0.3347,0.3536)
(0.3434,0.3609)
(0.3520,0.3682)
(0.3607,0.3755)
(0.3694,0.3828)
(0.3780,0.3901)
(0.3867,0.3974)
(0.3953,0.4048)
(0.4040,0.4121)
(0.4126,0.4194)
(0.4213,0.4267)
(0.4299,0.4340)
(0.4386,0.4413)
(0.4473,0.4486)
(0.4559,0.4560)
(0.4646,0.4633)
(0.4732,0.4706)
(0.4819,0.4779)
(0.4905,0.4852)
(0.4992,0.4925)
(0.5079,0.4998)
(0.5165,0.5072)
(0.5252,0.5145)
(0.5338,0.5218)
(0.5425,0.5291)
(0.5511,0.5364)
(0.5598,0.5437)
(0.5685,0.5510)
(0.5771,0.5584)
(0.5858,0.5657)
(0.5944,0.5730)
(0.6031,0.5803)
(0.6117,0.5876)
(0.6204,0.5949)
(0.6291,0.6022)
(0.6377,0.6096)
(0.6464,0.6169)
(0.6550,0.6242)
(0.6637,0.6315)
(0.6723,0.6388)
(0.6810,0.6461)
(0.6896,0.6534)
(0.6983,0.6608)
(0.7070,0.6681)
(0.7156,0.6754)
(0.7243,0.6827)
(0.7329,0.6900)
(0.7416,0.6973)
(0.7502,0.7046)
(0.7589,0.7120)
(0.7676,0.7193)
(0.7762,0.7266)
(0.7849,0.7339)
(0.7935,0.7412)
(0.8022,0.7485)
(0.8108,0.7558)
(0.8195,0.7632)
(0.8282,0.7705)
(0.8368,0.7778)
(0.8455,0.7851)
(0.8541,0.7924)
(0.8628,0.7997)
(0.8714,0.8070)
(0.8801,0.8144)
(0.8887,0.8217)
(0.8974,0.8290)
(0.9061,0.8363)
(0.9147,0.8436)
(0.9234,0.8509)
(0.9320,0.8582)
(0.9407,0.8656)
(0.9493,0.8729)
(0.9580,0.8802)

\PST@Dashed(0.8152,0.7565)
(0.8152,0.7622)

\PST@Dashed(0.8077,0.7565)
(0.8227,0.7565)

\PST@Dashed(0.8077,0.7622)
(0.8227,0.7622)

\PST@Dashed(0.6723,0.6367)
(0.6723,0.6416)

\PST@Dashed(0.6648,0.6367)
(0.6798,0.6367)

\PST@Dashed(0.6648,0.6416)
(0.6798,0.6416)

\PST@Dashed(0.5295,0.5165)
(0.5295,0.5214)

\PST@Dashed(0.5220,0.5165)
(0.5370,0.5165)

\PST@Dashed(0.5220,0.5214)
(0.5370,0.5214)

\PST@Dashed(0.3867,0.3962)
(0.3867,0.4012)

\PST@Dashed(0.3792,0.3962)
(0.3942,0.3962)

\PST@Dashed(0.3792,0.4012)
(0.3942,0.4012)

\PST@Dashed(0.2438,0.2760)
(0.2438,0.2810)

\PST@Dashed(0.2363,0.2760)
(0.2513,0.2760)

\PST@Dashed(0.2363,0.2810)
(0.2513,0.2810)

\PST@Dashed(0.3153,0.3326)
(0.3153,0.3375)

\PST@Dashed(0.3078,0.3326)
(0.3228,0.3326)

\PST@Dashed(0.3078,0.3375)
(0.3228,0.3375)

\PST@Dashed(0.4581,0.4528)
(0.4581,0.4578)

\PST@Dashed(0.4506,0.4528)
(0.4656,0.4528)

\PST@Dashed(0.4506,0.4578)
(0.4656,0.4578)

\PST@Dashed(0.6009,0.5766)
(0.6009,0.5815)

\PST@Dashed(0.5934,0.5766)
(0.6084,0.5766)

\PST@Dashed(0.5934,0.5815)
(0.6084,0.5815)

\PST@Diamond(0.8152,0.7594)
\PST@Diamond(0.6723,0.6392)
\PST@Diamond(0.5295,0.5189)
\PST@Diamond(0.3867,0.3987)
\PST@Diamond(0.2438,0.2785)
\PST@Diamond(0.3153,0.3351)
\PST@Diamond(0.4581,0.4553)
\PST@Diamond(0.6009,0.5790)
\PST@Border(0.1010,0.9680)
(0.1010,0.0840)
(0.9580,0.0840)
(0.9580,0.9680)
(0.1010,0.9680)

\catcode`@=12
\fi
\endpspicture

\end{figure}
\end{document}
