\documentclass[italian,a4paper]{article}
\usepackage[tight,nice]{units} %unità di misura
\usepackage{babel,amsmath,amssymb,amsthm,graphicx,url}
\usepackage[text={5.5in,9in},centering]{geometry}
\usepackage[utf8x]{inputenc}
\usepackage[T1]{fontenc}
\usepackage{ae,aecompl}
\usepackage{pstricks}
\usepackage[footnotesize,bf]{caption}
\usepackage{textcomp}
\usepackage{gensymb}
\frenchspacing
\pagestyle{plain}
%------------- eliminare prime e ultime linee isolate
\clubpenalty=9999%
\widowpenalty=9999
%--- definizione numerazioni
\renewcommand{\theequation}{\thesection.\arabic{equation}}
\renewcommand{\thefigure}{\arabic{figure}}
\renewcommand{\thetable}{\arabic{table}}
\addto\captionsitalian{\renewcommand{\figurename}{Grafico}}
%
%------------- ridefinizione simbolo per elenchi puntati: en dash
%\renewcommand{\labelitemi}{\textbf{--}}
\renewcommand{\labelenumi}{\textbf{\arabic{enumi}.}}
\setlength{\abovecaptionskip}{\baselineskip}   % 0.5cm as an example
\setlength{\floatsep}{2\baselineskip}
\setlength{\belowcaptionskip}{\baselineskip}   % 0.5cm as an example
%--------- comandi insiemi numeri complessi, naturali, reali e altre abbreviazioni
\renewcommand{\leq}{\leqslant}
%--------- porzione dedicata ai float in una pagina:
\renewcommand{\textfraction}{0.05}
\renewcommand{\topfraction}{0.95}
\renewcommand{\bottomfraction}{0.95}
\renewcommand{\floatpagefraction}{0.35}
\setcounter{totalnumber}{5}
%---------
%
%---------
\begin{document}
\title{Relazione di laboratorio: diodo con condensatore}
\author{\normalsize Ilaria Brivio (582116)\\%
\normalsize \url{brivio.ilaria@tiscali.it}%
\and %
\normalsize Matteo Abis (584206)\\ %
\normalsize \url{webmaster@latinblog.org}
\and %
\normalsize Lorenzo Rossato (579393)\\ %
\normalsize \url{supergiovane05@hotmail.com}}
\date{\today}
\maketitle
%------------------
\noindent
Connettere un condensatore $C$ di \unit[1]{nF} in parallelo alla resistenza.
Utilizzando il segnale triangolare del precedente esercizio, verificare se
l’andamento dell’uscita è consistente con l’andamento previsto.

Riportiamo le caratteristiche di $V_{\text{in}}$:
\begin{table}[h]
    \centering
    \begin{tabular}{rl}
        forma: & triangolare\\
        frequenza: & \unit[123.3]{kHz}\\
        ampiezza pp: & \unit[10]{V}\\
        valor medio: & \unit[0]{V}
    \end{tabular}
\end{table}\\
Il valore nominale della resistenza $R$ è \unit[$100$]{k\ohm}, quello del condensatore $C$ \unit[1]{nF}. Abbiamo collegato $V_{\text{in}}$ e $V_{\text{out}}$ ai canali 1 e 2 dell'oscilloscopio rispettivamente, mediante le sonde precedentemente compensate.  L'andamento qualitativo dell'ingresso e dell'uscita in funzione del tempo per la durata di un
periodo è riportato in figura~\ref{fig:vinvout}.
\begin{figure}[h]\caption{Rappresentazione schematica del circuito
    realizzato}
    \centering
    \psset{unit=1in,cornersize=absolute,dimen=middle}%
\begin{pspicture}(-1.395,-0.9625)(0.383333,1.254722)%
% dpic version 16.Jan.09 for PSTricks 0.93a or later
\psset{linewidth=0.8pt}%
\psset{linewidth=0.8pt}%
\makeatletter\@ifundefined{MPSTPatchA}{\def\psbezier@ii{\addto@pscode{%
\ifshowpoints true \else false \fi\tx@OpenBezier%
\ifshowpoints\tx@BezierShowPoints\fi}\end@OpenObj}%
\global\def\MPSTPatchA{}}{}\makeatother%
\psset{arrowsize=1.1pt 4,arrowlength=1.64,arrowinset=0}%
\psline(0,0)(0,0.15)
(0,0.15)(-0.01107,0.15)
\psline(0,0.6)(0,0.45)
(0,0.45)(-0.01107,0.45)
\psline(-0.2,0.2)(-0.2,0.4)
\psline(-0.325,0.3)(-0.2,0.3)
\psline(0,0.15)(-0.2,0.24)
\psline[arrowsize=0.055556in 0,arrowlength=1.5,arrowinset=0]{<-}(-0.05,0.1725)(-0.15,0.2175)
\psline(0,0.45)(-0.2,0.36)
\psline(0,0.6)(0,0.775)
(0,0.775)(-0.041667,0.795833)
(-0.041667,0.795833)(0.041667,0.8375)
(0.041667,0.8375)(-0.041667,0.879167)
(-0.041667,0.879167)(0.041667,0.920833)
(0.041667,0.920833)(-0.041667,0.9625)
(-0.041667,0.9625)(0.041667,1.004167)
(0.041667,1.004167)(0,1.025)
(0,1.025)(0,1.2)
\uput{0.501875ex}[r](0.041667,0.9){\rlap{$ R_c$}}
\pscircle[fillstyle=solid,fillcolor=black](0,0.6){0.02}
\uput{0.501875ex}[r](0.02,0.6){\rlap{$ V_\text{out}$}}
\pscircle[fillstyle=solid,fillcolor=black](0,1.2){0.02}
\uput{0.501875ex}[u](0,1.22){$ +\unit[15]{V}$}
\psline(-0.325,0.3)(-0.775,0.3)
\psline(-0.775,0.3)(-0.775,0.6)
\psline(-0.775,0.6)(-0.775,0.775)
(-0.775,0.775)(-0.816667,0.795833)
(-0.816667,0.795833)(-0.733333,0.8375)
(-0.733333,0.8375)(-0.816667,0.879167)
(-0.816667,0.879167)(-0.733333,0.920833)
(-0.733333,0.920833)(-0.816667,0.9625)
(-0.816667,0.9625)(-0.733333,1.004167)
(-0.733333,1.004167)(-0.775,1.025)
(-0.775,1.025)(-0.775,1.2)
\uput{0.501875ex}[r](-0.733333,0.9){\rlap{$ R_1$}}
\pscircle[fillstyle=solid,fillcolor=black](-0.775,1.2){0.02}
\uput{0.501875ex}[u](-0.775,1.22){$ +\unit[15]{V}$}
\psline(-0.775,0.3)(-0.775,0)
\psline(-0.775,0)(-0.775,-0.325)
(-0.775,-0.325)(-0.733333,-0.345833)
(-0.733333,-0.345833)(-0.816667,-0.3875)
(-0.816667,-0.3875)(-0.733333,-0.429167)
(-0.733333,-0.429167)(-0.816667,-0.470833)
(-0.816667,-0.470833)(-0.733333,-0.5125)
(-0.733333,-0.5125)(-0.816667,-0.554167)
(-0.816667,-0.554167)(-0.775,-0.575)
(-0.775,-0.575)(-0.775,-0.9)
\uput{0.501875ex}[r](-0.733333,-0.45){\rlap{$ R_2$}}
\psline(-0.691667,-0.9)(-0.858333,-0.9)
\psline(-0.719444,-0.93125)(-0.830556,-0.93125)
\psline(-0.739286,-0.9625)(-0.810714,-0.9625)
\psline(-0.775,0.3)(-1.05,0.3)
\psline(-1.05,0.383333)(-1.05,0.216667)
\psline(-1.1,0.383333)(-1.1,0.216667)
\psline(-1.1,0.3)(-1.375,0.3)
\uput{0.501875ex}[u](-1.075,0.383333){$ C_1$}
\pscircle[fillstyle=solid,fillcolor=black](-1.375,0.3){0.02}
\uput{0.501875ex}[d](-1.375,0.28){$ V_\text{in}$}
\psline(0,0)(-0,-0.1)
(-0,-0.1)(0.041667,-0.120833)
(0.041667,-0.120833)(-0.041667,-0.1625)
(-0.041667,-0.1625)(0.041667,-0.204167)
(0.041667,-0.204167)(-0.041667,-0.245833)
(-0.041667,-0.245833)(0.041667,-0.2875)
(0.041667,-0.2875)(-0.041667,-0.329167)
(-0.041667,-0.329167)(0,-0.35)
(0,-0.35)(0,-0.45)
\uput{0.501875ex}[l](-0.125,-0.225){\llap{$ R_{e1}$}}
\psline(0,-0.45)(-0,-0.55)
(-0,-0.55)(0.041667,-0.570833)
(0.041667,-0.570833)(-0.041667,-0.6125)
(-0.041667,-0.6125)(0.041667,-0.654167)
(0.041667,-0.654167)(-0.041667,-0.695833)
(-0.041667,-0.695833)(0.041667,-0.7375)
(0.041667,-0.7375)(-0.041667,-0.779167)
(-0.041667,-0.779167)(0,-0.8)
(0,-0.8)(0,-0.9)
\uput{0.501875ex}[l](-0.125,-0.675){\llap{$ R_{e2}$}}
\psline(0.083333,-0.9)(-0.083333,-0.9)
\psline(0.055556,-0.93125)(-0.055556,-0.93125)
\psline(0.035714,-0.9625)(-0.035714,-0.9625)
\psline(0,-0.45)(0.3,-0.45)
\psline(0.3,-0.45)(0.3,-0.65)
\psline(0.216667,-0.65)(0.383333,-0.65)
\psline(0.216667,-0.7)(0.383333,-0.7)
\psline(0.3,-0.7)(0.3,-0.9)
\uput{0.501875ex}[ur](0.325,-0.675){\rlap{$ C_2$}}
\psline(0.383333,-0.9)(0.216667,-0.9)
\psline(0.355556,-0.93125)(0.244444,-0.93125)
\psline(0.335714,-0.9625)(0.264286,-0.9625)
\end{pspicture}%

\end{figure}\\
\begin{figure}[h]\caption{Andamento qualitativo di $V_{\text{in}}$ e $V_{\text{out}}$ (\unit{V}) in funzione del tempo (\unit{ms}).}
        \centering                                     
        % GNUPLOT: LaTeX picture using PSTRICKS macros
% Define new PST objects, if not already defined
\ifx\PSTloaded\undefined
\def\PSTloaded{t}
\psset{arrowsize=.01 3.2 1.4 .3}
\psset{dotsize=.01}
\catcode`@=11

\newpsobject{PST@Border}{psline}{linewidth=.0015,linestyle=solid}
\newpsobject{PST@Axes}{psline}{linewidth=.0015,linestyle=dotted,dotsep=.004}
\newpsobject{PST@Solid}{psline}{linewidth=.0015,linestyle=solid}
\newpsobject{PST@Dashed}{psline}{linewidth=.0015,linestyle=dashed,dash=.01 .01}
\newpsobject{PST@Dotted}{psline}{linewidth=.0025,linestyle=dotted,dotsep=.008}
\newpsobject{PST@LongDash}{psline}{linewidth=.0015,linestyle=dashed,dash=.02 .01}
\newpsobject{PST@Diamond}{psdots}{linewidth=.001,linestyle=solid,dotstyle=square,dotangle=45}
\newpsobject{PST@Filldiamond}{psdots}{linewidth=.001,linestyle=solid,dotstyle=square*,dotangle=45}
\newpsobject{PST@Cross}{psdots}{linewidth=.001,linestyle=solid,dotstyle=+,dotangle=45}
\newpsobject{PST@Plus}{psdots}{linewidth=.001,linestyle=solid,dotstyle=+}
\newpsobject{PST@Square}{psdots}{linewidth=.001,linestyle=solid,dotstyle=square}
\newpsobject{PST@Circle}{psdots}{linewidth=.001,linestyle=solid,dotstyle=o}
\newpsobject{PST@Triangle}{psdots}{linewidth=.001,linestyle=solid,dotstyle=triangle}
\newpsobject{PST@Pentagon}{psdots}{linewidth=.001,linestyle=solid,dotstyle=pentagon}
\newpsobject{PST@Fillsquare}{psdots}{linewidth=.001,linestyle=solid,dotstyle=square*}
\newpsobject{PST@Fillcircle}{psdots}{linewidth=.001,linestyle=solid,dotstyle=*}
\newpsobject{PST@Filltriangle}{psdots}{linewidth=.001,linestyle=solid,dotstyle=triangle*}
\newpsobject{PST@Fillpentagon}{psdots}{linewidth=.001,linestyle=solid,dotstyle=pentagon*}
\newpsobject{PST@Arrow}{psline}{linewidth=.001,linestyle=solid}
\catcode`@=12

\fi
\psset{unit=5.0in,xunit=5.0in,yunit=3.0in}
\pspicture(0.000000,0.000000)(1.000000,1.000000)
\ifx\nofigs\undefined
\catcode`@=11

\PST@Border(0.1170,0.0840)
(0.1320,0.0840)

\rput[r](0.1010,0.0840){-6}
\PST@Border(0.1170,0.2313)
(0.1320,0.2313)

\rput[r](0.1010,0.2313){-4}
\PST@Border(0.1170,0.3787)
(0.1320,0.3787)

\rput[r](0.1010,0.3787){-2}
\PST@Border(0.1170,0.5260)
(0.1320,0.5260)

\rput[r](0.1010,0.5260){ 0}
\PST@Border(0.1170,0.6733)
(0.1320,0.6733)

\rput[r](0.1010,0.6733){ 2}
\PST@Border(0.1170,0.8207)
(0.1320,0.8207)

\rput[r](0.1010,0.8207){ 4}
\PST@Border(0.1170,0.9680)
(0.1320,0.9680)

\rput[r](0.1010,0.9680){ 6}
\PST@Border(0.1170,0.0840)
(0.1170,0.1040)

\rput(0.1170,0.0420){ 0}
\PST@Border(0.2110,0.0840)
(0.2110,0.1040)

\rput(0.2110,0.0420){ 2}
\PST@Border(0.3050,0.0840)
(0.3050,0.1040)

\rput(0.3050,0.0420){ 4}
\PST@Border(0.3990,0.0840)
(0.3990,0.1040)

\rput(0.3990,0.0420){ 6}
\PST@Border(0.4930,0.0840)
(0.4930,0.1040)

\rput(0.4930,0.0420){ 8}
\PST@Border(0.5870,0.0840)
(0.5870,0.1040)

\rput(0.5870,0.0420){ 10}
\PST@Border(0.6810,0.0840)
(0.6810,0.1040)

\rput(0.6810,0.0420){ 12}
\PST@Border(0.7750,0.0840)
(0.7750,0.1040)

\rput(0.7750,0.0420){ 14}
\PST@Border(0.8690,0.0840)
(0.8690,0.1040)

\rput(0.8690,0.0420){ 16}
\PST@Border(0.9630,0.0840)
(0.9630,0.1040)

\rput(0.9630,0.0420){ 18}
\PST@Border(0.1170,0.9680)
(0.1170,0.0840)
(0.9630,0.0840)
(0.9630,0.9680)
(0.1170,0.9680)

\PST@Solid(0.1170,0.9091)
(0.1170,0.9091)
(0.3077,0.1547)
(0.4985,0.9091)
(0.6892,0.1547)
(0.8799,0.9091)

\PST@Dashed(0.1170,0.8736)
(0.1170,0.8736)
(0.4874,0.8280)
(0.4985,0.8736)
(0.8688,0.8280)
(0.8799,0.8736)

\PST@Border(0.1170,0.9680)
(0.1170,0.0840)
(0.9630,0.0840)
(0.9630,0.9680)
(0.1170,0.9680)

\catcode`@=12
\fi
\endpspicture

    \label{fig:vinvout}
\end{figure}\\
La differenza di potenziale tra le due curve, nei rispettivi massimi, è
$\Delta V = \unit[482]{mV}$. 
Non si arriva a una differenza di
\unit[0.7]{V} a causa del dimensionamento del circuito, ma l'andamento
corrisponde con quanto previsto.
\end{document}
