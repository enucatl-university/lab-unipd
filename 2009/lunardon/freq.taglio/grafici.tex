\documentclass[italian,a4paper]{article}
\usepackage[tight,nice]{units}
\usepackage{babel,amsmath,amssymb,amsthm,graphicx,url,gensymb}
\usepackage[text={5.5in,9in},centering]{geometry}
\usepackage[utf8x]{inputenc}
%\usepackage[T1]{fontenc}
\usepackage{ae,aecompl}
\usepackage[footnotesize,bf]{caption}
\usepackage[usenames]{color}
\include{pstricks}
\frenchspacing
\pagestyle{plain}
%------------- eliminare prime e ultime linee isolate
\clubpenalty=9999%
\widowpenalty=9999
%--- definizione numerazioni
\renewcommand{\theequation}{\thesection.\arabic{equation}}
\renewcommand{\thefigure}{\arabic{figure}}
\renewcommand{\thetable}{\arabic{table}}
\addto\captionsitalian{%
  \renewcommand{\figurename}%
{Grafico}%
}
%
%------------- ridefinizione simbolo per elenchi puntati: en dash
%\renewcommand{\labelitemi}{\textbf{--}}
\renewcommand{\labelenumi}{\textbf{\arabic{enumi}.}}
\setlength{\abovecaptionskip}{\baselineskip}   % 0.5cm as an example
\setlength{\floatsep}{2\baselineskip}
\setlength{\belowcaptionskip}{\baselineskip}   % 0.5cm as an example
%--------- comandi insiemi numeri complessi, naturali, reali e altre abbreviazioni
\renewcommand{\leq}{\leqslant}
%--------- porzione dedicata ai float in una pagina:
\renewcommand{\textfraction}{0.05}
\renewcommand{\topfraction}{0.95}
\renewcommand{\bottomfraction}{0.95}
\renewcommand{\floatpagefraction}{0.35}
\setcounter{totalnumber}{5}
%---------
%
%---------
\begin{document}
\title{Grafici sulle misure relative alla determinazione della frequenza di taglio}
\author{\normalsize Ilaria Brivio (582116)\\%
\normalsize \url{brivio.ilaria@tiscali.it}%
\and %
\normalsize Matteo Abis (584206)\\ %
\normalsize \url{webmaster@latinblog.org}}
\date{\today}
\maketitle
%------------------
\begin{figure}[p]\caption{Frequenza in ascissa, amplificazione in ordinata.}
\centering
% GNUPLOT: LaTeX picture using PSTRICKS macros
% Define new PST objects, if not already defined
\ifx\PSTloaded\undefined
\def\PSTloaded{t}
\psset{arrowsize=.01 3.2 1.4 .3}
\psset{dotsize=.08}
\catcode`@=11

\newpsobject{PST@Border}{psline}{linewidth=.0015,linestyle=solid}
\newpsobject{PST@Axes}{psline}{linewidth=.0015,linestyle=dotted,dotsep=.004}
\newpsobject{PST@Solid}{psline}{linewidth=.0015,linestyle=solid}
\newpsobject{PST@Dashed}{psline}{linewidth=.0015,linestyle=dashed,dash=.01 .01}
\newpsobject{PST@Dotted}{psline}{linewidth=.0025,linestyle=dotted,dotsep=.008}
\newpsobject{PST@LongDash}{psline}{linewidth=.0015,linestyle=dashed,dash=.02 .01}
\newpsobject{PST@Diamond}{psdots}{linewidth=.001,linestyle=solid,dotstyle=square,dotangle=45}
\newpsobject{PST@Filldiamond}{psdots}{linewidth=.001,linestyle=solid,dotstyle=square*,dotangle=45}
\newpsobject{PST@Cross}{psdots}{linewidth=.001,linestyle=solid,dotstyle=+,dotangle=45}
\newpsobject{PST@Plus}{psdots}{linewidth=.001,linestyle=solid,dotstyle=+}
\newpsobject{PST@Square}{psdots}{linewidth=.001,linestyle=solid,dotstyle=square}
\newpsobject{PST@Circle}{psdots}{linewidth=.001,linestyle=solid,dotstyle=o}
\newpsobject{PST@Triangle}{psdots}{linewidth=.001,linestyle=solid,dotstyle=triangle}
\newpsobject{PST@Pentagon}{psdots}{linewidth=.001,linestyle=solid,dotstyle=pentagon}
\newpsobject{PST@Fillsquare}{psdots}{linewidth=.001,linestyle=solid,dotstyle=square*}
\newpsobject{PST@Fillcircle}{psdots}{linewidth=.001,linestyle=solid,dotstyle=*}
\newpsobject{PST@Filltriangle}{psdots}{linewidth=.001,linestyle=solid,dotstyle=triangle*}
\newpsobject{PST@Fillpentagon}{psdots}{linewidth=.001,linestyle=solid,dotstyle=pentagon*}
\newpsobject{PST@Arrow}{psline}{linewidth=.001,linestyle=solid}
\catcode`@=12

\fi
\psset{unit=5.0in,xunit=5.0in,yunit=3.0in}
\pspicture(0.000000,0.000000)(1.000000,1.000000)
\ifx\nofigs\undefined
\catcode`@=11

\PST@Border(0.1430,0.1260)
(0.1580,0.1260)

\rput[r](0.1270,0.1260){0.0}
\PST@Border(0.1430,0.2102)
(0.1580,0.2102)

\rput[r](0.1270,0.2102){0.1}
\PST@Border(0.1430,0.2944)
(0.1580,0.2944)

\rput[r](0.1270,0.2944){0.2}
\PST@Border(0.1430,0.3786)
(0.1580,0.3786)

\rput[r](0.1270,0.3786){0.3}
\PST@Border(0.1430,0.4628)
(0.1580,0.4628)

\rput[r](0.1270,0.4628){0.4}
\PST@Border(0.1430,0.5470)
(0.1580,0.5470)

\rput[r](0.1270,0.5470){0.5}
\PST@Border(0.1430,0.6312)
(0.1580,0.6312)

\rput[r](0.1270,0.6312){0.6}
\PST@Border(0.1430,0.7154)
(0.1580,0.7154)

\rput[r](0.1270,0.7154){0.7}
\PST@Border(0.1430,0.7996)
(0.1580,0.7996)

\rput[r](0.1270,0.7996){0.8}
\PST@Border(0.1430,0.8838)
(0.1580,0.8838)

\rput[r](0.1270,0.8838){0.9}
\PST@Border(0.1430,0.9680)
(0.1580,0.9680)

\rput[r](0.1270,0.9680){1.0}
\PST@Border(0.1430,0.1260)
(0.1430,0.1460)

\rput(0.1430,0.0840){0}
\PST@Border(0.2770,0.1260)
(0.2770,0.1460)

\rput(0.2770,0.0840){5}
\PST@Border(0.4110,0.1260)
(0.4110,0.1460)

\rput(0.4110,0.0840){10}
\PST@Border(0.5450,0.1260)
(0.5450,0.1460)

\rput(0.5450,0.0840){15}
\PST@Border(0.6790,0.1260)
(0.6790,0.1460)

\rput(0.6790,0.0840){20}
\PST@Border(0.8130,0.1260)
(0.8130,0.1460)

\rput(0.8130,0.0840){25}
\PST@Border(0.9470,0.1260)
(0.9470,0.1460)

\rput(0.9470,0.0840){30}
\PST@Border(0.1430,0.9680)
(0.1430,0.1260)
(0.9470,0.1260)
(0.9470,0.9680)
(0.1430,0.9680)

\rput{L}(0.0420,0.5470){$A$}
\rput(0.5450,0.0210){$f (\unit{kHz})$}
\PST@Diamond(0.1465,0.8557)
\PST@Diamond(0.1472,0.8277)
\PST@Diamond(0.1479,0.8052)
\PST@Diamond(0.1487,0.7715)
\PST@Diamond(0.1494,0.7491)
\PST@Diamond(0.1500,0.7210)
\PST@Diamond(0.1507,0.6986)
\PST@Diamond(0.1514,0.6784)
\PST@Diamond(0.1522,0.6514)
\PST@Diamond(0.1529,0.6290)
\PST@Diamond(0.2133,0.2192)
\PST@Diamond(0.3553,0.1575)
\PST@Diamond(0.4986,0.1442)
\PST@Diamond(0.6385,0.1391)
\PST@Diamond(0.7707,0.1366)
\PST@Diamond(0.8478,0.1355)
\PST@Diamond(0.1437,0.9212)
\PST@Diamond(0.1444,0.9111)
\PST@Diamond(0.1451,0.8945)
\PST@Border(0.1430,0.9680)
(0.1430,0.1260)
(0.9470,0.1260)
(0.9470,0.9680)
(0.1430,0.9680)

\catcode`@=12
\fi
\endpspicture

\end{figure}

\begin{figure}[p]\caption{Frequenza in ascissa, differenza di fase in ordinata.}
\centering
% GNUPLOT: LaTeX picture using PSTRICKS macros
% Define new PST objects, if not already defined
\ifx\PSTloaded\undefined
\def\PSTloaded{t}
\psset{arrowsize=.01 3.2 1.4 .3}
\psset{dotsize=.08}
\catcode`@=11

\newpsobject{PST@Border}{psline}{linewidth=.0015,linestyle=solid}
\newpsobject{PST@Axes}{psline}{linewidth=.0015,linestyle=dotted,dotsep=.004}
\newpsobject{PST@Solid}{psline}{linewidth=.0015,linestyle=solid}
\newpsobject{PST@Dashed}{psline}{linewidth=.0015,linestyle=dashed,dash=.01 .01}
\newpsobject{PST@Dotted}{psline}{linewidth=.0025,linestyle=dotted,dotsep=.008}
\newpsobject{PST@LongDash}{psline}{linewidth=.0015,linestyle=dashed,dash=.02 .01}
\newpsobject{PST@Diamond}{psdots}{linewidth=.001,linestyle=solid,dotstyle=square,dotangle=45}
\newpsobject{PST@Filldiamond}{psdots}{linewidth=.001,linestyle=solid,dotstyle=square*,dotangle=45}
\newpsobject{PST@Cross}{psdots}{linewidth=.001,linestyle=solid,dotstyle=+,dotangle=45}
\newpsobject{PST@Plus}{psdots}{linewidth=.001,linestyle=solid,dotstyle=+}
\newpsobject{PST@Square}{psdots}{linewidth=.001,linestyle=solid,dotstyle=square}
\newpsobject{PST@Circle}{psdots}{linewidth=.001,linestyle=solid,dotstyle=o}
\newpsobject{PST@Triangle}{psdots}{linewidth=.001,linestyle=solid,dotstyle=triangle}
\newpsobject{PST@Pentagon}{psdots}{linewidth=.001,linestyle=solid,dotstyle=pentagon}
\newpsobject{PST@Fillsquare}{psdots}{linewidth=.001,linestyle=solid,dotstyle=square*}
\newpsobject{PST@Fillcircle}{psdots}{linewidth=.001,linestyle=solid,dotstyle=*}
\newpsobject{PST@Filltriangle}{psdots}{linewidth=.001,linestyle=solid,dotstyle=triangle*}
\newpsobject{PST@Fillpentagon}{psdots}{linewidth=.001,linestyle=solid,dotstyle=pentagon*}
\newpsobject{PST@Arrow}{psline}{linewidth=.001,linestyle=solid}
\catcode`@=12

\fi
\psset{unit=5.0in,xunit=5.0in,yunit=3.0in}
\pspicture(0.000000,0.000000)(1.000000,1.000000)
\ifx\nofigs\undefined
\catcode`@=11

\PST@Border(0.1430,0.1260)
(0.1580,0.1260)

\rput[r](0.1270,0.1260){0.0}
\PST@Border(0.1430,0.2196)
(0.1580,0.2196)

\rput[r](0.1270,0.2196){0.2}
\PST@Border(0.1430,0.3131)
(0.1580,0.3131)

\rput[r](0.1270,0.3131){0.4}
\PST@Border(0.1430,0.4067)
(0.1580,0.4067)

\rput[r](0.1270,0.4067){0.6}
\PST@Border(0.1430,0.5002)
(0.1580,0.5002)

\rput[r](0.1270,0.5002){0.8}
\PST@Border(0.1430,0.5938)
(0.1580,0.5938)

\rput[r](0.1270,0.5938){1.0}
\PST@Border(0.1430,0.6873)
(0.1580,0.6873)

\rput[r](0.1270,0.6873){1.2}
\PST@Border(0.1430,0.7809)
(0.1580,0.7809)

\rput[r](0.1270,0.7809){1.4}
\PST@Border(0.1430,0.8744)
(0.1580,0.8744)

\rput[r](0.1270,0.8744){1.6}
\PST@Border(0.1430,0.9680)
(0.1580,0.9680)

\rput[r](0.1270,0.9680){1.8}
\PST@Border(0.1430,0.1260)
(0.1430,0.1460)

\rput(0.1430,0.0840){0}
\PST@Border(0.2770,0.1260)
(0.2770,0.1460)

\rput(0.2770,0.0840){5}
\PST@Border(0.4110,0.1260)
(0.4110,0.1460)

\rput(0.4110,0.0840){10}
\PST@Border(0.5450,0.1260)
(0.5450,0.1460)

\rput(0.5450,0.0840){15}
\PST@Border(0.6790,0.1260)
(0.6790,0.1460)

\rput(0.6790,0.0840){20}
\PST@Border(0.8130,0.1260)
(0.8130,0.1460)

\rput(0.8130,0.0840){25}
\PST@Border(0.9470,0.1260)
(0.9470,0.1460)

\rput(0.9470,0.0840){30}
\PST@Border(0.1430,0.9680)
(0.1430,0.1260)
(0.9470,0.1260)
(0.9470,0.9680)
(0.1430,0.9680)

\rput{L}(0.0420,0.5470){$\phi$}
\rput(0.5450,0.0210){$f (\unit{kHz})$}
\PST@Diamond(0.1465,0.3117)
\PST@Diamond(0.1472,0.3417)
\PST@Diamond(0.1479,0.3700)
\PST@Diamond(0.1487,0.4048)
\PST@Diamond(0.1494,0.4301)
\PST@Diamond(0.1500,0.4534)
\PST@Diamond(0.1507,0.4720)
\PST@Diamond(0.1514,0.4840)
\PST@Diamond(0.1522,0.5047)
\PST@Diamond(0.1529,0.5245)
\PST@Diamond(0.2133,0.8294)
\PST@Diamond(0.3553,0.8989)
\PST@Diamond(0.4986,0.8553)
\PST@Diamond(0.6385,0.8597)
\PST@Diamond(0.7707,0.8694)
\PST@Diamond(0.8478,0.8279)
\PST@Diamond(0.1437,0.1772)
\PST@Diamond(0.1444,0.2033)
\PST@Diamond(0.1451,0.2412)
\PST@Border(0.1430,0.9680)
(0.1430,0.1260)
(0.9470,0.1260)
(0.9470,0.9680)
(0.1430,0.9680)

\catcode`@=12
\fi
\endpspicture

\end{figure}

\begin{figure}[p]\caption{Quadrato della frequenza in ascissa, inverso del quadrato dell'amplificazione in ordinata.}
\centering
% GNUPLOT: LaTeX picture using PSTRICKS macros
% Define new PST objects, if not already defined
\ifx\PSTloaded\undefined
\def\PSTloaded{t}
\psset{arrowsize=.01 3.2 1.4 .3}
\psset{dotsize=.08}
\catcode`@=11

\newpsobject{PST@Border}{psline}{linewidth=.0015,linestyle=solid}
\newpsobject{PST@Axes}{psline}{linewidth=.0015,linestyle=dotted,dotsep=.004}
\newpsobject{PST@Solid}{psline}{linewidth=.0015,linestyle=solid}
\newpsobject{PST@Dashed}{psline}{linewidth=.0015,linestyle=dashed,dash=.01 .01}
\newpsobject{PST@Dotted}{psline}{linewidth=.0025,linestyle=dotted,dotsep=.008}
\newpsobject{PST@LongDash}{psline}{linewidth=.0015,linestyle=dashed,dash=.02 .01}
\newpsobject{PST@Diamond}{psdots}{linewidth=.001,linestyle=solid,dotstyle=square,dotangle=45}
\newpsobject{PST@Filldiamond}{psdots}{linewidth=.001,linestyle=solid,dotstyle=square*,dotangle=45}
\newpsobject{PST@Cross}{psdots}{linewidth=.001,linestyle=solid,dotstyle=+,dotangle=45}
\newpsobject{PST@Plus}{psdots}{linewidth=.001,linestyle=solid,dotstyle=+}
\newpsobject{PST@Square}{psdots}{linewidth=.001,linestyle=solid,dotstyle=square}
\newpsobject{PST@Circle}{psdots}{linewidth=.001,linestyle=solid,dotstyle=o}
\newpsobject{PST@Triangle}{psdots}{linewidth=.001,linestyle=solid,dotstyle=triangle}
\newpsobject{PST@Pentagon}{psdots}{linewidth=.001,linestyle=solid,dotstyle=pentagon}
\newpsobject{PST@Fillsquare}{psdots}{linewidth=.001,linestyle=solid,dotstyle=square*}
\newpsobject{PST@Fillcircle}{psdots}{linewidth=.001,linestyle=solid,dotstyle=*}
\newpsobject{PST@Filltriangle}{psdots}{linewidth=.001,linestyle=solid,dotstyle=triangle*}
\newpsobject{PST@Fillpentagon}{psdots}{linewidth=.001,linestyle=solid,dotstyle=pentagon*}
\newpsobject{PST@Arrow}{psline}{linewidth=.001,linestyle=solid}
\catcode`@=12

\fi
\psset{unit=5.0in,xunit=5.0in,yunit=3.0in}
\pspicture(0.000000,0.000000)(1.000000,1.000000)
\ifx\nofigs\undefined
\catcode`@=11

\PST@Border(0.1430,0.1260)
(0.1580,0.1260)

\rput[r](0.1270,0.1260){1.2}
\PST@Border(0.1430,0.2196)
(0.1580,0.2196)

\rput[r](0.1270,0.2196){1.4}
\PST@Border(0.1430,0.3131)
(0.1580,0.3131)

\rput[r](0.1270,0.3131){1.6}
\PST@Border(0.1430,0.4067)
(0.1580,0.4067)

\rput[r](0.1270,0.4067){1.8}
\PST@Border(0.1430,0.5002)
(0.1580,0.5002)

\rput[r](0.1270,0.5002){2.0}
\PST@Border(0.1430,0.5938)
(0.1580,0.5938)

\rput[r](0.1270,0.5938){2.2}
\PST@Border(0.1430,0.6873)
(0.1580,0.6873)

\rput[r](0.1270,0.6873){2.4}
\PST@Border(0.1430,0.7809)
(0.1580,0.7809)

\rput[r](0.1270,0.7809){2.6}
\PST@Border(0.1430,0.8744)
(0.1580,0.8744)

\rput[r](0.1270,0.8744){2.8}
\PST@Border(0.1430,0.9680)
(0.1580,0.9680)

\rput[r](0.1270,0.9680){3.0}
\PST@Border(0.1430,0.1260)
(0.1430,0.1460)

\rput(0.1430,0.0840){0}
\PST@Border(0.2579,0.1260)
(0.2579,0.1460)

\rput(0.2579,0.0840){20}
\PST@Border(0.3727,0.1260)
(0.3727,0.1460)

\rput(0.3727,0.0840){40}
\PST@Border(0.4876,0.1260)
(0.4876,0.1460)

\rput(0.4876,0.0840){60}
\PST@Border(0.6024,0.1260)
(0.6024,0.1460)

\rput(0.6024,0.0840){80}
\PST@Border(0.7173,0.1260)
(0.7173,0.1460)

\rput(0.7173,0.0840){100}
\PST@Border(0.8321,0.1260)
(0.8321,0.1460)

\rput(0.8321,0.0840){120}
\PST@Border(0.9470,0.1260)
(0.9470,0.1460)

\rput(0.9470,0.0840){140}
\PST@Border(0.1430,0.9680)
(0.1430,0.1260)
(0.9470,0.1260)
(0.9470,0.9680)
(0.1430,0.9680)

\rput{L}(0.0420,0.5470){$A^{-2}$}
\rput(0.5450,0.0210){$f^2 (\unit{kHz^2})$}
\PST@Diamond(0.2425,0.1874)
\PST@Diamond(0.2842,0.2383)
\PST@Diamond(0.3368,0.2835)
\PST@Diamond(0.4004,0.3605)
\PST@Diamond(0.4664,0.4189)
\PST@Diamond(0.5393,0.5014)
\PST@Diamond(0.6210,0.5763)
\PST@Diamond(0.7089,0.6517)
\PST@Diamond(0.8175,0.7660)
\PST@Diamond(0.9224,0.8757)
\PST@Dashed(0.2425,0.1947)
(0.2425,0.1947)
(0.2493,0.2016)
(0.2562,0.2084)
(0.2631,0.2153)
(0.2699,0.2221)
(0.2768,0.2290)
(0.2837,0.2359)
(0.2905,0.2427)
(0.2974,0.2496)
(0.3043,0.2565)
(0.3111,0.2633)
(0.3180,0.2702)
(0.3249,0.2771)
(0.3317,0.2839)
(0.3386,0.2908)
(0.3455,0.2976)
(0.3523,0.3045)
(0.3592,0.3114)
(0.3661,0.3182)
(0.3730,0.3251)
(0.3798,0.3320)
(0.3867,0.3388)
(0.3936,0.3457)
(0.4004,0.3526)
(0.4073,0.3594)
(0.4142,0.3663)
(0.4210,0.3731)
(0.4279,0.3800)
(0.4348,0.3869)
(0.4416,0.3937)
(0.4485,0.4006)
(0.4554,0.4075)
(0.4622,0.4143)
(0.4691,0.4212)
(0.4760,0.4281)
(0.4828,0.4349)
(0.4897,0.4418)
(0.4966,0.4487)
(0.5035,0.4555)
(0.5103,0.4624)
(0.5172,0.4692)
(0.5241,0.4761)
(0.5309,0.4830)
(0.5378,0.4898)
(0.5447,0.4967)
(0.5515,0.5036)
(0.5584,0.5104)
(0.5653,0.5173)
(0.5721,0.5242)
(0.5790,0.5310)
(0.5859,0.5379)
(0.5927,0.5447)
(0.5996,0.5516)
(0.6065,0.5585)
(0.6133,0.5653)
(0.6202,0.5722)
(0.6271,0.5791)
(0.6339,0.5859)
(0.6408,0.5928)
(0.6477,0.5997)
(0.6546,0.6065)
(0.6614,0.6134)
(0.6683,0.6202)
(0.6752,0.6271)
(0.6820,0.6340)
(0.6889,0.6408)
(0.6958,0.6477)
(0.7026,0.6546)
(0.7095,0.6614)
(0.7164,0.6683)
(0.7232,0.6752)
(0.7301,0.6820)
(0.7370,0.6889)
(0.7438,0.6957)
(0.7507,0.7026)
(0.7576,0.7095)
(0.7644,0.7163)
(0.7713,0.7232)
(0.7782,0.7301)
(0.7850,0.7369)
(0.7919,0.7438)
(0.7988,0.7507)
(0.8057,0.7575)
(0.8125,0.7644)
(0.8194,0.7713)
(0.8263,0.7781)
(0.8331,0.7850)
(0.8400,0.7918)
(0.8469,0.7987)
(0.8537,0.8056)
(0.8606,0.8124)
(0.8675,0.8193)
(0.8743,0.8262)
(0.8812,0.8330)
(0.8881,0.8399)
(0.8949,0.8468)
(0.9018,0.8536)
(0.9087,0.8605)
(0.9155,0.8673)
(0.9224,0.8742)

\PST@Border(0.1430,0.9680)
(0.1430,0.1260)
(0.9470,0.1260)
(0.9470,0.9680)
(0.1430,0.9680)

\catcode`@=12
\fi
\endpspicture

\end{figure}

\begin{figure}[p]\caption{Quadrato della frequenza in ascissa, inverso del quadrato dell'amplificazione in ordinata. Grafico dei residui.}
\centering
% GNUPLOT: LaTeX picture using PSTRICKS macros
% Define new PST objects, if not already defined
\ifx\PSTloaded\undefined
\def\PSTloaded{t}
\psset{arrowsize=.01 3.2 1.4 .3}
\psset{dotsize=.08}
\catcode`@=11

\newpsobject{PST@Border}{psline}{linewidth=.0015,linestyle=solid}
\newpsobject{PST@Axes}{psline}{linewidth=.0015,linestyle=dotted,dotsep=.004}
\newpsobject{PST@Solid}{psline}{linewidth=.0015,linestyle=solid}
\newpsobject{PST@Dashed}{psline}{linewidth=.0015,linestyle=dashed,dash=.01 .01}
\newpsobject{PST@Dotted}{psline}{linewidth=.0025,linestyle=dotted,dotsep=.008}
\newpsobject{PST@LongDash}{psline}{linewidth=.0015,linestyle=dashed,dash=.02 .01}
\newpsobject{PST@Diamond}{psdots}{linewidth=.001,linestyle=solid,dotstyle=square,dotangle=45}
\newpsobject{PST@Filldiamond}{psdots}{linewidth=.001,linestyle=solid,dotstyle=square*,dotangle=45}
\newpsobject{PST@Cross}{psdots}{linewidth=.001,linestyle=solid,dotstyle=+,dotangle=45}
\newpsobject{PST@Plus}{psdots}{linewidth=.001,linestyle=solid,dotstyle=+}
\newpsobject{PST@Square}{psdots}{linewidth=.001,linestyle=solid,dotstyle=square}
\newpsobject{PST@Circle}{psdots}{linewidth=.001,linestyle=solid,dotstyle=o}
\newpsobject{PST@Triangle}{psdots}{linewidth=.001,linestyle=solid,dotstyle=triangle}
\newpsobject{PST@Pentagon}{psdots}{linewidth=.001,linestyle=solid,dotstyle=pentagon}
\newpsobject{PST@Fillsquare}{psdots}{linewidth=.001,linestyle=solid,dotstyle=square*}
\newpsobject{PST@Fillcircle}{psdots}{linewidth=.001,linestyle=solid,dotstyle=*}
\newpsobject{PST@Filltriangle}{psdots}{linewidth=.001,linestyle=solid,dotstyle=triangle*}
\newpsobject{PST@Fillpentagon}{psdots}{linewidth=.001,linestyle=solid,dotstyle=pentagon*}
\newpsobject{PST@Arrow}{psline}{linewidth=.001,linestyle=solid}
\catcode`@=12

\fi
\psset{unit=5.0in,xunit=5.0in,yunit=3.0in}
\pspicture(0.000000,0.000000)(1.000000,1.000000)
\ifx\nofigs\undefined
\catcode`@=11

\PST@Border(0.1750,0.1260)
(0.1900,0.1260)

\rput[r](0.1590,0.1260){-0.04}
\PST@Border(0.1750,0.2313)
(0.1900,0.2313)

\rput[r](0.1590,0.2313){-0.03}
\PST@Border(0.1750,0.3365)
(0.1900,0.3365)

\rput[r](0.1590,0.3365){-0.02}
\PST@Border(0.1750,0.4418)
(0.1900,0.4418)

\rput[r](0.1590,0.4418){-0.01}
\PST@Border(0.1750,0.5470)
(0.1900,0.5470)

\rput[r](0.1590,0.5470){0.00}
\PST@Border(0.1750,0.6523)
(0.1900,0.6523)

\rput[r](0.1590,0.6523){0.01}
\PST@Border(0.1750,0.7575)
(0.1900,0.7575)

\rput[r](0.1590,0.7575){0.02}
\PST@Border(0.1750,0.8628)
(0.1900,0.8628)

\rput[r](0.1590,0.8628){0.03}
\PST@Border(0.1750,0.9680)
(0.1900,0.9680)

\rput[r](0.1590,0.9680){0.04}
\PST@Border(0.1750,0.1260)
(0.1750,0.1460)

\rput(0.1750,0.0840){0}
\PST@Border(0.2853,0.1260)
(0.2853,0.1460)

\rput(0.2853,0.0840){20}
\PST@Border(0.3956,0.1260)
(0.3956,0.1460)

\rput(0.3956,0.0840){40}
\PST@Border(0.5059,0.1260)
(0.5059,0.1460)

\rput(0.5059,0.0840){60}
\PST@Border(0.6161,0.1260)
(0.6161,0.1460)

\rput(0.6161,0.0840){80}
\PST@Border(0.7264,0.1260)
(0.7264,0.1460)

\rput(0.7264,0.0840){100}
\PST@Border(0.8367,0.1260)
(0.8367,0.1460)

\rput(0.8367,0.0840){120}
\PST@Border(0.9470,0.1260)
(0.9470,0.1460)

\rput(0.9470,0.0840){140}
\PST@Border(0.1750,0.9680)
(0.1750,0.1260)
(0.9470,0.1260)
(0.9470,0.9680)
(0.1750,0.9680)

\rput{L}(0.0420,0.5470){$A^{-2}$}
\rput(0.5610,0.0210){$f^2 (\unit{kHz^2})$}
\PST@Solid(0.2705,0.2336)
(0.2705,0.5344)

\PST@Solid(0.2630,0.2336)
(0.2780,0.2336)

\PST@Solid(0.2630,0.5344)
(0.2780,0.5344)

\PST@Solid(0.3106,0.4386)
(0.3106,0.7394)

\PST@Solid(0.3031,0.4386)
(0.3181,0.4386)

\PST@Solid(0.3031,0.7394)
(0.3181,0.7394)

\PST@Solid(0.3611,0.2744)
(0.3611,0.5752)

\PST@Solid(0.3536,0.2744)
(0.3686,0.2744)

\PST@Solid(0.3536,0.5752)
(0.3686,0.5752)

\PST@Solid(0.4221,0.5766)
(0.4221,0.8774)

\PST@Solid(0.4146,0.5766)
(0.4296,0.5766)

\PST@Solid(0.4146,0.8774)
(0.4296,0.8774)

\PST@Solid(0.4855,0.4061)
(0.4855,0.7069)

\PST@Solid(0.4780,0.4061)
(0.4930,0.4061)

\PST@Solid(0.4780,0.7069)
(0.4930,0.7069)

\PST@Solid(0.5555,0.6222)
(0.5555,0.9229)

\PST@Solid(0.5480,0.6222)
(0.5630,0.6222)

\PST@Solid(0.5480,0.9229)
(0.5630,0.9229)

\PST@Solid(0.6340,0.4712)
(0.6340,0.7720)

\PST@Solid(0.6265,0.4712)
(0.6415,0.4712)

\PST@Solid(0.6265,0.7720)
(0.6415,0.7720)

\PST@Solid(0.7183,0.1914)
(0.7183,0.4921)

\PST@Solid(0.7108,0.1914)
(0.7258,0.1914)

\PST@Solid(0.7108,0.4921)
(0.7258,0.4921)

\PST@Solid(0.8226,0.3223)
(0.8226,0.6231)

\PST@Solid(0.8151,0.3223)
(0.8301,0.3223)

\PST@Solid(0.8151,0.6231)
(0.8301,0.6231)

\PST@Solid(0.9234,0.4297)
(0.9234,0.7304)

\PST@Solid(0.9159,0.4297)
(0.9309,0.4297)

\PST@Solid(0.9159,0.7304)
(0.9309,0.7304)

\PST@Diamond(0.2705,0.3840)
\PST@Diamond(0.3106,0.5890)
\PST@Diamond(0.3611,0.4248)
\PST@Diamond(0.4221,0.7270)
\PST@Diamond(0.4855,0.5565)
\PST@Diamond(0.5555,0.7725)
\PST@Diamond(0.6340,0.6216)
\PST@Diamond(0.7183,0.3418)
\PST@Diamond(0.8226,0.4727)
\PST@Diamond(0.9234,0.5801)
\PST@Border(0.1750,0.9680)
(0.1750,0.1260)
(0.9470,0.1260)
(0.9470,0.9680)
(0.1750,0.9680)

\catcode`@=12
\fi
\endpspicture

\end{figure}

\begin{figure}[p]\caption{Grafico di Bode.}
\centering
% GNUPLOT: LaTeX picture using PSTRICKS macros
% Define new PST objects, if not already defined
\ifx\PSTloaded\undefined
\def\PSTloaded{t}
\psset{arrowsize=.01 3.2 1.4 .3}
\psset{dotsize=.08}
\catcode`@=11

\newpsobject{PST@Border}{psline}{linewidth=.0015,linestyle=solid}
\newpsobject{PST@Axes}{psline}{linewidth=.0015,linestyle=dotted,dotsep=.004}
\newpsobject{PST@Solid}{psline}{linewidth=.0015,linestyle=solid}
\newpsobject{PST@Dashed}{psline}{linewidth=.0015,linestyle=dashed,dash=.01 .01}
\newpsobject{PST@Dotted}{psline}{linewidth=.0025,linestyle=dotted,dotsep=.008}
\newpsobject{PST@LongDash}{psline}{linewidth=.0015,linestyle=dashed,dash=.02 .01}
\newpsobject{PST@Diamond}{psdots}{linewidth=.001,linestyle=solid,dotstyle=square,dotangle=45}
\newpsobject{PST@Filldiamond}{psdots}{linewidth=.001,linestyle=solid,dotstyle=square*,dotangle=45}
\newpsobject{PST@Cross}{psdots}{linewidth=.001,linestyle=solid,dotstyle=+,dotangle=45}
\newpsobject{PST@Plus}{psdots}{linewidth=.001,linestyle=solid,dotstyle=+}
\newpsobject{PST@Square}{psdots}{linewidth=.001,linestyle=solid,dotstyle=square}
\newpsobject{PST@Circle}{psdots}{linewidth=.001,linestyle=solid,dotstyle=o}
\newpsobject{PST@Triangle}{psdots}{linewidth=.001,linestyle=solid,dotstyle=triangle}
\newpsobject{PST@Pentagon}{psdots}{linewidth=.001,linestyle=solid,dotstyle=pentagon}
\newpsobject{PST@Fillsquare}{psdots}{linewidth=.001,linestyle=solid,dotstyle=square*}
\newpsobject{PST@Fillcircle}{psdots}{linewidth=.001,linestyle=solid,dotstyle=*}
\newpsobject{PST@Filltriangle}{psdots}{linewidth=.001,linestyle=solid,dotstyle=triangle*}
\newpsobject{PST@Fillpentagon}{psdots}{linewidth=.001,linestyle=solid,dotstyle=pentagon*}
\newpsobject{PST@Arrow}{psline}{linewidth=.001,linestyle=solid}
\catcode`@=12

\fi
\psset{unit=5.0in,xunit=5.0in,yunit=3.0in}
\pspicture(0.000000,0.000000)(1.000000,1.000000)
\ifx\nofigs\undefined
\catcode`@=11

\PST@Border(0.1590,0.1260)
(0.1740,0.1260)

\rput[r](0.1430,0.1260){-4.5}
\PST@Border(0.1590,0.2944)
(0.1740,0.2944)

\rput[r](0.1430,0.2944){-4.0}
\PST@Border(0.1590,0.4628)
(0.1740,0.4628)

\rput[r](0.1430,0.4628){-3.5}
\PST@Border(0.1590,0.6312)
(0.1740,0.6312)

\rput[r](0.1430,0.6312){-3.0}
\PST@Border(0.1590,0.7996)
(0.1740,0.7996)

\rput[r](0.1430,0.7996){-2.5}
\PST@Border(0.1590,0.9680)
(0.1740,0.9680)

\rput[r](0.1430,0.9680){-2.0}
\PST@Border(0.1590,0.1260)
(0.1590,0.1460)

\rput(0.1590,0.0840){7.5}
\PST@Border(0.2903,0.1260)
(0.2903,0.1460)

\rput(0.2903,0.0840){8.0}
\PST@Border(0.4217,0.1260)
(0.4217,0.1460)

\rput(0.4217,0.0840){8.5}
\PST@Border(0.5530,0.1260)
(0.5530,0.1460)

\rput(0.5530,0.0840){9.0}
\PST@Border(0.6843,0.1260)
(0.6843,0.1460)

\rput(0.6843,0.0840){9.5}
\PST@Border(0.8157,0.1260)
(0.8157,0.1460)

\rput(0.8157,0.0840){10.0}
\PST@Border(0.9470,0.1260)
(0.9470,0.1460)

\rput(0.9470,0.0840){10.5}
\PST@Border(0.1590,0.9680)
(0.1590,0.1260)
(0.9470,0.1260)
(0.9470,0.9680)
(0.1590,0.9680)

\rput{L}(0.0420,0.5470){$\log(A)$}
\rput(0.5530,0.0210){$\log(f)$}
\PST@Diamond(0.2568,0.9004)
\PST@Diamond(0.5470,0.5354)
\PST@Diamond(0.6826,0.3500)
\PST@Diamond(0.7697,0.2403)
\PST@Diamond(0.8318,0.1671)
\PST@Diamond(0.8622,0.1310)
\PST@Dashed(0.2568,0.9009)
(0.2568,0.9009)
(0.2629,0.8930)
(0.2691,0.8852)
(0.2752,0.8774)
(0.2813,0.8696)
(0.2874,0.8618)
(0.2935,0.8539)
(0.2996,0.8461)
(0.3058,0.8383)
(0.3119,0.8305)
(0.3180,0.8227)
(0.3241,0.8148)
(0.3302,0.8070)
(0.3363,0.7992)
(0.3424,0.7914)
(0.3486,0.7835)
(0.3547,0.7757)
(0.3608,0.7679)
(0.3669,0.7601)
(0.3730,0.7523)
(0.3791,0.7444)
(0.3853,0.7366)
(0.3914,0.7288)
(0.3975,0.7210)
(0.4036,0.7132)
(0.4097,0.7053)
(0.4158,0.6975)
(0.4219,0.6897)
(0.4281,0.6819)
(0.4342,0.6741)
(0.4403,0.6662)
(0.4464,0.6584)
(0.4525,0.6506)
(0.4586,0.6428)
(0.4648,0.6350)
(0.4709,0.6271)
(0.4770,0.6193)
(0.4831,0.6115)
(0.4892,0.6037)
(0.4953,0.5959)
(0.5014,0.5880)
(0.5076,0.5802)
(0.5137,0.5724)
(0.5198,0.5646)
(0.5259,0.5568)
(0.5320,0.5489)
(0.5381,0.5411)
(0.5442,0.5333)
(0.5504,0.5255)
(0.5565,0.5177)
(0.5626,0.5098)
(0.5687,0.5020)
(0.5748,0.4942)
(0.5809,0.4864)
(0.5871,0.4786)
(0.5932,0.4707)
(0.5993,0.4629)
(0.6054,0.4551)
(0.6115,0.4473)
(0.6176,0.4395)
(0.6237,0.4316)
(0.6299,0.4238)
(0.6360,0.4160)
(0.6421,0.4082)
(0.6482,0.4004)
(0.6543,0.3925)
(0.6604,0.3847)
(0.6666,0.3769)
(0.6727,0.3691)
(0.6788,0.3613)
(0.6849,0.3534)
(0.6910,0.3456)
(0.6971,0.3378)
(0.7032,0.3300)
(0.7094,0.3222)
(0.7155,0.3143)
(0.7216,0.3065)
(0.7277,0.2987)
(0.7338,0.2909)
(0.7399,0.2831)
(0.7461,0.2752)
(0.7522,0.2674)
(0.7583,0.2596)
(0.7644,0.2518)
(0.7705,0.2440)
(0.7766,0.2361)
(0.7827,0.2283)
(0.7889,0.2205)
(0.7950,0.2127)
(0.8011,0.2049)
(0.8072,0.1970)
(0.8133,0.1892)
(0.8194,0.1814)
(0.8256,0.1736)
(0.8317,0.1657)
(0.8378,0.1579)
(0.8439,0.1501)
(0.8500,0.1423)
(0.8561,0.1345)
(0.8622,0.1266)

\PST@Border(0.1590,0.9680)
(0.1590,0.1260)
(0.9470,0.1260)
(0.9470,0.9680)
(0.1590,0.9680)

\catcode`@=12
\fi
\endpspicture

\end{figure}

\begin{figure}[p]\caption{Grafico di Bode, residui.}
\centering
% GNUPLOT: LaTeX picture using PSTRICKS macros
% Define new PST objects, if not already defined
\ifx\PSTloaded\undefined
\def\PSTloaded{t}
\psset{arrowsize=.01 3.2 1.4 .3}
\psset{dotsize=.08}
\catcode`@=11

\newpsobject{PST@Border}{psline}{linewidth=.0015,linestyle=solid}
\newpsobject{PST@Axes}{psline}{linewidth=.0015,linestyle=dotted,dotsep=.004}
\newpsobject{PST@Solid}{psline}{linewidth=.0015,linestyle=solid}
\newpsobject{PST@Dashed}{psline}{linewidth=.0015,linestyle=dashed,dash=.01 .01}
\newpsobject{PST@Dotted}{psline}{linewidth=.0025,linestyle=dotted,dotsep=.008}
\newpsobject{PST@LongDash}{psline}{linewidth=.0015,linestyle=dashed,dash=.02 .01}
\newpsobject{PST@Diamond}{psdots}{linewidth=.001,linestyle=solid,dotstyle=square,dotangle=45}
\newpsobject{PST@Filldiamond}{psdots}{linewidth=.001,linestyle=solid,dotstyle=square*,dotangle=45}
\newpsobject{PST@Cross}{psdots}{linewidth=.001,linestyle=solid,dotstyle=+,dotangle=45}
\newpsobject{PST@Plus}{psdots}{linewidth=.001,linestyle=solid,dotstyle=+}
\newpsobject{PST@Square}{psdots}{linewidth=.001,linestyle=solid,dotstyle=square}
\newpsobject{PST@Circle}{psdots}{linewidth=.001,linestyle=solid,dotstyle=o}
\newpsobject{PST@Triangle}{psdots}{linewidth=.001,linestyle=solid,dotstyle=triangle}
\newpsobject{PST@Pentagon}{psdots}{linewidth=.001,linestyle=solid,dotstyle=pentagon}
\newpsobject{PST@Fillsquare}{psdots}{linewidth=.001,linestyle=solid,dotstyle=square*}
\newpsobject{PST@Fillcircle}{psdots}{linewidth=.001,linestyle=solid,dotstyle=*}
\newpsobject{PST@Filltriangle}{psdots}{linewidth=.001,linestyle=solid,dotstyle=triangle*}
\newpsobject{PST@Fillpentagon}{psdots}{linewidth=.001,linestyle=solid,dotstyle=pentagon*}
\newpsobject{PST@Arrow}{psline}{linewidth=.001,linestyle=solid}
\catcode`@=12

\fi
\psset{unit=5.0in,xunit=5.0in,yunit=3.0in}
\pspicture(0.000000,0.000000)(1.000000,1.000000)
\ifx\nofigs\undefined
\catcode`@=11

\PST@Border(0.1430,0.1260)
(0.1580,0.1260)

\rput[r](0.1270,0.1260){-40}
\PST@Border(0.1430,0.2313)
(0.1580,0.2313)

\rput[r](0.1270,0.2313){-30}
\PST@Border(0.1430,0.3365)
(0.1580,0.3365)

\rput[r](0.1270,0.3365){-20}
\PST@Border(0.1430,0.4418)
(0.1580,0.4418)

\rput[r](0.1270,0.4418){-10}
\PST@Border(0.1430,0.5470)
(0.1580,0.5470)

\rput[r](0.1270,0.5470){0}
\PST@Border(0.1430,0.6523)
(0.1580,0.6523)

\rput[r](0.1270,0.6523){10}
\PST@Border(0.1430,0.7575)
(0.1580,0.7575)

\rput[r](0.1270,0.7575){20}
\PST@Border(0.1430,0.8628)
(0.1580,0.8628)

\rput[r](0.1270,0.8628){30}
\PST@Border(0.1430,0.9680)
(0.1580,0.9680)

\rput[r](0.1270,0.9680){40}
\PST@Border(0.1430,0.1260)
(0.1430,0.1460)

\rput(0.1430,0.0840){7.5}
\PST@Border(0.2770,0.1260)
(0.2770,0.1460)

\rput(0.2770,0.0840){8.0}
\PST@Border(0.4110,0.1260)
(0.4110,0.1460)

\rput(0.4110,0.0840){8.5}
\PST@Border(0.5450,0.1260)
(0.5450,0.1460)

\rput(0.5450,0.0840){9.0}
\PST@Border(0.6790,0.1260)
(0.6790,0.1460)

\rput(0.6790,0.0840){9.5}
\PST@Border(0.8130,0.1260)
(0.8130,0.1460)

\rput(0.8130,0.0840){10.0}
\PST@Border(0.9470,0.1260)
(0.9470,0.1460)

\rput(0.9470,0.0840){10.5}
\PST@Border(0.1430,0.9680)
(0.1430,0.1260)
(0.9470,0.1260)
(0.9470,0.9680)
(0.1430,0.9680)

\rput{L}(0.0420,0.5470){$\log(A)\cdot 10^{-3}$}
\rput(0.5450,0.0210){$\log(f)$}
\PST@Solid(0.2428,0.3649)
(0.2428,0.7034)

\PST@Solid(0.2353,0.3649)
(0.2503,0.3649)

\PST@Solid(0.2353,0.7034)
(0.2503,0.7034)

\PST@Solid(0.5389,0.5559)
(0.5389,0.8944)

\PST@Solid(0.5314,0.5559)
(0.5464,0.5559)

\PST@Solid(0.5314,0.8944)
(0.5464,0.8944)

\PST@Solid(0.6772,0.1754)
(0.6772,0.5139)

\PST@Solid(0.6697,0.1754)
(0.6847,0.1754)

\PST@Solid(0.6697,0.5139)
(0.6847,0.5139)

\PST@Solid(0.7661,0.2324)
(0.7661,0.5709)

\PST@Solid(0.7586,0.2324)
(0.7736,0.2324)

\PST@Solid(0.7586,0.5709)
(0.7736,0.5709)

\PST@Solid(0.8294,0.4240)
(0.8294,0.7625)

\PST@Solid(0.8219,0.4240)
(0.8369,0.4240)

\PST@Solid(0.8219,0.7625)
(0.8369,0.7625)

\PST@Solid(0.8605,0.5137)
(0.8605,0.8523)

\PST@Solid(0.8530,0.5137)
(0.8680,0.5137)

\PST@Solid(0.8530,0.8523)
(0.8680,0.8523)

\PST@Diamond(0.2428,0.5342)
\PST@Diamond(0.5389,0.7252)
\PST@Diamond(0.6772,0.3447)
\PST@Diamond(0.7661,0.4017)
\PST@Diamond(0.8294,0.5933)
\PST@Diamond(0.8605,0.6830)
\PST@Border(0.1430,0.9680)
(0.1430,0.1260)
(0.9470,0.1260)
(0.9470,0.9680)
(0.1430,0.9680)

\catcode`@=12
\fi
\endpspicture

\end{figure}

\begin{figure}[p]\caption{Tangente dello sfasamento in ordinata, frequenza in ascissa.}
\centering
% GNUPLOT: LaTeX picture using PSTRICKS macros
% Define new PST objects, if not already defined
\ifx\PSTloaded\undefined
\def\PSTloaded{t}
\psset{arrowsize=.01 3.2 1.4 .3}
\psset{dotsize=.08}
\catcode`@=11

\newpsobject{PST@Border}{psline}{linewidth=.0015,linestyle=solid}
\newpsobject{PST@Axes}{psline}{linewidth=.0015,linestyle=dotted,dotsep=.004}
\newpsobject{PST@Solid}{psline}{linewidth=.0015,linestyle=solid}
\newpsobject{PST@Dashed}{psline}{linewidth=.0015,linestyle=dashed,dash=.01 .01}
\newpsobject{PST@Dotted}{psline}{linewidth=.0025,linestyle=dotted,dotsep=.008}
\newpsobject{PST@LongDash}{psline}{linewidth=.0015,linestyle=dashed,dash=.02 .01}
\newpsobject{PST@Diamond}{psdots}{linewidth=.001,linestyle=solid,dotstyle=square,dotangle=45}
\newpsobject{PST@Filldiamond}{psdots}{linewidth=.001,linestyle=solid,dotstyle=square*,dotangle=45}
\newpsobject{PST@Cross}{psdots}{linewidth=.001,linestyle=solid,dotstyle=+,dotangle=45}
\newpsobject{PST@Plus}{psdots}{linewidth=.001,linestyle=solid,dotstyle=+}
\newpsobject{PST@Square}{psdots}{linewidth=.001,linestyle=solid,dotstyle=square}
\newpsobject{PST@Circle}{psdots}{linewidth=.001,linestyle=solid,dotstyle=o}
\newpsobject{PST@Triangle}{psdots}{linewidth=.001,linestyle=solid,dotstyle=triangle}
\newpsobject{PST@Pentagon}{psdots}{linewidth=.001,linestyle=solid,dotstyle=pentagon}
\newpsobject{PST@Fillsquare}{psdots}{linewidth=.001,linestyle=solid,dotstyle=square*}
\newpsobject{PST@Fillcircle}{psdots}{linewidth=.001,linestyle=solid,dotstyle=*}
\newpsobject{PST@Filltriangle}{psdots}{linewidth=.001,linestyle=solid,dotstyle=triangle*}
\newpsobject{PST@Fillpentagon}{psdots}{linewidth=.001,linestyle=solid,dotstyle=pentagon*}
\newpsobject{PST@Arrow}{psline}{linewidth=.001,linestyle=solid}
\catcode`@=12

\fi
\psset{unit=5.0in,xunit=5.0in,yunit=3.0in}
\pspicture(0.000000,0.000000)(1.000000,1.000000)
\ifx\nofigs\undefined
\catcode`@=11

\PST@Border(0.1430,0.1260)
(0.1580,0.1260)

\rput[r](0.1270,0.1260){0.4}
\PST@Border(0.1430,0.2312)
(0.1580,0.2312)

\rput[r](0.1270,0.2312){0.5}
\PST@Border(0.1430,0.3365)
(0.1580,0.3365)

\rput[r](0.1270,0.3365){0.6}
\PST@Border(0.1430,0.4417)
(0.1580,0.4417)

\rput[r](0.1270,0.4417){0.7}
\PST@Border(0.1430,0.5470)
(0.1580,0.5470)

\rput[r](0.1270,0.5470){0.8}
\PST@Border(0.1430,0.6522)
(0.1580,0.6522)

\rput[r](0.1270,0.6522){0.9}
\PST@Border(0.1430,0.7575)
(0.1580,0.7575)

\rput[r](0.1270,0.7575){1.0}
\PST@Border(0.1430,0.8627)
(0.1580,0.8627)

\rput[r](0.1270,0.8627){1.1}
\PST@Border(0.1430,0.9680)
(0.1580,0.9680)

\rput[r](0.1270,0.9680){1.2}
\PST@Border(0.1430,0.1260)
(0.1430,0.1460)

\rput(0.1430,0.0840){100}
\PST@Border(0.2770,0.1260)
(0.2770,0.1460)

\rput(0.2770,0.0840){150}
\PST@Border(0.4110,0.1260)
(0.4110,0.1460)

\rput(0.4110,0.0840){200}
\PST@Border(0.5450,0.1260)
(0.5450,0.1460)

\rput(0.5450,0.0840){250}
\PST@Border(0.6790,0.1260)
(0.6790,0.1460)

\rput(0.6790,0.0840){300}
\PST@Border(0.8130,0.1260)
(0.8130,0.1460)

\rput(0.8130,0.0840){350}
\PST@Border(0.9470,0.1260)
(0.9470,0.1460)

\rput(0.9470,0.0840){400}
\PST@Border(0.1430,0.9680)
(0.1430,0.1260)
(0.9470,0.1260)
(0.9470,0.9680)
(0.1430,0.9680)

\rput{L}(0.0420,0.5470){$\tan(\phi)$}
\rput(0.5450,0.0210){$f (\unit{kHz})$}
\PST@Diamond(0.2277,0.1461)
\PST@Diamond(0.2952,0.2279)
\PST@Diamond(0.3673,0.3100)
\PST@Diamond(0.4424,0.4187)
\PST@Diamond(0.5110,0.5052)
\PST@Diamond(0.5790,0.5912)
\PST@Diamond(0.6482,0.6652)
\PST@Diamond(0.7163,0.7159)
\PST@Diamond(0.7934,0.8098)
\PST@Diamond(0.8623,0.9075)
\PST@Dashed(0.2277,0.1545)
(0.2277,0.1545)
(0.2341,0.1621)
(0.2405,0.1697)
(0.2469,0.1773)
(0.2533,0.1849)
(0.2597,0.1925)
(0.2662,0.2001)
(0.2726,0.2077)
(0.2790,0.2153)
(0.2854,0.2229)
(0.2918,0.2305)
(0.2982,0.2381)
(0.3046,0.2457)
(0.3110,0.2533)
(0.3174,0.2609)
(0.3238,0.2685)
(0.3303,0.2761)
(0.3367,0.2837)
(0.3431,0.2913)
(0.3495,0.2989)
(0.3559,0.3065)
(0.3623,0.3141)
(0.3687,0.3217)
(0.3751,0.3293)
(0.3815,0.3369)
(0.3879,0.3445)
(0.3944,0.3521)
(0.4008,0.3597)
(0.4072,0.3673)
(0.4136,0.3749)
(0.4200,0.3825)
(0.4264,0.3901)
(0.4328,0.3977)
(0.4392,0.4053)
(0.4456,0.4129)
(0.4521,0.4205)
(0.4585,0.4281)
(0.4649,0.4357)
(0.4713,0.4433)
(0.4777,0.4508)
(0.4841,0.4584)
(0.4905,0.4660)
(0.4969,0.4736)
(0.5033,0.4812)
(0.5097,0.4888)
(0.5162,0.4964)
(0.5226,0.5040)
(0.5290,0.5116)
(0.5354,0.5192)
(0.5418,0.5268)
(0.5482,0.5344)
(0.5546,0.5420)
(0.5610,0.5496)
(0.5674,0.5572)
(0.5738,0.5648)
(0.5803,0.5724)
(0.5867,0.5800)
(0.5931,0.5876)
(0.5995,0.5952)
(0.6059,0.6028)
(0.6123,0.6104)
(0.6187,0.6180)
(0.6251,0.6256)
(0.6315,0.6332)
(0.6379,0.6408)
(0.6444,0.6484)
(0.6508,0.6560)
(0.6572,0.6636)
(0.6636,0.6712)
(0.6700,0.6788)
(0.6764,0.6864)
(0.6828,0.6940)
(0.6892,0.7016)
(0.6956,0.7092)
(0.7021,0.7168)
(0.7085,0.7244)
(0.7149,0.7320)
(0.7213,0.7396)
(0.7277,0.7472)
(0.7341,0.7547)
(0.7405,0.7623)
(0.7469,0.7699)
(0.7533,0.7775)
(0.7597,0.7851)
(0.7662,0.7927)
(0.7726,0.8003)
(0.7790,0.8079)
(0.7854,0.8155)
(0.7918,0.8231)
(0.7982,0.8307)
(0.8046,0.8383)
(0.8110,0.8459)
(0.8174,0.8535)
(0.8238,0.8611)
(0.8303,0.8687)
(0.8367,0.8763)
(0.8431,0.8839)
(0.8495,0.8915)
(0.8559,0.8991)
(0.8623,0.9067)

\PST@Border(0.1430,0.9680)
(0.1430,0.1260)
(0.9470,0.1260)
(0.9470,0.9680)
(0.1430,0.9680)

\catcode`@=12
\fi
\endpspicture

\end{figure}

\begin{figure}[p]\caption{Tangente dello sfasamento in ordinata, frequenza in ascissa. Grafico dei residui.}
\centering
% GNUPLOT: LaTeX picture using PSTRICKS macros
% Define new PST objects, if not already defined
\ifx\PSTloaded\undefined
\def\PSTloaded{t}
\psset{arrowsize=.01 3.2 1.4 .3}
\psset{dotsize=.08}
\catcode`@=11

\newpsobject{PST@Border}{psline}{linewidth=.0015,linestyle=solid}
\newpsobject{PST@Axes}{psline}{linewidth=.0015,linestyle=dotted,dotsep=.004}
\newpsobject{PST@Solid}{psline}{linewidth=.0015,linestyle=solid}
\newpsobject{PST@Dashed}{psline}{linewidth=.0015,linestyle=dashed,dash=.01 .01}
\newpsobject{PST@Dotted}{psline}{linewidth=.0025,linestyle=dotted,dotsep=.008}
\newpsobject{PST@LongDash}{psline}{linewidth=.0015,linestyle=dashed,dash=.02 .01}
\newpsobject{PST@Diamond}{psdots}{linewidth=.001,linestyle=solid,dotstyle=square,dotangle=45}
\newpsobject{PST@Filldiamond}{psdots}{linewidth=.001,linestyle=solid,dotstyle=square*,dotangle=45}
\newpsobject{PST@Cross}{psdots}{linewidth=.001,linestyle=solid,dotstyle=+,dotangle=45}
\newpsobject{PST@Plus}{psdots}{linewidth=.001,linestyle=solid,dotstyle=+}
\newpsobject{PST@Square}{psdots}{linewidth=.001,linestyle=solid,dotstyle=square}
\newpsobject{PST@Circle}{psdots}{linewidth=.001,linestyle=solid,dotstyle=o}
\newpsobject{PST@Triangle}{psdots}{linewidth=.001,linestyle=solid,dotstyle=triangle}
\newpsobject{PST@Pentagon}{psdots}{linewidth=.001,linestyle=solid,dotstyle=pentagon}
\newpsobject{PST@Fillsquare}{psdots}{linewidth=.001,linestyle=solid,dotstyle=square*}
\newpsobject{PST@Fillcircle}{psdots}{linewidth=.001,linestyle=solid,dotstyle=*}
\newpsobject{PST@Filltriangle}{psdots}{linewidth=.001,linestyle=solid,dotstyle=triangle*}
\newpsobject{PST@Fillpentagon}{psdots}{linewidth=.001,linestyle=solid,dotstyle=pentagon*}
\newpsobject{PST@Arrow}{psline}{linewidth=.001,linestyle=solid}
\catcode`@=12

\fi
\psset{unit=5.0in,xunit=5.0in,yunit=3.0in}
\pspicture(0.000000,0.000000)(1.000000,1.000000)
\ifx\nofigs\undefined
\catcode`@=11

\PST@Border(0.1750,0.1260)
(0.1900,0.1260)

\rput[r](0.1590,0.1260){-0.04}
\PST@Border(0.1750,0.2313)
(0.1900,0.2313)

\rput[r](0.1590,0.2313){-0.03}
\PST@Border(0.1750,0.3365)
(0.1900,0.3365)

\rput[r](0.1590,0.3365){-0.02}
\PST@Border(0.1750,0.4418)
(0.1900,0.4418)

\rput[r](0.1590,0.4418){-0.01}
\PST@Border(0.1750,0.5470)
(0.1900,0.5470)

\rput[r](0.1590,0.5470){0.00}
\PST@Border(0.1750,0.6523)
(0.1900,0.6523)

\rput[r](0.1590,0.6523){0.01}
\PST@Border(0.1750,0.7575)
(0.1900,0.7575)

\rput[r](0.1590,0.7575){0.02}
\PST@Border(0.1750,0.8628)
(0.1900,0.8628)

\rput[r](0.1590,0.8628){0.03}
\PST@Border(0.1750,0.9680)
(0.1900,0.9680)

\rput[r](0.1590,0.9680){0.04}
\PST@Border(0.1750,0.1260)
(0.1750,0.1460)

\rput(0.1750,0.0840){100}
\PST@Border(0.3037,0.1260)
(0.3037,0.1460)

\rput(0.3037,0.0840){150}
\PST@Border(0.4323,0.1260)
(0.4323,0.1460)

\rput(0.4323,0.0840){200}
\PST@Border(0.5610,0.1260)
(0.5610,0.1460)

\rput(0.5610,0.0840){250}
\PST@Border(0.6897,0.1260)
(0.6897,0.1460)

\rput(0.6897,0.0840){300}
\PST@Border(0.8183,0.1260)
(0.8183,0.1460)

\rput(0.8183,0.0840){350}
\PST@Border(0.9470,0.1260)
(0.9470,0.1460)

\rput(0.9470,0.0840){400}
\PST@Border(0.1750,0.9680)
(0.1750,0.1260)
(0.9470,0.1260)
(0.9470,0.9680)
(0.1750,0.9680)

\rput{L}(0.0420,0.5470){$\tan(\phi)$}
\rput(0.5610,0.0210){$f (\unit{kHz})$}
\PST@Solid(0.2563,0.3196)
(0.2563,0.6064)

\PST@Solid(0.2488,0.3196)
(0.2638,0.3196)

\PST@Solid(0.2488,0.6064)
(0.2638,0.6064)

\PST@Solid(0.3212,0.3364)
(0.3212,0.6232)

\PST@Solid(0.3137,0.3364)
(0.3287,0.3364)

\PST@Solid(0.3137,0.6232)
(0.3287,0.6232)

\PST@Solid(0.3904,0.3034)
(0.3904,0.5902)

\PST@Solid(0.3829,0.3034)
(0.3979,0.3034)

\PST@Solid(0.3829,0.5902)
(0.3979,0.5902)

\PST@Solid(0.4624,0.5014)
(0.4624,0.7882)

\PST@Solid(0.4549,0.5014)
(0.4699,0.5014)

\PST@Solid(0.4549,0.7882)
(0.4699,0.7882)

\PST@Solid(0.5283,0.5532)
(0.5283,0.8400)

\PST@Solid(0.5208,0.5532)
(0.5358,0.5532)

\PST@Solid(0.5208,0.8400)
(0.5358,0.8400)

\PST@Solid(0.5937,0.6064)
(0.5937,0.8932)

\PST@Solid(0.5862,0.6064)
(0.6012,0.6064)

\PST@Solid(0.5862,0.8932)
(0.6012,0.8932)

\PST@Solid(0.6601,0.5267)
(0.6601,0.8135)

\PST@Solid(0.6526,0.5267)
(0.6676,0.5267)

\PST@Solid(0.6526,0.8135)
(0.6676,0.8135)

\PST@Solid(0.7254,0.2269)
(0.7254,0.5137)

\PST@Solid(0.7179,0.2269)
(0.7329,0.2269)

\PST@Solid(0.7179,0.5137)
(0.7329,0.5137)

\PST@Solid(0.7995,0.2506)
(0.7995,0.5374)

\PST@Solid(0.7920,0.2506)
(0.8070,0.2506)

\PST@Solid(0.7920,0.5374)
(0.8070,0.5374)

\PST@Solid(0.8657,0.4116)
(0.8657,0.6984)

\PST@Solid(0.8582,0.4116)
(0.8732,0.4116)

\PST@Solid(0.8582,0.6984)
(0.8732,0.6984)

\PST@Diamond(0.2563,0.4630)
\PST@Diamond(0.3212,0.4798)
\PST@Diamond(0.3904,0.4468)
\PST@Diamond(0.4624,0.6448)
\PST@Diamond(0.5283,0.6966)
\PST@Diamond(0.5937,0.7498)
\PST@Diamond(0.6601,0.6701)
\PST@Diamond(0.7254,0.3703)
\PST@Diamond(0.7995,0.3940)
\PST@Diamond(0.8657,0.5550)
\PST@Border(0.1750,0.9680)
(0.1750,0.1260)
(0.9470,0.1260)
(0.9470,0.9680)
(0.1750,0.9680)

\catcode`@=12
\fi
\endpspicture

\end{figure}
\end{document}
