\documentclass[italian,a4paper]{article}
\usepackage[tight,nice]{units}
\usepackage{babel,amsmath,amssymb,amsthm,graphicx,url,gensymb}
\usepackage[text={5.5in,9in},centering]{geometry}
\usepackage[utf8x]{inputenc}
%\usepackage[T1]{fontenc}
\usepackage{ae,aecompl}
\usepackage[footnotesize,bf]{caption}
\usepackage[usenames]{color}
\include{pstricks}
\frenchspacing
\pagestyle{plain}
%------------- eliminare prime e ultime linee isolate
\clubpenalty=9999%
\widowpenalty=9999
%--- definizione numerazioni
\renewcommand{\theequation}{\thesection.\arabic{equation}}
\renewcommand{\thefigure}{\arabic{figure}}
\renewcommand{\thetable}{\arabic{table}}
\addto\captionsitalian{%
  \renewcommand{\figurename}%
{Grafico}%
}
%
%------------- ridefinizione simbolo per elenchi puntati: en dash
%\renewcommand{\labelitemi}{\textbf{--}}
\renewcommand{\labelenumi}{\textbf{\arabic{enumi}.}}
\setlength{\abovecaptionskip}{\baselineskip}   % 0.5cm as an example
\setlength{\floatsep}{2\baselineskip}
\setlength{\belowcaptionskip}{\baselineskip}   % 0.5cm as an example
%--------- comandi insiemi numeri complessi, naturali, reali e altre abbreviazioni
\renewcommand{\leq}{\leqslant}
%--------- porzione dedicata ai float in una pagina:
\renewcommand{\textfraction}{0.05}
\renewcommand{\topfraction}{0.95}
\renewcommand{\bottomfraction}{0.95}
\renewcommand{\floatpagefraction}{0.35}
\setcounter{totalnumber}{5}
%---------
%
%---------
\begin{document}
\title{Grafici sulle misure di resistività}
\author{\normalsize Ilaria Brivio (582116)\\%
\normalsize \url{brivio.ilaria@tiscali.it}%
\and %
\normalsize Matteo Abis (584206)\\ %
\normalsize \url{webmaster@latinblog.org}}
\date{\today}
\maketitle
\begin{figure}[p]\caption{Corrente in Ampère in ascissa, tensione in Volt in ordinata. Filo di stablohm del diametro di 0.508mm e lunghezza di 1 metro.}
\centering
% GNUPLOT: LaTeX picture using PSTRICKS macros
% Define new PST objects, if not already defined
\ifx\PSTloaded\undefined
\def\PSTloaded{t}
\psset{arrowsize=.01 3.2 1.4 .3}
\psset{dotsize=.08}
\catcode`@=11

\newpsobject{PST@Border}{psline}{linewidth=.0015,linestyle=solid}
\newpsobject{PST@Axes}{psline}{linewidth=.0015,linestyle=dotted,dotsep=.004}
\newpsobject{PST@Solid}{psline}{linewidth=.0015,linestyle=solid}
\newpsobject{PST@Dashed}{psline}{linewidth=.0015,linestyle=dashed,dash=.01 .01}
\newpsobject{PST@Dotted}{psline}{linewidth=.0025,linestyle=dotted,dotsep=.008}
\newpsobject{PST@LongDash}{psline}{linewidth=.0015,linestyle=dashed,dash=.02 .01}
\newpsobject{PST@Diamond}{psdots}{linewidth=.001,linestyle=solid,dotstyle=square,dotangle=45}
\newpsobject{PST@Filldiamond}{psdots}{linewidth=.001,linestyle=solid,dotstyle=square*,dotangle=45}
\newpsobject{PST@Cross}{psdots}{linewidth=.001,linestyle=solid,dotstyle=+,dotangle=45}
\newpsobject{PST@Plus}{psdots}{linewidth=.001,linestyle=solid,dotstyle=+}
\newpsobject{PST@Square}{psdots}{linewidth=.001,linestyle=solid,dotstyle=square}
\newpsobject{PST@Circle}{psdots}{linewidth=.001,linestyle=solid,dotstyle=o}
\newpsobject{PST@Triangle}{psdots}{linewidth=.001,linestyle=solid,dotstyle=triangle}
\newpsobject{PST@Pentagon}{psdots}{linewidth=.001,linestyle=solid,dotstyle=pentagon}
\newpsobject{PST@Fillsquare}{psdots}{linewidth=.001,linestyle=solid,dotstyle=square*}
\newpsobject{PST@Fillcircle}{psdots}{linewidth=.001,linestyle=solid,dotstyle=*}
\newpsobject{PST@Filltriangle}{psdots}{linewidth=.001,linestyle=solid,dotstyle=triangle*}
\newpsobject{PST@Fillpentagon}{psdots}{linewidth=.001,linestyle=solid,dotstyle=pentagon*}
\newpsobject{PST@Arrow}{psline}{linewidth=.001,linestyle=solid}
\catcode`@=12

\fi
\psset{unit=5.0in,xunit=5.0in,yunit=3.0in}
\pspicture(0.000000,0.000000)(1.000000,1.000000)
\ifx\nofigs\undefined
\catcode`@=11

\PST@Border(0.1010,0.0840)
(0.1160,0.0840)

\rput[r](0.0850,0.0840){1.5}
\PST@Border(0.1010,0.2608)
(0.1160,0.2608)

\rput[r](0.0850,0.2608){2.0}
\PST@Border(0.1010,0.4376)
(0.1160,0.4376)

\rput[r](0.0850,0.4376){2.5}
\PST@Border(0.1010,0.6144)
(0.1160,0.6144)

\rput[r](0.0850,0.6144){3.0}
\PST@Border(0.1010,0.7912)
(0.1160,0.7912)

\rput[r](0.0850,0.7912){3.5}
\PST@Border(0.1010,0.9680)
(0.1160,0.9680)

\rput[r](0.0850,0.9680){4.0}
\PST@Border(0.1010,0.0840)
(0.1010,0.1040)

\rput(0.1010,0.0420){0.25}
\PST@Border(0.2438,0.0840)
(0.2438,0.1040)

\rput(0.2438,0.0420){0.30}
\PST@Border(0.3867,0.0840)
(0.3867,0.1040)

\rput(0.3867,0.0420){0.35}
\PST@Border(0.5295,0.0840)
(0.5295,0.1040)

\rput(0.5295,0.0420){0.40}
\PST@Border(0.6723,0.0840)
(0.6723,0.1040)

\rput(0.6723,0.0420){0.45}
\PST@Border(0.8152,0.0840)
(0.8152,0.1040)

\rput(0.8152,0.0420){0.50}
\PST@Border(0.9580,0.0840)
(0.9580,0.1040)

\rput(0.9580,0.0420){0.55}
\PST@Border(0.1010,0.9680)
(0.1010,0.0840)
(0.9580,0.0840)
(0.9580,0.9680)
(0.1010,0.9680)

\PST@Solid(0.1010,0.1561)
(0.1010,0.1561)
(0.1097,0.1634)
(0.1183,0.1707)
(0.1270,0.1780)
(0.1356,0.1853)
(0.1443,0.1926)
(0.1529,0.2000)
(0.1616,0.2073)
(0.1703,0.2146)
(0.1789,0.2219)
(0.1876,0.2292)
(0.1962,0.2365)
(0.2049,0.2438)
(0.2135,0.2512)
(0.2222,0.2585)
(0.2308,0.2658)
(0.2395,0.2731)
(0.2482,0.2804)
(0.2568,0.2877)
(0.2655,0.2950)
(0.2741,0.3024)
(0.2828,0.3097)
(0.2914,0.3170)
(0.3001,0.3243)
(0.3088,0.3316)
(0.3174,0.3389)
(0.3261,0.3462)
(0.3347,0.3536)
(0.3434,0.3609)
(0.3520,0.3682)
(0.3607,0.3755)
(0.3694,0.3828)
(0.3780,0.3901)
(0.3867,0.3974)
(0.3953,0.4048)
(0.4040,0.4121)
(0.4126,0.4194)
(0.4213,0.4267)
(0.4299,0.4340)
(0.4386,0.4413)
(0.4473,0.4486)
(0.4559,0.4560)
(0.4646,0.4633)
(0.4732,0.4706)
(0.4819,0.4779)
(0.4905,0.4852)
(0.4992,0.4925)
(0.5079,0.4998)
(0.5165,0.5072)
(0.5252,0.5145)
(0.5338,0.5218)
(0.5425,0.5291)
(0.5511,0.5364)
(0.5598,0.5437)
(0.5685,0.5510)
(0.5771,0.5584)
(0.5858,0.5657)
(0.5944,0.5730)
(0.6031,0.5803)
(0.6117,0.5876)
(0.6204,0.5949)
(0.6291,0.6022)
(0.6377,0.6096)
(0.6464,0.6169)
(0.6550,0.6242)
(0.6637,0.6315)
(0.6723,0.6388)
(0.6810,0.6461)
(0.6896,0.6534)
(0.6983,0.6608)
(0.7070,0.6681)
(0.7156,0.6754)
(0.7243,0.6827)
(0.7329,0.6900)
(0.7416,0.6973)
(0.7502,0.7046)
(0.7589,0.7120)
(0.7676,0.7193)
(0.7762,0.7266)
(0.7849,0.7339)
(0.7935,0.7412)
(0.8022,0.7485)
(0.8108,0.7558)
(0.8195,0.7632)
(0.8282,0.7705)
(0.8368,0.7778)
(0.8455,0.7851)
(0.8541,0.7924)
(0.8628,0.7997)
(0.8714,0.8070)
(0.8801,0.8144)
(0.8887,0.8217)
(0.8974,0.8290)
(0.9061,0.8363)
(0.9147,0.8436)
(0.9234,0.8509)
(0.9320,0.8582)
(0.9407,0.8656)
(0.9493,0.8729)
(0.9580,0.8802)

\PST@Dashed(0.8152,0.7565)
(0.8152,0.7622)

\PST@Dashed(0.8077,0.7565)
(0.8227,0.7565)

\PST@Dashed(0.8077,0.7622)
(0.8227,0.7622)

\PST@Dashed(0.6723,0.6367)
(0.6723,0.6416)

\PST@Dashed(0.6648,0.6367)
(0.6798,0.6367)

\PST@Dashed(0.6648,0.6416)
(0.6798,0.6416)

\PST@Dashed(0.5295,0.5165)
(0.5295,0.5214)

\PST@Dashed(0.5220,0.5165)
(0.5370,0.5165)

\PST@Dashed(0.5220,0.5214)
(0.5370,0.5214)

\PST@Dashed(0.3867,0.3962)
(0.3867,0.4012)

\PST@Dashed(0.3792,0.3962)
(0.3942,0.3962)

\PST@Dashed(0.3792,0.4012)
(0.3942,0.4012)

\PST@Dashed(0.2438,0.2760)
(0.2438,0.2810)

\PST@Dashed(0.2363,0.2760)
(0.2513,0.2760)

\PST@Dashed(0.2363,0.2810)
(0.2513,0.2810)

\PST@Dashed(0.3153,0.3326)
(0.3153,0.3375)

\PST@Dashed(0.3078,0.3326)
(0.3228,0.3326)

\PST@Dashed(0.3078,0.3375)
(0.3228,0.3375)

\PST@Dashed(0.4581,0.4528)
(0.4581,0.4578)

\PST@Dashed(0.4506,0.4528)
(0.4656,0.4528)

\PST@Dashed(0.4506,0.4578)
(0.4656,0.4578)

\PST@Dashed(0.6009,0.5766)
(0.6009,0.5815)

\PST@Dashed(0.5934,0.5766)
(0.6084,0.5766)

\PST@Dashed(0.5934,0.5815)
(0.6084,0.5815)

\PST@Diamond(0.8152,0.7594)
\PST@Diamond(0.6723,0.6392)
\PST@Diamond(0.5295,0.5189)
\PST@Diamond(0.3867,0.3987)
\PST@Diamond(0.2438,0.2785)
\PST@Diamond(0.3153,0.3351)
\PST@Diamond(0.4581,0.4553)
\PST@Diamond(0.6009,0.5790)
\PST@Border(0.1010,0.9680)
(0.1010,0.0840)
(0.9580,0.0840)
(0.9580,0.9680)
(0.1010,0.9680)

\catcode`@=12
\fi
\endpspicture

\end{figure}
\begin{figure}[p]\caption{Grafico dei residui. Corrente in Ampère in ascissa, tensione in Volt in ordinata. Filo di stablohm del diametro di 0.508mm e lunghezza di 1 metro.}
\centering
% GNUPLOT: LaTeX picture using PSTRICKS macros
% Define new PST objects, if not already defined
\ifx\PSTloaded\undefined
\def\PSTloaded{t}
\psset{arrowsize=.01 3.2 1.4 .3}
\psset{dotsize=.08}
\catcode`@=11

\newpsobject{PST@Border}{psline}{linewidth=.0015,linestyle=solid}
\newpsobject{PST@Axes}{psline}{linewidth=.0015,linestyle=dotted,dotsep=.004}
\newpsobject{PST@Solid}{psline}{linewidth=.0015,linestyle=solid}
\newpsobject{PST@Dashed}{psline}{linewidth=.0015,linestyle=dashed,dash=.01 .01}
\newpsobject{PST@Dotted}{psline}{linewidth=.0025,linestyle=dotted,dotsep=.008}
\newpsobject{PST@LongDash}{psline}{linewidth=.0015,linestyle=dashed,dash=.02 .01}
\newpsobject{PST@Diamond}{psdots}{linewidth=.001,linestyle=solid,dotstyle=square,dotangle=45}
\newpsobject{PST@Filldiamond}{psdots}{linewidth=.001,linestyle=solid,dotstyle=square*,dotangle=45}
\newpsobject{PST@Cross}{psdots}{linewidth=.001,linestyle=solid,dotstyle=+,dotangle=45}
\newpsobject{PST@Plus}{psdots}{linewidth=.001,linestyle=solid,dotstyle=+}
\newpsobject{PST@Square}{psdots}{linewidth=.001,linestyle=solid,dotstyle=square}
\newpsobject{PST@Circle}{psdots}{linewidth=.001,linestyle=solid,dotstyle=o}
\newpsobject{PST@Triangle}{psdots}{linewidth=.001,linestyle=solid,dotstyle=triangle}
\newpsobject{PST@Pentagon}{psdots}{linewidth=.001,linestyle=solid,dotstyle=pentagon}
\newpsobject{PST@Fillsquare}{psdots}{linewidth=.001,linestyle=solid,dotstyle=square*}
\newpsobject{PST@Fillcircle}{psdots}{linewidth=.001,linestyle=solid,dotstyle=*}
\newpsobject{PST@Filltriangle}{psdots}{linewidth=.001,linestyle=solid,dotstyle=triangle*}
\newpsobject{PST@Fillpentagon}{psdots}{linewidth=.001,linestyle=solid,dotstyle=pentagon*}
\newpsobject{PST@Arrow}{psline}{linewidth=.001,linestyle=solid}
\catcode`@=12

\fi
\psset{unit=5.0in,xunit=5.0in,yunit=3.0in}
\pspicture(0.000000,0.000000)(1.000000,1.000000)
\ifx\nofigs\undefined
\catcode`@=11

\PST@Border(0.1330,0.0840)
(0.1480,0.0840)

\rput[r](0.1170,0.0840){-0.01}
\PST@Border(0.1330,0.2608)
(0.1480,0.2608)

\rput[r](0.1170,0.2608){-0.01}
\PST@Border(0.1330,0.4376)
(0.1480,0.4376)

\rput[r](0.1170,0.4376){-0.00}
\PST@Border(0.1330,0.6144)
(0.1480,0.6144)

\rput[r](0.1170,0.6144){0.00}
\PST@Border(0.1330,0.7912)
(0.1480,0.7912)

\rput[r](0.1170,0.7912){0.01}
\PST@Border(0.1330,0.9680)
(0.1480,0.9680)

\rput[r](0.1170,0.9680){0.01}
\PST@Border(0.1330,0.0840)
(0.1330,0.1040)

\rput(0.1330,0.0420){0.25}
\PST@Border(0.2980,0.0840)
(0.2980,0.1040)

\rput(0.2980,0.0420){0.30}
\PST@Border(0.4630,0.0840)
(0.4630,0.1040)

\rput(0.4630,0.0420){0.35}
\PST@Border(0.6280,0.0840)
(0.6280,0.1040)

\rput(0.6280,0.0420){0.40}
\PST@Border(0.7930,0.0840)
(0.7930,0.1040)

\rput(0.7930,0.0420){0.45}
\PST@Border(0.9580,0.0840)
(0.9580,0.1040)

\rput(0.9580,0.0420){0.50}
\PST@Border(0.1330,0.9680)
(0.1330,0.0840)
(0.9580,0.0840)
(0.9580,0.9680)
(0.1330,0.9680)

\PST@Solid(0.9580,0.3208)
(0.9580,0.6958)

\PST@Solid(0.9505,0.3208)
(0.9655,0.3208)

\PST@Solid(0.9505,0.6958)
(0.9655,0.6958)

\PST@Solid(0.7930,0.3739)
(0.7930,0.7488)

\PST@Solid(0.7855,0.3739)
(0.8005,0.3739)

\PST@Solid(0.7855,0.7488)
(0.8005,0.7488)

\PST@Solid(0.6280,0.4269)
(0.6280,0.8019)

\PST@Solid(0.6205,0.4269)
(0.6355,0.4269)

\PST@Solid(0.6205,0.8019)
(0.6355,0.8019)

\PST@Solid(0.4630,0.4800)
(0.4630,0.8549)

\PST@Solid(0.4555,0.4800)
(0.4705,0.4800)

\PST@Solid(0.4555,0.8549)
(0.4705,0.8549)

\PST@Solid(0.2980,0.5330)
(0.2980,0.9080)

\PST@Solid(0.2905,0.5330)
(0.3055,0.5330)

\PST@Solid(0.2905,0.9080)
(0.3055,0.9080)

\PST@Solid(0.3805,0.1529)
(0.3805,0.5278)

\PST@Solid(0.3730,0.1529)
(0.3880,0.1529)

\PST@Solid(0.3730,0.5278)
(0.3880,0.5278)

\PST@Solid(0.5455,0.0998)
(0.5455,0.4748)

\PST@Solid(0.5380,0.0998)
(0.5530,0.0998)

\PST@Solid(0.5380,0.4748)
(0.5530,0.4748)

\PST@Solid(0.7105,0.4004)
(0.7105,0.7754)

\PST@Solid(0.7030,0.4004)
(0.7180,0.4004)

\PST@Solid(0.7030,0.7754)
(0.7180,0.7754)

\PST@Diamond(0.9580,0.5083)
\PST@Diamond(0.7930,0.5614)
\PST@Diamond(0.6280,0.6144)
\PST@Diamond(0.4630,0.6674)
\PST@Diamond(0.2980,0.7205)
\PST@Diamond(0.3805,0.3404)
\PST@Diamond(0.5455,0.2873)
\PST@Diamond(0.7105,0.5879)
\PST@Border(0.1330,0.9680)
(0.1330,0.0840)
(0.9580,0.0840)
(0.9580,0.9680)
(0.1330,0.9680)

\catcode`@=12
\fi
\endpspicture

\end{figure}
\begin{figure}[p]\caption{Corrente in Ampère in ascissa, tensione in Volt in ordinata. Filo di stablohm del diametro di 0.254mm e lunghezza di 1 metro.}
\centering
% GNUPLOT: LaTeX picture using PSTRICKS macros
% Define new PST objects, if not already defined
\ifx\PSTloaded\undefined
\def\PSTloaded{t}
\psset{arrowsize=.01 3.2 1.4 .3}
\psset{dotsize=.08}
\catcode`@=11

\newpsobject{PST@Border}{psline}{linewidth=.0015,linestyle=solid}
\newpsobject{PST@Axes}{psline}{linewidth=.0015,linestyle=dotted,dotsep=.004}
\newpsobject{PST@Solid}{psline}{linewidth=.0015,linestyle=solid}
\newpsobject{PST@Dashed}{psline}{linewidth=.0015,linestyle=dashed,dash=.01 .01}
\newpsobject{PST@Dotted}{psline}{linewidth=.0025,linestyle=dotted,dotsep=.008}
\newpsobject{PST@LongDash}{psline}{linewidth=.0015,linestyle=dashed,dash=.02 .01}
\newpsobject{PST@Diamond}{psdots}{linewidth=.001,linestyle=solid,dotstyle=square,dotangle=45}
\newpsobject{PST@Filldiamond}{psdots}{linewidth=.001,linestyle=solid,dotstyle=square*,dotangle=45}
\newpsobject{PST@Cross}{psdots}{linewidth=.001,linestyle=solid,dotstyle=+,dotangle=45}
\newpsobject{PST@Plus}{psdots}{linewidth=.001,linestyle=solid,dotstyle=+}
\newpsobject{PST@Square}{psdots}{linewidth=.001,linestyle=solid,dotstyle=square}
\newpsobject{PST@Circle}{psdots}{linewidth=.001,linestyle=solid,dotstyle=o}
\newpsobject{PST@Triangle}{psdots}{linewidth=.001,linestyle=solid,dotstyle=triangle}
\newpsobject{PST@Pentagon}{psdots}{linewidth=.001,linestyle=solid,dotstyle=pentagon}
\newpsobject{PST@Fillsquare}{psdots}{linewidth=.001,linestyle=solid,dotstyle=square*}
\newpsobject{PST@Fillcircle}{psdots}{linewidth=.001,linestyle=solid,dotstyle=*}
\newpsobject{PST@Filltriangle}{psdots}{linewidth=.001,linestyle=solid,dotstyle=triangle*}
\newpsobject{PST@Fillpentagon}{psdots}{linewidth=.001,linestyle=solid,dotstyle=pentagon*}
\newpsobject{PST@Arrow}{psline}{linewidth=.001,linestyle=solid}
\catcode`@=12

\fi
\psset{unit=5.0in,xunit=5.0in,yunit=3.0in}
\pspicture(0.000000,0.000000)(1.000000,1.000000)
\ifx\nofigs\undefined
\catcode`@=11

\PST@Border(0.1010,0.0840)
(0.1160,0.0840)

\rput[r](0.0850,0.0840){1.6}
\PST@Border(0.1010,0.1822)
(0.1160,0.1822)

\rput[r](0.0850,0.1822){1.8}
\PST@Border(0.1010,0.2804)
(0.1160,0.2804)

\rput[r](0.0850,0.2804){2.0}
\PST@Border(0.1010,0.3787)
(0.1160,0.3787)

\rput[r](0.0850,0.3787){2.2}
\PST@Border(0.1010,0.4769)
(0.1160,0.4769)

\rput[r](0.0850,0.4769){2.4}
\PST@Border(0.1010,0.5751)
(0.1160,0.5751)

\rput[r](0.0850,0.5751){2.6}
\PST@Border(0.1010,0.6733)
(0.1160,0.6733)

\rput[r](0.0850,0.6733){2.8}
\PST@Border(0.1010,0.7716)
(0.1160,0.7716)

\rput[r](0.0850,0.7716){3.0}
\PST@Border(0.1010,0.8698)
(0.1160,0.8698)

\rput[r](0.0850,0.8698){3.2}
\PST@Border(0.1010,0.9680)
(0.1160,0.9680)

\rput[r](0.0850,0.9680){3.4}
\PST@Border(0.1724,0.0840)
(0.1724,0.1040)

\rput(0.1724,0.0420){0.07}
\PST@Border(0.3152,0.0840)
(0.3152,0.1040)

\rput(0.3152,0.0420){0.08}
\PST@Border(0.4581,0.0840)
(0.4581,0.1040)

\rput(0.4581,0.0420){0.09}
\PST@Border(0.6009,0.0840)
(0.6009,0.1040)

\rput(0.6009,0.0420){0.10}
\PST@Border(0.7437,0.0840)
(0.7437,0.1040)

\rput(0.7437,0.0420){0.11}
\PST@Border(0.8866,0.0840)
(0.8866,0.1040)

\rput(0.8866,0.0420){0.12}
\PST@Border(0.1010,0.9680)
(0.1010,0.0840)
(0.9580,0.0840)
(0.9580,0.9680)
(0.1010,0.9680)

\PST@Solid(0.1010,0.1614)
(0.1010,0.1614)
(0.1097,0.1695)
(0.1183,0.1775)
(0.1270,0.1856)
(0.1356,0.1937)
(0.1443,0.2018)
(0.1529,0.2099)
(0.1616,0.2180)
(0.1703,0.2260)
(0.1789,0.2341)
(0.1876,0.2422)
(0.1962,0.2503)
(0.2049,0.2584)
(0.2135,0.2664)
(0.2222,0.2745)
(0.2308,0.2826)
(0.2395,0.2907)
(0.2482,0.2988)
(0.2568,0.3069)
(0.2655,0.3149)
(0.2741,0.3230)
(0.2828,0.3311)
(0.2914,0.3392)
(0.3001,0.3473)
(0.3088,0.3554)
(0.3174,0.3634)
(0.3261,0.3715)
(0.3347,0.3796)
(0.3434,0.3877)
(0.3520,0.3958)
(0.3607,0.4038)
(0.3694,0.4119)
(0.3780,0.4200)
(0.3867,0.4281)
(0.3953,0.4362)
(0.4040,0.4443)
(0.4126,0.4523)
(0.4213,0.4604)
(0.4299,0.4685)
(0.4386,0.4766)
(0.4473,0.4847)
(0.4559,0.4928)
(0.4646,0.5008)
(0.4732,0.5089)
(0.4819,0.5170)
(0.4905,0.5251)
(0.4992,0.5332)
(0.5079,0.5412)
(0.5165,0.5493)
(0.5252,0.5574)
(0.5338,0.5655)
(0.5425,0.5736)
(0.5511,0.5817)
(0.5598,0.5897)
(0.5685,0.5978)
(0.5771,0.6059)
(0.5858,0.6140)
(0.5944,0.6221)
(0.6031,0.6302)
(0.6117,0.6382)
(0.6204,0.6463)
(0.6291,0.6544)
(0.6377,0.6625)
(0.6464,0.6706)
(0.6550,0.6786)
(0.6637,0.6867)
(0.6723,0.6948)
(0.6810,0.7029)
(0.6896,0.7110)
(0.6983,0.7191)
(0.7070,0.7271)
(0.7156,0.7352)
(0.7243,0.7433)
(0.7329,0.7514)
(0.7416,0.7595)
(0.7502,0.7676)
(0.7589,0.7756)
(0.7676,0.7837)
(0.7762,0.7918)
(0.7849,0.7999)
(0.7935,0.8080)
(0.8022,0.8160)
(0.8108,0.8241)
(0.8195,0.8322)
(0.8282,0.8403)
(0.8368,0.8484)
(0.8455,0.8565)
(0.8541,0.8645)
(0.8628,0.8726)
(0.8714,0.8807)
(0.8801,0.8888)
(0.8887,0.8969)
(0.8974,0.9050)
(0.9061,0.9130)
(0.9147,0.9211)
(0.9234,0.9292)
(0.9320,0.9373)
(0.9407,0.9454)
(0.9493,0.9534)
(0.9580,0.9615)

\PST@Dashed(0.8866,0.8909)
(0.8866,0.8978)

\PST@Dashed(0.8791,0.8909)
(0.8941,0.8909)

\PST@Dashed(0.8791,0.8978)
(0.8941,0.8978)

\PST@Dashed(0.7438,0.7583)
(0.7438,0.7652)

\PST@Dashed(0.7363,0.7583)
(0.7513,0.7583)

\PST@Dashed(0.7363,0.7652)
(0.7513,0.7652)

\PST@Dashed(0.6009,0.6257)
(0.6009,0.6326)

\PST@Dashed(0.5934,0.6257)
(0.6084,0.6257)

\PST@Dashed(0.5934,0.6326)
(0.6084,0.6326)

\PST@Dashed(0.4724,0.5029)
(0.4724,0.5098)

\PST@Dashed(0.4649,0.5029)
(0.4799,0.5029)

\PST@Dashed(0.4649,0.5098)
(0.4799,0.5098)

\PST@Dashed(0.3867,0.4293)
(0.3867,0.4361)

\PST@Dashed(0.3792,0.4293)
(0.3942,0.4293)

\PST@Dashed(0.3792,0.4361)
(0.3942,0.4361)

\PST@Dashed(0.3153,0.3556)
(0.3153,0.3625)

\PST@Dashed(0.3078,0.3556)
(0.3228,0.3556)

\PST@Dashed(0.3078,0.3625)
(0.3228,0.3625)

\PST@Dashed(0.2438,0.2873)
(0.2438,0.2932)

\PST@Dashed(0.2363,0.2873)
(0.2513,0.2873)

\PST@Dashed(0.2363,0.2932)
(0.2513,0.2932)

\PST@Dashed(0.1724,0.2284)
(0.1724,0.2343)

\PST@Dashed(0.1649,0.2284)
(0.1799,0.2284)

\PST@Dashed(0.1649,0.2343)
(0.1799,0.2343)

\PST@Diamond(0.8866,0.8943)
\PST@Diamond(0.7438,0.7617)
\PST@Diamond(0.6009,0.6291)
\PST@Diamond(0.4724,0.5064)
\PST@Diamond(0.3867,0.4327)
\PST@Diamond(0.3153,0.3590)
\PST@Diamond(0.2438,0.2903)
\PST@Diamond(0.1724,0.2313)
\PST@Border(0.1010,0.9680)
(0.1010,0.0840)
(0.9580,0.0840)
(0.9580,0.9680)
(0.1010,0.9680)

\catcode`@=12
\fi
\endpspicture

\end{figure}
\begin{figure}[p]\caption{Grafico dei residui. Corrente in Ampère in ascissa, tensione in Volt in ordinata. Filo di stablohm del diametro di 0.254mm e lunghezza di 1 metro.}
\centering
% GNUPLOT: LaTeX picture using PSTRICKS macros
% Define new PST objects, if not already defined
\ifx\PSTloaded\undefined
\def\PSTloaded{t}
\psset{arrowsize=.01 3.2 1.4 .3}
\psset{dotsize=.08}
\catcode`@=11

\newpsobject{PST@Border}{psline}{linewidth=.0015,linestyle=solid}
\newpsobject{PST@Axes}{psline}{linewidth=.0015,linestyle=dotted,dotsep=.004}
\newpsobject{PST@Solid}{psline}{linewidth=.0015,linestyle=solid}
\newpsobject{PST@Dashed}{psline}{linewidth=.0015,linestyle=dashed,dash=.01 .01}
\newpsobject{PST@Dotted}{psline}{linewidth=.0025,linestyle=dotted,dotsep=.008}
\newpsobject{PST@LongDash}{psline}{linewidth=.0015,linestyle=dashed,dash=.02 .01}
\newpsobject{PST@Diamond}{psdots}{linewidth=.001,linestyle=solid,dotstyle=square,dotangle=45}
\newpsobject{PST@Filldiamond}{psdots}{linewidth=.001,linestyle=solid,dotstyle=square*,dotangle=45}
\newpsobject{PST@Cross}{psdots}{linewidth=.001,linestyle=solid,dotstyle=+,dotangle=45}
\newpsobject{PST@Plus}{psdots}{linewidth=.001,linestyle=solid,dotstyle=+}
\newpsobject{PST@Square}{psdots}{linewidth=.001,linestyle=solid,dotstyle=square}
\newpsobject{PST@Circle}{psdots}{linewidth=.001,linestyle=solid,dotstyle=o}
\newpsobject{PST@Triangle}{psdots}{linewidth=.001,linestyle=solid,dotstyle=triangle}
\newpsobject{PST@Pentagon}{psdots}{linewidth=.001,linestyle=solid,dotstyle=pentagon}
\newpsobject{PST@Fillsquare}{psdots}{linewidth=.001,linestyle=solid,dotstyle=square*}
\newpsobject{PST@Fillcircle}{psdots}{linewidth=.001,linestyle=solid,dotstyle=*}
\newpsobject{PST@Filltriangle}{psdots}{linewidth=.001,linestyle=solid,dotstyle=triangle*}
\newpsobject{PST@Fillpentagon}{psdots}{linewidth=.001,linestyle=solid,dotstyle=pentagon*}
\newpsobject{PST@Arrow}{psline}{linewidth=.001,linestyle=solid}
\catcode`@=12

\fi
\psset{unit=5.0in,xunit=5.0in,yunit=3.0in}
\pspicture(0.000000,0.000000)(1.000000,1.000000)
\ifx\nofigs\undefined
\catcode`@=11

\PST@Border(0.1490,0.0840)
(0.1640,0.0840)

\rput[r](0.1330,0.0840){-0.020}
\PST@Border(0.1490,0.1945)
(0.1640,0.1945)

\rput[r](0.1330,0.1945){-0.015}
\PST@Border(0.1490,0.3050)
(0.1640,0.3050)

\rput[r](0.1330,0.3050){-0.010}
\PST@Border(0.1490,0.4155)
(0.1640,0.4155)

\rput[r](0.1330,0.4155){-0.005}
\PST@Border(0.1490,0.5260)
(0.1640,0.5260)

\rput[r](0.1330,0.5260){0.000}
\PST@Border(0.1490,0.6365)
(0.1640,0.6365)

\rput[r](0.1330,0.6365){0.005}
\PST@Border(0.1490,0.7470)
(0.1640,0.7470)

\rput[r](0.1330,0.7470){0.010}
\PST@Border(0.1490,0.8575)
(0.1640,0.8575)

\rput[r](0.1330,0.8575){0.015}
\PST@Border(0.1490,0.9680)
(0.1640,0.9680)

\rput[r](0.1330,0.9680){0.020}
\PST@Border(0.1490,0.0840)
(0.1490,0.1040)

\rput(0.1490,0.0420){0.06}
\PST@Border(0.2646,0.0840)
(0.2646,0.1040)

\rput(0.2646,0.0420){0.07}
\PST@Border(0.3801,0.0840)
(0.3801,0.1040)

\rput(0.3801,0.0420){0.08}
\PST@Border(0.4957,0.0840)
(0.4957,0.1040)

\rput(0.4957,0.0420){0.09}
\PST@Border(0.6113,0.0840)
(0.6113,0.1040)

\rput(0.6113,0.0420){0.10}
\PST@Border(0.7269,0.0840)
(0.7269,0.1040)

\rput(0.7269,0.0420){0.11}
\PST@Border(0.8424,0.0840)
(0.8424,0.1040)

\rput(0.8424,0.0420){0.12}
\PST@Border(0.9580,0.0840)
(0.9580,0.1040)

\rput(0.9580,0.0420){0.13}
\PST@Border(0.1490,0.9680)
(0.1490,0.0840)
(0.9580,0.0840)
(0.9580,0.9680)
(0.1490,0.9680)

\PST@Solid(0.8424,0.3593)
(0.8424,0.6485)

\PST@Solid(0.8349,0.3593)
(0.8499,0.3593)

\PST@Solid(0.8349,0.6485)
(0.8499,0.6485)

\PST@Solid(0.7269,0.3924)
(0.7269,0.6817)

\PST@Solid(0.7194,0.3924)
(0.7344,0.3924)

\PST@Solid(0.7194,0.6817)
(0.7344,0.6817)

\PST@Solid(0.6113,0.4256)
(0.6113,0.7148)

\PST@Solid(0.6038,0.4256)
(0.6188,0.4256)

\PST@Solid(0.6038,0.7148)
(0.6188,0.7148)

\PST@Solid(0.5073,0.3007)
(0.5073,0.5900)

\PST@Solid(0.4998,0.3007)
(0.5148,0.3007)

\PST@Solid(0.4998,0.5900)
(0.5148,0.5900)

\PST@Solid(0.4379,0.5858)
(0.4379,0.8750)

\PST@Solid(0.4304,0.5858)
(0.4454,0.5858)

\PST@Solid(0.4304,0.8750)
(0.4454,0.8750)

\PST@Solid(0.3801,0.2709)
(0.3801,0.5601)

\PST@Solid(0.3726,0.2709)
(0.3876,0.2709)

\PST@Solid(0.3726,0.5601)
(0.3876,0.5601)

\PST@Solid(0.3224,0.1770)
(0.3224,0.4662)

\PST@Solid(0.3149,0.1770)
(0.3299,0.1770)

\PST@Solid(0.3149,0.4662)
(0.3299,0.4662)

\PST@Solid(0.2646,0.5250)
(0.2646,0.8143)

\PST@Solid(0.2571,0.5250)
(0.2721,0.5250)

\PST@Solid(0.2571,0.8143)
(0.2721,0.8143)

\PST@Diamond(0.8424,0.5039)
\PST@Diamond(0.7269,0.5371)
\PST@Diamond(0.6113,0.5702)
\PST@Diamond(0.5073,0.4453)
\PST@Diamond(0.4379,0.7304)
\PST@Diamond(0.3801,0.4155)
\PST@Diamond(0.3224,0.3216)
\PST@Diamond(0.2646,0.6697)
\PST@Border(0.1490,0.9680)
(0.1490,0.0840)
(0.9580,0.0840)
(0.9580,0.9680)
(0.1490,0.9680)

\catcode`@=12
\fi
\endpspicture

\end{figure}
\begin{figure}[p]\caption{Corrente in Ampère in ascissa, tensione in Volt in ordinata. Filo di stablohm del diametro di 1.016mm e lunghezza di 1 metro.}
\centering
% GNUPLOT: LaTeX picture using PSTRICKS macros
% Define new PST objects, if not already defined
\ifx\PSTloaded\undefined
\def\PSTloaded{t}
\psset{arrowsize=.01 3.2 1.4 .3}
\psset{dotsize=.08}
\catcode`@=11

\newpsobject{PST@Border}{psline}{linewidth=.0015,linestyle=solid}
\newpsobject{PST@Axes}{psline}{linewidth=.0015,linestyle=dotted,dotsep=.004}
\newpsobject{PST@Solid}{psline}{linewidth=.0015,linestyle=solid}
\newpsobject{PST@Dashed}{psline}{linewidth=.0015,linestyle=dashed,dash=.01 .01}
\newpsobject{PST@Dotted}{psline}{linewidth=.0025,linestyle=dotted,dotsep=.008}
\newpsobject{PST@LongDash}{psline}{linewidth=.0015,linestyle=dashed,dash=.02 .01}
\newpsobject{PST@Diamond}{psdots}{linewidth=.001,linestyle=solid,dotstyle=square,dotangle=45}
\newpsobject{PST@Filldiamond}{psdots}{linewidth=.001,linestyle=solid,dotstyle=square*,dotangle=45}
\newpsobject{PST@Cross}{psdots}{linewidth=.001,linestyle=solid,dotstyle=+,dotangle=45}
\newpsobject{PST@Plus}{psdots}{linewidth=.001,linestyle=solid,dotstyle=+}
\newpsobject{PST@Square}{psdots}{linewidth=.001,linestyle=solid,dotstyle=square}
\newpsobject{PST@Circle}{psdots}{linewidth=.001,linestyle=solid,dotstyle=o}
\newpsobject{PST@Triangle}{psdots}{linewidth=.001,linestyle=solid,dotstyle=triangle}
\newpsobject{PST@Pentagon}{psdots}{linewidth=.001,linestyle=solid,dotstyle=pentagon}
\newpsobject{PST@Fillsquare}{psdots}{linewidth=.001,linestyle=solid,dotstyle=square*}
\newpsobject{PST@Fillcircle}{psdots}{linewidth=.001,linestyle=solid,dotstyle=*}
\newpsobject{PST@Filltriangle}{psdots}{linewidth=.001,linestyle=solid,dotstyle=triangle*}
\newpsobject{PST@Fillpentagon}{psdots}{linewidth=.001,linestyle=solid,dotstyle=pentagon*}
\newpsobject{PST@Arrow}{psline}{linewidth=.001,linestyle=solid}
\catcode`@=12

\fi
\psset{unit=5.0in,xunit=5.0in,yunit=3.0in}
\pspicture(0.000000,0.000000)(1.000000,1.000000)
\ifx\nofigs\undefined
\catcode`@=11

\PST@Border(0.1010,0.0840)
(0.1160,0.0840)

\rput[r](0.0850,0.0840){0.4}
\PST@Border(0.1010,0.2313)
(0.1160,0.2313)

\rput[r](0.0850,0.2313){0.5}
\PST@Border(0.1010,0.3787)
(0.1160,0.3787)

\rput[r](0.0850,0.3787){0.6}
\PST@Border(0.1010,0.5260)
(0.1160,0.5260)

\rput[r](0.0850,0.5260){0.7}
\PST@Border(0.1010,0.6733)
(0.1160,0.6733)

\rput[r](0.0850,0.6733){0.8}
\PST@Border(0.1010,0.8207)
(0.1160,0.8207)

\rput[r](0.0850,0.8207){0.9}
\PST@Border(0.1010,0.9680)
(0.1160,0.9680)

\rput[r](0.0850,0.9680){1.0}
\PST@Border(0.1010,0.0840)
(0.1010,0.1040)

\rput(0.1010,0.0420){0.25}
\PST@Border(0.2438,0.0840)
(0.2438,0.1040)

\rput(0.2438,0.0420){0.30}
\PST@Border(0.3867,0.0840)
(0.3867,0.1040)

\rput(0.3867,0.0420){0.35}
\PST@Border(0.5295,0.0840)
(0.5295,0.1040)

\rput(0.5295,0.0420){0.40}
\PST@Border(0.6723,0.0840)
(0.6723,0.1040)

\rput(0.6723,0.0420){0.45}
\PST@Border(0.8152,0.0840)
(0.8152,0.1040)

\rput(0.8152,0.0420){0.50}
\PST@Border(0.9580,0.0840)
(0.9580,0.1040)

\rput(0.9580,0.0420){0.55}
\PST@Border(0.1010,0.9680)
(0.1010,0.0840)
(0.9580,0.0840)
(0.9580,0.9680)
(0.1010,0.9680)

\PST@Solid(0.1010,0.1246)
(0.1010,0.1246)
(0.1097,0.1323)
(0.1183,0.1400)
(0.1270,0.1478)
(0.1356,0.1555)
(0.1443,0.1632)
(0.1529,0.1709)
(0.1616,0.1787)
(0.1703,0.1864)
(0.1789,0.1941)
(0.1876,0.2018)
(0.1962,0.2096)
(0.2049,0.2173)
(0.2135,0.2250)
(0.2222,0.2328)
(0.2308,0.2405)
(0.2395,0.2482)
(0.2482,0.2559)
(0.2568,0.2637)
(0.2655,0.2714)
(0.2741,0.2791)
(0.2828,0.2868)
(0.2914,0.2946)
(0.3001,0.3023)
(0.3088,0.3100)
(0.3174,0.3177)
(0.3261,0.3255)
(0.3347,0.3332)
(0.3434,0.3409)
(0.3520,0.3486)
(0.3607,0.3564)
(0.3694,0.3641)
(0.3780,0.3718)
(0.3867,0.3795)
(0.3953,0.3873)
(0.4040,0.3950)
(0.4126,0.4027)
(0.4213,0.4104)
(0.4299,0.4182)
(0.4386,0.4259)
(0.4473,0.4336)
(0.4559,0.4414)
(0.4646,0.4491)
(0.4732,0.4568)
(0.4819,0.4645)
(0.4905,0.4723)
(0.4992,0.4800)
(0.5079,0.4877)
(0.5165,0.4954)
(0.5252,0.5032)
(0.5338,0.5109)
(0.5425,0.5186)
(0.5511,0.5263)
(0.5598,0.5341)
(0.5685,0.5418)
(0.5771,0.5495)
(0.5858,0.5572)
(0.5944,0.5650)
(0.6031,0.5727)
(0.6117,0.5804)
(0.6204,0.5881)
(0.6291,0.5959)
(0.6377,0.6036)
(0.6464,0.6113)
(0.6550,0.6190)
(0.6637,0.6268)
(0.6723,0.6345)
(0.6810,0.6422)
(0.6896,0.6500)
(0.6983,0.6577)
(0.7070,0.6654)
(0.7156,0.6731)
(0.7243,0.6809)
(0.7329,0.6886)
(0.7416,0.6963)
(0.7502,0.7040)
(0.7589,0.7118)
(0.7676,0.7195)
(0.7762,0.7272)
(0.7849,0.7349)
(0.7935,0.7427)
(0.8022,0.7504)
(0.8108,0.7581)
(0.8195,0.7658)
(0.8282,0.7736)
(0.8368,0.7813)
(0.8455,0.7890)
(0.8541,0.7967)
(0.8628,0.8045)
(0.8714,0.8122)
(0.8801,0.8199)
(0.8887,0.8276)
(0.8974,0.8354)
(0.9061,0.8431)
(0.9147,0.8508)
(0.9234,0.8586)
(0.9320,0.8663)
(0.9407,0.8740)
(0.9493,0.8817)
(0.9580,0.8895)

\PST@Dashed(0.8152,0.7641)
(0.8152,0.7682)

\PST@Dashed(0.8077,0.7641)
(0.8227,0.7641)

\PST@Dashed(0.8077,0.7682)
(0.8227,0.7682)

\PST@Dashed(0.7437,0.6935)
(0.7437,0.6973)

\PST@Dashed(0.7362,0.6935)
(0.7512,0.6935)

\PST@Dashed(0.7362,0.6973)
(0.7512,0.6973)

\PST@Dashed(0.6723,0.6316)
(0.6723,0.6355)

\PST@Dashed(0.6648,0.6316)
(0.6798,0.6316)

\PST@Dashed(0.6648,0.6355)
(0.6798,0.6355)

\PST@Dashed(0.6009,0.5684)
(0.6009,0.5720)

\PST@Dashed(0.5934,0.5684)
(0.6084,0.5684)

\PST@Dashed(0.5934,0.5720)
(0.6084,0.5720)

\PST@Dashed(0.5295,0.5051)
(0.5295,0.5086)

\PST@Dashed(0.5220,0.5051)
(0.5370,0.5051)

\PST@Dashed(0.5220,0.5086)
(0.5370,0.5086)

\PST@Dashed(0.4581,0.4389)
(0.4581,0.4422)

\PST@Dashed(0.4506,0.4389)
(0.4656,0.4389)

\PST@Dashed(0.4506,0.4422)
(0.4656,0.4422)

\PST@Dashed(0.3867,0.3787)
(0.3867,0.3816)

\PST@Dashed(0.3792,0.3787)
(0.3942,0.3787)

\PST@Dashed(0.3792,0.3816)
(0.3942,0.3816)

\PST@Dashed(0.3153,0.3168)
(0.3153,0.3197)

\PST@Dashed(0.3078,0.3168)
(0.3228,0.3168)

\PST@Dashed(0.3078,0.3197)
(0.3228,0.3197)

\PST@Diamond(0.8152,0.7662)
\PST@Diamond(0.7437,0.6954)
\PST@Diamond(0.6723,0.6336)
\PST@Diamond(0.6009,0.5702)
\PST@Diamond(0.5295,0.5068)
\PST@Diamond(0.4581,0.4405)
\PST@Diamond(0.3867,0.3801)
\PST@Diamond(0.3153,0.3183)
\PST@Border(0.1010,0.9680)
(0.1010,0.0840)
(0.9580,0.0840)
(0.9580,0.9680)
(0.1010,0.9680)

\catcode`@=12
\fi
\endpspicture

\end{figure}
\begin{figure}[p]\caption{Grafico dei residui. Corrente in Ampère in ascissa, tensione in Volt in ordinata. Filo di stablohm del diametro di 1.016mm e lunghezza di 1 metro.}
\centering
% GNUPLOT: LaTeX picture using PSTRICKS macros
% Define new PST objects, if not already defined
\ifx\PSTloaded\undefined
\def\PSTloaded{t}
\psset{arrowsize=.01 3.2 1.4 .3}
\psset{dotsize=.08}
\catcode`@=11

\newpsobject{PST@Border}{psline}{linewidth=.0015,linestyle=solid}
\newpsobject{PST@Axes}{psline}{linewidth=.0015,linestyle=dotted,dotsep=.004}
\newpsobject{PST@Solid}{psline}{linewidth=.0015,linestyle=solid}
\newpsobject{PST@Dashed}{psline}{linewidth=.0015,linestyle=dashed,dash=.01 .01}
\newpsobject{PST@Dotted}{psline}{linewidth=.0025,linestyle=dotted,dotsep=.008}
\newpsobject{PST@LongDash}{psline}{linewidth=.0015,linestyle=dashed,dash=.02 .01}
\newpsobject{PST@Diamond}{psdots}{linewidth=.001,linestyle=solid,dotstyle=square,dotangle=45}
\newpsobject{PST@Filldiamond}{psdots}{linewidth=.001,linestyle=solid,dotstyle=square*,dotangle=45}
\newpsobject{PST@Cross}{psdots}{linewidth=.001,linestyle=solid,dotstyle=+,dotangle=45}
\newpsobject{PST@Plus}{psdots}{linewidth=.001,linestyle=solid,dotstyle=+}
\newpsobject{PST@Square}{psdots}{linewidth=.001,linestyle=solid,dotstyle=square}
\newpsobject{PST@Circle}{psdots}{linewidth=.001,linestyle=solid,dotstyle=o}
\newpsobject{PST@Triangle}{psdots}{linewidth=.001,linestyle=solid,dotstyle=triangle}
\newpsobject{PST@Pentagon}{psdots}{linewidth=.001,linestyle=solid,dotstyle=pentagon}
\newpsobject{PST@Fillsquare}{psdots}{linewidth=.001,linestyle=solid,dotstyle=square*}
\newpsobject{PST@Fillcircle}{psdots}{linewidth=.001,linestyle=solid,dotstyle=*}
\newpsobject{PST@Filltriangle}{psdots}{linewidth=.001,linestyle=solid,dotstyle=triangle*}
\newpsobject{PST@Fillpentagon}{psdots}{linewidth=.001,linestyle=solid,dotstyle=pentagon*}
\newpsobject{PST@Arrow}{psline}{linewidth=.001,linestyle=solid}
\catcode`@=12

\fi
\psset{unit=5.0in,xunit=5.0in,yunit=3.0in}
\pspicture(0.000000,0.000000)(1.000000,1.000000)
\ifx\nofigs\undefined
\catcode`@=11

\PST@Border(0.1490,0.0840)
(0.1640,0.0840)

\rput[r](0.1330,0.0840){-0.004}
\PST@Border(0.1490,0.1822)
(0.1640,0.1822)

\rput[r](0.1330,0.1822){-0.003}
\PST@Border(0.1490,0.2804)
(0.1640,0.2804)

\rput[r](0.1330,0.2804){-0.002}
\PST@Border(0.1490,0.3787)
(0.1640,0.3787)

\rput[r](0.1330,0.3787){-0.001}
\PST@Border(0.1490,0.4769)
(0.1640,0.4769)

\rput[r](0.1330,0.4769){0.000}
\PST@Border(0.1490,0.5751)
(0.1640,0.5751)

\rput[r](0.1330,0.5751){0.001}
\PST@Border(0.1490,0.6733)
(0.1640,0.6733)

\rput[r](0.1330,0.6733){0.002}
\PST@Border(0.1490,0.7716)
(0.1640,0.7716)

\rput[r](0.1330,0.7716){0.003}
\PST@Border(0.1490,0.8698)
(0.1640,0.8698)

\rput[r](0.1330,0.8698){0.004}
\PST@Border(0.1490,0.9680)
(0.1640,0.9680)

\rput[r](0.1330,0.9680){0.005}
\PST@Border(0.1490,0.0840)
(0.1490,0.1040)

\rput(0.1490,0.0420){0.30}
\PST@Border(0.3108,0.0840)
(0.3108,0.1040)

\rput(0.3108,0.0420){0.35}
\PST@Border(0.4726,0.0840)
(0.4726,0.1040)

\rput(0.4726,0.0420){0.40}
\PST@Border(0.6344,0.0840)
(0.6344,0.1040)

\rput(0.6344,0.0420){0.45}
\PST@Border(0.7962,0.0840)
(0.7962,0.1040)

\rput(0.7962,0.0420){0.50}
\PST@Border(0.9580,0.0840)
(0.9580,0.1040)

\rput(0.9580,0.0420){0.55}
\PST@Border(0.1490,0.9680)
(0.1490,0.0840)
(0.9580,0.0840)
(0.9580,0.9680)
(0.1490,0.9680)

\PST@Solid(0.7962,0.5979)
(0.7962,0.9452)

\PST@Solid(0.7887,0.5979)
(0.8037,0.5979)

\PST@Solid(0.7887,0.9452)
(0.8037,0.9452)

\PST@Solid(0.7153,0.1314)
(0.7153,0.4786)

\PST@Solid(0.7078,0.1314)
(0.7228,0.1314)

\PST@Solid(0.7078,0.4786)
(0.7228,0.4786)

\PST@Solid(0.6344,0.2541)
(0.6344,0.6014)

\PST@Solid(0.6269,0.2541)
(0.6419,0.2541)

\PST@Solid(0.6269,0.6014)
(0.6419,0.6014)

\PST@Solid(0.5535,0.2787)
(0.5535,0.6260)

\PST@Solid(0.5460,0.2787)
(0.5610,0.2787)

\PST@Solid(0.5460,0.6260)
(0.5610,0.6260)

\PST@Solid(0.4726,0.3033)
(0.4726,0.6505)

\PST@Solid(0.4651,0.3033)
(0.4801,0.3033)

\PST@Solid(0.4651,0.6505)
(0.4801,0.6505)

\PST@Solid(0.3917,0.1314)
(0.3917,0.4786)

\PST@Solid(0.3842,0.1314)
(0.3992,0.1314)

\PST@Solid(0.3842,0.4786)
(0.3992,0.4786)

\PST@Solid(0.3108,0.3524)
(0.3108,0.6996)

\PST@Solid(0.3033,0.3524)
(0.3183,0.3524)

\PST@Solid(0.3033,0.6996)
(0.3183,0.6996)

\PST@Solid(0.2299,0.4751)
(0.2299,0.8224)

\PST@Solid(0.2224,0.4751)
(0.2374,0.4751)

\PST@Solid(0.2224,0.8224)
(0.2374,0.8224)

\PST@Diamond(0.7962,0.7716)
\PST@Diamond(0.7153,0.3050)
\PST@Diamond(0.6344,0.4278)
\PST@Diamond(0.5535,0.4523)
\PST@Diamond(0.4726,0.4769)
\PST@Diamond(0.3917,0.3050)
\PST@Diamond(0.3108,0.5260)
\PST@Diamond(0.2299,0.6488)
\PST@Border(0.1490,0.9680)
(0.1490,0.0840)
(0.9580,0.0840)
(0.9580,0.9680)
(0.1490,0.9680)

\catcode`@=12
\fi
\endpspicture

\end{figure}
\begin{figure}[p]\caption{Misura della differenza di potenziale a vari intervalli di lunghezza del filo, con corrente costante. Posizione in ascissa (metri), differenza di potenziale in ordinata (Volt).}
\centering
% GNUPLOT: LaTeX picture using PSTRICKS macros
% Define new PST objects, if not already defined
\ifx\PSTloaded\undefined
\def\PSTloaded{t}
\psset{arrowsize=.01 3.2 1.4 .3}
\psset{dotsize=.08}
\catcode`@=11

\newpsobject{PST@Border}{psline}{linewidth=.0015,linestyle=solid}
\newpsobject{PST@Axes}{psline}{linewidth=.0015,linestyle=dotted,dotsep=.004}
\newpsobject{PST@Solid}{psline}{linewidth=.0015,linestyle=solid}
\newpsobject{PST@Dashed}{psline}{linewidth=.0015,linestyle=dashed,dash=.01 .01}
\newpsobject{PST@Dotted}{psline}{linewidth=.0025,linestyle=dotted,dotsep=.008}
\newpsobject{PST@LongDash}{psline}{linewidth=.0015,linestyle=dashed,dash=.02 .01}
\newpsobject{PST@Diamond}{psdots}{linewidth=.001,linestyle=solid,dotstyle=square,dotangle=45}
\newpsobject{PST@Filldiamond}{psdots}{linewidth=.001,linestyle=solid,dotstyle=square*,dotangle=45}
\newpsobject{PST@Cross}{psdots}{linewidth=.001,linestyle=solid,dotstyle=+,dotangle=45}
\newpsobject{PST@Plus}{psdots}{linewidth=.001,linestyle=solid,dotstyle=+}
\newpsobject{PST@Square}{psdots}{linewidth=.001,linestyle=solid,dotstyle=square}
\newpsobject{PST@Circle}{psdots}{linewidth=.001,linestyle=solid,dotstyle=o}
\newpsobject{PST@Triangle}{psdots}{linewidth=.001,linestyle=solid,dotstyle=triangle}
\newpsobject{PST@Pentagon}{psdots}{linewidth=.001,linestyle=solid,dotstyle=pentagon}
\newpsobject{PST@Fillsquare}{psdots}{linewidth=.001,linestyle=solid,dotstyle=square*}
\newpsobject{PST@Fillcircle}{psdots}{linewidth=.001,linestyle=solid,dotstyle=*}
\newpsobject{PST@Filltriangle}{psdots}{linewidth=.001,linestyle=solid,dotstyle=triangle*}
\newpsobject{PST@Fillpentagon}{psdots}{linewidth=.001,linestyle=solid,dotstyle=pentagon*}
\newpsobject{PST@Arrow}{psline}{linewidth=.001,linestyle=solid}
\catcode`@=12

\fi
\psset{unit=5.0in,xunit=5.0in,yunit=3.0in}
\pspicture(0.000000,0.000000)(1.000000,1.000000)
\ifx\nofigs\undefined
\catcode`@=11

\PST@Border(0.1010,0.0840)
(0.1160,0.0840)

\rput[r](0.0850,0.0840){0.0}
\PST@Border(0.1010,0.2103)
(0.1160,0.2103)

\rput[r](0.0850,0.2103){0.5}
\PST@Border(0.1010,0.3366)
(0.1160,0.3366)

\rput[r](0.0850,0.3366){1.0}
\PST@Border(0.1010,0.4629)
(0.1160,0.4629)

\rput[r](0.0850,0.4629){1.5}
\PST@Border(0.1010,0.5891)
(0.1160,0.5891)

\rput[r](0.0850,0.5891){2.0}
\PST@Border(0.1010,0.7154)
(0.1160,0.7154)

\rput[r](0.0850,0.7154){2.5}
\PST@Border(0.1010,0.8417)
(0.1160,0.8417)

\rput[r](0.0850,0.8417){3.0}
\PST@Border(0.1010,0.9680)
(0.1160,0.9680)

\rput[r](0.0850,0.9680){3.5}
\PST@Border(0.1010,0.0840)
(0.1010,0.1040)

\rput(0.1010,0.0420){0.1}
\PST@Border(0.1962,0.0840)
(0.1962,0.1040)

\rput(0.1962,0.0420){0.2}
\PST@Border(0.2914,0.0840)
(0.2914,0.1040)

\rput(0.2914,0.0420){0.3}
\PST@Border(0.3867,0.0840)
(0.3867,0.1040)

\rput(0.3867,0.0420){0.4}
\PST@Border(0.4819,0.0840)
(0.4819,0.1040)

\rput(0.4819,0.0420){0.5}
\PST@Border(0.5771,0.0840)
(0.5771,0.1040)

\rput(0.5771,0.0420){0.6}
\PST@Border(0.6723,0.0840)
(0.6723,0.1040)

\rput(0.6723,0.0420){0.7}
\PST@Border(0.7676,0.0840)
(0.7676,0.1040)

\rput(0.7676,0.0420){0.8}
\PST@Border(0.8628,0.0840)
(0.8628,0.1040)

\rput(0.8628,0.0420){0.9}
\PST@Border(0.9580,0.0840)
(0.9580,0.1040)

\rput(0.9580,0.0420){1.0}
\PST@Border(0.1010,0.9680)
(0.1010,0.0840)
(0.9580,0.0840)
(0.9580,0.9680)
(0.1010,0.9680)

\PST@Solid(0.1010,0.1688)
(0.1010,0.1688)
(0.1097,0.1761)
(0.1183,0.1835)
(0.1270,0.1909)
(0.1356,0.1983)
(0.1443,0.2056)
(0.1529,0.2130)
(0.1616,0.2204)
(0.1703,0.2277)
(0.1789,0.2351)
(0.1876,0.2425)
(0.1962,0.2498)
(0.2049,0.2572)
(0.2135,0.2646)
(0.2222,0.2719)
(0.2308,0.2793)
(0.2395,0.2867)
(0.2482,0.2941)
(0.2568,0.3014)
(0.2655,0.3088)
(0.2741,0.3162)
(0.2828,0.3235)
(0.2914,0.3309)
(0.3001,0.3383)
(0.3088,0.3456)
(0.3174,0.3530)
(0.3261,0.3604)
(0.3347,0.3678)
(0.3434,0.3751)
(0.3520,0.3825)
(0.3607,0.3899)
(0.3694,0.3972)
(0.3780,0.4046)
(0.3867,0.4120)
(0.3953,0.4193)
(0.4040,0.4267)
(0.4126,0.4341)
(0.4213,0.4415)
(0.4299,0.4488)
(0.4386,0.4562)
(0.4473,0.4636)
(0.4559,0.4709)
(0.4646,0.4783)
(0.4732,0.4857)
(0.4819,0.4930)
(0.4905,0.5004)
(0.4992,0.5078)
(0.5079,0.5152)
(0.5165,0.5225)
(0.5252,0.5299)
(0.5338,0.5373)
(0.5425,0.5446)
(0.5511,0.5520)
(0.5598,0.5594)
(0.5685,0.5667)
(0.5771,0.5741)
(0.5858,0.5815)
(0.5944,0.5889)
(0.6031,0.5962)
(0.6117,0.6036)
(0.6204,0.6110)
(0.6291,0.6183)
(0.6377,0.6257)
(0.6464,0.6331)
(0.6550,0.6404)
(0.6637,0.6478)
(0.6723,0.6552)
(0.6810,0.6625)
(0.6896,0.6699)
(0.6983,0.6773)
(0.7070,0.6847)
(0.7156,0.6920)
(0.7243,0.6994)
(0.7329,0.7068)
(0.7416,0.7141)
(0.7502,0.7215)
(0.7589,0.7289)
(0.7676,0.7362)
(0.7762,0.7436)
(0.7849,0.7510)
(0.7935,0.7584)
(0.8022,0.7657)
(0.8108,0.7731)
(0.8195,0.7805)
(0.8282,0.7878)
(0.8368,0.7952)
(0.8455,0.8026)
(0.8541,0.8099)
(0.8628,0.8173)
(0.8714,0.8247)
(0.8801,0.8321)
(0.8887,0.8394)
(0.8974,0.8468)
(0.9061,0.8542)
(0.9147,0.8615)
(0.9234,0.8689)
(0.9320,0.8763)
(0.9407,0.8836)
(0.9493,0.8910)
(0.9580,0.8984)

\PST@Dashed(0.9580,0.9031)
(0.9580,0.9066)

\PST@Dashed(0.9505,0.9031)
(0.9655,0.9031)

\PST@Dashed(0.9505,0.9066)
(0.9655,0.9066)

\PST@Dashed(0.8628,0.8096)
(0.8628,0.8132)

\PST@Dashed(0.8553,0.8096)
(0.8703,0.8096)

\PST@Dashed(0.8553,0.8132)
(0.8703,0.8132)

\PST@Dashed(0.7676,0.7313)
(0.7676,0.7349)

\PST@Dashed(0.7601,0.7313)
(0.7751,0.7313)

\PST@Dashed(0.7601,0.7349)
(0.7751,0.7349)

\PST@Dashed(0.6723,0.6556)
(0.6723,0.6591)

\PST@Dashed(0.6648,0.6556)
(0.6798,0.6556)

\PST@Dashed(0.6648,0.6591)
(0.6798,0.6591)

\PST@Dashed(0.5771,0.5750)
(0.5771,0.5780)

\PST@Dashed(0.5696,0.5750)
(0.5846,0.5750)

\PST@Dashed(0.5696,0.5780)
(0.5846,0.5780)

\PST@Dashed(0.4819,0.4891)
(0.4819,0.4922)

\PST@Dashed(0.4744,0.4891)
(0.4894,0.4891)

\PST@Dashed(0.4744,0.4922)
(0.4894,0.4922)

\PST@Dashed(0.3867,0.4083)
(0.3867,0.4113)

\PST@Dashed(0.3792,0.4083)
(0.3942,0.4083)

\PST@Dashed(0.3792,0.4113)
(0.3942,0.4113)

\PST@Dashed(0.2914,0.3300)
(0.2914,0.3330)

\PST@Dashed(0.2839,0.3300)
(0.2989,0.3300)

\PST@Dashed(0.2839,0.3330)
(0.2989,0.3330)

\PST@Dashed(0.1962,0.2492)
(0.1962,0.2522)

\PST@Dashed(0.1887,0.2492)
(0.2037,0.2492)

\PST@Dashed(0.1887,0.2522)
(0.2037,0.2522)

\PST@Dashed(0.1010,0.1684)
(0.1010,0.1714)

\PST@Dashed(0.0935,0.1684)
(0.1085,0.1684)

\PST@Dashed(0.0935,0.1714)
(0.1085,0.1714)

\PST@Diamond(0.9580,0.9049)
\PST@Diamond(0.8628,0.8114)
\PST@Diamond(0.7676,0.7331)
\PST@Diamond(0.6723,0.6573)
\PST@Diamond(0.5771,0.5765)
\PST@Diamond(0.4819,0.4906)
\PST@Diamond(0.3867,0.4098)
\PST@Diamond(0.2914,0.3315)
\PST@Diamond(0.1962,0.2507)
\PST@Diamond(0.1010,0.1699)
\PST@Border(0.1010,0.9680)
(0.1010,0.0840)
(0.9580,0.0840)
(0.9580,0.9680)
(0.1010,0.9680)

\catcode`@=12
\fi
\endpspicture

\end{figure}
\begin{figure}[p]\caption{Grafico dei residui della misura della differenza di potenziale a vari intervalli di lunghezza del filo, con corrente costante. Posizione in ascissa (metri), differenza di potenziale in ordinata (Volt).}
\centering
% GNUPLOT: LaTeX picture using PSTRICKS macros
% Define new PST objects, if not already defined
\ifx\PSTloaded\undefined
\def\PSTloaded{t}
\psset{arrowsize=.01 3.2 1.4 .3}
\psset{dotsize=.08}
\catcode`@=11

\newpsobject{PST@Border}{psline}{linewidth=.0015,linestyle=solid}
\newpsobject{PST@Axes}{psline}{linewidth=.0015,linestyle=dotted,dotsep=.004}
\newpsobject{PST@Solid}{psline}{linewidth=.0015,linestyle=solid}
\newpsobject{PST@Dashed}{psline}{linewidth=.0015,linestyle=dashed,dash=.01 .01}
\newpsobject{PST@Dotted}{psline}{linewidth=.0025,linestyle=dotted,dotsep=.008}
\newpsobject{PST@LongDash}{psline}{linewidth=.0015,linestyle=dashed,dash=.02 .01}
\newpsobject{PST@Diamond}{psdots}{linewidth=.001,linestyle=solid,dotstyle=square,dotangle=45}
\newpsobject{PST@Filldiamond}{psdots}{linewidth=.001,linestyle=solid,dotstyle=square*,dotangle=45}
\newpsobject{PST@Cross}{psdots}{linewidth=.001,linestyle=solid,dotstyle=+,dotangle=45}
\newpsobject{PST@Plus}{psdots}{linewidth=.001,linestyle=solid,dotstyle=+}
\newpsobject{PST@Square}{psdots}{linewidth=.001,linestyle=solid,dotstyle=square}
\newpsobject{PST@Circle}{psdots}{linewidth=.001,linestyle=solid,dotstyle=o}
\newpsobject{PST@Triangle}{psdots}{linewidth=.001,linestyle=solid,dotstyle=triangle}
\newpsobject{PST@Pentagon}{psdots}{linewidth=.001,linestyle=solid,dotstyle=pentagon}
\newpsobject{PST@Fillsquare}{psdots}{linewidth=.001,linestyle=solid,dotstyle=square*}
\newpsobject{PST@Fillcircle}{psdots}{linewidth=.001,linestyle=solid,dotstyle=*}
\newpsobject{PST@Filltriangle}{psdots}{linewidth=.001,linestyle=solid,dotstyle=triangle*}
\newpsobject{PST@Fillpentagon}{psdots}{linewidth=.001,linestyle=solid,dotstyle=pentagon*}
\newpsobject{PST@Arrow}{psline}{linewidth=.001,linestyle=solid}
\catcode`@=12

\fi
\psset{unit=5.0in,xunit=5.0in,yunit=3.0in}
\pspicture(0.000000,0.000000)(1.000000,1.000000)
\ifx\nofigs\undefined
\catcode`@=11

\PST@Border(0.1490,0.0840)
(0.1640,0.0840)

\rput[r](0.1330,0.0840){-0.040}
\PST@Border(0.1490,0.1822)
(0.1640,0.1822)

\rput[r](0.1330,0.1822){-0.030}
\PST@Border(0.1490,0.2804)
(0.1640,0.2804)

\rput[r](0.1330,0.2804){-0.020}
\PST@Border(0.1490,0.3787)
(0.1640,0.3787)

\rput[r](0.1330,0.3787){-0.010}
\PST@Border(0.1490,0.4769)
(0.1640,0.4769)

\rput[r](0.1330,0.4769){0.000}
\PST@Border(0.1490,0.5751)
(0.1640,0.5751)

\rput[r](0.1330,0.5751){0.010}
\PST@Border(0.1490,0.6733)
(0.1640,0.6733)

\rput[r](0.1330,0.6733){0.020}
\PST@Border(0.1490,0.7716)
(0.1640,0.7716)

\rput[r](0.1330,0.7716){0.030}
\PST@Border(0.1490,0.8698)
(0.1640,0.8698)

\rput[r](0.1330,0.8698){0.040}
\PST@Border(0.1490,0.9680)
(0.1640,0.9680)

\rput[r](0.1330,0.9680){0.050}
\PST@Border(0.1490,0.0840)
(0.1490,0.1040)

\rput(0.1490,0.0420){0.00}
\PST@Border(0.2961,0.0840)
(0.2961,0.1040)

\rput(0.2961,0.0420){0.20}
\PST@Border(0.4432,0.0840)
(0.4432,0.1040)

\rput(0.4432,0.0420){0.40}
\PST@Border(0.5903,0.0840)
(0.5903,0.1040)

\rput(0.5903,0.0420){0.60}
\PST@Border(0.7374,0.0840)
(0.7374,0.1040)

\rput(0.7374,0.0420){0.80}
\PST@Border(0.8845,0.0840)
(0.8845,0.1040)

\rput(0.8845,0.0420){1.00}
\PST@Border(0.1490,0.9680)
(0.1490,0.0840)
(0.9580,0.0840)
(0.9580,0.9680)
(0.1490,0.9680)

\PST@Solid(0.8845,0.6192)
(0.8845,0.9239)

\PST@Solid(0.8770,0.6192)
(0.8920,0.6192)

\PST@Solid(0.8770,0.9239)
(0.8920,0.9239)

\PST@Solid(0.8109,0.1379)
(0.8109,0.4426)

\PST@Solid(0.8034,0.1379)
(0.8184,0.1379)

\PST@Solid(0.8034,0.4426)
(0.8184,0.4426)

\PST@Solid(0.7374,0.2459)
(0.7374,0.5507)

\PST@Solid(0.7299,0.2459)
(0.7449,0.2459)

\PST@Solid(0.7299,0.5507)
(0.7449,0.5507)

\PST@Solid(0.6638,0.4522)
(0.6638,0.7569)

\PST@Solid(0.6563,0.4522)
(0.6713,0.4522)

\PST@Solid(0.6563,0.7569)
(0.6713,0.7569)

\PST@Solid(0.5903,0.4620)
(0.5903,0.7668)

\PST@Solid(0.5828,0.4620)
(0.5978,0.4620)

\PST@Solid(0.5828,0.7668)
(0.5978,0.7668)

\PST@Solid(0.5167,0.2754)
(0.5167,0.5801)

\PST@Solid(0.5092,0.2754)
(0.5242,0.2754)

\PST@Solid(0.5092,0.5801)
(0.5242,0.5801)

\PST@Solid(0.4432,0.2852)
(0.4432,0.5900)

\PST@Solid(0.4357,0.2852)
(0.4507,0.2852)

\PST@Solid(0.4357,0.5900)
(0.4507,0.5900)

\PST@Solid(0.3696,0.3933)
(0.3696,0.6980)

\PST@Solid(0.3621,0.3933)
(0.3771,0.3933)

\PST@Solid(0.3621,0.6980)
(0.3771,0.6980)

\PST@Solid(0.2961,0.4031)
(0.2961,0.7078)

\PST@Solid(0.2886,0.4031)
(0.3036,0.4031)

\PST@Solid(0.2886,0.7078)
(0.3036,0.7078)

\PST@Solid(0.2225,0.4129)
(0.2225,0.7177)

\PST@Solid(0.2150,0.4129)
(0.2300,0.4129)

\PST@Solid(0.2150,0.7177)
(0.2300,0.7177)

\PST@Diamond(0.8845,0.7716)
\PST@Diamond(0.8109,0.2903)
\PST@Diamond(0.7374,0.3983)
\PST@Diamond(0.6638,0.6046)
\PST@Diamond(0.5903,0.6144)
\PST@Diamond(0.5167,0.4278)
\PST@Diamond(0.4432,0.4376)
\PST@Diamond(0.3696,0.5456)
\PST@Diamond(0.2961,0.5555)
\PST@Diamond(0.2225,0.5653)
\PST@Border(0.1490,0.9680)
(0.1490,0.0840)
(0.9580,0.0840)
(0.9580,0.9680)
(0.1490,0.9680)

\catcode`@=12
\fi
\endpspicture

\end{figure}
\begin{figure}[p]\caption{Tensione (Volt) in ordinata rispetto alla corrente (A) in un filo di rame di lunghezza 1 metro e diametro 0.150mm.}
\centering
% GNUPLOT: LaTeX picture using PSTRICKS macros
% Define new PST objects, if not already defined
\ifx\PSTloaded\undefined
\def\PSTloaded{t}
\psset{arrowsize=.01 3.2 1.4 .3}
\psset{dotsize=.08}
\catcode`@=11

\newpsobject{PST@Border}{psline}{linewidth=.0015,linestyle=solid}
\newpsobject{PST@Axes}{psline}{linewidth=.0015,linestyle=dotted,dotsep=.004}
\newpsobject{PST@Solid}{psline}{linewidth=.0015,linestyle=solid}
\newpsobject{PST@Dashed}{psline}{linewidth=.0015,linestyle=dashed,dash=.01 .01}
\newpsobject{PST@Dotted}{psline}{linewidth=.0025,linestyle=dotted,dotsep=.008}
\newpsobject{PST@LongDash}{psline}{linewidth=.0015,linestyle=dashed,dash=.02 .01}
\newpsobject{PST@Diamond}{psdots}{linewidth=.001,linestyle=solid,dotstyle=square,dotangle=45}
\newpsobject{PST@Filldiamond}{psdots}{linewidth=.001,linestyle=solid,dotstyle=square*,dotangle=45}
\newpsobject{PST@Cross}{psdots}{linewidth=.001,linestyle=solid,dotstyle=+,dotangle=45}
\newpsobject{PST@Plus}{psdots}{linewidth=.001,linestyle=solid,dotstyle=+}
\newpsobject{PST@Square}{psdots}{linewidth=.001,linestyle=solid,dotstyle=square}
\newpsobject{PST@Circle}{psdots}{linewidth=.001,linestyle=solid,dotstyle=o}
\newpsobject{PST@Triangle}{psdots}{linewidth=.001,linestyle=solid,dotstyle=triangle}
\newpsobject{PST@Pentagon}{psdots}{linewidth=.001,linestyle=solid,dotstyle=pentagon}
\newpsobject{PST@Fillsquare}{psdots}{linewidth=.001,linestyle=solid,dotstyle=square*}
\newpsobject{PST@Fillcircle}{psdots}{linewidth=.001,linestyle=solid,dotstyle=*}
\newpsobject{PST@Filltriangle}{psdots}{linewidth=.001,linestyle=solid,dotstyle=triangle*}
\newpsobject{PST@Fillpentagon}{psdots}{linewidth=.001,linestyle=solid,dotstyle=pentagon*}
\newpsobject{PST@Arrow}{psline}{linewidth=.001,linestyle=solid}
\catcode`@=12

\fi
\psset{unit=5.0in,xunit=5.0in,yunit=3.0in}
\pspicture(0.000000,0.000000)(1.000000,1.000000)
\ifx\nofigs\undefined
\catcode`@=11

\PST@Border(0.1170,0.0840)
(0.1320,0.0840)

\rput[r](0.1010,0.0840){0.00}
\PST@Border(0.1170,0.1724)
(0.1320,0.1724)

\rput[r](0.1010,0.1724){0.05}
\PST@Border(0.1170,0.2608)
(0.1320,0.2608)

\rput[r](0.1010,0.2608){0.10}
\PST@Border(0.1170,0.3492)
(0.1320,0.3492)

\rput[r](0.1010,0.3492){0.15}
\PST@Border(0.1170,0.4376)
(0.1320,0.4376)

\rput[r](0.1010,0.4376){0.20}
\PST@Border(0.1170,0.5260)
(0.1320,0.5260)

\rput[r](0.1010,0.5260){0.25}
\PST@Border(0.1170,0.6144)
(0.1320,0.6144)

\rput[r](0.1010,0.6144){0.30}
\PST@Border(0.1170,0.7028)
(0.1320,0.7028)

\rput[r](0.1010,0.7028){0.35}
\PST@Border(0.1170,0.7912)
(0.1320,0.7912)

\rput[r](0.1010,0.7912){0.40}
\PST@Border(0.1170,0.8796)
(0.1320,0.8796)

\rput[r](0.1010,0.8796){0.45}
\PST@Border(0.1170,0.9680)
(0.1320,0.9680)

\rput[r](0.1010,0.9680){0.50}
\PST@Border(0.1170,0.0840)
(0.1170,0.1040)

\rput(0.1170,0.0420){0.00}
\PST@Border(0.2104,0.0840)
(0.2104,0.1040)

\rput(0.2104,0.0420){0.05}
\PST@Border(0.3039,0.0840)
(0.3039,0.1040)

\rput(0.3039,0.0420){0.10}
\PST@Border(0.3973,0.0840)
(0.3973,0.1040)

\rput(0.3973,0.0420){0.15}
\PST@Border(0.4908,0.0840)
(0.4908,0.1040)

\rput(0.4908,0.0420){0.20}
\PST@Border(0.5842,0.0840)
(0.5842,0.1040)

\rput(0.5842,0.0420){0.25}
\PST@Border(0.6777,0.0840)
(0.6777,0.1040)

\rput(0.6777,0.0420){0.30}
\PST@Border(0.7711,0.0840)
(0.7711,0.1040)

\rput(0.7711,0.0420){0.35}
\PST@Border(0.8646,0.0840)
(0.8646,0.1040)

\rput(0.8646,0.0420){0.40}
\PST@Border(0.9580,0.0840)
(0.9580,0.1040)

\rput(0.9580,0.0420){0.45}
\PST@Border(0.1170,0.9680)
(0.1170,0.0840)
(0.9580,0.0840)
(0.9580,0.9680)
(0.1170,0.9680)

\PST@Solid(0.1357,0.0985)
(0.1357,0.0985)
(0.1440,0.1064)
(0.1523,0.1143)
(0.1606,0.1222)
(0.1689,0.1302)
(0.1772,0.1381)
(0.1855,0.1460)
(0.1938,0.1539)
(0.2021,0.1619)
(0.2104,0.1698)
(0.2188,0.1777)
(0.2271,0.1856)
(0.2354,0.1936)
(0.2437,0.2015)
(0.2520,0.2094)
(0.2603,0.2173)
(0.2686,0.2253)
(0.2769,0.2332)
(0.2852,0.2411)
(0.2935,0.2490)
(0.3018,0.2570)
(0.3101,0.2649)
(0.3184,0.2728)
(0.3267,0.2807)
(0.3350,0.2887)
(0.3433,0.2966)
(0.3516,0.3045)
(0.3600,0.3124)
(0.3683,0.3204)
(0.3766,0.3283)
(0.3849,0.3362)
(0.3932,0.3441)
(0.4015,0.3521)
(0.4098,0.3600)
(0.4181,0.3679)
(0.4264,0.3758)
(0.4347,0.3837)
(0.4430,0.3917)
(0.4513,0.3996)
(0.4596,0.4075)
(0.4679,0.4154)
(0.4762,0.4234)
(0.4845,0.4313)
(0.4929,0.4392)
(0.5012,0.4471)
(0.5095,0.4551)
(0.5178,0.4630)
(0.5261,0.4709)
(0.5344,0.4788)
(0.5427,0.4868)
(0.5510,0.4947)
(0.5593,0.5026)
(0.5676,0.5105)
(0.5759,0.5185)
(0.5842,0.5264)
(0.5925,0.5343)
(0.6008,0.5422)
(0.6091,0.5502)
(0.6174,0.5581)
(0.6258,0.5660)
(0.6341,0.5739)
(0.6424,0.5819)
(0.6507,0.5898)
(0.6590,0.5977)
(0.6673,0.6056)
(0.6756,0.6136)
(0.6839,0.6215)
(0.6922,0.6294)
(0.7005,0.6373)
(0.7088,0.6453)
(0.7171,0.6532)
(0.7254,0.6611)
(0.7337,0.6690)
(0.7420,0.6770)
(0.7503,0.6849)
(0.7587,0.6928)
(0.7670,0.7007)
(0.7753,0.7087)
(0.7836,0.7166)
(0.7919,0.7245)
(0.8002,0.7324)
(0.8085,0.7403)
(0.8168,0.7483)
(0.8251,0.7562)
(0.8334,0.7641)
(0.8417,0.7720)
(0.8500,0.7800)
(0.8583,0.7879)
(0.8666,0.7958)
(0.8749,0.8037)
(0.8832,0.8117)
(0.8916,0.8196)
(0.8999,0.8275)
(0.9082,0.8354)
(0.9165,0.8434)
(0.9248,0.8513)
(0.9331,0.8592)
(0.9414,0.8671)
(0.9497,0.8751)
(0.9580,0.8830)

\PST@Dashed(0.9580,0.8851)
(0.9580,0.8883)

\PST@Dashed(0.9505,0.8851)
(0.9655,0.8851)

\PST@Dashed(0.9505,0.8883)
(0.9655,0.8883)

\PST@Dashed(0.9393,0.8674)
(0.9393,0.8706)

\PST@Dashed(0.9318,0.8674)
(0.9468,0.8674)

\PST@Dashed(0.9318,0.8706)
(0.9468,0.8706)

\PST@Dashed(0.9038,0.8304)
(0.9038,0.8333)

\PST@Dashed(0.8963,0.8304)
(0.9113,0.8304)

\PST@Dashed(0.8963,0.8333)
(0.9113,0.8333)

\PST@Dashed(0.8664,0.7951)
(0.8664,0.7979)

\PST@Dashed(0.8589,0.7951)
(0.8739,0.7951)

\PST@Dashed(0.8589,0.7979)
(0.8739,0.7979)

\PST@Dashed(0.8272,0.7562)
(0.8272,0.7590)

\PST@Dashed(0.8197,0.7562)
(0.8347,0.7562)

\PST@Dashed(0.8197,0.7590)
(0.8347,0.7590)

\PST@Dashed(0.7898,0.7208)
(0.7898,0.7237)

\PST@Dashed(0.7823,0.7208)
(0.7973,0.7208)

\PST@Dashed(0.7823,0.7237)
(0.7973,0.7237)

\PST@Dashed(0.7543,0.6855)
(0.7543,0.6883)

\PST@Dashed(0.7468,0.6855)
(0.7618,0.6855)

\PST@Dashed(0.7468,0.6883)
(0.7618,0.6883)

\PST@Dashed(0.7150,0.6485)
(0.7150,0.6510)

\PST@Dashed(0.7075,0.6485)
(0.7225,0.6485)

\PST@Dashed(0.7075,0.6510)
(0.7225,0.6510)

\PST@Dashed(0.6777,0.6132)
(0.6777,0.6156)

\PST@Dashed(0.6702,0.6132)
(0.6852,0.6132)

\PST@Dashed(0.6702,0.6156)
(0.6852,0.6156)

\PST@Dashed(0.6384,0.5743)
(0.6384,0.5767)

\PST@Dashed(0.6309,0.5743)
(0.6459,0.5743)

\PST@Dashed(0.6309,0.5767)
(0.6459,0.5767)

\PST@Dashed(0.6029,0.5407)
(0.6029,0.5431)

\PST@Dashed(0.5954,0.5407)
(0.6104,0.5407)

\PST@Dashed(0.5954,0.5431)
(0.6104,0.5431)

\PST@Dashed(0.5655,0.5053)
(0.5655,0.5078)

\PST@Dashed(0.5580,0.5053)
(0.5730,0.5053)

\PST@Dashed(0.5580,0.5078)
(0.5730,0.5078)

\PST@Dashed(0.5282,0.4717)
(0.5282,0.4742)

\PST@Dashed(0.5207,0.4717)
(0.5357,0.4717)

\PST@Dashed(0.5207,0.4742)
(0.5357,0.4742)

\PST@Dashed(0.4908,0.4348)
(0.4908,0.4369)

\PST@Dashed(0.4833,0.4348)
(0.4983,0.4348)

\PST@Dashed(0.4833,0.4369)
(0.4983,0.4369)

\PST@Dashed(0.4534,0.3994)
(0.4534,0.4015)

\PST@Dashed(0.4459,0.3994)
(0.4609,0.3994)

\PST@Dashed(0.4459,0.4015)
(0.4609,0.4015)

\PST@Dashed(0.4160,0.3641)
(0.4160,0.3662)

\PST@Dashed(0.4085,0.3641)
(0.4235,0.3641)

\PST@Dashed(0.4085,0.3662)
(0.4235,0.3662)

\PST@Dashed(0.3786,0.3285)
(0.3786,0.3292)

\PST@Dashed(0.3711,0.3285)
(0.3861,0.3285)

\PST@Dashed(0.3711,0.3292)
(0.3861,0.3292)

\PST@Dashed(0.3600,0.3114)
(0.3600,0.3121)

\PST@Dashed(0.3525,0.3114)
(0.3675,0.3114)

\PST@Dashed(0.3525,0.3121)
(0.3675,0.3121)

\PST@Dashed(0.3413,0.2939)
(0.3413,0.2946)

\PST@Dashed(0.3338,0.2939)
(0.3488,0.2939)

\PST@Dashed(0.3338,0.2946)
(0.3488,0.2946)

\PST@Dashed(0.3226,0.2769)
(0.3226,0.2776)

\PST@Dashed(0.3151,0.2769)
(0.3301,0.2769)

\PST@Dashed(0.3151,0.2776)
(0.3301,0.2776)

\PST@Dashed(0.3002,0.2566)
(0.3002,0.2573)

\PST@Dashed(0.2927,0.2566)
(0.3077,0.2566)

\PST@Dashed(0.2927,0.2573)
(0.3077,0.2573)

\PST@Dashed(0.2684,0.2254)
(0.2684,0.2258)

\PST@Dashed(0.2609,0.2254)
(0.2759,0.2254)

\PST@Dashed(0.2609,0.2258)
(0.2759,0.2258)

\PST@Dashed(0.2291,0.1890)
(0.2291,0.1894)

\PST@Dashed(0.2216,0.1890)
(0.2366,0.1890)

\PST@Dashed(0.2216,0.1894)
(0.2366,0.1894)

\PST@Dashed(0.1918,0.1533)
(0.1918,0.1537)

\PST@Dashed(0.1843,0.1533)
(0.1993,0.1533)

\PST@Dashed(0.1843,0.1537)
(0.1993,0.1537)

\PST@Dashed(0.1731,0.1363)
(0.1731,0.1367)

\PST@Dashed(0.1656,0.1363)
(0.1806,0.1363)

\PST@Dashed(0.1656,0.1367)
(0.1806,0.1367)

\PST@Dashed(0.1562,0.1199)
(0.1562,0.1202)

\PST@Dashed(0.1487,0.1199)
(0.1637,0.1199)

\PST@Dashed(0.1487,0.1202)
(0.1637,0.1202)

\PST@Dashed(0.1357,0.0971)
(0.1357,0.0974)

\PST@Dashed(0.1282,0.0971)
(0.1432,0.0971)

\PST@Dashed(0.1282,0.0974)
(0.1432,0.0974)

\PST@Diamond(0.9580,0.8867)
\PST@Diamond(0.9393,0.8690)
\PST@Diamond(0.9038,0.8319)
\PST@Diamond(0.8664,0.7965)
\PST@Diamond(0.8272,0.7576)
\PST@Diamond(0.7898,0.7222)
\PST@Diamond(0.7543,0.6869)
\PST@Diamond(0.7150,0.6498)
\PST@Diamond(0.6777,0.6144)
\PST@Diamond(0.6384,0.5755)
\PST@Diamond(0.6029,0.5419)
\PST@Diamond(0.5655,0.5066)
\PST@Diamond(0.5282,0.4730)
\PST@Diamond(0.4908,0.4358)
\PST@Diamond(0.4534,0.4005)
\PST@Diamond(0.4160,0.3651)
\PST@Diamond(0.3786,0.3289)
\PST@Diamond(0.3600,0.3117)
\PST@Diamond(0.3413,0.2942)
\PST@Diamond(0.3226,0.2772)
\PST@Diamond(0.3002,0.2569)
\PST@Diamond(0.2684,0.2256)
\PST@Diamond(0.2291,0.1892)
\PST@Diamond(0.1918,0.1535)
\PST@Diamond(0.1731,0.1365)
\PST@Diamond(0.1562,0.1201)
\PST@Diamond(0.1357,0.0973)
\PST@Border(0.1170,0.9680)
(0.1170,0.0840)
(0.9580,0.0840)
(0.9580,0.9680)
(0.1170,0.9680)

\catcode`@=12
\fi
\endpspicture

\end{figure}
\begin{figure}[p]\caption{Grafico dei residui relativo al precedente. La forma dei residui indica un andamento tutt'altro che casuale. L'effetto è dovuto al riscaldamento del filo sottoposto ad alte intensità di corrente e al conseguente aumento di resistività.}
\centering
% GNUPLOT: LaTeX picture using PSTRICKS macros
% Define new PST objects, if not already defined
\ifx\PSTloaded\undefined
\def\PSTloaded{t}
\psset{arrowsize=.01 3.2 1.4 .3}
\psset{dotsize=.08}
\catcode`@=11

\newpsobject{PST@Border}{psline}{linewidth=.0015,linestyle=solid}
\newpsobject{PST@Axes}{psline}{linewidth=.0015,linestyle=dotted,dotsep=.004}
\newpsobject{PST@Solid}{psline}{linewidth=.0015,linestyle=solid}
\newpsobject{PST@Dashed}{psline}{linewidth=.0015,linestyle=dashed,dash=.01 .01}
\newpsobject{PST@Dotted}{psline}{linewidth=.0025,linestyle=dotted,dotsep=.008}
\newpsobject{PST@LongDash}{psline}{linewidth=.0015,linestyle=dashed,dash=.02 .01}
\newpsobject{PST@Diamond}{psdots}{linewidth=.001,linestyle=solid,dotstyle=square,dotangle=45}
\newpsobject{PST@Filldiamond}{psdots}{linewidth=.001,linestyle=solid,dotstyle=square*,dotangle=45}
\newpsobject{PST@Cross}{psdots}{linewidth=.001,linestyle=solid,dotstyle=+,dotangle=45}
\newpsobject{PST@Plus}{psdots}{linewidth=.001,linestyle=solid,dotstyle=+}
\newpsobject{PST@Square}{psdots}{linewidth=.001,linestyle=solid,dotstyle=square}
\newpsobject{PST@Circle}{psdots}{linewidth=.001,linestyle=solid,dotstyle=o}
\newpsobject{PST@Triangle}{psdots}{linewidth=.001,linestyle=solid,dotstyle=triangle}
\newpsobject{PST@Pentagon}{psdots}{linewidth=.001,linestyle=solid,dotstyle=pentagon}
\newpsobject{PST@Fillsquare}{psdots}{linewidth=.001,linestyle=solid,dotstyle=square*}
\newpsobject{PST@Fillcircle}{psdots}{linewidth=.001,linestyle=solid,dotstyle=*}
\newpsobject{PST@Filltriangle}{psdots}{linewidth=.001,linestyle=solid,dotstyle=triangle*}
\newpsobject{PST@Fillpentagon}{psdots}{linewidth=.001,linestyle=solid,dotstyle=pentagon*}
\newpsobject{PST@Arrow}{psline}{linewidth=.001,linestyle=solid}
\catcode`@=12

\fi
\psset{unit=5.0in,xunit=5.0in,yunit=3.0in}
\pspicture(0.000000,0.000000)(1.000000,1.000000)
\ifx\nofigs\undefined
\catcode`@=11

\PST@Border(0.1490,0.0840)
(0.1640,0.0840)

\rput[r](0.1330,0.0840){-0.003}
\PST@Border(0.1490,0.2103)
(0.1640,0.2103)

\rput[r](0.1330,0.2103){-0.002}
\PST@Border(0.1490,0.3366)
(0.1640,0.3366)

\rput[r](0.1330,0.3366){-0.001}
\PST@Border(0.1490,0.4629)
(0.1640,0.4629)

\rput[r](0.1330,0.4629){0.000}
\PST@Border(0.1490,0.5891)
(0.1640,0.5891)

\rput[r](0.1330,0.5891){0.001}
\PST@Border(0.1490,0.7154)
(0.1640,0.7154)

\rput[r](0.1330,0.7154){0.002}
\PST@Border(0.1490,0.8417)
(0.1640,0.8417)

\rput[r](0.1330,0.8417){0.003}
\PST@Border(0.1490,0.9680)
(0.1640,0.9680)

\rput[r](0.1330,0.9680){0.004}
\PST@Border(0.1490,0.0840)
(0.1490,0.1040)

\rput(0.1490,0.0420){0.00}
\PST@Border(0.3108,0.0840)
(0.3108,0.1040)

\rput(0.3108,0.0420){0.10}
\PST@Border(0.4726,0.0840)
(0.4726,0.1040)

\rput(0.4726,0.0420){0.20}
\PST@Border(0.6344,0.0840)
(0.6344,0.1040)

\rput(0.6344,0.0420){0.30}
\PST@Border(0.7962,0.0840)
(0.7962,0.1040)

\rput(0.7962,0.0420){0.40}
\PST@Border(0.9580,0.0840)
(0.9580,0.1040)

\rput(0.9580,0.0420){0.50}
\PST@Border(0.1490,0.9680)
(0.1490,0.0840)
(0.9580,0.0840)
(0.9580,0.9680)
(0.1490,0.9680)

\PST@Solid(0.8771,0.6265)
(0.8771,0.8801)

\PST@Solid(0.8696,0.6265)
(0.8846,0.6265)

\PST@Solid(0.8696,0.8801)
(0.8846,0.8801)

\PST@Solid(0.8609,0.6366)
(0.8609,0.8902)

\PST@Solid(0.8534,0.6366)
(0.8684,0.6366)

\PST@Solid(0.8534,0.8902)
(0.8684,0.8902)

\PST@Solid(0.8302,0.4033)
(0.8302,0.6568)

\PST@Solid(0.8227,0.4033)
(0.8377,0.4033)

\PST@Solid(0.8227,0.6568)
(0.8377,0.6568)

\PST@Solid(0.7978,0.4235)
(0.7978,0.6770)

\PST@Solid(0.7903,0.4235)
(0.8053,0.4235)

\PST@Solid(0.7903,0.6770)
(0.8053,0.6770)

\PST@Solid(0.7638,0.3184)
(0.7638,0.5720)

\PST@Solid(0.7563,0.3184)
(0.7713,0.3184)

\PST@Solid(0.7563,0.5720)
(0.7713,0.5720)

\PST@Solid(0.7315,0.3386)
(0.7315,0.5922)

\PST@Solid(0.7240,0.3386)
(0.7390,0.3386)

\PST@Solid(0.7240,0.5922)
(0.7390,0.5922)

\PST@Solid(0.7007,0.2315)
(0.7007,0.4851)

\PST@Solid(0.6932,0.2315)
(0.7082,0.2315)

\PST@Solid(0.6932,0.4851)
(0.7082,0.4851)

\PST@Solid(0.6668,0.2527)
(0.6668,0.5063)

\PST@Solid(0.6593,0.2527)
(0.6743,0.2527)

\PST@Solid(0.6593,0.5063)
(0.6743,0.5063)

\PST@Solid(0.6344,0.2729)
(0.6344,0.5265)

\PST@Solid(0.6269,0.2729)
(0.6419,0.2729)

\PST@Solid(0.6269,0.5265)
(0.6419,0.5265)

\PST@Solid(0.6004,0.1679)
(0.6004,0.4214)

\PST@Solid(0.5929,0.1679)
(0.6079,0.1679)

\PST@Solid(0.5929,0.4214)
(0.6079,0.4214)

\PST@Solid(0.5697,0.1871)
(0.5697,0.4406)

\PST@Solid(0.5622,0.1871)
(0.5772,0.1871)

\PST@Solid(0.5622,0.4406)
(0.5772,0.4406)

\PST@Solid(0.5373,0.2073)
(0.5373,0.4608)

\PST@Solid(0.5298,0.2073)
(0.5448,0.2073)

\PST@Solid(0.5298,0.4608)
(0.5448,0.4608)

\PST@Solid(0.5050,0.3537)
(0.5050,0.6073)

\PST@Solid(0.4975,0.3537)
(0.5125,0.3537)

\PST@Solid(0.4975,0.6073)
(0.5125,0.6073)

\PST@Solid(0.4726,0.2477)
(0.4726,0.5012)

\PST@Solid(0.4651,0.2477)
(0.4801,0.2477)

\PST@Solid(0.4651,0.5012)
(0.4801,0.5012)

\PST@Solid(0.4402,0.2679)
(0.4402,0.5215)

\PST@Solid(0.4327,0.2679)
(0.4477,0.2679)

\PST@Solid(0.4327,0.5215)
(0.4477,0.5215)

\PST@Solid(0.4079,0.2881)
(0.4079,0.5417)

\PST@Solid(0.4004,0.2881)
(0.4154,0.2881)

\PST@Solid(0.4004,0.5417)
(0.4154,0.5417)

\PST@Solid(0.3755,0.2451)
(0.3755,0.4987)

\PST@Solid(0.3680,0.2451)
(0.3830,0.2451)

\PST@Solid(0.3680,0.4987)
(0.3830,0.4987)

\PST@Solid(0.3593,0.2931)
(0.3593,0.5467)

\PST@Solid(0.3518,0.2931)
(0.3668,0.2931)

\PST@Solid(0.3518,0.5467)
(0.3668,0.5467)

\PST@Solid(0.3432,0.3159)
(0.3432,0.5694)

\PST@Solid(0.3357,0.3159)
(0.3507,0.3159)

\PST@Solid(0.3357,0.5694)
(0.3507,0.5694)

\PST@Solid(0.3270,0.3765)
(0.3270,0.6301)

\PST@Solid(0.3195,0.3765)
(0.3345,0.3765)

\PST@Solid(0.3195,0.6301)
(0.3345,0.6301)

\PST@Solid(0.3076,0.4517)
(0.3076,0.7053)

\PST@Solid(0.3001,0.4517)
(0.3151,0.4517)

\PST@Solid(0.3001,0.7053)
(0.3151,0.7053)

\PST@Solid(0.2801,0.3805)
(0.2801,0.6341)

\PST@Solid(0.2726,0.3805)
(0.2876,0.3805)

\PST@Solid(0.2726,0.6341)
(0.2876,0.6341)

\PST@Solid(0.2461,0.4523)
(0.2461,0.7058)

\PST@Solid(0.2386,0.4523)
(0.2536,0.4523)

\PST@Solid(0.2386,0.7058)
(0.2536,0.7058)

\PST@Solid(0.2137,0.4472)
(0.2137,0.7008)

\PST@Solid(0.2062,0.4472)
(0.2212,0.4472)

\PST@Solid(0.2062,0.7008)
(0.2212,0.7008)

\PST@Solid(0.1975,0.5078)
(0.1975,0.7614)

\PST@Solid(0.1900,0.5078)
(0.2050,0.5078)

\PST@Solid(0.1900,0.7614)
(0.2050,0.7614)

\PST@Solid(0.1830,0.4790)
(0.1830,0.7326)

\PST@Solid(0.1755,0.4790)
(0.1905,0.4790)

\PST@Solid(0.1755,0.7326)
(0.1905,0.7326)

\PST@Solid(0.1652,0.2502)
(0.1652,0.5038)

\PST@Solid(0.1577,0.2502)
(0.1727,0.2502)

\PST@Solid(0.1577,0.5038)
(0.1727,0.5038)

\PST@Diamond(0.8771,0.7533)
\PST@Diamond(0.8609,0.7634)
\PST@Diamond(0.8302,0.5300)
\PST@Diamond(0.7978,0.5502)
\PST@Diamond(0.7638,0.4452)
\PST@Diamond(0.7315,0.4654)
\PST@Diamond(0.7007,0.3583)
\PST@Diamond(0.6668,0.3795)
\PST@Diamond(0.6344,0.3997)
\PST@Diamond(0.6004,0.2946)
\PST@Diamond(0.5697,0.3138)
\PST@Diamond(0.5373,0.3340)
\PST@Diamond(0.5050,0.4805)
\PST@Diamond(0.4726,0.3745)
\PST@Diamond(0.4402,0.3947)
\PST@Diamond(0.4079,0.4149)
\PST@Diamond(0.3755,0.3719)
\PST@Diamond(0.3593,0.4199)
\PST@Diamond(0.3432,0.4427)
\PST@Diamond(0.3270,0.5033)
\PST@Diamond(0.3076,0.5785)
\PST@Diamond(0.2801,0.5073)
\PST@Diamond(0.2461,0.5790)
\PST@Diamond(0.2137,0.5740)
\PST@Diamond(0.1975,0.6346)
\PST@Diamond(0.1830,0.6058)
\PST@Diamond(0.1652,0.3770)
\PST@Border(0.1490,0.9680)
(0.1490,0.0840)
(0.9580,0.0840)
(0.9580,0.9680)
(0.1490,0.9680)

\catcode`@=12
\fi
\endpspicture

\end{figure}
\end{document}
